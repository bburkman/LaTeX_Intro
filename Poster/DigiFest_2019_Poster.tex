\documentclass[final]{beamer}
  \mode<presentation>
  {
% you can chose your theme here:
\usetheme{Berlin}
% further beamerposter themes are available at 
% http://www-i6.informatik.rwth-aachen.de/~dreuw/latexbeamerposter.php
}
  \usepackage{type1cm}
  \usepackage{calc} 
  \usepackage{times}
  \usepackage{amsmath,amsthm, amssymb, latexsym}
	\usepackage{color}
	\usepackage{tikz}
	\usetikzlibrary{shadows}
	\usetikzlibrary{positioning}
\usepackage{adjustbox}	
%  \boldmath
  \usepackage[english]{babel}
  \usepackage[latin1]{inputenc}

\setbeamertemplate{itemize/enumerate body begin}{\normalsize}
\setbeamertemplate{itemize/enumerate subbody begin}{\normalsize}
\usepackage[orientation=portrait,size=custom,height=73.66,width=99.16,scale=1.1,debug]{beamerposter}
%\usepackage[orientation=portrait,size=custom,height=91.44,width=121.92,scale=1.352,debug]{beamerposter}
  \graphicspath{{figures/}}
	\setbeamertemplate{blocks}[rounded,shadow=false]


  \title[DigiFest]{DigiFest South 2019}
  \author[Burkman]{J. Bradford Burkman}
  \institute[LSMSA]{Louisiana School for Math, Science, and the Arts}

\setbeamertemplate{itemize item}{$\bullet$\ }
\setbeamertemplate{itemize subitem}{\tiny$\blacksquare$\ }
\setbeamertemplate{itemize subsubitem}{$\bullet$\ }

%\usepackage{amssymb}
%\renewcommand{\labelitemi}{\tiny$\blacksquare$}
%\renewcommand{\labelitemi}{$\bullet$}
%\renewcommand{\labelitemii}{$\cdot$}
%\renewcommand{\labelitemiii}{$\diamond$}
%\renewcommand{\labelitemiv}{$\ast$}

\setlength{\itemsep}{5mm}

  \begin{document}

%define some nice colors 
%	\definecolor{blue}{RGB}{25,74,188}
	\definecolor{blue}{RGB}{1,45,110} % Official
	\definecolor{gold}{RGB}{255,189,77}
%	\definecolor{gold}{RGB}{169,130,67} % Official

%	\definecolor{darkgreen}{RGB}{0,122,64}
%	\definecolor{lightgreen}{RGB}{89,184,130}
	\definecolor{darkgreen}{RGB}{128,0,128}
	\definecolor{lightgreen}{RGB}{128,0,128}
%	\colorlet{lightgreen}{white!50!darkgreen}	

%	\begin{Large}
\setbeamercolor{block body}{bg=white} 
\setbeamercolor{block body alerted}{fg=black, bg=white} 
\setbeamercolor{block body example}{fg=black, bg=white} 
\setbeamercolor{block title}{fg=black, bg=white} 
%\setbeamercolor{exampleblock body}{bg=white} 
  \begin{frame}[plain]{}

% Frame
\begin{tikzpicture}[remember picture,overlay]
% \fill [blue]
% (current page.south west) rectangle (current page.north east);
\draw [line width=4cm,rounded corners=3.5cm, blue] (current page.south west) rectangle (current page.north east);
\end{tikzpicture}


% Title
	\begin{tikzpicture}[remember picture, overlay]
		\node [xshift=4cm,yshift=-2.5cm] at (current page.north west)
		[below right]
		{
%		     \includegraphics[height=8cm]{LSMSA-Crest.jpg}
		     \includegraphics[height=8cm]{LSMSACrest_solid_blue.png}
		};
		\node [xshift=15cm,yshift=-3.5cm] at (current page.north west)
		[text width=78cm,fill=gold,rounded corners=1.5cm, below right,drop shadow]
		{
			\vskip 0.5cm
			\color{black}
			\Huge 
			\centerline{
				Census \ Tracts in \ Bossier \ Parish: \  Map \ Making \ is \  Both \ Science \ and \ Art
				}
			\Large
			\centerline{
				\vrule width 0pt height 2cm depth .5cm
				\Large
				Brad Burkman \qquad Louisiana School for Math, Science, and the Arts
			}


		};
	\end{tikzpicture}

	\vskip 7cm

	\begin{columns}[t]
	\begin{column}{5mm}
	\end{column}
	\begin{column}{.3\linewidth}  {\color{white} xxx}

% How to Make a Map

	\begin{block}{}
		\begin{tikzpicture}
			\node (0,0) [text width=28cm,fill=blue,rounded corners=.5cm, below right,drop shadow]
			{
				\Large
				\centerline{
				\vrule width 0pt height 40pt depth 12pt
				\color{gold}
				Steps \ in \ Making \ a \ Map: \  Science \ or \ Art?
				}
			};
		\end{tikzpicture}

		\vskip 12pt

\begin{tikzpicture}[x=10mm, y=15mm]
	\path (1,1) node [rotate=45] {Science};
	\path (5,1) node [rotate=45] {Art};
	\foreach \i in {1,2,3,4,5}
		\path (\i,0) node {\i};
	\foreach \i in {-1, -2.5, -4, -5, -6, -7, -8, -9.5, -11, -12, -13}{
		\draw [thin, dashed] (0.5,\i) -- (5.5,\i);
		\foreach \j in {1,2,3,4,5}
			\draw [thin] (\j, \i-0.1) -- (\j, \i+0.1);
	}

	\path (6,-1) node [right] {Decide what story you want your map to tell.};
	\fill (5,-1) circle (9pt);

	\path (6,-2.5) node [right] {
		\begin{tabular}{@{}l}
			Find some relevant data, and decide whether \cr
			the data is accurate, up to date, and unbiased.\cr
		\end{tabular}
		};
	\fill (3,-2.5) circle (9pt);
	
	\path (6,-4) node [right] {Plot the points.};
	\fill (1,-4) circle (9pt);

	\path (6,-5) node [right] {Choose appropriate colors for the polygons.};
	\fill (4,-5) circle (9pt);

	\path (7,-6) node [right] {What do you want the colors to represent?};
	\fill (5,-6) circle (9pt);

	\path (7,-7) node [right] {How many colors do you need?};
	\fill (2,-7) circle (9pt);
	
	\path (7,-8) node [right] {Can most viewers easily tell the colors apart?};
	\fill (2,-8) circle (9pt);
	
	\path (7,-9.5) node [right] {
		\begin{tabular}{@{}l}
			Can a viewer with one of several common \cr
			color blindnesses easily tell the colors apart?\cr
		\end{tabular}
		};
	\fill (1,-9.5) circle (9pt);
	
	\path (7,-11) node [right] {Is the color scheme aesthetically pleasing?};
	\fill (5,-11) circle (9pt);
	
	\path (6,-12) node [right] {Choose appropriate symbols (colors and sizes).};
	\fill (4,-12) circle (9pt);
	
	\path (6,-13) node [right] {Choose appropriate text (colors, fonts, and sizes).};
	\fill (2,-13) circle (9pt);
\end{tikzpicture}


	\end{block}

% Plot Some Points	

	\begin{block}{}
	
		\begin{tikzpicture}
			\node (0,0) [text width=28cm,fill=gold,rounded corners=.5cm, below right,drop shadow]
			{
				\Large
				\centerline{
				\vrule width 0pt height 40pt depth 12pt
				Plot \ Some \ Points \ (194 \ Vertices)
				}
			};
		\end{tikzpicture}

		\

		These maps show census tracts in Bossier Parish.  A census tract is a geographic area with 1000-8000 people, and is the smallest area on which the Census Bureau publicly reports its data.  The figure below is Census Tract 106.02, Bossier Parish, Louisiana, where you are today.  The boundaries are defined as a polygon with vertices given as longitude and latitude.  

\

		\begin{tikzpicture}[x=3.0mm, y=3.45mm]
	\fill (1.054000,-24.647000) circle (0.4);
	\fill (1.075000,-24.670000) circle (0.4);
	\fill (1.243000,-24.875000) circle (0.4);
	\fill (1.378000,-25.058000) circle (0.4);
	\fill (1.426000,-25.129000) circle (0.4);
	\fill (1.619000,-25.454000) circle (0.4);
	\fill (2.288000,-26.582000) circle (0.4);
	\fill (2.397000,-26.766000) circle (0.4);
	\fill (2.723000,-27.318000) circle (0.4);
	\fill (2.831000,-27.501000) circle (0.4);
	\fill (2.972000,-27.739000) circle (0.4);
	\fill (3.186000,-28.102000) circle (0.4);
	\fill (3.279000,-28.271000) circle (0.4);
	\fill (3.367000,-28.468000) circle (0.4);
	\fill (3.460000,-28.726000) circle (0.4);
	\fill (2.943000,-28.865000) circle (0.4);
	\fill (1.392000,-29.281000) circle (0.4);
	\fill (0.874000,-29.419000) circle (0.4);
	\fill (0.515000,-29.516000) circle (0.4);
	\fill (0.202000,-29.599000) circle (0.4);
	\fill (-0.565000,-29.805000) circle (0.4);
	\fill (-0.925000,-29.901000) circle (0.4);
	\fill (-1.175000,-29.969000) circle (0.4);
	\fill (-1.925000,-30.170000) circle (0.4);
	\fill (-2.176000,-30.237000) circle (0.4);
	\fill (-3.290000,-30.535000) circle (0.4);
	\fill (-3.492000,-30.589000) circle (0.4);
	\fill (-3.939000,-30.709000) circle (0.4);
	\fill (-5.581000,-31.150000) circle (0.4);
	\fill (-6.259000,-31.331000) circle (0.4);
	\fill (-7.161000,-31.573000) circle (0.4);
	\fill (-8.026000,-31.805000) circle (0.4);
	\fill (-8.446000,-31.918000) circle (0.4);
	\fill (-9.307000,-32.149000) circle (0.4);
	\fill (-9.849000,-32.294000) circle (0.4);
	\fill (-10.295000,-32.413000) circle (0.4);
	\fill (-11.109000,-32.632000) circle (0.4);
	\fill (-12.107000,-32.899000) circle (0.4);
	\fill (-12.464000,-32.996000) circle (0.4);
	\fill (-13.536000,-33.281000) circle (0.4);
	\fill (-13.894000,-33.376000) circle (0.4);
	\fill (-14.151000,-33.441000) circle (0.4);
	\fill (-14.488000,-33.524000) circle (0.4);
	\fill (-14.872000,-33.610000) circle (0.4);
	\fill (-14.927000,-33.622000) circle (0.4);
	\fill (-15.187000,-33.676000) circle (0.4);
	\fill (-15.407000,-33.720000) circle (0.4);
	\fill (-15.645000,-33.766000) circle (0.4);
	\fill (-16.034000,-33.836000) circle (0.4);
	\fill (-16.071000,-33.843000) circle (0.4);
	\fill (-16.294000,-33.878000) circle (0.4);
	\fill (-16.424000,-33.899000) circle (0.4);
	\fill (-16.450000,-33.903000) circle (0.4);
	\fill (-16.816000,-33.957000) circle (0.4);
	\fill (-16.921000,-33.972000) circle (0.4);
	\fill (-17.079000,-33.992000) circle (0.4);
	\fill (-17.181000,-34.006000) circle (0.4);
	\fill (-17.488000,-34.043000) circle (0.4);
	\fill (-17.591000,-34.055000) circle (0.4);
	\fill (-17.602000,-34.056000) circle (0.4);
	\fill (-17.971000,-34.094000) circle (0.4);
	\fill (-17.997000,-34.096000) circle (0.4);
	\fill (-18.392000,-34.131000) circle (0.4);
	\fill (-18.788000,-34.160000) circle (0.4);
	\fill (-19.116000,-34.178000) circle (0.4);
	\fill (-19.499000,-34.196000) circle (0.4);
	\fill (-19.601000,-34.150000) circle (0.4);
	\fill (-19.739000,-34.087000) circle (0.4);
	\fill (-19.908000,-34.015000) circle (0.4);
	\fill (-20.012000,-33.970000) circle (0.4);
	\fill (-19.693000,-33.501000) circle (0.4);
	\fill (-19.650000,-33.436000) circle (0.4);
	\fill (-19.250000,-32.718000) circle (0.4);
	\fill (-18.985000,-32.139000) circle (0.4);
	\fill (-18.935000,-31.984000) circle (0.4);
	\fill (-18.764000,-31.443000) circle (0.4);
	\fill (-18.666000,-30.984000) circle (0.4);
	\fill (-18.591000,-30.630000) circle (0.4);
	\fill (-18.550000,-29.839000) circle (0.4);
	\fill (-18.543000,-29.583000) circle (0.4);
	\fill (-18.535000,-29.247000) circle (0.4);
	\fill (-18.545000,-29.113000) circle (0.4);
	\fill (-18.554000,-28.990000) circle (0.4);
	\fill (-18.586000,-28.694000) circle (0.4);
	\fill (-18.601000,-28.565000) circle (0.4);
	\fill (-18.614000,-28.451000) circle (0.4);
	\fill (-18.743000,-27.880000) circle (0.4);
	\fill (-18.859000,-27.465000) circle (0.4);
	\fill (-18.860000,-27.463000) circle (0.4);
	\fill (-19.011000,-27.069000) circle (0.4);
	\fill (-19.168000,-26.692000) circle (0.4);
	\fill (-19.214000,-26.575000) circle (0.4);
	\fill (-19.339000,-26.262000) circle (0.4);
	\fill (-19.399000,-26.100000) circle (0.4);
	\fill (-19.613000,-25.576000) circle (0.4);
	\fill (-19.673000,-25.417000) circle (0.4);
	\fill (-19.806000,-25.083000) circle (0.4);
	\fill (-19.888000,-24.881000) circle (0.4);
	\fill (-19.928000,-24.766000) circle (0.4);
	\fill (-20.002000,-24.585000) circle (0.4);
	\fill (-20.129000,-24.275000) circle (0.4);
	\fill (-20.211000,-24.076000) circle (0.4);
	\fill (-20.506000,-23.341000) circle (0.4);
	\fill (-20.608000,-23.088000) circle (0.4);
	\fill (-20.631000,-23.028000) circle (0.4);
	\fill (-20.706000,-22.834000) circle (0.4);
	\fill (-20.885000,-22.373000) circle (0.4);
	\fill (-20.938000,-22.251000) circle (0.4);
	\fill (-20.990000,-22.131000) circle (0.4);
	\fill (-21.018000,-22.057000) circle (0.4);
	\fill (-21.052000,-21.968000) circle (0.4);
	\fill (-21.173000,-21.684000) circle (0.4);
	\fill (-21.186000,-21.644000) circle (0.4);
	\fill (-21.222000,-20.031000) circle (0.4);
	\fill (-21.201000,-19.850000) circle (0.4);
	\fill (-21.038000,-19.237000) circle (0.4);
	\fill (-20.947000,-18.996000) circle (0.4);
	\fill (-20.852000,-18.745000) circle (0.4);
	\fill (-20.834000,-18.694000) circle (0.4);
	\fill (-20.547000,-18.200000) circle (0.4);
	\fill (-20.441000,-18.055000) circle (0.4);
	\fill (-20.283000,-17.838000) circle (0.4);
	\fill (-20.277000,-17.829000) circle (0.4);
	\fill (-20.225000,-17.779000) circle (0.4);
	\fill (-20.047000,-17.602000) circle (0.4);
	\fill (-19.989000,-17.543000) circle (0.4);
	\fill (-19.674000,-17.214000) circle (0.4);
	\fill (-19.205000,-16.831000) circle (0.4);
	\fill (-19.018000,-16.698000) circle (0.4);
	\fill (-18.753000,-16.505000) circle (0.4);
	\fill (-18.215000,-16.214000) circle (0.4);
	\fill (-17.908000,-16.058000) circle (0.4);
	\fill (-17.256000,-15.774000) circle (0.4);
	\fill (-15.511000,-15.077000) circle (0.4);
	\fill (-14.428000,-14.642000) circle (0.4);
	\fill (-14.312000,-14.597000) circle (0.4);
	\fill (-13.578000,-14.310000) circle (0.4);
	\fill (-12.174000,-13.760000) circle (0.4);
	\fill (-11.382000,-13.442000) circle (0.4);
	\fill (-10.652000,-13.147000) circle (0.4);
	\fill (-10.097000,-12.924000) circle (0.4);
	\fill (-9.503000,-12.684000) circle (0.4);
	\fill (-8.429000,-12.260000) circle (0.4);
	\fill (-7.873000,-12.040000) circle (0.4);
	\fill (-7.860000,-12.131000) circle (0.4);
	\fill (-7.822000,-12.404000) circle (0.4);
	\fill (-7.810000,-12.494000) circle (0.4);
	\fill (-7.774000,-12.753000) circle (0.4);
	\fill (-7.667000,-13.527000) circle (0.4);
	\fill (-7.632000,-13.785000) circle (0.4);
	\fill (-7.622000,-13.856000) circle (0.4);
	\fill (-7.573000,-14.128000) circle (0.4);
	\fill (-7.565000,-14.157000) circle (0.4);
	\fill (-7.521000,-14.330000) circle (0.4);
	\fill (-7.458000,-14.528000) circle (0.4);
	\fill (-7.382000,-14.724000) circle (0.4);
	\fill (-7.294000,-14.917000) circle (0.4);
	\fill (-7.194000,-15.105000) circle (0.4);
	\fill (-7.137000,-15.199000) circle (0.4);
	\fill (-7.083000,-15.288000) circle (0.4);
	\fill (-6.928000,-15.513000) circle (0.4);
	\fill (-6.792000,-15.702000) circle (0.4);
	\fill (-6.384000,-16.269000) circle (0.4);
	\fill (-6.249000,-16.457000) circle (0.4);
	\fill (-6.071000,-16.704000) circle (0.4);
	\fill (-5.540000,-17.444000) circle (0.4);
	\fill (-5.363000,-17.689000) circle (0.4);
	\fill (-5.227000,-17.880000) circle (0.4);
	\fill (-4.819000,-18.447000) circle (0.4);
	\fill (-4.683000,-18.636000) circle (0.4);
	\fill (-4.466000,-18.933000) circle (0.4);
	\fill (-4.341000,-19.104000) circle (0.4);
	\fill (-4.167000,-19.322000) circle (0.4);
	\fill (-3.985000,-19.534000) circle (0.4);
	\fill (-3.758000,-19.775000) circle (0.4);
	\fill (-3.709000,-19.828000) circle (0.4);
	\fill (-3.502000,-20.037000) circle (0.4);
	\fill (-3.265000,-20.276000) circle (0.4);
	\fill (-2.557000,-20.990000) circle (0.4);
	\fill (-2.321000,-21.228000) circle (0.4);
	\fill (-2.152000,-21.399000) circle (0.4);
	\fill (-1.872000,-21.682000) circle (0.4);
	\fill (-0.525000,-23.042000) circle (0.4);
	\fill (-0.223000,-23.345000) circle (0.4);
	\fill (-0.076000,-23.493000) circle (0.4);
	\fill (0.010000,-23.580000) circle (0.4);
	\fill (0.268000,-23.841000) circle (0.4);
	\fill (0.354000,-23.927000) circle (0.4);
	\fill (0.438000,-24.011000) circle (0.4);
	\fill (0.687000,-24.263000) circle (0.4);
	\fill (0.770000,-24.346000) circle (0.4);
	\fill (0.899000,-24.478000) circle (0.4);
	\fill (1.017000,-24.606000) circle (0.4);
	\fill (1.054000,-24.647000) circle (0.4);
	%\node () at (-11.264376,-24.586624) {194 Vertices};
	\draw [] (1.619000, -25.454000) -- (10.822188, -25.887688) node [right] {$(-93.698,32.524)$};
	\draw [] (-7.161000, -31.573000) -- (10.822188, -35.066188) node [right] {$(-93.707,32.518)$};
	\draw [] (-17.079000, -33.992000) -- (-26.200812, -38.694688) node [left] {$(-93.717,32.516)$};
	\draw [] (-18.535000, -29.247000) -- (-26.200812, -31.577188) node [left] {$(-93.718,32.520)$};
	\draw [] (-20.706000, -22.834000) -- (-26.200812, -21.957688) node [left] {$(-93.720,32.527)$};
	\draw [] (-18.215000, -16.214000) -- (-26.200812, -12.027688) node [left] {$(-93.718,32.533)$};
	\draw [] (-7.382000, -14.724000) -- (10.822188, -9.792688) node [right] {$(-93.707,32.535)$};
	\draw [] (-2.152000, -21.399000) -- (10.822188, -19.805188) node [right] {$(-93.702,32.528)$};
	\end{tikzpicture}

		
\

	Geography Challenge:  Why are the longitudes (the $x$-values) negative?	
	\end{block}




	
	\end{column}
%%%%%%%%%%%%%%%%%
%
% Second Column
%
%%%%%%%%%%%%%%%	
	\begin{column}{.3\linewidth}  {\color{white} xxx}
	

% Connect the Dots
	\begin{block}{}
		\begin{tikzpicture}
			\node (0,0) [text width=28cm,fill=gold,rounded corners=.5cm, below right,drop shadow]
			{
				\centerline{
				\vrule width 0pt height 40pt depth 12pt
				\Large
				Connect \ the \ Dots \ and \ Add \ Labels
				}
			};
		\end{tikzpicture}

		\vskip -12pt
		\begin{center}
		\begin{tabular}{p{0.1in}p{3in}p{2in}p{2in}}		
		&
		\adjustbox{valign=T}{
			\begin{tikzpicture}[x=3.0mm, y=3.45mm]
	\filldraw [draw=black, ultra thick, fill=gold]
		(1.054000,-24.647000) -- 
		(1.075000,-24.670000) -- 
		(1.243000,-24.875000) -- 
		(1.378000,-25.058000) -- 
		(1.426000,-25.129000) -- 
		(1.619000,-25.454000) -- 
		(2.288000,-26.582000) -- 
		(2.397000,-26.766000) -- 
		(2.723000,-27.318000) -- 
		(2.831000,-27.501000) -- 
		(2.972000,-27.739000) -- 
		(3.186000,-28.102000) -- 
		(3.279000,-28.271000) -- 
		(3.367000,-28.468000) -- 
		(3.460000,-28.726000) -- 
		(2.943000,-28.865000) -- 
		(1.392000,-29.281000) -- 
		(0.874000,-29.419000) -- 
		(0.515000,-29.516000) -- 
		(0.202000,-29.599000) -- 
		(-0.565000,-29.805000) -- 
		(-0.925000,-29.901000) -- 
		(-1.175000,-29.969000) -- 
		(-1.925000,-30.170000) -- 
		(-2.176000,-30.237000) -- 
		(-3.290000,-30.535000) -- 
		(-3.492000,-30.589000) -- 
		(-3.939000,-30.709000) -- 
		(-5.581000,-31.150000) -- 
		(-6.259000,-31.331000) -- 
		(-7.161000,-31.573000) -- 
		(-8.026000,-31.805000) -- 
		(-8.446000,-31.918000) -- 
		(-9.307000,-32.149000) -- 
		(-9.849000,-32.294000) -- 
		(-10.295000,-32.413000) -- 
		(-11.109000,-32.632000) -- 
		(-12.107000,-32.899000) -- 
		(-12.464000,-32.996000) -- 
		(-13.536000,-33.281000) -- 
		(-13.894000,-33.376000) -- 
		(-14.151000,-33.441000) -- 
		(-14.488000,-33.524000) -- 
		(-14.872000,-33.610000) -- 
		(-14.927000,-33.622000) -- 
		(-15.187000,-33.676000) -- 
		(-15.407000,-33.720000) -- 
		(-15.645000,-33.766000) -- 
		(-16.034000,-33.836000) -- 
		(-16.071000,-33.843000) -- 
		(-16.294000,-33.878000) -- 
		(-16.424000,-33.899000) -- 
		(-16.450000,-33.903000) -- 
		(-16.816000,-33.957000) -- 
		(-16.921000,-33.972000) -- 
		(-17.079000,-33.992000) -- 
		(-17.181000,-34.006000) -- 
		(-17.488000,-34.043000) -- 
		(-17.591000,-34.055000) -- 
		(-17.602000,-34.056000) -- 
		(-17.971000,-34.094000) -- 
		(-17.997000,-34.096000) -- 
		(-18.392000,-34.131000) -- 
		(-18.788000,-34.160000) -- 
		(-19.116000,-34.178000) -- 
		(-19.499000,-34.196000) -- 
		(-19.601000,-34.150000) -- 
		(-19.739000,-34.087000) -- 
		(-19.908000,-34.015000) -- 
		(-20.012000,-33.970000) -- 
		(-19.693000,-33.501000) -- 
		(-19.650000,-33.436000) -- 
		(-19.250000,-32.718000) -- 
		(-18.985000,-32.139000) -- 
		(-18.935000,-31.984000) -- 
		(-18.764000,-31.443000) -- 
		(-18.666000,-30.984000) -- 
		(-18.591000,-30.630000) -- 
		(-18.550000,-29.839000) -- 
		(-18.543000,-29.583000) -- 
		(-18.535000,-29.247000) -- 
		(-18.545000,-29.113000) -- 
		(-18.554000,-28.990000) -- 
		(-18.586000,-28.694000) -- 
		(-18.601000,-28.565000) -- 
		(-18.614000,-28.451000) -- 
		(-18.743000,-27.880000) -- 
		(-18.859000,-27.465000) -- 
		(-18.860000,-27.463000) -- 
		(-19.011000,-27.069000) -- 
		(-19.168000,-26.692000) -- 
		(-19.214000,-26.575000) -- 
		(-19.339000,-26.262000) -- 
		(-19.399000,-26.100000) -- 
		(-19.613000,-25.576000) -- 
		(-19.673000,-25.417000) -- 
		(-19.806000,-25.083000) -- 
		(-19.888000,-24.881000) -- 
		(-19.928000,-24.766000) -- 
		(-20.002000,-24.585000) -- 
		(-20.129000,-24.275000) -- 
		(-20.211000,-24.076000) -- 
		(-20.506000,-23.341000) -- 
		(-20.608000,-23.088000) -- 
		(-20.631000,-23.028000) -- 
		(-20.706000,-22.834000) -- 
		(-20.885000,-22.373000) -- 
		(-20.938000,-22.251000) -- 
		(-20.990000,-22.131000) -- 
		(-21.018000,-22.057000) -- 
		(-21.052000,-21.968000) -- 
		(-21.173000,-21.684000) -- 
		(-21.186000,-21.644000) -- 
		(-21.222000,-20.031000) -- 
		(-21.201000,-19.850000) -- 
		(-21.038000,-19.237000) -- 
		(-20.947000,-18.996000) -- 
		(-20.852000,-18.745000) -- 
		(-20.834000,-18.694000) -- 
		(-20.547000,-18.200000) -- 
		(-20.441000,-18.055000) -- 
		(-20.283000,-17.838000) -- 
		(-20.277000,-17.829000) -- 
		(-20.225000,-17.779000) -- 
		(-20.047000,-17.602000) -- 
		(-19.989000,-17.543000) -- 
		(-19.674000,-17.214000) -- 
		(-19.205000,-16.831000) -- 
		(-19.018000,-16.698000) -- 
		(-18.753000,-16.505000) -- 
		(-18.215000,-16.214000) -- 
		(-17.908000,-16.058000) -- 
		(-17.256000,-15.774000) -- 
		(-15.511000,-15.077000) -- 
		(-14.428000,-14.642000) -- 
		(-14.312000,-14.597000) -- 
		(-13.578000,-14.310000) -- 
		(-12.174000,-13.760000) -- 
		(-11.382000,-13.442000) -- 
		(-10.652000,-13.147000) -- 
		(-10.097000,-12.924000) -- 
		(-9.503000,-12.684000) -- 
		(-8.429000,-12.260000) -- 
		(-7.873000,-12.040000) -- 
		(-7.860000,-12.131000) -- 
		(-7.822000,-12.404000) -- 
		(-7.810000,-12.494000) -- 
		(-7.774000,-12.753000) -- 
		(-7.667000,-13.527000) -- 
		(-7.632000,-13.785000) -- 
		(-7.622000,-13.856000) -- 
		(-7.573000,-14.128000) -- 
		(-7.565000,-14.157000) -- 
		(-7.521000,-14.330000) -- 
		(-7.458000,-14.528000) -- 
		(-7.382000,-14.724000) -- 
		(-7.294000,-14.917000) -- 
		(-7.194000,-15.105000) -- 
		(-7.137000,-15.199000) -- 
		(-7.083000,-15.288000) -- 
		(-6.928000,-15.513000) -- 
		(-6.792000,-15.702000) -- 
		(-6.384000,-16.269000) -- 
		(-6.249000,-16.457000) -- 
		(-6.071000,-16.704000) -- 
		(-5.540000,-17.444000) -- 
		(-5.363000,-17.689000) -- 
		(-5.227000,-17.880000) -- 
		(-4.819000,-18.447000) -- 
		(-4.683000,-18.636000) -- 
		(-4.466000,-18.933000) -- 
		(-4.341000,-19.104000) -- 
		(-4.167000,-19.322000) -- 
		(-3.985000,-19.534000) -- 
		(-3.758000,-19.775000) -- 
		(-3.709000,-19.828000) -- 
		(-3.502000,-20.037000) -- 
		(-3.265000,-20.276000) -- 
		(-2.557000,-20.990000) -- 
		(-2.321000,-21.228000) -- 
		(-2.152000,-21.399000) -- 
		(-1.872000,-21.682000) -- 
		(-0.525000,-23.042000) -- 
		(-0.223000,-23.345000) -- 
		(-0.076000,-23.493000) -- 
		(0.010000,-23.580000) -- 
		(0.268000,-23.841000) -- 
		(0.354000,-23.927000) -- 
		(0.438000,-24.011000) -- 
		(0.687000,-24.263000) -- 
		(0.770000,-24.346000) -- 
		(0.899000,-24.478000) -- 
		(1.017000,-24.606000) -- 
		(1.054000,-24.647000);
\node  (Here) at (-9.00000,-24.700000) {\begin{tabular}{r}Bossier\cr Civic\cr Center\cr\end{tabular}};
\node [draw, circle, fill=blue] (BCC) at (-13.500000,-24.700000) { \ };
\end{tikzpicture}

					}
		&
		\adjustbox{valign=T}{
%			\begin{tikzpicture}[x=0.20mm, y=0.23mm]
	\filldraw [draw=black, ultra thick, fill=lightgreen]
		(-22.792000,-17.661000) -- 
		(-22.718000,-17.850000) -- 
		(-22.595000,-18.166000) -- 
		(-22.590000,-18.177000) -- 
		(-22.347000,-18.746000) -- 
		(-22.189000,-19.150000) -- 
		(-22.112000,-19.350000) -- 
		(-22.058000,-19.475000) -- 
		(-22.043000,-19.510000) -- 
		(-21.999000,-19.611000) -- 
		(-21.985000,-19.643000) -- 
		(-21.881000,-19.883000) -- 
		(-21.496000,-20.874000) -- 
		(-21.277000,-21.049000) -- 
		(-21.195000,-21.497000) -- 
		(-21.186000,-21.644000) -- 
		(-21.173000,-21.684000) -- 
		(-21.052000,-21.968000) -- 
		(-21.018000,-22.057000) -- 
		(-21.362000,-22.116000) -- 
		(-22.398000,-22.290000) -- 
		(-22.743000,-22.347000) -- 
		(-22.989000,-22.389000) -- 
		(-23.730000,-22.514000) -- 
		(-23.977000,-22.555000) -- 
		(-25.049000,-22.735000) -- 
		(-26.138000,-22.919000) -- 
		(-26.298000,-22.946000) -- 
		(-28.686000,-23.348000) -- 
		(-29.198000,-23.434000) -- 
		(-29.411000,-23.477000) -- 
		(-29.621000,-23.529000) -- 
		(-29.828000,-23.590000) -- 
		(-30.098000,-23.686000) -- 
		(-30.358000,-23.797000) -- 
		(-30.609000,-23.924000) -- 
		(-30.641000,-23.943000) -- 
		(-30.849000,-24.065000) -- 
		(-31.077000,-24.220000) -- 
		(-31.245000,-24.350000) -- 
		(-31.363000,-24.500000) -- 
		(-31.562000,-24.785000) -- 
		(-31.574000,-24.781000) -- 
		(-31.658000,-24.740000) -- 
		(-31.804000,-24.911000) -- 
		(-31.923000,-25.067000) -- 
		(-32.264000,-25.571000) -- 
		(-32.573000,-26.030000) -- 
		(-32.576000,-26.042000) -- 
		(-32.693000,-26.208000) -- 
		(-33.187000,-26.941000) -- 
		(-33.803000,-27.857000) -- 
		(-33.881000,-27.975000) -- 
		(-34.079000,-28.265000) -- 
		(-34.102000,-28.304000) -- 
		(-34.117000,-28.327000) -- 
		(-34.199000,-28.440000) -- 
		(-34.389000,-28.749000) -- 
		(-34.827000,-29.456000) -- 
		(-34.973000,-29.667000) -- 
		(-35.181000,-29.964000) -- 
		(-35.312000,-30.140000) -- 
		(-35.506000,-30.398000) -- 
		(-35.710000,-30.663000) -- 
		(-35.844000,-30.835000) -- 
		(-35.911000,-30.919000) -- 
		(-35.916000,-30.925000) -- 
		(-36.083000,-31.091000) -- 
		(-36.147000,-31.146000) -- 
		(-36.170000,-31.161000) -- 
		(-36.216000,-31.188000) -- 
		(-36.273000,-31.213000) -- 
		(-36.316000,-31.226000) -- 
		(-36.393000,-31.240000) -- 
		(-36.424000,-31.242000) -- 
		(-36.529000,-31.247000) -- 
		(-36.887000,-31.251000) -- 
		(-37.042000,-31.251000) -- 
		(-37.020000,-31.355000) -- 
		(-36.993000,-31.486000) -- 
		(-36.940000,-31.662000) -- 
		(-36.929000,-31.699000) -- 
		(-36.905000,-31.760000) -- 
		(-36.880000,-31.825000) -- 
		(-36.848000,-31.908000) -- 
		(-36.797000,-32.014000) -- 
		(-36.767000,-32.074000) -- 
		(-36.856000,-32.138000) -- 
		(-37.284000,-32.451000) -- 
		(-37.702000,-32.756000) -- 
		(-38.530000,-33.337000) -- 
		(-38.855000,-33.556000) -- 
		(-39.388000,-33.910000) -- 
		(-39.427000,-33.937000) -- 
		(-39.544000,-34.015000) -- 
		(-39.584000,-34.041000) -- 
		(-39.763000,-34.161000) -- 
		(-39.944000,-34.281000) -- 
		(-40.294000,-34.535000) -- 
		(-40.411000,-34.618000) -- 
		(-40.451000,-34.647000) -- 
		(-40.470000,-34.661000) -- 
		(-40.495000,-34.679000) -- 
		(-41.117000,-35.127000) -- 
		(-41.305000,-35.270000) -- 
		(-41.909000,-35.729000) -- 
		(-41.910000,-35.729000) -- 
		(-42.195000,-35.690000) -- 
		(-42.958000,-35.490000) -- 
		(-43.306000,-35.409000) -- 
		(-43.809000,-35.292000) -- 
		(-44.483000,-35.069000) -- 
		(-44.623000,-34.962000) -- 
		(-44.835000,-34.798000) -- 
		(-44.864000,-34.303000) -- 
		(-44.747000,-33.784000) -- 
		(-44.729000,-33.586000) -- 
		(-44.693000,-33.195000) -- 
		(-44.630000,-32.523000) -- 
		(-44.499000,-31.865000) -- 
		(-44.453000,-31.635000) -- 
		(-44.260000,-30.667000) -- 
		(-44.234000,-30.331000) -- 
		(-44.233000,-30.330000) -- 
		(-44.149000,-29.267000) -- 
		(-44.115000,-28.836000) -- 
		(-43.714000,-20.618000) -- 
		(-43.689000,-20.061000) -- 
		(-43.689000,-20.013000) -- 
		(-43.962000,-19.288000) -- 
		(-44.228000,-18.763000) -- 
		(-45.096000,-17.932000) -- 
		(-45.226000,-17.832000) -- 
		(-45.420000,-17.682000) -- 
		(-46.013000,-17.300000) -- 
		(-46.640000,-17.036000) -- 
		(-47.471000,-16.789000) -- 
		(-48.327000,-16.759000) -- 
		(-48.923000,-17.023000) -- 
		(-49.164000,-17.214000) -- 
		(-49.170000,-17.217000) -- 
		(-49.896000,-17.790000) -- 
		(-50.492000,-18.451000) -- 
		(-51.026000,-19.510000) -- 
		(-51.497000,-20.567000) -- 
		(-51.935000,-21.202000) -- 
		(-52.437000,-21.413000) -- 
		(-52.773000,-21.516000) -- 
		(-52.799000,-21.511000) -- 
		(-52.868000,-21.452000) -- 
		(-52.901000,-21.424000) -- 
		(-52.975000,-21.363000) -- 
		(-53.157000,-21.212000) -- 
		(-53.210000,-21.167000) -- 
		(-53.356000,-21.045000) -- 
		(-54.455000,-20.088000) -- 
		(-55.108000,-19.761000) -- 
		(-55.660000,-19.615000) -- 
		(-56.795000,-19.523000) -- 
		(-57.319000,-19.310000) -- 
		(-57.631000,-19.058000) -- 
		(-58.231000,-18.278000) -- 
		(-58.417000,-18.052000) -- 
		(-59.480000,-17.177000) -- 
		(-61.854000,-15.474000) -- 
		(-63.448000,-14.503000) -- 
		(-64.498000,-13.870000) -- 
		(-65.424000,-13.228000) -- 
		(-65.389000,-13.122000) -- 
		(-67.156000,-11.958000) -- 
		(-67.444000,-11.599000) -- 
		(-67.457000,-11.566000) -- 
		(-67.502000,-11.424000) -- 
		(-67.607000,-11.095000) -- 
		(-67.656000,-9.040000) -- 
		(-67.659000,-8.895000) -- 
		(-67.577000,-8.529000) -- 
		(-67.225000,-8.025000) -- 
		(-67.060000,-8.019000) -- 
		(-66.657000,-8.004000) -- 
		(-66.561000,-8.051000) -- 
		(-66.279000,-8.051000) -- 
		(-66.187000,-8.118000) -- 
		(-65.920000,-8.142000) -- 
		(-64.848000,-8.236000) -- 
		(-64.799000,-8.239000) -- 
		(-64.679000,-8.250000) -- 
		(-63.699000,-8.601000) -- 
		(-63.137000,-9.232000) -- 
		(-62.838000,-9.421000) -- 
		(-62.039000,-9.924000) -- 
		(-61.852000,-10.042000) -- 
		(-60.650000,-10.800000) -- 
		(-60.607000,-10.828000) -- 
		(-59.908000,-11.214000) -- 
		(-59.459000,-11.464000) -- 
		(-56.811000,-12.584000) -- 
		(-56.294000,-12.702000) -- 
		(-55.443000,-12.317000) -- 
		(-54.847000,-11.478000) -- 
		(-54.178000,-10.749000) -- 
		(-52.777000,-9.530000) -- 
		(-52.720000,-8.791000) -- 
		(-52.717000,-8.753000) -- 
		(-52.679000,-8.270000) -- 
		(-52.678000,-8.250000) -- 
		(-52.529000,-8.101000) -- 
		(-51.075000,-6.647000) -- 
		(-50.984000,-6.505000) -- 
		(-50.146000,-5.227000) -- 
		(-47.678000,-3.350000) -- 
		(-44.562000,-3.466000) -- 
		(-43.678000,-3.750000) -- 
		(-43.605000,-3.825000) -- 
		(-42.700000,-4.133000) -- 
		(-41.744000,-4.789000) -- 
		(-40.858000,-5.774000) -- 
		(-40.398000,-6.610000) -- 
		(-40.118000,-7.399000) -- 
		(-39.688000,-11.018000) -- 
		(-39.502000,-13.161000) -- 
		(-38.819000,-15.199000) -- 
		(-37.796000,-16.669000) -- 
		(-37.763000,-16.715000) -- 
		(-37.318000,-16.924000) -- 
		(-36.647000,-17.236000) -- 
		(-35.618000,-17.192000) -- 
		(-35.421000,-17.183000) -- 
		(-32.420000,-17.037000) -- 
		(-30.533000,-16.775000) -- 
		(-30.447000,-16.763000) -- 
		(-30.397000,-16.742000) -- 
		(-29.676000,-16.484000) -- 
		(-29.246000,-16.057000) -- 
		(-29.135000,-15.753000) -- 
		(-28.794000,-15.440000) -- 
		(-28.791000,-15.392000) -- 
		(-28.873000,-15.041000) -- 
		(-28.972000,-14.777000) -- 
		(-28.988000,-14.726000) -- 
		(-29.186000,-14.564000) -- 
		(-29.414000,-14.206000) -- 
		(-30.675000,-12.964000) -- 
		(-32.166000,-11.262000) -- 
		(-34.341000,-8.212000) -- 
		(-34.693000,-7.442000) -- 
		(-35.962000,-4.696000) -- 
		(-38.173000,-1.729000) -- 
		(-38.869000,-0.310000) -- 
		(-39.121000,0.262000) -- 
		(-39.137000,0.299000) -- 
		(-39.347000,0.691000) -- 
		(-39.557000,1.085000) -- 
		(-39.626000,1.214000) -- 
		(-39.695000,1.342000) -- 
		(-39.345000,1.398000) -- 
		(-36.358000,1.986000) -- 
		(-35.994000,2.037000) -- 
		(-35.459000,2.114000) -- 
		(-34.063000,2.144000) -- 
		(-32.553000,1.983000) -- 
		(-31.317000,1.726000) -- 
		(-31.043000,1.422000) -- 
		(-30.975000,1.350000) -- 
		(-30.496000,0.845000) -- 
		(-29.692000,0.075000) -- 
		(-28.485000,-1.190000) -- 
		(-28.180000,-1.607000) -- 
		(-28.139000,-1.693000) -- 
		(-27.976000,-2.032000) -- 
		(-27.812000,-2.370000) -- 
		(-27.662000,-2.680000) -- 
		(-27.336000,-3.417000) -- 
		(-27.204000,-3.718000) -- 
		(-27.097000,-3.961000) -- 
		(-27.081000,-4.042000) -- 
		(-26.802000,-5.484000) -- 
		(-26.369000,-7.045000) -- 
		(-26.284000,-7.045000) -- 
		(-26.105000,-7.046000) -- 
		(-25.669000,-7.048000) -- 
		(-25.316000,-7.051000) -- 
		(-25.053000,-7.052000) -- 
		(-24.979000,-7.054000) -- 
		(-24.757000,-7.054000) -- 
		(-24.684000,-7.054000) -- 
		(-24.676000,-7.114000) -- 
		(-24.652000,-7.291000) -- 
		(-24.645000,-7.350000) -- 
		(-24.526000,-8.240000) -- 
		(-24.511000,-8.495000) -- 
		(-24.492000,-8.835000) -- 
		(-24.397000,-9.964000) -- 
		(-24.169000,-11.843000) -- 
		(-24.159000,-11.934000) -- 
		(-24.034000,-13.078000) -- 
		(-24.009000,-13.305000) -- 
		(-23.952000,-13.831000) -- 
		(-23.931000,-13.985000) -- 
		(-23.902000,-14.209000) -- 
		(-23.862000,-14.509000) -- 
		(-23.799000,-14.992000) -- 
		(-23.675000,-15.358000) -- 
		(-23.666000,-15.389000) -- 
		(-23.582000,-15.678000) -- 
		(-23.412000,-16.114000) -- 
		(-23.345000,-16.286000) -- 
		(-22.890000,-17.416000) -- 
		(-22.792000,-17.661000);
	\filldraw [draw=black, ultra thick, fill=blue]
		(-31.029000,8.853000) -- 
		(-31.030000,8.854000) -- 
		(-31.033000,8.872000) -- 
		(-31.038000,8.900000) -- 
		(-30.875000,8.844000) -- 
		(-30.669000,8.832000) -- 
		(-30.283000,8.793000) -- 
		(-29.889000,8.740000) -- 
		(-29.578000,8.684000) -- 
		(-29.023000,8.562000) -- 
		(-26.525000,7.939000) -- 
		(-24.981000,7.558000) -- 
		(-24.273000,7.401000) -- 
		(-23.975000,7.344000) -- 
		(-23.705000,7.299000) -- 
		(-23.342000,7.249000) -- 
		(-22.829000,7.194000) -- 
		(-22.550000,7.170000) -- 
		(-22.272000,7.157000) -- 
		(-21.856000,7.154000) -- 
		(-20.757000,7.149000) -- 
		(-20.728000,7.149000) -- 
		(-19.603000,7.157000) -- 
		(-17.675000,7.163000) -- 
		(-16.270000,7.178000) -- 
		(-15.146000,7.179000) -- 
		(-14.371000,7.175000) -- 
		(-13.603000,7.179000) -- 
		(-11.773000,7.178000) -- 
		(-10.528000,7.183000) -- 
		(-10.491000,6.920000) -- 
		(-10.474000,6.801000) -- 
		(-10.425000,6.446000) -- 
		(-10.409000,6.328000) -- 
		(-10.377000,6.103000) -- 
		(-10.365000,6.011000) -- 
		(-10.284000,5.434000) -- 
		(-10.254000,5.211000) -- 
		(-10.176000,4.647000) -- 
		(-10.021000,3.525000) -- 
		(-9.943000,2.954000) -- 
		(-9.942000,2.947000) -- 
		(-9.865000,2.391000) -- 
		(-9.837000,2.189000) -- 
		(-9.753000,1.586000) -- 
		(-9.729000,1.407000) -- 
		(-9.726000,1.385000) -- 
		(-9.712000,1.285000) -- 
		(-9.671000,0.991000) -- 
		(-9.658000,0.893000) -- 
		(-9.613000,0.567000) -- 
		(-9.478000,-0.409000) -- 
		(-9.465000,-0.507000) -- 
		(-9.434000,-0.734000) -- 
		(-9.369000,-1.203000) -- 
		(-9.174000,-2.607000) -- 
		(-9.110000,-3.073000) -- 
		(-9.015000,-3.761000) -- 
		(-8.731000,-5.818000) -- 
		(-8.637000,-6.504000) -- 
		(-8.570000,-6.988000) -- 
		(-8.329000,-8.733000) -- 
		(-8.209000,-9.602000) -- 
		(-8.201000,-9.661000) -- 
		(-8.039000,-10.835000) -- 
		(-7.873000,-12.040000) -- 
		(-8.429000,-12.260000) -- 
		(-9.503000,-12.684000) -- 
		(-10.097000,-12.924000) -- 
		(-10.652000,-13.147000) -- 
		(-11.382000,-13.442000) -- 
		(-12.174000,-13.760000) -- 
		(-13.578000,-14.310000) -- 
		(-14.312000,-14.597000) -- 
		(-14.428000,-14.642000) -- 
		(-15.511000,-15.077000) -- 
		(-17.256000,-15.774000) -- 
		(-17.908000,-16.058000) -- 
		(-18.215000,-16.214000) -- 
		(-18.753000,-16.505000) -- 
		(-19.018000,-16.698000) -- 
		(-19.205000,-16.831000) -- 
		(-19.674000,-17.214000) -- 
		(-19.989000,-17.543000) -- 
		(-20.047000,-17.602000) -- 
		(-20.225000,-17.779000) -- 
		(-20.277000,-17.829000) -- 
		(-20.283000,-17.838000) -- 
		(-20.441000,-18.055000) -- 
		(-20.547000,-18.200000) -- 
		(-20.834000,-18.694000) -- 
		(-20.852000,-18.745000) -- 
		(-20.947000,-18.996000) -- 
		(-21.038000,-19.237000) -- 
		(-21.201000,-19.850000) -- 
		(-21.222000,-20.031000) -- 
		(-21.186000,-21.644000) -- 
		(-21.195000,-21.497000) -- 
		(-21.277000,-21.049000) -- 
		(-21.496000,-20.874000) -- 
		(-21.881000,-19.883000) -- 
		(-21.985000,-19.643000) -- 
		(-21.999000,-19.611000) -- 
		(-22.043000,-19.510000) -- 
		(-22.058000,-19.475000) -- 
		(-22.112000,-19.350000) -- 
		(-22.189000,-19.150000) -- 
		(-22.347000,-18.746000) -- 
		(-22.590000,-18.177000) -- 
		(-22.595000,-18.166000) -- 
		(-22.718000,-17.850000) -- 
		(-22.792000,-17.661000) -- 
		(-22.890000,-17.416000) -- 
		(-23.345000,-16.286000) -- 
		(-23.412000,-16.114000) -- 
		(-23.582000,-15.678000) -- 
		(-23.666000,-15.389000) -- 
		(-23.675000,-15.358000) -- 
		(-23.799000,-14.992000) -- 
		(-23.862000,-14.509000) -- 
		(-23.902000,-14.209000) -- 
		(-23.931000,-13.985000) -- 
		(-23.952000,-13.831000) -- 
		(-24.009000,-13.305000) -- 
		(-24.034000,-13.078000) -- 
		(-24.159000,-11.934000) -- 
		(-24.169000,-11.843000) -- 
		(-24.397000,-9.964000) -- 
		(-24.492000,-8.835000) -- 
		(-24.511000,-8.495000) -- 
		(-24.526000,-8.240000) -- 
		(-24.645000,-7.350000) -- 
		(-24.652000,-7.291000) -- 
		(-24.676000,-7.114000) -- 
		(-24.684000,-7.054000) -- 
		(-24.834000,-6.026000) -- 
		(-24.909000,-5.629000) -- 
		(-25.023000,-5.218000) -- 
		(-25.084000,-5.061000) -- 
		(-25.376000,-4.304000) -- 
		(-26.175000,-2.450000) -- 
		(-26.427000,-1.812000) -- 
		(-26.672000,-1.063000) -- 
		(-26.935000,-0.162000) -- 
		(-27.143000,0.691000) -- 
		(-27.160000,0.757000) -- 
		(-27.300000,1.243000) -- 
		(-27.457000,1.708000) -- 
		(-27.622000,2.137000) -- 
		(-27.850000,2.600000) -- 
		(-27.895000,2.691000) -- 
		(-29.016000,4.663000) -- 
		(-29.661000,5.792000) -- 
		(-30.010000,6.382000) -- 
		(-30.674000,7.687000) -- 
		(-30.973000,8.511000) -- 
		(-30.984000,8.579000) -- 
		(-30.988000,8.600000) -- 
		(-31.029000,8.853000);
	\filldraw [draw=black, ultra thick, fill=lightgreen]
		(1.378000,-25.058000) -- 
		(1.243000,-24.875000) -- 
		(1.075000,-24.670000) -- 
		(1.054000,-24.647000) -- 
		(1.017000,-24.606000) -- 
		(0.899000,-24.478000) -- 
		(0.770000,-24.346000) -- 
		(0.687000,-24.263000) -- 
		(0.438000,-24.011000) -- 
		(0.354000,-23.927000) -- 
		(0.268000,-23.841000) -- 
		(0.010000,-23.580000) -- 
		(-0.076000,-23.493000) -- 
		(-0.223000,-23.345000) -- 
		(-0.525000,-23.042000) -- 
		(-1.872000,-21.682000) -- 
		(-2.152000,-21.399000) -- 
		(-2.321000,-21.228000) -- 
		(-2.557000,-20.990000) -- 
		(-3.265000,-20.276000) -- 
		(-3.502000,-20.037000) -- 
		(-3.709000,-19.828000) -- 
		(-3.758000,-19.775000) -- 
		(-3.985000,-19.534000) -- 
		(-4.167000,-19.322000) -- 
		(-4.341000,-19.104000) -- 
		(-4.466000,-18.933000) -- 
		(-4.683000,-18.636000) -- 
		(-4.819000,-18.447000) -- 
		(-5.227000,-17.880000) -- 
		(-5.363000,-17.689000) -- 
		(-5.540000,-17.444000) -- 
		(-6.071000,-16.704000) -- 
		(-6.249000,-16.457000) -- 
		(-6.384000,-16.269000) -- 
		(-6.792000,-15.702000) -- 
		(-6.928000,-15.513000) -- 
		(-7.083000,-15.288000) -- 
		(-7.137000,-15.199000) -- 
		(-7.194000,-15.105000) -- 
		(-7.294000,-14.917000) -- 
		(-7.382000,-14.724000) -- 
		(-7.458000,-14.528000) -- 
		(-7.521000,-14.330000) -- 
		(-7.565000,-14.157000) -- 
		(-7.573000,-14.128000) -- 
		(-7.622000,-13.856000) -- 
		(-7.632000,-13.785000) -- 
		(-7.667000,-13.527000) -- 
		(-7.774000,-12.753000) -- 
		(-7.810000,-12.494000) -- 
		(-7.822000,-12.404000) -- 
		(-7.860000,-12.131000) -- 
		(-7.873000,-12.040000) -- 
		(-8.039000,-10.835000) -- 
		(-8.201000,-9.661000) -- 
		(-8.209000,-9.602000) -- 
		(-8.329000,-8.733000) -- 
		(-8.570000,-6.988000) -- 
		(-8.637000,-6.504000) -- 
		(-8.731000,-5.818000) -- 
		(-9.015000,-3.761000) -- 
		(-9.110000,-3.073000) -- 
		(-9.174000,-2.607000) -- 
		(-9.369000,-1.203000) -- 
		(-9.434000,-0.734000) -- 
		(-9.465000,-0.507000) -- 
		(-9.478000,-0.409000) -- 
		(-9.613000,0.567000) -- 
		(-9.658000,0.893000) -- 
		(-9.671000,0.991000) -- 
		(-9.712000,1.285000) -- 
		(-9.726000,1.385000) -- 
		(-9.729000,1.407000) -- 
		(-9.753000,1.586000) -- 
		(-9.837000,2.189000) -- 
		(-9.865000,2.391000) -- 
		(-9.942000,2.947000) -- 
		(-9.943000,2.954000) -- 
		(-10.021000,3.525000) -- 
		(-10.176000,4.647000) -- 
		(-10.254000,5.211000) -- 
		(-10.284000,5.434000) -- 
		(-10.365000,6.011000) -- 
		(-10.377000,6.103000) -- 
		(-10.409000,6.328000) -- 
		(-10.425000,6.446000) -- 
		(-10.474000,6.801000) -- 
		(-10.491000,6.920000) -- 
		(-10.528000,7.183000) -- 
		(-6.048000,7.200000) -- 
		(-5.879000,7.201000) -- 
		(-5.375000,7.203000) -- 
		(-5.207000,7.205000) -- 
		(-5.049000,7.206000) -- 
		(-4.606000,7.209000) -- 
		(-4.034000,7.217000) -- 
		(-3.931000,7.220000) -- 
		(-3.562000,7.234000) -- 
		(-2.622000,7.281000) -- 
		(-2.113000,7.314000) -- 
		(-0.113000,7.454000) -- 
		(1.158000,7.545000) -- 
		(2.308000,7.626000) -- 
		(5.755000,7.871000) -- 
		(6.903000,7.954000) -- 
		(8.238000,8.047000) -- 
		(10.972000,8.240000) -- 
		(11.640000,8.279000) -- 
		(12.243000,8.301000) -- 
		(12.831000,8.323000) -- 
		(13.048000,8.326000) -- 
		(13.373000,8.339000) -- 
		(13.579000,8.342000) -- 
		(14.054000,8.349000) -- 
		(14.177000,8.346000) -- 
		(14.694000,8.340000) -- 
		(15.715000,8.311000) -- 
		(15.971000,8.300000) -- 
		(16.568000,8.277000) -- 
		(17.532000,8.238000) -- 
		(18.030000,8.218000) -- 
		(18.702000,8.197000) -- 
		(19.378000,8.168000) -- 
		(20.421000,8.134000) -- 
		(21.384000,8.103000) -- 
		(25.651000,7.963000) -- 
		(29.874000,7.823000) -- 
		(30.810000,7.800000) -- 
		(32.388000,7.772000) -- 
		(33.738000,7.767000) -- 
		(33.982000,7.770000) -- 
		(34.238000,7.139000) -- 
		(34.291000,7.140000) -- 
		(34.290000,7.083000) -- 
		(34.289000,7.044000) -- 
		(34.328000,6.945000) -- 
		(34.349000,6.935000) -- 
		(34.399000,6.912000) -- 
		(34.522000,6.868000) -- 
		(34.834000,6.747000) -- 
		(35.289000,6.692000) -- 
		(35.581000,6.615000) -- 
		(35.602000,6.603000) -- 
		(35.717000,6.538000) -- 
		(35.931000,6.373000) -- 
		(35.983000,6.175000) -- 
		(36.054000,4.927000) -- 
		(36.087000,4.619000) -- 
		(36.054000,4.432000) -- 
		(35.989000,4.157000) -- 
		(35.872000,3.921000) -- 
		(35.775000,3.805000) -- 
		(35.548000,3.651000) -- 
		(35.180000,3.449000) -- 
		(35.022000,3.365000) -- 
		(34.846000,3.250000) -- 
		(34.807000,3.244000) -- 
		(34.788000,3.223000) -- 
		(34.431000,2.953000) -- 
		(34.359000,2.860000) -- 
		(34.229000,2.674000) -- 
		(34.132000,2.535000) -- 
		(34.099000,2.480000) -- 
		(33.879000,2.277000) -- 
		(33.723000,2.139000) -- 
		(33.290000,1.870000) -- 
		(33.281000,1.865000) -- 
		(33.125000,1.810000) -- 
		(32.392000,1.474000) -- 
		(31.944000,1.293000) -- 
		(31.457000,1.161000) -- 
		(31.190000,1.046000) -- 
		(30.742000,0.820000) -- 
		(30.431000,0.628000) -- 
		(30.119000,0.479000) -- 
		(29.851000,0.396000) -- 
		(29.613000,0.325000) -- 
		(29.538000,0.320000) -- 
		(29.275000,0.303000) -- 
		(29.035000,0.281000) -- 
		(28.768000,0.325000) -- 
		(28.620000,0.363000) -- 
		(28.574000,0.375000) -- 
		(28.379000,0.369000) -- 
		(27.989000,0.254000) -- 
		(27.970000,0.243000) -- 
		(27.587000,0.034000) -- 
		(27.548000,0.029000) -- 
		(27.353000,-0.114000) -- 
		(26.976000,-0.411000) -- 
		(26.626000,-0.736000) -- 
		(26.489000,-0.890000) -- 
		(26.433000,-1.012000) -- 
		(26.327000,-1.236000) -- 
		(26.243000,-1.533000) -- 
		(26.246000,-1.648000) -- 
		(26.249000,-1.731000) -- 
		(26.282000,-1.896000) -- 
		(26.376000,-2.027000) -- 
		(26.385000,-2.039000) -- 
		(26.522000,-2.154000) -- 
		(27.028000,-2.396000) -- 
		(27.255000,-2.462000) -- 
		(27.492000,-2.517000) -- 
		(27.893000,-2.606000) -- 
		(28.086000,-2.649000) -- 
		(28.511000,-2.633000) -- 
		(28.937000,-2.617000) -- 
		(29.125000,-2.606000) -- 
		(29.201000,-2.597000) -- 
		(29.222000,-2.610000) -- 
		(29.235000,-2.618000) -- 
		(29.574000,-2.606000) -- 
		(29.785000,-2.593000) -- 
		(29.869000,-2.584000) -- 
		(30.584000,-2.532000) -- 
		(30.669000,-2.531000) -- 
		(30.964000,-2.511000) -- 
		(31.619000,-2.486000) -- 
		(32.153000,-2.465000) -- 
		(33.007000,-2.456000) -- 
		(33.231000,-2.467000) -- 
		(33.304000,-2.470000) -- 
		(33.603000,-2.474000) -- 
		(33.774000,-2.482000) -- 
		(33.987000,-2.483000) -- 
		(34.070000,-2.493000) -- 
		(34.198000,-2.500000) -- 
		(34.269000,-2.471000) -- 
		(34.313000,-2.480000) -- 
		(34.561000,-2.488000) -- 
		(34.637000,-2.491000) -- 
		(34.674000,-2.500000) -- 
		(34.708000,-2.478000) -- 
		(34.747000,-2.462000) -- 
		(34.789000,-2.460000) -- 
		(34.996000,-2.429000) -- 
		(35.119000,-2.405000) -- 
		(35.289000,-2.399000) -- 
		(35.460000,-2.400000) -- 
		(35.544000,-2.409000) -- 
		(35.669000,-2.430000) -- 
		(35.739000,-2.458000) -- 
		(35.746000,-2.459000) -- 
		(35.859000,-2.509000) -- 
		(35.929000,-2.550000) -- 
		(35.987000,-2.599000) -- 
		(36.120000,-2.734000) -- 
		(36.142000,-2.764000) -- 
		(36.268000,-2.905000) -- 
		(36.318000,-3.003000) -- 
		(36.391000,-3.132000) -- 
		(36.419000,-3.190000) -- 
		(36.471000,-3.297000) -- 
		(36.507000,-3.362000) -- 
		(36.554000,-3.421000) -- 
		(36.706000,-3.589000) -- 
		(36.795000,-3.665000) -- 
		(37.082000,-3.880000) -- 
		(37.205000,-3.979000) -- 
		(37.415000,-4.157000) -- 
		(37.447000,-4.180000) -- 
		(37.468000,-4.207000) -- 
		(37.493000,-4.273000) -- 
		(37.532000,-4.410000) -- 
		(37.581000,-4.465000) -- 
		(37.630000,-4.521000) -- 
		(37.695000,-4.685000) -- 
		(37.745000,-4.743000) -- 
		(37.711000,-4.947000) -- 
		(37.693000,-5.059000) -- 
		(37.671000,-5.165000) -- 
		(37.647000,-5.213000) -- 
		(37.637000,-5.230000) -- 
		(37.583000,-5.285000) -- 
		(37.462000,-5.473000) -- 
		(37.436000,-5.502000) -- 
		(37.393000,-5.562000) -- 
		(37.394000,-5.597000) -- 
		(37.404000,-5.632000) -- 
		(37.404000,-5.660000) -- 
		(37.357000,-5.686000) -- 
		(37.181000,-5.961000) -- 
		(37.156000,-6.247000) -- 
		(37.148000,-6.340000) -- 
		(37.142000,-6.401000) -- 
		(37.189000,-6.612000) -- 
		(37.208000,-6.702000) -- 
		(37.246000,-6.874000) -- 
		(37.246000,-7.132000) -- 
		(37.155000,-7.440000) -- 
		(37.070000,-7.600000) -- 
		(37.018000,-7.671000) -- 
		(36.769000,-7.879000) -- 
		(36.720000,-7.919000) -- 
		(36.473000,-8.078000) -- 
		(36.239000,-8.150000) -- 
		(35.837000,-8.287000) -- 
		(35.733000,-8.314000) -- 
		(35.259000,-8.485000) -- 
		(34.986000,-8.639000) -- 
		(34.259000,-9.106000) -- 
		(34.194000,-9.134000) -- 
		(34.058000,-9.232000) -- 
		(33.967000,-9.320000) -- 
		(33.785000,-9.458000) -- 
		(33.642000,-9.705000) -- 
		(33.562000,-10.051000) -- 
		(33.402000,-10.739000) -- 
		(33.356000,-10.838000) -- 
		(33.142000,-10.992000) -- 
		(32.914000,-11.135000) -- 
		(32.713000,-11.261000) -- 
		(32.460000,-11.470000) -- 
		(32.402000,-11.635000) -- 
		(32.301000,-11.883000) -- 
		(32.259000,-11.982000) -- 
		(31.895000,-12.284000) -- 
		(31.493000,-12.575000) -- 
		(31.292000,-12.691000) -- 
		(31.142000,-12.812000) -- 
		(30.993000,-13.032000) -- 
		(30.746000,-13.774000) -- 
		(30.558000,-14.060000) -- 
		(30.481000,-14.161000) -- 
		(30.285000,-14.412000) -- 
		(29.915000,-14.780000) -- 
		(29.751000,-14.805000) -- 
		(29.757000,-14.822000) -- 
		(29.772000,-14.870000) -- 
		(29.777000,-14.886000) -- 
		(29.788000,-14.921000) -- 
		(29.824000,-15.038000) -- 
		(29.912000,-15.327000) -- 
		(29.959000,-15.491000) -- 
		(30.002000,-15.641000) -- 
		(29.875000,-15.682000) -- 
		(29.492000,-15.804000) -- 
		(29.364000,-15.844000) -- 
		(29.331000,-15.857000) -- 
		(29.230000,-15.888000) -- 
		(29.196000,-15.898000) -- 
		(29.100000,-16.118000) -- 
		(28.920000,-16.282000) -- 
		(28.908000,-16.299000) -- 
		(28.859000,-16.357000) -- 
		(28.808000,-16.456000) -- 
		(28.729000,-16.653000) -- 
		(28.672000,-16.794000) -- 
		(28.643000,-16.860000) -- 
		(28.554000,-16.999000) -- 
		(28.411000,-17.220000) -- 
		(27.960000,-17.912000) -- 
		(27.607000,-18.453000) -- 
		(27.397000,-18.801000) -- 
		(27.056000,-19.383000) -- 
		(26.680000,-20.033000) -- 
		(26.265000,-20.747000) -- 
		(26.200000,-20.861000) -- 
		(26.013000,-21.187000) -- 
		(26.005000,-21.201000) -- 
		(25.939000,-21.314000) -- 
		(25.844000,-21.480000) -- 
		(25.767000,-21.611000) -- 
		(25.767000,-21.612000) -- 
		(25.557000,-21.972000) -- 
		(25.461000,-22.136000) -- 
		(25.454000,-22.150000) -- 
		(25.430000,-22.190000) -- 
		(25.422000,-22.203000) -- 
		(25.371000,-22.293000) -- 
		(25.215000,-22.560000) -- 
		(25.199000,-22.586000) -- 
		(25.160000,-22.647000) -- 
		(25.055000,-22.813000) -- 
		(25.011000,-22.896000) -- 
		(24.964000,-22.972000) -- 
		(24.875000,-23.082000) -- 
		(24.763000,-23.188000) -- 
		(24.697000,-23.240000) -- 
		(24.615000,-23.297000) -- 
		(24.480000,-23.364000) -- 
		(24.412000,-23.389000) -- 
		(24.269000,-23.434000) -- 
		(24.111000,-23.469000) -- 
		(24.034000,-23.478000) -- 
		(23.746000,-23.498000) -- 
		(23.093000,-23.544000) -- 
		(21.134000,-23.681000) -- 
		(20.862000,-23.700000) -- 
		(20.704000,-23.731000) -- 
		(20.626000,-23.765000) -- 
		(20.553000,-23.819000) -- 
		(20.524000,-23.854000) -- 
		(20.512000,-23.879000) -- 
		(20.472000,-23.954000) -- 
		(20.459000,-23.979000) -- 
		(20.414000,-24.079000) -- 
		(20.386000,-24.119000) -- 
		(20.352000,-24.155000) -- 
		(20.317000,-24.184000) -- 
		(20.234000,-24.233000) -- 
		(20.151000,-24.268000) -- 
		(20.041000,-24.302000) -- 
		(19.886000,-24.339000) -- 
		(19.495000,-24.432000) -- 
		(19.099000,-24.531000) -- 
		(18.946000,-24.569000) -- 
		(16.331000,-25.277000) -- 
		(14.681000,-25.721000) -- 
		(13.208000,-26.114000) -- 
		(11.918000,-26.460000) -- 
		(9.139000,-27.303000) -- 
		(8.370000,-27.536000) -- 
		(8.323000,-27.549000) -- 
		(7.840000,-27.551000) -- 
		(7.812000,-27.560000) -- 
		(7.381000,-27.677000) -- 
		(6.632000,-27.876000) -- 
		(6.003000,-28.045000) -- 
		(5.543000,-28.168000) -- 
		(5.127000,-28.280000) -- 
		(3.877000,-28.615000) -- 
		(3.460000,-28.726000) -- 
		(3.367000,-28.468000) -- 
		(3.279000,-28.271000) -- 
		(3.186000,-28.102000) -- 
		(2.972000,-27.739000) -- 
		(2.831000,-27.501000) -- 
		(2.723000,-27.318000) -- 
		(2.397000,-26.766000) -- 
		(2.288000,-26.582000) -- 
		(1.619000,-25.454000) -- 
		(1.426000,-25.129000) -- 
		(1.378000,-25.058000);
	\filldraw [draw=black, ultra thick, fill=gold]
		(1.054000,-24.647000) -- 
		(1.075000,-24.670000) -- 
		(1.243000,-24.875000) -- 
		(1.378000,-25.058000) -- 
		(1.426000,-25.129000) -- 
		(1.619000,-25.454000) -- 
		(2.288000,-26.582000) -- 
		(2.397000,-26.766000) -- 
		(2.723000,-27.318000) -- 
		(2.831000,-27.501000) -- 
		(2.972000,-27.739000) -- 
		(3.186000,-28.102000) -- 
		(3.279000,-28.271000) -- 
		(3.367000,-28.468000) -- 
		(3.460000,-28.726000) -- 
		(2.943000,-28.865000) -- 
		(1.392000,-29.281000) -- 
		(0.874000,-29.419000) -- 
		(0.515000,-29.516000) -- 
		(0.202000,-29.599000) -- 
		(-0.565000,-29.805000) -- 
		(-0.925000,-29.901000) -- 
		(-1.175000,-29.969000) -- 
		(-1.925000,-30.170000) -- 
		(-2.176000,-30.237000) -- 
		(-3.290000,-30.535000) -- 
		(-3.492000,-30.589000) -- 
		(-3.939000,-30.709000) -- 
		(-5.581000,-31.150000) -- 
		(-6.259000,-31.331000) -- 
		(-7.161000,-31.573000) -- 
		(-8.026000,-31.805000) -- 
		(-8.446000,-31.918000) -- 
		(-9.307000,-32.149000) -- 
		(-9.849000,-32.294000) -- 
		(-10.295000,-32.413000) -- 
		(-11.109000,-32.632000) -- 
		(-12.107000,-32.899000) -- 
		(-12.464000,-32.996000) -- 
		(-13.536000,-33.281000) -- 
		(-13.894000,-33.376000) -- 
		(-14.151000,-33.441000) -- 
		(-14.488000,-33.524000) -- 
		(-14.872000,-33.610000) -- 
		(-14.927000,-33.622000) -- 
		(-15.187000,-33.676000) -- 
		(-15.407000,-33.720000) -- 
		(-15.645000,-33.766000) -- 
		(-16.034000,-33.836000) -- 
		(-16.071000,-33.843000) -- 
		(-16.294000,-33.878000) -- 
		(-16.424000,-33.899000) -- 
		(-16.450000,-33.903000) -- 
		(-16.816000,-33.957000) -- 
		(-16.921000,-33.972000) -- 
		(-17.079000,-33.992000) -- 
		(-17.181000,-34.006000) -- 
		(-17.488000,-34.043000) -- 
		(-17.591000,-34.055000) -- 
		(-17.602000,-34.056000) -- 
		(-17.971000,-34.094000) -- 
		(-17.997000,-34.096000) -- 
		(-18.392000,-34.131000) -- 
		(-18.788000,-34.160000) -- 
		(-19.116000,-34.178000) -- 
		(-19.499000,-34.196000) -- 
		(-19.601000,-34.150000) -- 
		(-19.739000,-34.087000) -- 
		(-19.908000,-34.015000) -- 
		(-20.012000,-33.970000) -- 
		(-19.693000,-33.501000) -- 
		(-19.650000,-33.436000) -- 
		(-19.250000,-32.718000) -- 
		(-18.985000,-32.139000) -- 
		(-18.935000,-31.984000) -- 
		(-18.764000,-31.443000) -- 
		(-18.666000,-30.984000) -- 
		(-18.591000,-30.630000) -- 
		(-18.550000,-29.839000) -- 
		(-18.543000,-29.583000) -- 
		(-18.535000,-29.247000) -- 
		(-18.545000,-29.113000) -- 
		(-18.554000,-28.990000) -- 
		(-18.586000,-28.694000) -- 
		(-18.601000,-28.565000) -- 
		(-18.614000,-28.451000) -- 
		(-18.743000,-27.880000) -- 
		(-18.859000,-27.465000) -- 
		(-18.860000,-27.463000) -- 
		(-19.011000,-27.069000) -- 
		(-19.168000,-26.692000) -- 
		(-19.214000,-26.575000) -- 
		(-19.339000,-26.262000) -- 
		(-19.399000,-26.100000) -- 
		(-19.613000,-25.576000) -- 
		(-19.673000,-25.417000) -- 
		(-19.806000,-25.083000) -- 
		(-19.888000,-24.881000) -- 
		(-19.928000,-24.766000) -- 
		(-20.002000,-24.585000) -- 
		(-20.129000,-24.275000) -- 
		(-20.211000,-24.076000) -- 
		(-20.506000,-23.341000) -- 
		(-20.608000,-23.088000) -- 
		(-20.631000,-23.028000) -- 
		(-20.706000,-22.834000) -- 
		(-20.885000,-22.373000) -- 
		(-20.938000,-22.251000) -- 
		(-20.990000,-22.131000) -- 
		(-21.018000,-22.057000) -- 
		(-21.052000,-21.968000) -- 
		(-21.173000,-21.684000) -- 
		(-21.186000,-21.644000) -- 
		(-21.222000,-20.031000) -- 
		(-21.201000,-19.850000) -- 
		(-21.038000,-19.237000) -- 
		(-20.947000,-18.996000) -- 
		(-20.852000,-18.745000) -- 
		(-20.834000,-18.694000) -- 
		(-20.547000,-18.200000) -- 
		(-20.441000,-18.055000) -- 
		(-20.283000,-17.838000) -- 
		(-20.277000,-17.829000) -- 
		(-20.225000,-17.779000) -- 
		(-20.047000,-17.602000) -- 
		(-19.989000,-17.543000) -- 
		(-19.674000,-17.214000) -- 
		(-19.205000,-16.831000) -- 
		(-19.018000,-16.698000) -- 
		(-18.753000,-16.505000) -- 
		(-18.215000,-16.214000) -- 
		(-17.908000,-16.058000) -- 
		(-17.256000,-15.774000) -- 
		(-15.511000,-15.077000) -- 
		(-14.428000,-14.642000) -- 
		(-14.312000,-14.597000) -- 
		(-13.578000,-14.310000) -- 
		(-12.174000,-13.760000) -- 
		(-11.382000,-13.442000) -- 
		(-10.652000,-13.147000) -- 
		(-10.097000,-12.924000) -- 
		(-9.503000,-12.684000) -- 
		(-8.429000,-12.260000) -- 
		(-7.873000,-12.040000) -- 
		(-7.860000,-12.131000) -- 
		(-7.822000,-12.404000) -- 
		(-7.810000,-12.494000) -- 
		(-7.774000,-12.753000) -- 
		(-7.667000,-13.527000) -- 
		(-7.632000,-13.785000) -- 
		(-7.622000,-13.856000) -- 
		(-7.573000,-14.128000) -- 
		(-7.565000,-14.157000) -- 
		(-7.521000,-14.330000) -- 
		(-7.458000,-14.528000) -- 
		(-7.382000,-14.724000) -- 
		(-7.294000,-14.917000) -- 
		(-7.194000,-15.105000) -- 
		(-7.137000,-15.199000) -- 
		(-7.083000,-15.288000) -- 
		(-6.928000,-15.513000) -- 
		(-6.792000,-15.702000) -- 
		(-6.384000,-16.269000) -- 
		(-6.249000,-16.457000) -- 
		(-6.071000,-16.704000) -- 
		(-5.540000,-17.444000) -- 
		(-5.363000,-17.689000) -- 
		(-5.227000,-17.880000) -- 
		(-4.819000,-18.447000) -- 
		(-4.683000,-18.636000) -- 
		(-4.466000,-18.933000) -- 
		(-4.341000,-19.104000) -- 
		(-4.167000,-19.322000) -- 
		(-3.985000,-19.534000) -- 
		(-3.758000,-19.775000) -- 
		(-3.709000,-19.828000) -- 
		(-3.502000,-20.037000) -- 
		(-3.265000,-20.276000) -- 
		(-2.557000,-20.990000) -- 
		(-2.321000,-21.228000) -- 
		(-2.152000,-21.399000) -- 
		(-1.872000,-21.682000) -- 
		(-0.525000,-23.042000) -- 
		(-0.223000,-23.345000) -- 
		(-0.076000,-23.493000) -- 
		(0.010000,-23.580000) -- 
		(0.268000,-23.841000) -- 
		(0.354000,-23.927000) -- 
		(0.438000,-24.011000) -- 
		(0.687000,-24.263000) -- 
		(0.770000,-24.346000) -- 
		(0.899000,-24.478000) -- 
		(1.017000,-24.606000) -- 
		(1.054000,-24.647000);
	\filldraw [draw=black, ultra thick, fill=blue]
		(16.989000,-38.371000) -- 
		(16.988000,-38.370000) -- 
		(16.864000,-38.231000) -- 
		(16.438000,-37.776000) -- 
		(16.158000,-37.494000) -- 
		(15.769000,-37.122000) -- 
		(15.459000,-36.840000) -- 
		(15.002000,-36.448000) -- 
		(14.955000,-36.410000) -- 
		(14.836000,-36.313000) -- 
		(14.352000,-35.920000) -- 
		(13.914000,-35.563000) -- 
		(13.870000,-35.527000) -- 
		(13.403000,-35.148000) -- 
		(12.829000,-34.681000) -- 
		(12.435000,-34.361000) -- 
		(12.300000,-34.250000) -- 
		(12.270000,-34.226000) -- 
		(12.144000,-34.123000) -- 
		(11.594000,-33.677000) -- 
		(11.343000,-33.900000) -- 
		(10.586000,-34.566000) -- 
		(10.333000,-34.786000) -- 
		(10.034000,-35.039000) -- 
		(9.926000,-35.131000) -- 
		(9.864000,-35.180000) -- 
		(9.756000,-35.259000) -- 
		(9.667000,-35.318000) -- 
		(9.466000,-35.447000) -- 
		(9.283000,-35.552000) -- 
		(9.013000,-35.688000) -- 
		(8.836000,-35.766000) -- 
		(8.545000,-35.877000) -- 
		(8.338000,-35.944000) -- 
		(8.223000,-35.977000) -- 
		(7.974000,-36.046000) -- 
		(7.627000,-36.117000) -- 
		(7.538000,-36.131000) -- 
		(7.529000,-36.131000) -- 
		(7.266000,-36.155000) -- 
		(7.175000,-36.162000) -- 
		(7.080000,-36.166000) -- 
		(6.935000,-36.167000) -- 
		(6.739000,-36.167000) -- 
		(6.215000,-36.157000) -- 
		(5.974000,-36.152000) -- 
		(5.718000,-36.146000) -- 
		(4.950000,-36.128000) -- 
		(4.694000,-36.121000) -- 
		(4.657000,-36.120000) -- 
		(4.546000,-36.117000) -- 
		(4.509000,-36.116000) -- 
		(4.356000,-36.114000) -- 
		(3.895000,-36.103000) -- 
		(3.741000,-36.098000) -- 
		(3.678000,-36.097000) -- 
		(3.487000,-36.093000) -- 
		(3.423000,-36.091000) -- 
		(2.381000,-36.066000) -- 
		(1.494000,-36.044000) -- 
		(1.342000,-36.041000) -- 
		(0.303000,-36.016000) -- 
		(-0.754000,-35.991000) -- 
		(-0.745000,-36.015000) -- 
		(-1.821000,-35.986000) -- 
		(-2.056000,-35.979000) -- 
		(-2.121000,-36.123000) -- 
		(-2.318000,-36.551000) -- 
		(-2.384000,-36.694000) -- 
		(-2.465000,-36.874000) -- 
		(-2.712000,-37.412000) -- 
		(-2.795000,-37.591000) -- 
		(-2.813000,-37.634000) -- 
		(-2.871000,-37.758000) -- 
		(-2.891000,-37.799000) -- 
		(-2.971000,-37.975000) -- 
		(-3.212000,-38.502000) -- 
		(-3.293000,-38.677000) -- 
		(-3.311000,-38.718000) -- 
		(-3.338000,-38.774000) -- 
		(-3.368000,-38.841000) -- 
		(-3.387000,-38.882000) -- 
		(-3.569000,-39.281000) -- 
		(-4.117000,-40.474000) -- 
		(-4.300000,-40.872000) -- 
		(-4.336000,-40.908000) -- 
		(-4.445000,-41.014000) -- 
		(-4.482000,-41.049000) -- 
		(-4.633000,-40.981000) -- 
		(-5.086000,-40.776000) -- 
		(-5.238000,-40.707000) -- 
		(-6.468000,-40.152000) -- 
		(-6.950000,-39.934000) -- 
		(-7.262000,-39.790000) -- 
		(-8.757000,-39.103000) -- 
		(-9.854000,-38.620000) -- 
		(-11.738000,-37.746000) -- 
		(-12.345000,-37.469000) -- 
		(-13.872000,-36.755000) -- 
		(-14.263000,-36.579000) -- 
		(-14.277000,-36.573000) -- 
		(-14.321000,-36.553000) -- 
		(-14.336000,-36.546000) -- 
		(-14.585000,-36.435000) -- 
		(-15.335000,-36.097000) -- 
		(-15.575000,-35.988000) -- 
		(-15.585000,-35.984000) -- 
		(-15.691000,-35.938000) -- 
		(-15.873000,-35.857000) -- 
		(-15.993000,-35.805000) -- 
		(-16.010000,-35.797000) -- 
		(-16.117000,-35.749000) -- 
		(-16.147000,-35.735000) -- 
		(-16.238000,-35.693000) -- 
		(-16.269000,-35.679000) -- 
		(-16.915000,-35.385000) -- 
		(-17.752000,-35.003000) -- 
		(-18.851000,-34.490000) -- 
		(-18.862000,-34.484000) -- 
		(-19.499000,-34.196000) -- 
		(-19.116000,-34.178000) -- 
		(-18.788000,-34.160000) -- 
		(-18.392000,-34.131000) -- 
		(-17.997000,-34.096000) -- 
		(-17.971000,-34.094000) -- 
		(-17.602000,-34.056000) -- 
		(-17.591000,-34.055000) -- 
		(-17.488000,-34.043000) -- 
		(-17.181000,-34.006000) -- 
		(-17.079000,-33.992000) -- 
		(-16.921000,-33.972000) -- 
		(-16.816000,-33.957000) -- 
		(-16.450000,-33.903000) -- 
		(-16.424000,-33.899000) -- 
		(-16.294000,-33.878000) -- 
		(-16.071000,-33.843000) -- 
		(-16.034000,-33.836000) -- 
		(-15.645000,-33.766000) -- 
		(-15.407000,-33.720000) -- 
		(-15.187000,-33.676000) -- 
		(-14.927000,-33.622000) -- 
		(-14.872000,-33.610000) -- 
		(-14.488000,-33.524000) -- 
		(-14.151000,-33.441000) -- 
		(-13.894000,-33.376000) -- 
		(-13.536000,-33.281000) -- 
		(-12.464000,-32.996000) -- 
		(-12.107000,-32.899000) -- 
		(-11.109000,-32.632000) -- 
		(-10.295000,-32.413000) -- 
		(-9.849000,-32.294000) -- 
		(-9.307000,-32.149000) -- 
		(-8.446000,-31.918000) -- 
		(-8.026000,-31.805000) -- 
		(-7.161000,-31.573000) -- 
		(-6.259000,-31.331000) -- 
		(-5.581000,-31.150000) -- 
		(-3.939000,-30.709000) -- 
		(-3.492000,-30.589000) -- 
		(-3.290000,-30.535000) -- 
		(-2.176000,-30.237000) -- 
		(-1.925000,-30.170000) -- 
		(-1.175000,-29.969000) -- 
		(-0.925000,-29.901000) -- 
		(-0.565000,-29.805000) -- 
		(0.202000,-29.599000) -- 
		(0.515000,-29.516000) -- 
		(0.874000,-29.419000) -- 
		(1.392000,-29.281000) -- 
		(2.943000,-28.865000) -- 
		(3.460000,-28.726000) -- 
		(3.877000,-28.615000) -- 
		(5.127000,-28.280000) -- 
		(5.543000,-28.168000) -- 
		(6.003000,-28.045000) -- 
		(6.632000,-27.876000) -- 
		(7.381000,-27.677000) -- 
		(7.812000,-27.560000) -- 
		(7.840000,-27.551000) -- 
		(8.323000,-27.549000) -- 
		(8.370000,-27.536000) -- 
		(9.139000,-27.303000) -- 
		(11.918000,-26.460000) -- 
		(13.208000,-26.114000) -- 
		(14.681000,-25.721000) -- 
		(16.331000,-25.277000) -- 
		(18.946000,-24.569000) -- 
		(19.099000,-24.531000) -- 
		(19.495000,-24.432000) -- 
		(19.886000,-24.339000) -- 
		(20.041000,-24.302000) -- 
		(20.151000,-24.268000) -- 
		(20.234000,-24.233000) -- 
		(20.317000,-24.184000) -- 
		(20.352000,-24.155000) -- 
		(20.386000,-24.119000) -- 
		(20.414000,-24.079000) -- 
		(20.459000,-23.979000) -- 
		(20.472000,-23.954000) -- 
		(20.512000,-23.879000) -- 
		(20.524000,-23.854000) -- 
		(20.553000,-23.819000) -- 
		(20.626000,-23.765000) -- 
		(20.704000,-23.731000) -- 
		(20.862000,-23.700000) -- 
		(21.134000,-23.681000) -- 
		(23.093000,-23.544000) -- 
		(23.746000,-23.498000) -- 
		(23.721000,-23.545000) -- 
		(23.644000,-23.686000) -- 
		(23.618000,-23.732000) -- 
		(23.604000,-23.759000) -- 
		(23.604000,-23.765000) -- 
		(23.603000,-23.834000) -- 
		(23.603000,-23.873000) -- 
		(23.602000,-23.909000) -- 
		(23.441000,-23.920000) -- 
		(23.402000,-23.922000) -- 
		(23.198000,-23.969000) -- 
		(23.157000,-23.971000) -- 
		(23.052000,-23.908000) -- 
		(22.986000,-23.942000) -- 
		(22.841000,-24.015000) -- 
		(22.825000,-24.023000) -- 
		(22.436000,-24.364000) -- 
		(22.180000,-24.529000) -- 
		(21.949000,-24.678000) -- 
		(21.748000,-24.837000) -- 
		(21.267000,-25.101000) -- 
		(21.053000,-25.198000) -- 
		(20.988000,-25.227000) -- 
		(20.421000,-25.565000) -- 
		(18.853000,-26.498000) -- 
		(18.418000,-26.707000) -- 
		(17.717000,-26.943000) -- 
		(16.977000,-27.091000) -- 
		(16.711000,-27.141000) -- 
		(16.282000,-27.240000) -- 
		(16.088000,-27.311000) -- 
		(15.815000,-27.498000) -- 
		(15.266000,-27.907000) -- 
		(15.075000,-28.048000) -- 
		(14.802000,-28.197000) -- 
		(14.452000,-28.307000) -- 
		(14.324000,-28.317000) -- 
		(13.947000,-28.370000) -- 
		(13.863000,-28.365000) -- 
		(13.525000,-28.368000) -- 
		(13.440000,-28.373000) -- 
		(13.218000,-28.399000) -- 
		(13.187000,-28.403000) -- 
		(13.105000,-28.423000) -- 
		(12.950000,-28.483000) -- 
		(12.877000,-28.519000) -- 
		(12.873000,-28.522000) -- 
		(12.809000,-28.562000) -- 
		(12.712000,-28.631000) -- 
		(12.685000,-28.659000) -- 
		(12.340000,-28.911000) -- 
		(12.238000,-28.972000) -- 
		(12.164000,-29.007000) -- 
		(12.087000,-29.038000) -- 
		(11.936000,-29.092000) -- 
		(11.854000,-29.120000) -- 
		(11.574000,-29.157000) -- 
		(11.526000,-29.182000) -- 
		(11.487000,-29.196000) -- 
		(11.448000,-29.204000) -- 
		(11.408000,-29.192000) -- 
		(11.226000,-29.236000) -- 
		(11.018000,-29.434000) -- 
		(11.010000,-29.453000) -- 
		(10.889000,-29.703000) -- 
		(10.759000,-30.022000) -- 
		(10.623000,-30.418000) -- 
		(10.655000,-30.787000) -- 
		(10.824000,-31.227000) -- 
		(10.891000,-31.344000) -- 
		(11.005000,-31.540000) -- 
		(11.161000,-31.694000) -- 
		(11.338000,-31.832000) -- 
		(11.492000,-31.952000) -- 
		(11.622000,-32.090000) -- 
		(11.771000,-32.315000) -- 
		(11.849000,-32.412000) -- 
		(11.970000,-32.564000) -- 
		(12.044000,-32.656000) -- 
		(12.141000,-32.876000) -- 
		(12.375000,-33.179000) -- 
		(12.557000,-33.322000) -- 
		(12.985000,-33.586000) -- 
		(13.030000,-33.624000) -- 
		(13.323000,-33.789000) -- 
		(13.472000,-33.987000) -- 
		(13.621000,-34.311000) -- 
		(13.679000,-34.504000) -- 
		(13.783000,-34.658000) -- 
		(13.978000,-34.828000) -- 
		(14.199000,-34.927000) -- 
		(14.393000,-35.065000) -- 
		(14.614000,-35.142000) -- 
		(14.880000,-35.208000) -- 
		(15.062000,-35.307000) -- 
		(15.308000,-35.477000) -- 
		(15.490000,-35.697000) -- 
		(15.763000,-36.005000) -- 
		(15.957000,-36.203000) -- 
		(16.425000,-36.478000) -- 
		(16.678000,-36.599000) -- 
		(17.009000,-36.704000) -- 
		(17.158000,-36.797000) -- 
		(17.359000,-36.973000) -- 
		(17.411000,-37.006000) -- 
		(17.826000,-37.430000) -- 
		(17.989000,-37.682000) -- 
		(18.125000,-37.979000) -- 
		(18.190000,-38.309000) -- 
		(18.177000,-38.568000) -- 
		(18.138000,-38.832000) -- 
		(18.079000,-38.997000) -- 
		(18.066000,-39.118000) -- 
		(17.982000,-39.206000) -- 
		(17.763000,-39.316000) -- 
		(17.591000,-39.090000) -- 
		(17.277000,-38.704000) -- 
		(16.989000,-38.371000);
	\filldraw [draw=black, ultra thick, fill=lightgreen]
		(15.769000,-37.122000) -- 
		(16.158000,-37.494000) -- 
		(16.438000,-37.776000) -- 
		(16.864000,-38.231000) -- 
		(16.988000,-38.370000) -- 
		(16.989000,-38.371000) -- 
		(17.277000,-38.704000) -- 
		(17.591000,-39.090000) -- 
		(17.763000,-39.316000) -- 
		(17.677000,-39.360000) -- 
		(17.184000,-39.519000) -- 
		(16.970000,-39.623000) -- 
		(16.898000,-39.679000) -- 
		(16.749000,-39.766000) -- 
		(16.684000,-39.827000) -- 
		(16.638000,-40.023000) -- 
		(16.600000,-40.184000) -- 
		(16.587000,-40.388000) -- 
		(16.639000,-40.685000) -- 
		(16.697000,-41.075000) -- 
		(16.678000,-41.284000) -- 
		(16.600000,-41.592000) -- 
		(16.533000,-41.729000) -- 
		(16.496000,-41.801000) -- 
		(16.204000,-42.159000) -- 
		(16.074000,-42.257000) -- 
		(15.788000,-42.422000) -- 
		(15.457000,-42.532000) -- 
		(14.581000,-42.912000) -- 
		(14.455000,-42.956000) -- 
		(14.056000,-43.088000) -- 
		(13.666000,-43.231000) -- 
		(13.498000,-43.341000) -- 
		(13.420000,-43.440000) -- 
		(13.342000,-43.671000) -- 
		(13.342000,-43.775000) -- 
		(13.472000,-43.990000) -- 
		(13.848000,-44.386000) -- 
		(13.906000,-44.501000) -- 
		(13.916000,-44.523000) -- 
		(14.004000,-44.721000) -- 
		(14.004000,-44.957000) -- 
		(13.965000,-45.227000) -- 
		(13.893000,-45.546000) -- 
		(13.738000,-46.035000) -- 
		(13.601000,-46.272000) -- 
		(13.517000,-46.365000) -- 
		(13.322000,-46.437000) -- 
		(13.154000,-46.453000) -- 
		(12.985000,-46.404000) -- 
		(12.615000,-46.211000) -- 
		(12.422000,-46.094000) -- 
		(12.407000,-46.085000) -- 
		(12.148000,-45.975000) -- 
		(12.057000,-45.920000) -- 
		(11.745000,-45.969000) -- 
		(11.629000,-46.030000) -- 
		(11.564000,-46.129000) -- 
		(11.527000,-46.204000) -- 
		(11.328000,-46.600000) -- 
		(11.304000,-46.646000) -- 
		(11.213000,-46.772000) -- 
		(10.934000,-47.075000) -- 
		(10.785000,-47.151000) -- 
		(10.526000,-47.201000) -- 
		(10.396000,-47.195000) -- 
		(10.253000,-47.166000) -- 
		(10.188000,-47.152000) -- 
		(9.877000,-46.998000) -- 
		(9.857000,-46.976000) -- 
		(9.630000,-46.937000) -- 
		(9.442000,-46.992000) -- 
		(9.307000,-47.091000) -- 
		(9.208000,-47.163000) -- 
		(9.065000,-47.355000) -- 
		(8.955000,-47.542000) -- 
		(8.871000,-47.740000) -- 
		(8.871000,-47.949000) -- 
		(8.916000,-48.152000) -- 
		(9.027000,-48.301000) -- 
		(9.342000,-48.525000) -- 
		(9.345000,-48.526000) -- 
		(9.533000,-48.614000) -- 
		(9.734000,-48.680000) -- 
		(9.903000,-48.697000) -- 
		(9.928000,-48.836000) -- 
		(9.974000,-49.098000) -- 
		(9.968000,-49.252000) -- 
		(9.918000,-49.384000) -- 
		(9.903000,-49.423000) -- 
		(9.747000,-49.522000) -- 
		(9.546000,-49.566000) -- 
		(9.409000,-49.558000) -- 
		(9.189000,-49.544000) -- 
		(8.495000,-49.379000) -- 
		(8.151000,-49.379000) -- 
		(7.988000,-49.417000) -- 
		(7.807000,-49.494000) -- 
		(7.743000,-49.513000) -- 
		(7.489000,-49.582000) -- 
		(7.215000,-49.720000) -- 
		(7.140000,-49.626000) -- 
		(7.108000,-49.589000) -- 
		(6.776000,-49.201000) -- 
		(6.707000,-49.121000) -- 
		(6.666000,-49.070000) -- 
		(6.200000,-48.497000) -- 
		(6.156000,-48.443000) -- 
		(5.530000,-47.631000) -- 
		(5.029000,-47.041000) -- 
		(4.789000,-46.784000) -- 
		(4.451000,-46.422000) -- 
		(4.275000,-46.252000) -- 
		(4.157000,-46.130000) -- 
		(4.140000,-46.112000) -- 
		(3.787000,-45.777000) -- 
		(3.663000,-45.659000) -- 
		(3.438000,-45.445000) -- 
		(3.422000,-45.429000) -- 
		(2.988000,-45.098000) -- 
		(2.695000,-44.879000) -- 
		(2.535000,-44.760000) -- 
		(2.446000,-44.691000) -- 
		(2.205000,-44.505000) -- 
		(1.985000,-44.353000) -- 
		(1.734000,-44.179000) -- 
		(1.077000,-43.745000) -- 
		(0.656000,-43.499000) -- 
		(0.534000,-43.432000) -- 
		(0.031000,-43.155000) -- 
		(-0.054000,-43.108000) -- 
		(-0.260000,-43.003000) -- 
		(-0.731000,-42.762000) -- 
		(-1.145000,-42.570000) -- 
		(-1.172000,-42.556000) -- 
		(-1.444000,-42.432000) -- 
		(-1.867000,-42.239000) -- 
		(-2.632000,-41.890000) -- 
		(-3.073000,-41.696000) -- 
		(-3.141000,-41.664000) -- 
		(-3.564000,-41.467000) -- 
		(-3.746000,-41.382000) -- 
		(-3.769000,-41.371000) -- 
		(-4.298000,-41.133000) -- 
		(-4.482000,-41.049000) -- 
		(-4.445000,-41.014000) -- 
		(-4.336000,-40.908000) -- 
		(-4.300000,-40.872000) -- 
		(-4.117000,-40.474000) -- 
		(-3.569000,-39.281000) -- 
		(-3.387000,-38.882000) -- 
		(-3.368000,-38.841000) -- 
		(-3.338000,-38.774000) -- 
		(-3.311000,-38.718000) -- 
		(-3.293000,-38.677000) -- 
		(-3.212000,-38.502000) -- 
		(-2.971000,-37.975000) -- 
		(-2.891000,-37.799000) -- 
		(-2.871000,-37.758000) -- 
		(-2.813000,-37.634000) -- 
		(-2.795000,-37.591000) -- 
		(-2.712000,-37.412000) -- 
		(-2.465000,-36.874000) -- 
		(-2.384000,-36.694000) -- 
		(-2.318000,-36.551000) -- 
		(-2.121000,-36.123000) -- 
		(-2.056000,-35.979000) -- 
		(-1.821000,-35.986000) -- 
		(-0.745000,-36.015000) -- 
		(-0.754000,-35.991000) -- 
		(0.303000,-36.016000) -- 
		(1.342000,-36.041000) -- 
		(1.494000,-36.044000) -- 
		(2.381000,-36.066000) -- 
		(3.423000,-36.091000) -- 
		(3.487000,-36.093000) -- 
		(3.678000,-36.097000) -- 
		(3.741000,-36.098000) -- 
		(3.895000,-36.103000) -- 
		(4.356000,-36.114000) -- 
		(4.509000,-36.116000) -- 
		(4.546000,-36.117000) -- 
		(4.657000,-36.120000) -- 
		(4.694000,-36.121000) -- 
		(4.950000,-36.128000) -- 
		(5.718000,-36.146000) -- 
		(5.974000,-36.152000) -- 
		(6.215000,-36.157000) -- 
		(6.739000,-36.167000) -- 
		(6.935000,-36.167000) -- 
		(7.080000,-36.166000) -- 
		(7.175000,-36.162000) -- 
		(7.266000,-36.155000) -- 
		(7.529000,-36.131000) -- 
		(7.538000,-36.131000) -- 
		(7.627000,-36.117000) -- 
		(7.974000,-36.046000) -- 
		(8.223000,-35.977000) -- 
		(8.338000,-35.944000) -- 
		(8.545000,-35.877000) -- 
		(8.836000,-35.766000) -- 
		(9.013000,-35.688000) -- 
		(9.283000,-35.552000) -- 
		(9.466000,-35.447000) -- 
		(9.667000,-35.318000) -- 
		(9.756000,-35.259000) -- 
		(9.864000,-35.180000) -- 
		(9.926000,-35.131000) -- 
		(10.034000,-35.039000) -- 
		(10.333000,-34.786000) -- 
		(10.586000,-34.566000) -- 
		(11.343000,-33.900000) -- 
		(11.594000,-33.677000) -- 
		(12.144000,-34.123000) -- 
		(12.270000,-34.226000) -- 
		(12.300000,-34.250000) -- 
		(12.435000,-34.361000) -- 
		(12.829000,-34.681000) -- 
		(13.403000,-35.148000) -- 
		(13.870000,-35.527000) -- 
		(13.914000,-35.563000) -- 
		(14.352000,-35.920000) -- 
		(14.836000,-36.313000) -- 
		(14.955000,-36.410000) -- 
		(15.002000,-36.448000) -- 
		(15.459000,-36.840000) -- 
		(15.769000,-37.122000);
	\filldraw [draw=black, ultra thick, fill=gold]
		(-22.003000,-37.022000) -- 
		(-21.496000,-36.202000) -- 
		(-21.302000,-35.887000) -- 
		(-21.052000,-35.540000) -- 
		(-20.957000,-35.408000) -- 
		(-20.187000,-34.227000) -- 
		(-20.176000,-34.211000) -- 
		(-20.142000,-34.164000) -- 
		(-20.044000,-34.020000) -- 
		(-20.012000,-33.970000) -- 
		(-19.908000,-34.015000) -- 
		(-19.739000,-34.087000) -- 
		(-19.601000,-34.150000) -- 
		(-19.499000,-34.196000) -- 
		(-18.862000,-34.484000) -- 
		(-18.851000,-34.490000) -- 
		(-17.752000,-35.003000) -- 
		(-16.915000,-35.385000) -- 
		(-16.269000,-35.679000) -- 
		(-16.238000,-35.693000) -- 
		(-16.147000,-35.735000) -- 
		(-16.117000,-35.749000) -- 
		(-16.010000,-35.797000) -- 
		(-15.993000,-35.805000) -- 
		(-15.873000,-35.857000) -- 
		(-15.691000,-35.938000) -- 
		(-15.585000,-35.984000) -- 
		(-15.575000,-35.988000) -- 
		(-15.335000,-36.097000) -- 
		(-14.585000,-36.435000) -- 
		(-14.336000,-36.546000) -- 
		(-14.321000,-36.553000) -- 
		(-14.277000,-36.573000) -- 
		(-14.263000,-36.579000) -- 
		(-13.872000,-36.755000) -- 
		(-12.345000,-37.469000) -- 
		(-11.738000,-37.746000) -- 
		(-9.854000,-38.620000) -- 
		(-8.757000,-39.103000) -- 
		(-7.262000,-39.790000) -- 
		(-6.950000,-39.934000) -- 
		(-6.468000,-40.152000) -- 
		(-5.238000,-40.707000) -- 
		(-5.086000,-40.776000) -- 
		(-4.633000,-40.981000) -- 
		(-4.482000,-41.049000) -- 
		(-4.298000,-41.133000) -- 
		(-3.769000,-41.371000) -- 
		(-3.746000,-41.382000) -- 
		(-3.564000,-41.467000) -- 
		(-3.141000,-41.664000) -- 
		(-3.073000,-41.696000) -- 
		(-2.632000,-41.890000) -- 
		(-1.867000,-42.239000) -- 
		(-1.444000,-42.432000) -- 
		(-1.172000,-42.556000) -- 
		(-1.145000,-42.570000) -- 
		(-0.731000,-42.762000) -- 
		(-0.260000,-43.003000) -- 
		(-0.054000,-43.108000) -- 
		(0.031000,-43.155000) -- 
		(0.534000,-43.432000) -- 
		(0.656000,-43.499000) -- 
		(1.077000,-43.745000) -- 
		(1.734000,-44.179000) -- 
		(1.985000,-44.353000) -- 
		(2.205000,-44.505000) -- 
		(2.446000,-44.691000) -- 
		(2.535000,-44.760000) -- 
		(2.695000,-44.879000) -- 
		(2.988000,-45.098000) -- 
		(3.422000,-45.429000) -- 
		(3.438000,-45.445000) -- 
		(3.663000,-45.659000) -- 
		(3.787000,-45.777000) -- 
		(4.140000,-46.112000) -- 
		(4.157000,-46.130000) -- 
		(4.275000,-46.252000) -- 
		(4.451000,-46.422000) -- 
		(4.789000,-46.784000) -- 
		(5.029000,-47.041000) -- 
		(5.530000,-47.631000) -- 
		(6.156000,-48.443000) -- 
		(6.200000,-48.497000) -- 
		(6.666000,-49.070000) -- 
		(6.707000,-49.121000) -- 
		(6.776000,-49.201000) -- 
		(7.108000,-49.589000) -- 
		(7.140000,-49.626000) -- 
		(7.215000,-49.720000) -- 
		(7.378000,-49.924000) -- 
		(7.429000,-49.988000) -- 
		(7.856000,-50.543000) -- 
		(8.015000,-50.749000) -- 
		(8.193000,-50.980000) -- 
		(8.727000,-51.673000) -- 
		(8.904000,-51.904000) -- 
		(8.924000,-51.930000) -- 
		(8.984000,-52.007000) -- 
		(9.003000,-52.032000) -- 
		(8.953000,-52.057000) -- 
		(8.802000,-52.126000) -- 
		(8.751000,-52.149000) -- 
		(8.498000,-52.265000) -- 
		(7.738000,-52.611000) -- 
		(7.484000,-52.726000) -- 
		(7.331000,-52.803000) -- 
		(7.173000,-52.877000) -- 
		(6.958000,-52.975000) -- 
		(6.780000,-53.059000) -- 
		(6.493000,-53.205000) -- 
		(6.356000,-53.285000) -- 
		(6.249000,-53.349000) -- 
		(6.246000,-53.349000) -- 
		(6.132000,-53.421000) -- 
		(5.960000,-53.538000) -- 
		(5.922000,-53.564000) -- 
		(5.710000,-53.723000) -- 
		(5.559000,-53.848000) -- 
		(5.546000,-53.858000) -- 
		(5.309000,-54.075000) -- 
		(5.071000,-54.316000) -- 
		(4.789000,-54.633000) -- 
		(4.652000,-54.773000) -- 
		(4.496000,-54.943000) -- 
		(4.334000,-55.112000) -- 
		(4.091000,-55.374000) -- 
		(4.029000,-55.441000) -- 
		(3.814000,-55.630000) -- 
		(3.524000,-55.874000) -- 
		(3.294000,-56.067000) -- 
		(3.043000,-56.257000) -- 
		(3.031000,-56.148000) -- 
		(3.018000,-56.034000) -- 
		(2.985000,-55.742000) -- 
		(2.985000,-55.739000) -- 
		(2.964000,-55.426000) -- 
		(2.867000,-54.508000) -- 
		(2.865000,-54.490000) -- 
		(2.815000,-54.006000) -- 
		(2.745000,-53.354000) -- 
		(2.782000,-52.996000) -- 
		(2.505000,-52.893000) -- 
		(2.101000,-52.489000) -- 
		(1.638000,-52.369000) -- 
		(1.043000,-52.360000) -- 
		(0.494000,-52.520000) -- 
		(0.301000,-52.650000) -- 
		(-0.827000,-53.444000) -- 
		(-0.970000,-53.551000) -- 
		(-1.956000,-54.287000) -- 
		(-2.261000,-54.513000) -- 
		(-2.556000,-54.733000) -- 
		(-2.576000,-54.748000) -- 
		(-2.806000,-54.919000) -- 
		(-5.683000,-55.978000) -- 
		(-5.779000,-56.050000) -- 
		(-5.907000,-56.147000) -- 
		(-5.922000,-56.158000) -- 
		(-6.103000,-56.295000) -- 
		(-6.205000,-56.325000) -- 
		(-7.584000,-56.702000) -- 
		(-7.643000,-56.846000) -- 
		(-7.952000,-57.664000) -- 
		(-8.107000,-57.782000) -- 
		(-8.201000,-57.854000) -- 
		(-10.678000,-59.749000) -- 
		(-12.067000,-60.601000) -- 
		(-12.133000,-60.653000) -- 
		(-12.037000,-60.807000) -- 
		(-12.383000,-61.136000) -- 
		(-12.785000,-61.150000) -- 
		(-12.786000,-61.149000) -- 
		(-13.579000,-61.840000) -- 
		(-14.394000,-62.246000) -- 
		(-16.485000,-62.350000) -- 
		(-17.578000,-62.048000) -- 
		(-17.881000,-61.687000) -- 
		(-18.385000,-61.628000) -- 
		(-18.807000,-60.926000) -- 
		(-18.807000,-58.324000) -- 
		(-18.127000,-57.807000) -- 
		(-18.123000,-57.821000) -- 
		(-18.111000,-57.860000) -- 
		(-18.108000,-57.872000) -- 
		(-14.918000,-56.183000) -- 
		(-13.990000,-55.692000) -- 
		(-13.818000,-55.660000) -- 
		(-12.215000,-54.865000) -- 
		(-10.006000,-54.865000) -- 
		(-9.303000,-54.546000) -- 
		(-9.059000,-54.435000) -- 
		(-8.858000,-54.344000) -- 
		(-6.591000,-53.301000) -- 
		(-5.200000,-52.792000) -- 
		(-5.043000,-52.649000) -- 
		(-5.184000,-52.326000) -- 
		(-5.245000,-52.187000) -- 
		(-5.679000,-51.252000) -- 
		(-6.592000,-51.056000) -- 
		(-8.070000,-50.711000) -- 
		(-8.069000,-50.711000) -- 
		(-7.751000,-50.207000) -- 
		(-7.686000,-50.096000) -- 
		(-7.534000,-49.834000) -- 
		(-7.593000,-49.786000) -- 
		(-7.662000,-49.733000) -- 
		(-7.967000,-49.496000) -- 
		(-8.057000,-49.426000) -- 
		(-8.130000,-49.368000) -- 
		(-8.308000,-49.210000) -- 
		(-14.448000,-45.788000) -- 
		(-15.155000,-44.778000) -- 
		(-22.199000,-41.979000) -- 
		(-24.777000,-41.247000) -- 
		(-24.807000,-41.234000) -- 
		(-24.772000,-41.183000) -- 
		(-24.738000,-41.132000) -- 
		(-24.625000,-40.966000) -- 
		(-24.080000,-40.159000) -- 
		(-24.049000,-40.112000) -- 
		(-23.896000,-39.892000) -- 
		(-23.662000,-39.557000) -- 
		(-23.448000,-39.251000) -- 
		(-23.237000,-38.909000) -- 
		(-23.213000,-38.868000) -- 
		(-23.156000,-38.776000) -- 
		(-23.132000,-38.738000) -- 
		(-22.639000,-37.938000) -- 
		(-22.003000,-37.022000);
	\filldraw [draw=black, ultra thick, fill=blue]
		(36.978000,-88.710000) -- 
		(36.226000,-87.742000) -- 
		(35.801000,-87.196000) -- 
		(35.602000,-86.940000) -- 
		(35.002000,-86.169000) -- 
		(34.802000,-85.911000) -- 
		(34.731000,-85.820000) -- 
		(34.518000,-85.546000) -- 
		(34.447000,-85.454000) -- 
		(34.146000,-85.067000) -- 
		(33.303000,-83.983000) -- 
		(33.242000,-83.904000) -- 
		(32.940000,-83.516000) -- 
		(32.866000,-83.421000) -- 
		(32.642000,-83.133000) -- 
		(32.567000,-83.036000) -- 
		(31.709000,-81.934000) -- 
		(31.420000,-81.561000) -- 
		(30.377000,-80.220000) -- 
		(29.135000,-78.624000) -- 
		(28.277000,-77.519000) -- 
		(27.181000,-76.108000) -- 
		(25.628000,-74.110000) -- 
		(25.103000,-73.437000) -- 
		(23.885000,-71.878000) -- 
		(23.571000,-71.476000) -- 
		(22.788000,-70.466000) -- 
		(22.865000,-70.426000) -- 
		(22.951000,-70.378000) -- 
		(22.951000,-70.377000) -- 
		(23.035000,-70.332000) -- 
		(23.099000,-70.313000) -- 
		(23.124000,-70.305000) -- 
		(23.159000,-70.301000) -- 
		(23.184000,-70.301000) -- 
		(23.317000,-70.303000) -- 
		(23.716000,-70.308000) -- 
		(23.849000,-70.309000) -- 
		(24.104000,-70.313000) -- 
		(24.866000,-70.322000) -- 
		(25.120000,-70.324000) -- 
		(25.351000,-70.327000) -- 
		(26.041000,-70.336000) -- 
		(26.271000,-70.339000) -- 
		(26.503000,-70.342000) -- 
		(27.196000,-70.350000) -- 
		(27.427000,-70.352000) -- 
		(27.654000,-70.356000) -- 
		(28.332000,-70.365000) -- 
		(28.558000,-70.367000) -- 
		(28.788000,-70.370000) -- 
		(29.478000,-70.379000) -- 
		(29.708000,-70.382000) -- 
		(29.952000,-70.385000) -- 
		(30.681000,-70.395000) -- 
		(30.924000,-70.396000) -- 
		(31.171000,-70.401000) -- 
		(31.912000,-70.410000) -- 
		(32.158000,-70.412000) -- 
		(32.400000,-70.415000) -- 
		(33.125000,-70.424000) -- 
		(33.366000,-70.426000) -- 
		(33.713000,-70.430000) -- 
		(34.754000,-70.443000) -- 
		(35.100000,-70.446000) -- 
		(35.195000,-70.448000) -- 
		(35.478000,-70.452000) -- 
		(35.572000,-70.453000) -- 
		(35.652000,-70.453000) -- 
		(35.664000,-70.452000) -- 
		(35.710000,-70.447000) -- 
		(35.766000,-70.436000) -- 
		(35.825000,-70.419000) -- 
		(35.888000,-70.394000) -- 
		(35.925000,-70.375000) -- 
		(35.945000,-70.364000) -- 
		(36.002000,-70.326000) -- 
		(36.086000,-70.385000) -- 
		(36.832000,-70.517000) -- 
		(37.227000,-70.633000) -- 
		(37.486000,-70.676000) -- 
		(40.030000,-71.090000) -- 
		(41.970000,-71.667000) -- 
		(42.326000,-71.772000) -- 
		(42.534000,-71.794000) -- 
		(42.910000,-71.734000) -- 
		(43.462000,-71.584000) -- 
		(43.475000,-71.580000) -- 
		(43.682000,-71.569000) -- 
		(44.279000,-71.619000) -- 
		(45.103000,-71.520000) -- 
		(45.564000,-71.388000) -- 
		(45.733000,-71.304000) -- 
		(45.927000,-71.207000) -- 
		(46.329000,-70.954000) -- 
		(46.777000,-70.586000) -- 
		(46.842000,-70.509000) -- 
		(46.991000,-70.426000) -- 
		(47.374000,-70.278000) -- 
		(47.880000,-69.942000) -- 
		(48.081000,-69.855000) -- 
		(48.717000,-69.624000) -- 
		(48.827000,-69.547000) -- 
		(48.938000,-69.360000) -- 
		(49.295000,-68.464000) -- 
		(49.250000,-68.271000) -- 
		(49.152000,-68.106000) -- 
		(48.945000,-67.600000) -- 
		(48.939000,-67.402000) -- 
		(49.029000,-67.341000) -- 
		(49.107000,-67.287000) -- 
		(49.877000,-66.945000) -- 
		(50.160000,-66.818000) -- 
		(50.191000,-66.803000) -- 
		(50.522000,-66.605000) -- 
		(50.658000,-66.418000) -- 
		(50.769000,-66.226000) -- 
		(50.872000,-66.100000) -- 
		(51.288000,-65.907000) -- 
		(51.385000,-65.819000) -- 
		(51.392000,-65.588000) -- 
		(51.359000,-65.440000) -- 
		(52.421000,-65.347000) -- 
		(52.496000,-65.352000) -- 
		(55.911000,-65.556000) -- 
		(57.049000,-65.623000) -- 
		(58.835000,-65.600000) -- 
		(60.694000,-65.576000) -- 
		(60.717000,-67.099000) -- 
		(60.721000,-67.347000) -- 
		(60.742000,-69.071000) -- 
		(60.762000,-70.856000) -- 
		(60.907000,-71.003000) -- 
		(60.939000,-71.074000) -- 
		(60.939000,-71.532000) -- 
		(60.939000,-71.580000) -- 
		(61.179000,-71.728000) -- 
		(61.341000,-71.915000) -- 
		(61.846000,-72.906000) -- 
		(61.885000,-73.021000) -- 
		(62.041000,-73.197000) -- 
		(62.182000,-73.309000) -- 
		(62.352000,-73.445000) -- 
		(62.644000,-73.880000) -- 
		(62.648000,-74.155000) -- 
		(62.653000,-74.689000) -- 
		(62.931000,-74.774000) -- 
		(63.194000,-74.853000) -- 
		(63.828000,-74.850000) -- 
		(65.727000,-74.839000) -- 
		(66.236000,-74.835000) -- 
		(66.359000,-74.834000) -- 
		(66.057000,-75.649000) -- 
		(65.902000,-76.063000) -- 
		(65.741000,-76.434000) -- 
		(65.674000,-76.565000) -- 
		(65.614000,-76.659000) -- 
		(65.217000,-77.204000) -- 
		(64.701000,-77.850000) -- 
		(64.648000,-77.914000) -- 
		(64.169000,-78.536000) -- 
		(64.112000,-78.618000) -- 
		(64.054000,-78.700000) -- 
		(63.954000,-78.842000) -- 
		(63.907000,-78.901000) -- 
		(63.887000,-78.933000) -- 
		(63.811000,-79.060000) -- 
		(63.779000,-79.126000) -- 
		(63.768000,-79.146000) -- 
		(63.703000,-79.253000) -- 
		(63.502000,-79.684000) -- 
		(62.960000,-81.180000) -- 
		(62.843000,-81.500000) -- 
		(62.753000,-81.739000) -- 
		(62.733000,-81.771000) -- 
		(62.708000,-81.800000) -- 
		(62.663000,-81.838000) -- 
		(62.666000,-81.882000) -- 
		(62.636000,-82.058000) -- 
		(62.278000,-83.083000) -- 
		(62.250000,-83.150000) -- 
		(62.178000,-83.353000) -- 
		(62.108000,-83.596000) -- 
		(62.083000,-83.664000) -- 
		(61.855000,-84.382000) -- 
		(61.826000,-84.448000) -- 
		(61.815000,-84.492000) -- 
		(61.731000,-84.795000) -- 
		(61.709000,-84.875000) -- 
		(61.680000,-85.015000) -- 
		(61.660000,-85.085000) -- 
		(61.652000,-85.156000) -- 
		(61.648000,-85.333000) -- 
		(61.671000,-85.617000) -- 
		(61.693000,-85.793000) -- 
		(61.696000,-85.864000) -- 
		(61.707000,-85.934000) -- 
		(61.771000,-86.499000) -- 
		(61.849000,-87.027000) -- 
		(61.871000,-87.238000) -- 
		(61.878000,-87.344000) -- 
		(61.877000,-87.414000) -- 
		(61.895000,-87.696000) -- 
		(61.904000,-87.766000) -- 
		(61.931000,-88.258000) -- 
		(61.932000,-88.435000) -- 
		(61.937000,-88.470000) -- 
		(61.927000,-88.717000) -- 
		(61.929000,-88.788000) -- 
		(61.920000,-88.894000) -- 
		(61.875000,-89.177000) -- 
		(61.853000,-89.246000) -- 
		(61.710000,-89.505000) -- 
		(61.678000,-89.571000) -- 
		(61.628000,-89.629000) -- 
		(61.521000,-89.782000) -- 
		(61.413000,-89.977000) -- 
		(61.337000,-90.181000) -- 
		(61.295000,-90.356000) -- 
		(61.292000,-90.462000) -- 
		(61.299000,-90.639000) -- 
		(61.332000,-90.816000) -- 
		(61.382000,-90.988000) -- 
		(61.413000,-91.053000) -- 
		(61.422000,-91.087000) -- 
		(61.491000,-91.217000) -- 
		(61.622000,-91.439000) -- 
		(62.038000,-92.060000) -- 
		(62.216000,-92.358000) -- 
		(62.252000,-92.423000) -- 
		(62.270000,-92.455000) -- 
		(62.415000,-92.791000) -- 
		(62.451000,-92.893000) -- 
		(62.501000,-93.101000) -- 
		(62.511000,-93.172000) -- 
		(62.522000,-93.206000) -- 
		(62.526000,-93.242000) -- 
		(62.485000,-93.870000) -- 
		(62.459000,-93.974000) -- 
		(62.446000,-94.008000) -- 
		(62.428000,-94.078000) -- 
		(62.365000,-94.210000) -- 
		(62.312000,-94.308000) -- 
		(61.985000,-94.807000) -- 
		(61.879000,-95.002000) -- 
		(61.856000,-95.071000) -- 
		(61.800000,-95.315000) -- 
		(61.776000,-95.527000) -- 
		(61.758000,-95.633000) -- 
		(61.741000,-95.703000) -- 
		(61.719000,-95.844000) -- 
		(61.704000,-95.914000) -- 
		(61.672000,-96.017000) -- 
		(61.646000,-96.085000) -- 
		(61.536000,-96.428000) -- 
		(61.472000,-96.597000) -- 
		(61.463000,-96.632000) -- 
		(61.412000,-96.730000) -- 
		(61.139000,-97.212000) -- 
		(61.062000,-97.340000) -- 
		(60.986000,-97.425000) -- 
		(60.900000,-97.503000) -- 
		(60.867000,-97.527000) -- 
		(60.762000,-97.586000) -- 
		(60.608000,-97.646000) -- 
		(60.245000,-97.745000) -- 
		(60.162000,-97.759000) -- 
		(60.039000,-97.787000) -- 
		(59.803000,-97.868000) -- 
		(59.591000,-97.988000) -- 
		(59.403000,-98.072000) -- 
		(59.332000,-98.110000) -- 
		(59.149000,-98.248000) -- 
		(59.142000,-98.254000) -- 
		(59.040000,-98.369000) -- 
		(59.011000,-98.395000) -- 
		(58.987000,-98.424000) -- 
		(58.902000,-98.560000) -- 
		(58.848000,-98.646000) -- 
		(58.771000,-98.849000) -- 
		(58.754000,-98.956000) -- 
		(58.743000,-99.062000) -- 
		(58.743000,-99.133000) -- 
		(58.791000,-99.415000) -- 
		(58.819000,-99.519000) -- 
		(58.921000,-99.791000) -- 
		(59.215000,-100.423000) -- 
		(59.248000,-100.511000) -- 
		(59.381000,-100.866000) -- 
		(59.487000,-101.151000) -- 
		(59.520000,-101.240000) -- 
		(59.580000,-101.483000) -- 
		(59.595000,-101.626000) -- 
		(59.592000,-101.697000) -- 
		(59.536000,-101.943000) -- 
		(59.529000,-101.963000) -- 
		(59.443000,-102.180000) -- 
		(59.421000,-102.322000) -- 
		(59.417000,-102.467000) -- 
		(59.415000,-102.500000) -- 
		(59.435000,-102.677000) -- 
		(59.441000,-102.856000) -- 
		(59.449000,-102.896000) -- 
		(59.455000,-102.926000) -- 
		(59.472000,-102.959000) -- 
		(59.516000,-103.020000) -- 
		(59.458000,-103.054000) -- 
		(59.421000,-103.077000) -- 
		(59.398000,-103.089000) -- 
		(59.191000,-103.144000) -- 
		(59.101000,-103.121000) -- 
		(59.087000,-103.117000) -- 
		(58.990000,-103.109000) -- 
		(58.886000,-103.100000) -- 
		(58.186000,-102.973000) -- 
		(57.985000,-102.962000) -- 
		(57.453000,-102.874000) -- 
		(57.238000,-102.814000) -- 
		(57.219000,-102.808000) -- 
		(57.024000,-102.741000) -- 
		(56.947000,-102.714000) -- 
		(56.727000,-102.599000) -- 
		(56.474000,-102.461000) -- 
		(55.806000,-101.955000) -- 
		(55.735000,-101.873000) -- 
		(55.624000,-101.702000) -- 
		(55.586000,-101.532000) -- 
		(55.586000,-100.938000) -- 
		(55.469000,-100.531000) -- 
		(55.340000,-100.267000) -- 
		(55.178000,-100.052000) -- 
		(54.964000,-99.827000) -- 
		(54.724000,-99.678000) -- 
		(54.471000,-99.497000) -- 
		(54.341000,-99.376000) -- 
		(54.076000,-99.178000) -- 
		(53.940000,-99.059000) -- 
		(53.810000,-98.946000) -- 
		(53.168000,-98.446000) -- 
		(52.520000,-97.985000) -- 
		(52.442000,-97.929000) -- 
		(52.053000,-97.714000) -- 
		(51.483000,-97.486000) -- 
		(51.268000,-97.400000) -- 
		(50.477000,-97.153000) -- 
		(50.224000,-97.032000) -- 
		(50.004000,-96.960000) -- 
		(49.615000,-96.883000) -- 
		(48.538000,-96.734000) -- 
		(48.138000,-96.658000) -- 
		(47.423000,-96.520000) -- 
		(47.000000,-96.462000) -- 
		(46.774000,-96.431000) -- 
		(46.755000,-96.430000) -- 
		(46.281000,-96.376000) -- 
		(46.017000,-96.391000) -- 
		(45.769000,-96.404000) -- 
		(45.108000,-96.281000) -- 
		(45.036000,-96.266000) -- 
		(44.900000,-96.260000) -- 
		(44.614000,-96.178000) -- 
		(44.407000,-96.062000) -- 
		(43.830000,-95.435000) -- 
		(43.642000,-95.342000) -- 
		(43.434000,-95.314000) -- 
		(43.432000,-95.316000) -- 
		(42.954000,-95.413000) -- 
		(42.804000,-95.521000) -- 
		(42.779000,-95.539000) -- 
		(42.740000,-95.576000) -- 
		(42.632000,-95.676000) -- 
		(42.554000,-95.749000) -- 
		(42.492000,-95.806000) -- 
		(42.417000,-95.710000) -- 
		(42.192000,-95.418000) -- 
		(42.117000,-95.320000) -- 
		(41.872000,-95.006000) -- 
		(41.135000,-94.058000) -- 
		(40.889000,-93.741000) -- 
		(40.296000,-92.978000) -- 
		(38.517000,-90.689000) -- 
		(37.923000,-89.926000) -- 
		(37.499000,-89.380000) -- 
		(36.978000,-88.710000);
	\filldraw [draw=black, ultra thick, fill=lightgreen]
		(36.978000,-88.710000) -- 
		(37.499000,-89.380000) -- 
		(37.923000,-89.926000) -- 
		(37.806000,-89.964000) -- 
		(36.845000,-90.327000) -- 
		(36.166000,-90.582000) -- 
		(35.929000,-90.665000) -- 
		(35.690000,-90.743000) -- 
		(35.597000,-90.771000) -- 
		(35.449000,-90.815000) -- 
		(35.029000,-90.929000) -- 
		(34.849000,-90.970000) -- 
		(34.620000,-91.013000) -- 
		(34.208000,-91.074000) -- 
		(33.986000,-91.100000) -- 
		(33.763000,-91.120000) -- 
		(33.540000,-91.134000) -- 
		(33.508000,-91.136000) -- 
		(33.207000,-91.145000) -- 
		(32.357000,-91.148000) -- 
		(32.178000,-91.149000) -- 
		(31.940000,-91.150000) -- 
		(30.688000,-91.154000) -- 
		(30.270000,-91.155000) -- 
		(28.010000,-91.163000) -- 
		(23.887000,-91.172000) -- 
		(23.616000,-91.172000) -- 
		(23.344000,-91.173000) -- 
		(23.030000,-91.174000) -- 
		(22.835000,-91.174000) -- 
		(22.789000,-91.175000) -- 
		(22.676000,-91.177000) -- 
		(22.564000,-91.179000) -- 
		(22.470000,-91.180000) -- 
		(22.187000,-91.197000) -- 
		(22.050000,-91.205000) -- 
		(21.914000,-91.214000) -- 
		(21.397000,-91.251000) -- 
		(20.204000,-91.183000) -- 
		(20.094000,-91.183000) -- 
		(19.200000,-91.185000) -- 
		(18.909000,-91.258000) -- 
		(18.559000,-91.256000) -- 
		(18.049000,-91.255000) -- 
		(17.761000,-91.190000) -- 
		(17.588000,-91.190000) -- 
		(17.139000,-91.190000) -- 
		(17.125000,-91.246000) -- 
		(17.062000,-91.190000) -- 
		(16.806000,-91.190000) -- 
		(16.291000,-91.192000) -- 
		(16.812000,-90.562000) -- 
		(17.055000,-90.268000) -- 
		(17.539000,-89.683000) -- 
		(17.605000,-89.603000) -- 
		(19.149000,-87.738000) -- 
		(19.666000,-87.113000) -- 
		(18.735000,-86.018000) -- 
		(17.679000,-84.771000) -- 
		(17.232000,-84.243000) -- 
		(17.221000,-84.230000) -- 
		(16.935000,-84.233000) -- 
		(16.794000,-84.233000) -- 
		(16.588000,-84.344000) -- 
		(16.101000,-84.605000) -- 
		(15.721000,-84.808000) -- 
		(15.421000,-84.360000) -- 
		(15.184000,-84.382000) -- 
		(15.096000,-84.433000) -- 
		(15.022000,-84.478000) -- 
		(14.979000,-84.504000) -- 
		(14.910000,-84.545000) -- 
		(14.834000,-84.590000) -- 
		(13.654000,-85.297000) -- 
		(13.600000,-85.329000) -- 
		(12.507000,-85.981000) -- 
		(11.448000,-86.572000) -- 
		(10.007000,-86.717000) -- 
		(9.743000,-86.694000) -- 
		(9.698000,-86.685000) -- 
		(9.176000,-84.166000) -- 
		(9.621000,-83.344000) -- 
		(9.890000,-82.843000) -- 
		(9.947000,-82.247000) -- 
		(9.983000,-81.856000) -- 
		(10.031000,-81.334000) -- 
		(10.038000,-81.254000) -- 
		(9.880000,-81.074000) -- 
		(9.547000,-80.694000) -- 
		(9.536000,-80.682000) -- 
		(9.328000,-80.445000) -- 
		(9.309000,-79.368000) -- 
		(7.499000,-76.624000) -- 
		(5.356000,-75.268000) -- 
		(4.602000,-75.110000) -- 
		(3.431000,-74.861000) -- 
		(0.213000,-72.376000) -- 
		(-0.323000,-71.518000) -- 
		(-0.834000,-70.015000) -- 
		(-0.899000,-69.870000) -- 
		(-1.280000,-69.010000) -- 
		(-1.446000,-68.649000) -- 
		(-1.458000,-68.365000) -- 
		(-0.215000,-67.962000) -- 
		(0.429000,-66.816000) -- 
		(1.728000,-65.738000) -- 
		(2.133000,-65.256000) -- 
		(2.725000,-63.262000) -- 
		(2.862000,-62.799000) -- 
		(2.951000,-62.522000) -- 
		(3.207000,-61.730000) -- 
		(3.236000,-61.018000) -- 
		(3.234000,-60.925000) -- 
		(3.231000,-60.655000) -- 
		(3.222000,-59.211000) -- 
		(3.221000,-59.028000) -- 
		(3.220000,-58.845000) -- 
		(3.223000,-57.864000) -- 
		(3.176000,-57.430000) -- 
		(3.150000,-57.206000) -- 
		(3.043000,-56.257000) -- 
		(3.294000,-56.067000) -- 
		(3.524000,-55.874000) -- 
		(3.814000,-55.630000) -- 
		(4.029000,-55.441000) -- 
		(4.091000,-55.374000) -- 
		(4.334000,-55.112000) -- 
		(4.496000,-54.943000) -- 
		(4.652000,-54.773000) -- 
		(4.789000,-54.633000) -- 
		(5.071000,-54.316000) -- 
		(5.309000,-54.075000) -- 
		(5.546000,-53.858000) -- 
		(5.559000,-53.848000) -- 
		(5.710000,-53.723000) -- 
		(5.922000,-53.564000) -- 
		(5.960000,-53.538000) -- 
		(6.132000,-53.421000) -- 
		(6.246000,-53.349000) -- 
		(6.249000,-53.349000) -- 
		(6.356000,-53.285000) -- 
		(6.493000,-53.205000) -- 
		(6.780000,-53.059000) -- 
		(6.958000,-52.975000) -- 
		(7.173000,-52.877000) -- 
		(7.331000,-52.803000) -- 
		(7.484000,-52.726000) -- 
		(7.738000,-52.611000) -- 
		(8.498000,-52.265000) -- 
		(8.751000,-52.149000) -- 
		(9.056000,-52.528000) -- 
		(9.463000,-53.032000) -- 
		(9.966000,-53.665000) -- 
		(10.268000,-54.045000) -- 
		(10.444000,-54.266000) -- 
		(10.680000,-54.562000) -- 
		(11.609000,-55.728000) -- 
		(11.910000,-56.118000) -- 
		(12.314000,-56.640000) -- 
		(13.446000,-58.109000) -- 
		(14.707000,-59.747000) -- 
		(14.950000,-60.071000) -- 
		(15.768000,-61.274000) -- 
		(15.971000,-61.574000) -- 
		(16.348000,-62.081000) -- 
		(16.743000,-62.584000) -- 
		(16.810000,-62.669000) -- 
		(17.880000,-64.047000) -- 
		(18.428000,-65.070000) -- 
		(18.610000,-65.269000) -- 
		(18.939000,-65.627000) -- 
		(20.015000,-66.848000) -- 
		(20.362000,-67.313000) -- 
		(20.780000,-67.873000) -- 
		(20.939000,-68.078000) -- 
		(21.309000,-68.556000) -- 
		(22.351000,-69.902000) -- 
		(22.419000,-69.990000) -- 
		(22.788000,-70.466000) -- 
		(23.571000,-71.476000) -- 
		(23.885000,-71.878000) -- 
		(25.103000,-73.437000) -- 
		(25.628000,-74.110000) -- 
		(27.181000,-76.108000) -- 
		(28.277000,-77.519000) -- 
		(29.135000,-78.624000) -- 
		(30.377000,-80.220000) -- 
		(31.420000,-81.561000) -- 
		(31.709000,-81.934000) -- 
		(32.567000,-83.036000) -- 
		(32.642000,-83.133000) -- 
		(32.866000,-83.421000) -- 
		(32.940000,-83.516000) -- 
		(33.242000,-83.904000) -- 
		(33.303000,-83.983000) -- 
		(34.146000,-85.067000) -- 
		(34.447000,-85.454000) -- 
		(34.518000,-85.546000) -- 
		(34.731000,-85.820000) -- 
		(34.802000,-85.911000) -- 
		(35.002000,-86.169000) -- 
		(35.602000,-86.940000) -- 
		(35.801000,-87.196000) -- 
		(36.226000,-87.742000) -- 
		(36.978000,-88.710000);
	\filldraw [draw=black, ultra thick, fill=white]
		(22.951000,-70.378000) -- 
		(22.865000,-70.426000) -- 
		(22.788000,-70.466000) -- 
		(22.419000,-69.990000) -- 
		(22.351000,-69.902000) -- 
		(21.309000,-68.556000) -- 
		(20.939000,-68.078000) -- 
		(20.780000,-67.873000) -- 
		(20.362000,-67.313000) -- 
		(20.015000,-66.848000) -- 
		(18.939000,-65.627000) -- 
		(18.610000,-65.269000) -- 
		(18.428000,-65.070000) -- 
		(17.880000,-64.047000) -- 
		(16.810000,-62.669000) -- 
		(16.743000,-62.584000) -- 
		(16.348000,-62.081000) -- 
		(15.971000,-61.574000) -- 
		(15.768000,-61.274000) -- 
		(14.950000,-60.071000) -- 
		(14.707000,-59.747000) -- 
		(13.446000,-58.109000) -- 
		(12.314000,-56.640000) -- 
		(11.910000,-56.118000) -- 
		(11.609000,-55.728000) -- 
		(10.680000,-54.562000) -- 
		(10.444000,-54.266000) -- 
		(10.268000,-54.045000) -- 
		(9.966000,-53.665000) -- 
		(9.463000,-53.032000) -- 
		(9.056000,-52.528000) -- 
		(8.751000,-52.149000) -- 
		(8.802000,-52.126000) -- 
		(8.953000,-52.057000) -- 
		(9.003000,-52.032000) -- 
		(9.170000,-52.247000) -- 
		(9.671000,-52.890000) -- 
		(9.838000,-53.102000) -- 
		(10.375000,-53.468000) -- 
		(10.787000,-53.707000) -- 
		(10.856000,-53.741000) -- 
		(11.109000,-53.864000) -- 
		(11.365000,-53.951000) -- 
		(11.622000,-53.986000) -- 
		(12.046000,-54.027000) -- 
		(13.890000,-54.036000) -- 
		(14.415000,-54.061000) -- 
		(15.616000,-54.112000) -- 
		(15.758000,-54.540000) -- 
		(16.182000,-55.821000) -- 
		(16.323000,-56.248000) -- 
		(16.671000,-57.657000) -- 
		(16.959000,-58.823000) -- 
		(19.245000,-60.993000) -- 
		(19.456000,-61.193000) -- 
		(20.911000,-60.712000) -- 
		(24.745000,-59.784000) -- 
		(27.383000,-59.168000) -- 
		(29.570000,-58.639000) -- 
		(29.593000,-58.717000) -- 
		(29.671000,-58.990000) -- 
		(29.776000,-59.260000) -- 
		(29.827000,-59.392000) -- 
		(30.132000,-59.958000) -- 
		(30.242000,-60.090000) -- 
		(30.846000,-60.557000) -- 
		(30.897000,-60.596000) -- 
		(31.313000,-60.888000) -- 
		(31.746000,-61.157000) -- 
		(32.591000,-61.680000) -- 
		(33.051000,-61.933000) -- 
		(33.058000,-61.939000) -- 
		(33.337000,-62.164000) -- 
		(33.437000,-62.276000) -- 
		(33.450000,-62.291000) -- 
		(33.489000,-62.335000) -- 
		(33.501000,-62.348000) -- 
		(33.519000,-62.369000) -- 
		(33.538000,-62.389000) -- 
		(33.558000,-62.436000) -- 
		(33.568000,-62.458000) -- 
		(33.560000,-62.515000) -- 
		(33.536000,-62.684000) -- 
		(33.527000,-62.740000) -- 
		(33.192000,-63.228000) -- 
		(33.110000,-63.346000) -- 
		(32.667000,-64.099000) -- 
		(32.271000,-64.409000) -- 
		(32.187000,-64.651000) -- 
		(31.991000,-65.208000) -- 
		(31.923000,-65.364000) -- 
		(32.397000,-65.320000) -- 
		(32.773000,-65.158000) -- 
		(33.269000,-65.208000) -- 
		(33.746000,-65.256000) -- 
		(34.758000,-65.220000) -- 
		(35.255000,-65.200000) -- 
		(35.324000,-65.186000) -- 
		(35.415000,-65.164000) -- 
		(35.532000,-65.186000) -- 
		(35.601000,-65.199000) -- 
		(36.174000,-65.100000) -- 
		(37.162000,-65.105000) -- 
		(38.958000,-65.112000) -- 
		(41.869000,-65.158000) -- 
		(43.438000,-65.180000) -- 
		(44.955000,-66.293000) -- 
		(46.916000,-67.731000) -- 
		(48.672000,-66.829000) -- 
		(48.726000,-66.906000) -- 
		(49.773000,-66.447000) -- 
		(49.947000,-66.369000) -- 
		(51.359000,-65.440000) -- 
		(51.392000,-65.588000) -- 
		(51.385000,-65.819000) -- 
		(51.288000,-65.907000) -- 
		(50.872000,-66.100000) -- 
		(50.769000,-66.226000) -- 
		(50.658000,-66.418000) -- 
		(50.522000,-66.605000) -- 
		(50.191000,-66.803000) -- 
		(50.160000,-66.818000) -- 
		(49.877000,-66.945000) -- 
		(49.107000,-67.287000) -- 
		(49.029000,-67.341000) -- 
		(48.939000,-67.402000) -- 
		(48.945000,-67.600000) -- 
		(49.152000,-68.106000) -- 
		(49.250000,-68.271000) -- 
		(49.295000,-68.464000) -- 
		(48.938000,-69.360000) -- 
		(48.827000,-69.547000) -- 
		(48.717000,-69.624000) -- 
		(48.081000,-69.855000) -- 
		(47.880000,-69.942000) -- 
		(47.374000,-70.278000) -- 
		(46.991000,-70.426000) -- 
		(46.842000,-70.509000) -- 
		(46.777000,-70.586000) -- 
		(46.329000,-70.954000) -- 
		(45.927000,-71.207000) -- 
		(45.733000,-71.304000) -- 
		(45.564000,-71.388000) -- 
		(45.103000,-71.520000) -- 
		(44.279000,-71.619000) -- 
		(43.682000,-71.569000) -- 
		(43.475000,-71.580000) -- 
		(43.462000,-71.584000) -- 
		(42.910000,-71.734000) -- 
		(42.534000,-71.794000) -- 
		(42.326000,-71.772000) -- 
		(41.970000,-71.667000) -- 
		(40.030000,-71.090000) -- 
		(37.486000,-70.676000) -- 
		(37.227000,-70.633000) -- 
		(36.832000,-70.517000) -- 
		(36.086000,-70.385000) -- 
		(36.002000,-70.326000) -- 
		(35.945000,-70.364000) -- 
		(35.925000,-70.375000) -- 
		(35.888000,-70.394000) -- 
		(35.825000,-70.419000) -- 
		(35.766000,-70.436000) -- 
		(35.710000,-70.447000) -- 
		(35.664000,-70.452000) -- 
		(35.652000,-70.453000) -- 
		(35.572000,-70.453000) -- 
		(35.478000,-70.452000) -- 
		(35.195000,-70.448000) -- 
		(35.100000,-70.446000) -- 
		(34.754000,-70.443000) -- 
		(33.713000,-70.430000) -- 
		(33.366000,-70.426000) -- 
		(33.125000,-70.424000) -- 
		(32.400000,-70.415000) -- 
		(32.158000,-70.412000) -- 
		(31.912000,-70.410000) -- 
		(31.171000,-70.401000) -- 
		(30.924000,-70.396000) -- 
		(30.681000,-70.395000) -- 
		(29.952000,-70.385000) -- 
		(29.708000,-70.382000) -- 
		(29.478000,-70.379000) -- 
		(28.788000,-70.370000) -- 
		(28.558000,-70.367000) -- 
		(28.332000,-70.365000) -- 
		(27.654000,-70.356000) -- 
		(27.427000,-70.352000) -- 
		(27.196000,-70.350000) -- 
		(26.503000,-70.342000) -- 
		(26.271000,-70.339000) -- 
		(26.041000,-70.336000) -- 
		(25.351000,-70.327000) -- 
		(25.120000,-70.324000) -- 
		(24.866000,-70.322000) -- 
		(24.104000,-70.313000) -- 
		(23.849000,-70.309000) -- 
		(23.716000,-70.308000) -- 
		(23.317000,-70.303000) -- 
		(23.184000,-70.301000) -- 
		(23.159000,-70.301000) -- 
		(23.124000,-70.305000) -- 
		(23.099000,-70.313000) -- 
		(23.035000,-70.332000) -- 
		(22.951000,-70.377000) -- 
		(22.951000,-70.378000);
	\filldraw [draw=black, ultra thick, fill=gold]
		(121.400000,-16.827000) -- 
		(130.637000,-16.639000) -- 
		(132.118000,-16.608000) -- 
		(132.443000,-16.605000) -- 
		(137.457000,-16.551000) -- 
		(141.672000,-16.922000) -- 
		(149.809000,-16.720000) -- 
		(158.069000,-18.823000) -- 
		(159.761000,-19.253000) -- 
		(160.450000,-19.304000) -- 
		(161.424000,-19.376000) -- 
		(163.778000,-19.662000) -- 
		(165.974000,-19.560000) -- 
		(167.227000,-19.501000) -- 
		(167.229000,-19.536000) -- 
		(167.235000,-19.639000) -- 
		(167.237000,-19.671000) -- 
		(167.258000,-20.878000) -- 
		(167.318000,-24.498000) -- 
		(167.319000,-24.597000) -- 
		(167.337000,-25.704000) -- 
		(167.300000,-28.085000) -- 
		(167.237000,-32.009000) -- 
		(167.186000,-35.223000) -- 
		(167.148000,-37.602000) -- 
		(167.145000,-37.826000) -- 
		(167.133000,-38.499000) -- 
		(167.129000,-38.722000) -- 
		(167.052000,-43.877000) -- 
		(167.017000,-47.730000) -- 
		(166.997000,-49.960000) -- 
		(167.097000,-50.180000) -- 
		(167.107000,-50.212000) -- 
		(167.230000,-50.603000) -- 
		(167.367000,-50.994000) -- 
		(167.373000,-51.013000) -- 
		(167.428000,-51.261000) -- 
		(167.426000,-51.350000) -- 
		(167.479000,-51.434000) -- 
		(167.708000,-51.553000) -- 
		(167.729000,-51.556000) -- 
		(169.086000,-51.641000) -- 
		(169.085000,-51.683000) -- 
		(169.068000,-52.150000) -- 
		(169.005000,-58.023000) -- 
		(168.993000,-59.087000) -- 
		(168.814000,-75.641000) -- 
		(168.750000,-81.512000) -- 
		(168.745000,-81.839000) -- 
		(167.136000,-81.836000) -- 
		(161.325000,-81.824000) -- 
		(159.388000,-81.819000) -- 
		(156.181000,-81.830000) -- 
		(146.557000,-81.855000) -- 
		(143.349000,-81.863000) -- 
		(143.211000,-81.613000) -- 
		(138.554000,-82.016000) -- 
		(127.458000,-82.976000) -- 
		(123.722000,-83.203000) -- 
		(123.116000,-83.239000) -- 
		(122.392000,-83.283000) -- 
		(118.779000,-83.609000) -- 
		(118.627000,-83.116000) -- 
		(118.170000,-81.631000) -- 
		(118.115000,-81.453000) -- 
		(117.786000,-81.415000) -- 
		(117.254000,-81.414000) -- 
		(115.656000,-81.410000) -- 
		(115.286000,-81.409000) -- 
		(115.203000,-81.408000) -- 
		(115.123000,-81.408000) -- 
		(115.028000,-81.408000) -- 
		(114.933000,-81.409000) -- 
		(114.725000,-81.411000) -- 
		(113.528000,-81.419000) -- 
		(113.348000,-81.419000) -- 
		(113.238000,-81.420000) -- 
		(113.128000,-81.420000) -- 
		(113.028000,-81.420000) -- 
		(112.927000,-81.420000) -- 
		(112.168000,-81.420000) -- 
		(109.285000,-81.417000) -- 
		(108.324000,-81.415000) -- 
		(108.096000,-81.414000) -- 
		(107.411000,-81.410000) -- 
		(107.277000,-81.409000) -- 
		(107.182000,-81.408000) -- 
		(107.097000,-81.409000) -- 
		(106.809000,-81.413000) -- 
		(105.689000,-81.422000) -- 
		(105.467000,-81.424000) -- 
		(105.391000,-81.424000) -- 
		(105.315000,-81.425000) -- 
		(105.180000,-81.629000) -- 
		(104.262000,-82.917000) -- 
		(104.046000,-83.218000) -- 
		(103.570000,-83.837000) -- 
		(102.776000,-84.844000) -- 
		(102.186000,-85.455000) -- 
		(101.829000,-85.916000) -- 
		(101.543000,-86.348000) -- 
		(101.422000,-86.692000) -- 
		(101.404000,-87.030000) -- 
		(101.415000,-87.420000) -- 
		(101.419000,-87.599000) -- 
		(101.750000,-89.085000) -- 
		(101.789000,-89.203000) -- 
		(101.663000,-89.205000) -- 
		(101.160000,-89.215000) -- 
		(99.272000,-89.252000) -- 
		(98.758000,-89.261000) -- 
		(98.642000,-89.263000) -- 
		(98.528000,-89.262000) -- 
		(98.444000,-89.262000) -- 
		(98.431000,-89.262000) -- 
		(97.850000,-89.251000) -- 
		(97.651000,-89.245000) -- 
		(97.581000,-89.245000) -- 
		(97.371000,-89.242000) -- 
		(97.300000,-89.241000) -- 
		(93.944000,-89.176000) -- 
		(93.808000,-89.241000) -- 
		(84.215000,-89.028000) -- 
		(82.844000,-88.997000) -- 
		(82.752000,-81.952000) -- 
		(82.751000,-81.865000) -- 
		(82.316000,-81.864000) -- 
		(81.008000,-81.861000) -- 
		(80.572000,-81.859000) -- 
		(80.262000,-81.885000) -- 
		(79.380000,-81.888000) -- 
		(78.535000,-81.889000) -- 
		(78.413000,-81.888000) -- 
		(77.763000,-81.896000) -- 
		(75.744000,-81.870000) -- 
		(74.966000,-81.874000) -- 
		(73.594000,-81.867000) -- 
		(72.420000,-81.855000) -- 
		(71.989000,-81.848000) -- 
		(70.538000,-81.832000) -- 
		(70.387000,-81.788000) -- 
		(70.325000,-81.776000) -- 
		(70.233000,-81.757000) -- 
		(70.137000,-81.771000) -- 
		(70.073000,-81.779000) -- 
		(70.042000,-81.763000) -- 
		(69.949000,-81.714000) -- 
		(69.918000,-81.696000) -- 
		(69.891000,-81.695000) -- 
		(69.807000,-81.690000) -- 
		(69.779000,-81.688000) -- 
		(69.695000,-81.686000) -- 
		(69.440000,-81.678000) -- 
		(69.355000,-81.674000) -- 
		(69.279000,-79.837000) -- 
		(69.089000,-75.242000) -- 
		(68.179000,-75.106000) -- 
		(66.473000,-74.851000) -- 
		(66.359000,-74.834000) -- 
		(66.236000,-74.835000) -- 
		(65.727000,-74.839000) -- 
		(63.828000,-74.850000) -- 
		(63.194000,-74.853000) -- 
		(62.931000,-74.774000) -- 
		(62.653000,-74.689000) -- 
		(62.648000,-74.155000) -- 
		(62.644000,-73.880000) -- 
		(62.352000,-73.445000) -- 
		(62.182000,-73.309000) -- 
		(62.041000,-73.197000) -- 
		(61.885000,-73.021000) -- 
		(61.846000,-72.906000) -- 
		(61.341000,-71.915000) -- 
		(61.179000,-71.728000) -- 
		(60.939000,-71.580000) -- 
		(60.939000,-71.532000) -- 
		(60.939000,-71.074000) -- 
		(60.907000,-71.003000) -- 
		(60.762000,-70.856000) -- 
		(60.742000,-69.071000) -- 
		(60.721000,-67.347000) -- 
		(60.717000,-67.099000) -- 
		(60.694000,-65.576000) -- 
		(58.835000,-65.600000) -- 
		(57.049000,-65.623000) -- 
		(55.911000,-65.556000) -- 
		(52.496000,-65.352000) -- 
		(52.421000,-65.347000) -- 
		(51.359000,-65.440000) -- 
		(49.947000,-66.369000) -- 
		(49.773000,-66.447000) -- 
		(48.726000,-66.906000) -- 
		(48.672000,-66.829000) -- 
		(46.916000,-67.731000) -- 
		(44.955000,-66.293000) -- 
		(43.438000,-65.180000) -- 
		(41.869000,-65.158000) -- 
		(38.958000,-65.112000) -- 
		(37.162000,-65.105000) -- 
		(36.174000,-65.100000) -- 
		(35.601000,-65.199000) -- 
		(35.532000,-65.186000) -- 
		(35.415000,-65.164000) -- 
		(35.324000,-65.186000) -- 
		(35.255000,-65.200000) -- 
		(34.758000,-65.220000) -- 
		(33.746000,-65.256000) -- 
		(33.269000,-65.208000) -- 
		(32.773000,-65.158000) -- 
		(32.397000,-65.320000) -- 
		(31.923000,-65.364000) -- 
		(31.991000,-65.208000) -- 
		(32.187000,-64.651000) -- 
		(32.271000,-64.409000) -- 
		(32.667000,-64.099000) -- 
		(33.110000,-63.346000) -- 
		(33.192000,-63.228000) -- 
		(33.527000,-62.740000) -- 
		(33.536000,-62.684000) -- 
		(33.560000,-62.515000) -- 
		(33.568000,-62.458000) -- 
		(33.558000,-62.436000) -- 
		(33.538000,-62.389000) -- 
		(33.519000,-62.369000) -- 
		(33.501000,-62.348000) -- 
		(33.489000,-62.335000) -- 
		(33.450000,-62.291000) -- 
		(33.437000,-62.276000) -- 
		(33.337000,-62.164000) -- 
		(33.058000,-61.939000) -- 
		(33.051000,-61.933000) -- 
		(32.591000,-61.680000) -- 
		(31.746000,-61.157000) -- 
		(31.313000,-60.888000) -- 
		(30.897000,-60.596000) -- 
		(30.846000,-60.557000) -- 
		(30.242000,-60.090000) -- 
		(30.132000,-59.958000) -- 
		(29.827000,-59.392000) -- 
		(29.776000,-59.260000) -- 
		(29.671000,-58.990000) -- 
		(29.593000,-58.717000) -- 
		(29.570000,-58.639000) -- 
		(27.383000,-59.168000) -- 
		(24.745000,-59.784000) -- 
		(20.911000,-60.712000) -- 
		(19.456000,-61.193000) -- 
		(19.245000,-60.993000) -- 
		(16.959000,-58.823000) -- 
		(16.671000,-57.657000) -- 
		(16.323000,-56.248000) -- 
		(16.182000,-55.821000) -- 
		(15.758000,-54.540000) -- 
		(15.616000,-54.112000) -- 
		(14.415000,-54.061000) -- 
		(13.890000,-54.036000) -- 
		(12.046000,-54.027000) -- 
		(11.622000,-53.986000) -- 
		(11.365000,-53.951000) -- 
		(11.109000,-53.864000) -- 
		(10.856000,-53.741000) -- 
		(10.787000,-53.707000) -- 
		(10.375000,-53.468000) -- 
		(9.838000,-53.102000) -- 
		(9.671000,-52.890000) -- 
		(9.170000,-52.247000) -- 
		(9.003000,-52.032000) -- 
		(8.984000,-52.007000) -- 
		(8.924000,-51.930000) -- 
		(8.904000,-51.904000) -- 
		(8.727000,-51.673000) -- 
		(8.193000,-50.980000) -- 
		(8.015000,-50.749000) -- 
		(7.856000,-50.543000) -- 
		(7.429000,-49.988000) -- 
		(7.378000,-49.924000) -- 
		(7.215000,-49.720000) -- 
		(7.489000,-49.582000) -- 
		(7.743000,-49.513000) -- 
		(7.807000,-49.494000) -- 
		(7.988000,-49.417000) -- 
		(8.151000,-49.379000) -- 
		(8.495000,-49.379000) -- 
		(9.189000,-49.544000) -- 
		(9.409000,-49.558000) -- 
		(9.546000,-49.566000) -- 
		(9.747000,-49.522000) -- 
		(9.903000,-49.423000) -- 
		(9.918000,-49.384000) -- 
		(9.968000,-49.252000) -- 
		(9.974000,-49.098000) -- 
		(9.928000,-48.836000) -- 
		(9.903000,-48.697000) -- 
		(9.734000,-48.680000) -- 
		(9.533000,-48.614000) -- 
		(9.345000,-48.526000) -- 
		(9.342000,-48.525000) -- 
		(9.027000,-48.301000) -- 
		(8.916000,-48.152000) -- 
		(8.871000,-47.949000) -- 
		(8.871000,-47.740000) -- 
		(8.955000,-47.542000) -- 
		(9.065000,-47.355000) -- 
		(9.208000,-47.163000) -- 
		(9.307000,-47.091000) -- 
		(9.442000,-46.992000) -- 
		(9.630000,-46.937000) -- 
		(9.857000,-46.976000) -- 
		(9.877000,-46.998000) -- 
		(10.188000,-47.152000) -- 
		(10.253000,-47.166000) -- 
		(10.396000,-47.195000) -- 
		(10.526000,-47.201000) -- 
		(10.785000,-47.151000) -- 
		(10.934000,-47.075000) -- 
		(11.213000,-46.772000) -- 
		(11.304000,-46.646000) -- 
		(11.328000,-46.600000) -- 
		(11.527000,-46.204000) -- 
		(11.564000,-46.129000) -- 
		(11.629000,-46.030000) -- 
		(11.745000,-45.969000) -- 
		(12.057000,-45.920000) -- 
		(12.148000,-45.975000) -- 
		(12.407000,-46.085000) -- 
		(12.422000,-46.094000) -- 
		(12.615000,-46.211000) -- 
		(12.985000,-46.404000) -- 
		(13.154000,-46.453000) -- 
		(13.322000,-46.437000) -- 
		(13.517000,-46.365000) -- 
		(13.601000,-46.272000) -- 
		(13.738000,-46.035000) -- 
		(13.893000,-45.546000) -- 
		(13.965000,-45.227000) -- 
		(14.004000,-44.957000) -- 
		(14.004000,-44.721000) -- 
		(13.916000,-44.523000) -- 
		(13.906000,-44.501000) -- 
		(13.848000,-44.386000) -- 
		(13.472000,-43.990000) -- 
		(13.342000,-43.775000) -- 
		(13.342000,-43.671000) -- 
		(13.420000,-43.440000) -- 
		(13.498000,-43.341000) -- 
		(13.666000,-43.231000) -- 
		(14.056000,-43.088000) -- 
		(14.455000,-42.956000) -- 
		(14.581000,-42.912000) -- 
		(15.457000,-42.532000) -- 
		(15.788000,-42.422000) -- 
		(16.074000,-42.257000) -- 
		(16.204000,-42.159000) -- 
		(16.496000,-41.801000) -- 
		(16.533000,-41.729000) -- 
		(16.600000,-41.592000) -- 
		(16.678000,-41.284000) -- 
		(16.697000,-41.075000) -- 
		(16.639000,-40.685000) -- 
		(16.587000,-40.388000) -- 
		(16.600000,-40.184000) -- 
		(16.638000,-40.023000) -- 
		(16.684000,-39.827000) -- 
		(16.749000,-39.766000) -- 
		(16.898000,-39.679000) -- 
		(16.970000,-39.623000) -- 
		(17.184000,-39.519000) -- 
		(17.677000,-39.360000) -- 
		(17.763000,-39.316000) -- 
		(17.982000,-39.206000) -- 
		(18.066000,-39.118000) -- 
		(18.079000,-38.997000) -- 
		(18.138000,-38.832000) -- 
		(18.177000,-38.568000) -- 
		(18.190000,-38.309000) -- 
		(18.125000,-37.979000) -- 
		(17.989000,-37.682000) -- 
		(17.826000,-37.430000) -- 
		(17.411000,-37.006000) -- 
		(17.359000,-36.973000) -- 
		(17.158000,-36.797000) -- 
		(17.009000,-36.704000) -- 
		(16.678000,-36.599000) -- 
		(16.425000,-36.478000) -- 
		(15.957000,-36.203000) -- 
		(15.763000,-36.005000) -- 
		(15.490000,-35.697000) -- 
		(15.308000,-35.477000) -- 
		(15.062000,-35.307000) -- 
		(14.880000,-35.208000) -- 
		(14.614000,-35.142000) -- 
		(14.393000,-35.065000) -- 
		(14.199000,-34.927000) -- 
		(13.978000,-34.828000) -- 
		(13.783000,-34.658000) -- 
		(13.679000,-34.504000) -- 
		(13.621000,-34.311000) -- 
		(13.472000,-33.987000) -- 
		(13.323000,-33.789000) -- 
		(13.030000,-33.624000) -- 
		(12.985000,-33.586000) -- 
		(12.557000,-33.322000) -- 
		(12.375000,-33.179000) -- 
		(12.141000,-32.876000) -- 
		(12.044000,-32.656000) -- 
		(11.970000,-32.564000) -- 
		(11.849000,-32.412000) -- 
		(11.771000,-32.315000) -- 
		(11.622000,-32.090000) -- 
		(11.492000,-31.952000) -- 
		(11.338000,-31.832000) -- 
		(11.161000,-31.694000) -- 
		(11.005000,-31.540000) -- 
		(10.891000,-31.344000) -- 
		(10.824000,-31.227000) -- 
		(10.655000,-30.787000) -- 
		(10.623000,-30.418000) -- 
		(10.759000,-30.022000) -- 
		(10.889000,-29.703000) -- 
		(11.010000,-29.453000) -- 
		(11.018000,-29.434000) -- 
		(11.226000,-29.236000) -- 
		(11.408000,-29.192000) -- 
		(11.448000,-29.204000) -- 
		(11.487000,-29.196000) -- 
		(11.526000,-29.182000) -- 
		(11.574000,-29.157000) -- 
		(11.854000,-29.120000) -- 
		(11.936000,-29.092000) -- 
		(12.087000,-29.038000) -- 
		(12.164000,-29.007000) -- 
		(12.238000,-28.972000) -- 
		(12.340000,-28.911000) -- 
		(12.685000,-28.659000) -- 
		(12.712000,-28.631000) -- 
		(12.809000,-28.562000) -- 
		(12.873000,-28.522000) -- 
		(12.877000,-28.519000) -- 
		(12.950000,-28.483000) -- 
		(13.105000,-28.423000) -- 
		(13.187000,-28.403000) -- 
		(13.218000,-28.399000) -- 
		(13.440000,-28.373000) -- 
		(13.525000,-28.368000) -- 
		(13.863000,-28.365000) -- 
		(13.947000,-28.370000) -- 
		(14.324000,-28.317000) -- 
		(14.452000,-28.307000) -- 
		(14.802000,-28.197000) -- 
		(15.075000,-28.048000) -- 
		(15.266000,-27.907000) -- 
		(15.815000,-27.498000) -- 
		(16.088000,-27.311000) -- 
		(16.282000,-27.240000) -- 
		(16.711000,-27.141000) -- 
		(16.977000,-27.091000) -- 
		(17.717000,-26.943000) -- 
		(18.418000,-26.707000) -- 
		(18.853000,-26.498000) -- 
		(20.421000,-25.565000) -- 
		(20.988000,-25.227000) -- 
		(21.053000,-25.198000) -- 
		(21.267000,-25.101000) -- 
		(21.748000,-24.837000) -- 
		(21.949000,-24.678000) -- 
		(22.180000,-24.529000) -- 
		(22.436000,-24.364000) -- 
		(22.825000,-24.023000) -- 
		(22.841000,-24.015000) -- 
		(22.986000,-23.942000) -- 
		(23.052000,-23.908000) -- 
		(23.157000,-23.971000) -- 
		(23.198000,-23.969000) -- 
		(23.402000,-23.922000) -- 
		(23.441000,-23.920000) -- 
		(23.602000,-23.909000) -- 
		(23.742000,-23.945000) -- 
		(23.817000,-23.980000) -- 
		(24.338000,-24.348000) -- 
		(24.383000,-24.374000) -- 
		(24.408000,-24.389000) -- 
		(24.446000,-24.404000) -- 
		(24.517000,-24.443000) -- 
		(24.648000,-24.535000) -- 
		(24.792000,-24.609000) -- 
		(24.903000,-24.659000) -- 
		(24.972000,-24.701000) -- 
		(25.008000,-24.718000) -- 
		(25.085000,-24.768000) -- 
		(25.244000,-24.870000) -- 
		(25.317000,-24.903000) -- 
		(25.494000,-25.000000) -- 
		(25.685000,-25.142000) -- 
		(25.706000,-25.173000) -- 
		(25.762000,-25.224000) -- 
		(25.801000,-25.236000) -- 
		(25.837000,-25.255000) -- 
		(26.051000,-25.552000) -- 
		(26.278000,-25.739000) -- 
		(26.479000,-25.855000) -- 
		(26.830000,-25.970000) -- 
		(27.115000,-26.146000) -- 
		(27.446000,-26.493000) -- 
		(27.544000,-26.570000) -- 
		(27.660000,-26.724000) -- 
		(27.764000,-27.032000) -- 
		(27.887000,-27.208000) -- 
		(28.017000,-27.340000) -- 
		(28.498000,-27.494000) -- 
		(28.640000,-27.582000) -- 
		(28.790000,-27.659000) -- 
		(29.005000,-27.882000) -- 
		(29.049000,-27.928000) -- 
		(29.127000,-27.983000) -- 
		(29.335000,-28.192000) -- 
		(29.510000,-28.335000) -- 
		(29.724000,-28.566000) -- 
		(29.906000,-28.808000) -- 
		(30.035000,-29.012000) -- 
		(30.144000,-29.160000) -- 
		(30.250000,-29.303000) -- 
		(30.354000,-29.369000) -- 
		(30.659000,-29.694000) -- 
		(30.726000,-29.737000) -- 
		(30.751000,-29.765000) -- 
		(30.941000,-29.909000) -- 
		(31.313000,-30.223000) -- 
		(31.429000,-30.313000) -- 
		(31.718000,-30.523000) -- 
		(31.782000,-30.569000) -- 
		(31.934000,-30.633000) -- 
		(32.079000,-30.706000) -- 
		(32.193000,-30.754000) -- 
		(32.233000,-30.767000) -- 
		(32.419000,-30.852000) -- 
		(32.575000,-30.907000) -- 
		(32.651000,-30.940000) -- 
		(32.807000,-30.995000) -- 
		(33.382000,-31.133000) -- 
		(33.467000,-31.138000) -- 
		(33.510000,-31.134000) -- 
		(33.804000,-31.088000) -- 
		(34.299000,-30.985000) -- 
		(34.847000,-30.915000) -- 
		(35.099000,-30.875000) -- 
		(35.267000,-30.855000) -- 
		(35.772000,-30.781000) -- 
		(35.930000,-30.728000) -- 
		(36.192000,-30.609000) -- 
		(36.329000,-30.524000) -- 
		(36.753000,-30.287000) -- 
		(37.048000,-30.142000) -- 
		(37.241000,-30.068000) -- 
		(37.402000,-30.020000) -- 
		(37.444000,-30.011000) -- 
		(37.563000,-29.971000) -- 
		(37.666000,-29.951000) -- 
		(37.687000,-29.945000) -- 
		(37.772000,-29.950000) -- 
		(37.885000,-29.985000) -- 
		(37.988000,-30.048000) -- 
		(38.016000,-30.075000) -- 
		(38.206000,-30.355000) -- 
		(38.457000,-30.846000) -- 
		(38.670000,-31.234000) -- 
		(38.833000,-31.572000) -- 
		(38.861000,-31.630000) -- 
		(38.904000,-31.731000) -- 
		(38.941000,-31.833000) -- 
		(38.967000,-31.939000) -- 
		(39.000000,-32.078000) -- 
		(39.005000,-32.150000) -- 
		(38.998000,-32.256000) -- 
		(38.985000,-32.363000) -- 
		(38.850000,-33.031000) -- 
		(38.849000,-33.094000) -- 
		(38.848000,-33.103000) -- 
		(38.863000,-33.173000) -- 
		(38.928000,-33.306000) -- 
		(38.979000,-33.363000) -- 
		(39.045000,-33.417000) -- 
		(39.187000,-33.416000) -- 
		(39.329000,-33.415000) -- 
		(44.066000,-33.379000) -- 
		(45.804000,-33.365000) -- 
		(45.815000,-31.577000) -- 
		(48.981000,-30.122000) -- 
		(52.977000,-28.285000) -- 
		(53.027000,-28.357000) -- 
		(53.800000,-29.476000) -- 
		(55.794000,-28.538000) -- 
		(56.633000,-28.141000) -- 
		(58.245000,-27.382000) -- 
		(58.336000,-27.339000) -- 
		(58.478000,-27.272000) -- 
		(58.621000,-27.204000) -- 
		(60.233000,-26.445000) -- 
		(60.335000,-26.397000) -- 
		(60.437000,-26.348000) -- 
		(62.995000,-25.119000) -- 
		(70.102000,-21.704000) -- 
		(70.669000,-20.920000) -- 
		(70.921000,-20.569000) -- 
		(72.317000,-20.148000) -- 
		(73.036000,-20.078000) -- 
		(73.195000,-20.377000) -- 
		(74.116000,-20.169000) -- 
		(75.224000,-19.918000) -- 
		(76.914000,-19.812000) -- 
		(77.348000,-19.784000) -- 
		(77.782000,-20.048000) -- 
		(80.321000,-19.697000) -- 
		(81.021000,-19.599000) -- 
		(83.262000,-19.472000) -- 
		(86.023000,-19.315000) -- 
		(86.879000,-19.266000) -- 
		(87.987000,-19.158000) -- 
		(90.418000,-18.915000) -- 
		(90.537000,-18.903000) -- 
		(90.654000,-18.887000) -- 
		(91.920000,-18.720000) -- 
		(93.256000,-18.478000) -- 
		(96.245000,-17.935000) -- 
		(99.794000,-17.367000) -- 
		(101.413000,-17.248000) -- 
		(101.421000,-17.247000) -- 
		(101.652000,-17.240000) -- 
		(104.163000,-17.164000) -- 
		(106.252000,-17.123000) -- 
		(112.519000,-16.998000) -- 
		(113.969000,-16.969000) -- 
		(114.607000,-16.956000) -- 
		(115.966000,-16.930000) -- 
		(120.042000,-16.850000) -- 
		(120.552000,-16.839000) -- 
		(121.400000,-16.827000);
	\filldraw [draw=black, ultra thick, fill=lightgreen]
		(189.130000,-47.926000) -- 
		(189.872000,-49.992000) -- 
		(190.042000,-50.436000) -- 
		(190.087000,-50.576000) -- 
		(190.088000,-50.577000) -- 
		(190.145000,-51.016000) -- 
		(190.232000,-51.881000) -- 
		(190.239000,-52.177000) -- 
		(190.267000,-53.327000) -- 
		(190.579000,-53.327000) -- 
		(191.513000,-53.327000) -- 
		(191.620000,-53.327000) -- 
		(191.777000,-53.352000) -- 
		(191.820000,-53.366000) -- 
		(192.268000,-53.508000) -- 
		(193.126000,-53.738000) -- 
		(193.333000,-53.793000) -- 
		(194.471000,-53.908000) -- 
		(195.244000,-53.899000) -- 
		(197.134000,-53.461000) -- 
		(197.866000,-53.290000) -- 
		(198.453000,-53.144000) -- 
		(198.988000,-53.012000) -- 
		(199.228000,-52.952000) -- 
		(199.899000,-52.782000) -- 
		(200.646000,-52.604000) -- 
		(201.528000,-52.394000) -- 
		(201.556000,-52.387000) -- 
		(201.947000,-52.285000) -- 
		(201.947000,-52.284000) -- 
		(202.081000,-52.249000) -- 
		(202.328000,-52.184000) -- 
		(202.694000,-52.088000) -- 
		(202.776000,-52.074000) -- 
		(203.223000,-51.986000) -- 
		(203.457000,-51.970000) -- 
		(203.654000,-51.956000) -- 
		(204.151000,-51.945000) -- 
		(204.612000,-51.934000) -- 
		(205.032000,-51.939000) -- 
		(206.111000,-51.950000) -- 
		(206.291000,-51.950000) -- 
		(206.710000,-51.949000) -- 
		(207.105000,-51.949000) -- 
		(207.106000,-51.950000) -- 
		(208.340000,-51.951000) -- 
		(208.571000,-51.953000) -- 
		(209.250000,-51.957000) -- 
		(210.361000,-51.946000) -- 
		(211.392000,-51.932000) -- 
		(214.152000,-51.995000) -- 
		(214.651000,-52.005000) -- 
		(216.009000,-52.091000) -- 
		(216.401000,-52.116000) -- 
		(216.530000,-52.116000) -- 
		(218.003000,-52.104000) -- 
		(218.092000,-52.099000) -- 
		(218.226000,-52.089000) -- 
		(218.413000,-52.077000) -- 
		(218.611000,-52.064000) -- 
		(219.035000,-52.036000) -- 
		(220.129000,-52.014000) -- 
		(220.939000,-52.008000) -- 
		(221.135000,-52.009000) -- 
		(221.974000,-52.012000) -- 
		(222.993000,-52.068000) -- 
		(223.639000,-52.110000) -- 
		(223.728000,-52.116000) -- 
		(224.230000,-52.142000) -- 
		(224.534000,-52.157000) -- 
		(225.192000,-52.160000) -- 
		(225.399000,-52.171000) -- 
		(225.514000,-52.177000) -- 
		(225.932000,-52.197000) -- 
		(226.518000,-52.261000) -- 
		(226.794000,-52.277000) -- 
		(227.395000,-52.267000) -- 
		(227.488000,-52.266000) -- 
		(228.112000,-52.213000) -- 
		(228.693000,-52.131000) -- 
		(228.753000,-52.122000) -- 
		(229.105000,-52.087000) -- 
		(229.776000,-52.115000) -- 
		(230.480000,-52.132000) -- 
		(230.898000,-52.158000) -- 
		(231.214000,-52.151000) -- 
		(231.351000,-52.205000) -- 
		(231.448000,-52.243000) -- 
		(231.762000,-52.358000) -- 
		(231.899000,-52.406000) -- 
		(232.473000,-52.613000) -- 
		(232.876000,-52.780000) -- 
		(233.155000,-52.895000) -- 
		(233.405000,-53.024000) -- 
		(233.406000,-53.025000) -- 
		(233.785000,-53.218000) -- 
		(234.052000,-53.359000) -- 
		(234.438000,-53.614000) -- 
		(234.897000,-53.942000) -- 
		(235.088000,-54.111000) -- 
		(235.435000,-54.531000) -- 
		(235.905000,-55.100000) -- 
		(236.094000,-55.342000) -- 
		(236.283000,-55.586000) -- 
		(236.492000,-55.884000) -- 
		(236.683000,-56.123000) -- 
		(236.827000,-56.256000) -- 
		(237.001000,-56.417000) -- 
		(237.215000,-56.550000) -- 
		(237.611000,-56.744000) -- 
		(238.083000,-56.963000) -- 
		(238.679000,-57.253000) -- 
		(239.930000,-57.897000) -- 
		(240.218000,-58.043000) -- 
		(240.725000,-58.302000) -- 
		(240.979000,-58.420000) -- 
		(241.544000,-58.683000) -- 
		(242.338000,-59.083000) -- 
		(242.837000,-59.324000) -- 
		(243.578000,-59.679000) -- 
		(244.181000,-59.994000) -- 
		(246.754000,-61.279000) -- 
		(247.346000,-61.560000) -- 
		(248.376000,-62.099000) -- 
		(248.411000,-62.117000) -- 
		(249.682000,-62.734000) -- 
		(250.230000,-63.007000) -- 
		(251.641000,-63.702000) -- 
		(251.879000,-63.804000) -- 
		(251.894000,-63.811000) -- 
		(252.003000,-63.857000) -- 
		(252.270000,-63.953000) -- 
		(253.001000,-64.099000) -- 
		(253.508000,-64.150000) -- 
		(253.768000,-64.185000) -- 
		(254.136000,-64.234000) -- 
		(254.452000,-64.297000) -- 
		(254.614000,-64.341000) -- 
		(254.676000,-64.357000) -- 
		(255.045000,-64.493000) -- 
		(255.859000,-64.849000) -- 
		(256.478000,-65.078000) -- 
		(256.953000,-65.300000) -- 
		(257.160000,-65.408000) -- 
		(257.355000,-65.510000) -- 
		(257.655000,-65.657000) -- 
		(258.777000,-66.299000) -- 
		(258.780000,-79.603000) -- 
		(258.781000,-85.892000) -- 
		(258.782000,-92.181000) -- 
		(258.783000,-95.904000) -- 
		(258.784000,-99.628000) -- 
		(258.784000,-100.120000) -- 
		(258.784000,-100.610000) -- 
		(258.784000,-101.567000) -- 
		(258.784000,-101.570000) -- 
		(258.784000,-102.034000) -- 
		(258.784000,-102.498000) -- 
		(258.786000,-111.212000) -- 
		(258.789000,-121.480000) -- 
		(258.789000,-122.472000) -- 
		(258.788000,-122.473000) -- 
		(257.794000,-122.295000) -- 
		(256.970000,-122.263000) -- 
		(256.814000,-122.256000) -- 
		(256.030000,-122.243000) -- 
		(255.065000,-122.244000) -- 
		(254.541000,-122.224000) -- 
		(254.075000,-122.183000) -- 
		(253.966000,-122.174000) -- 
		(253.318000,-122.075000) -- 
		(251.875000,-121.835000) -- 
		(251.495000,-121.825000) -- 
		(251.068000,-121.813000) -- 
		(250.179000,-121.829000) -- 
		(249.814000,-121.817000) -- 
		(249.661000,-121.832000) -- 
		(249.525000,-121.871000) -- 
		(249.333000,-121.943000) -- 
		(249.252000,-121.978000) -- 
		(248.228000,-122.411000) -- 
		(248.055000,-122.496000) -- 
		(247.664000,-122.686000) -- 
		(247.107000,-122.957000) -- 
		(246.977000,-123.020000) -- 
		(245.503000,-123.708000) -- 
		(244.908000,-124.002000) -- 
		(244.634000,-124.134000) -- 
		(244.219000,-124.331000) -- 
		(243.796000,-124.524000) -- 
		(243.335000,-124.714000) -- 
		(243.003000,-124.828000) -- 
		(242.736000,-124.876000) -- 
		(242.469000,-124.905000) -- 
		(242.176000,-124.906000) -- 
		(241.589000,-124.892000) -- 
		(233.387000,-124.690000) -- 
		(230.652000,-124.622000) -- 
		(230.505000,-124.619000) -- 
		(230.061000,-124.608000) -- 
		(229.913000,-124.604000) -- 
		(229.523000,-124.594000) -- 
		(229.513000,-124.594000) -- 
		(228.746000,-124.574000) -- 
		(228.313000,-124.565000) -- 
		(227.912000,-124.553000) -- 
		(224.328000,-124.462000) -- 
		(223.652000,-124.444000) -- 
		(220.834000,-124.376000) -- 
		(216.261000,-124.259000) -- 
		(214.524000,-124.203000) -- 
		(213.573000,-124.179000) -- 
		(209.988000,-124.088000) -- 
		(207.700000,-124.027000) -- 
		(207.332000,-124.017000) -- 
		(206.454000,-123.999000) -- 
		(206.125000,-123.991000) -- 
		(206.124000,-123.990000) -- 
		(205.512000,-123.978000) -- 
		(205.356000,-123.975000) -- 
		(204.972000,-123.968000) -- 
		(203.671000,-123.935000) -- 
		(202.301000,-123.898000) -- 
		(202.300000,-123.897000) -- 
		(201.822000,-123.882000) -- 
		(200.762000,-123.861000) -- 
		(200.761000,-123.860000) -- 
		(199.993000,-123.848000) -- 
		(199.593000,-123.823000) -- 
		(199.404000,-123.811000) -- 
		(197.257000,-123.766000) -- 
		(196.046000,-123.733000) -- 
		(195.850000,-123.729000) -- 
		(194.990000,-123.709000) -- 
		(194.068000,-123.678000) -- 
		(192.933000,-123.638000) -- 
		(192.441000,-123.655000) -- 
		(192.317000,-123.635000) -- 
		(192.094000,-123.708000) -- 
		(192.039000,-123.736000) -- 
		(191.561000,-123.976000) -- 
		(191.249000,-124.141000) -- 
		(190.986000,-124.279000) -- 
		(190.944000,-124.303000) -- 
		(190.817000,-124.369000) -- 
		(190.774000,-124.391000) -- 
		(190.079000,-124.758000) -- 
		(188.723000,-125.481000) -- 
		(188.723000,-125.482000) -- 
		(188.048000,-125.832000) -- 
		(187.180000,-126.281000) -- 
		(184.739000,-127.577000) -- 
		(182.500000,-128.746000) -- 
		(181.567000,-129.244000) -- 
		(179.863000,-130.137000) -- 
		(179.566000,-130.292000) -- 
		(177.136000,-131.577000) -- 
		(176.815000,-131.747000) -- 
		(176.586000,-131.868000) -- 
		(175.802000,-132.279000) -- 
		(174.930000,-132.736000) -- 
		(174.378000,-133.024000) -- 
		(173.687000,-133.391000) -- 
		(172.922000,-133.798000) -- 
		(170.681000,-134.987000) -- 
		(169.822000,-135.445000) -- 
		(169.488000,-135.623000) -- 
		(168.555000,-136.123000) -- 
		(167.099000,-136.897000) -- 
		(166.594000,-137.154000) -- 
		(166.322000,-137.299000) -- 
		(165.988000,-137.479000) -- 
		(165.218000,-137.890000) -- 
		(164.229000,-138.407000) -- 
		(162.663000,-139.242000) -- 
		(161.556000,-139.832000) -- 
		(160.744000,-140.264000) -- 
		(160.730000,-140.271000) -- 
		(159.804000,-140.720000) -- 
		(159.626000,-140.818000) -- 
		(157.365000,-142.049000) -- 
		(156.327000,-142.589000) -- 
		(155.095000,-143.259000) -- 
		(154.377000,-143.632000) -- 
		(153.892000,-143.888000) -- 
		(153.806000,-143.932000) -- 
		(153.806000,-143.933000) -- 
		(153.138000,-144.307000) -- 
		(152.952000,-144.443000) -- 
		(152.558000,-144.798000) -- 
		(152.306000,-145.090000) -- 
		(152.215000,-145.220000) -- 
		(152.117000,-145.360000) -- 
		(152.033000,-145.499000) -- 
		(151.579000,-146.243000) -- 
		(151.509000,-146.349000) -- 
		(151.324000,-146.625000) -- 
		(151.273000,-146.702000) -- 
		(151.118000,-146.930000) -- 
		(151.117000,-146.933000) -- 
		(151.065000,-147.009000) -- 
		(151.013000,-147.086000) -- 
		(150.812000,-147.385000) -- 
		(150.679000,-147.565000) -- 
		(150.652000,-147.594000) -- 
		(150.500000,-147.751000) -- 
		(150.380000,-147.853000) -- 
		(150.318000,-147.905000) -- 
		(149.622000,-148.368000) -- 
		(148.889000,-148.814000) -- 
		(148.276000,-149.185000) -- 
		(148.201000,-149.232000) -- 
		(148.077000,-149.305000) -- 
		(147.974000,-149.368000) -- 
		(147.898000,-149.413000) -- 
		(147.839000,-149.450000) -- 
		(147.660000,-149.557000) -- 
		(147.600000,-149.593000) -- 
		(146.507000,-150.255000) -- 
		(146.054000,-150.528000) -- 
		(144.428000,-151.511000) -- 
		(143.230000,-152.247000) -- 
		(143.058000,-152.352000) -- 
		(142.643000,-152.620000) -- 
		(142.417000,-152.723000) -- 
		(142.092000,-152.803000) -- 
		(142.023000,-152.818000) -- 
		(141.775000,-152.780000) -- 
		(141.503000,-152.703000) -- 
		(141.475000,-152.694000) -- 
		(140.272000,-152.274000) -- 
		(139.668000,-152.105000) -- 
		(139.455000,-152.045000) -- 
		(139.327000,-152.009000) -- 
		(139.054000,-151.945000) -- 
		(138.739000,-151.872000) -- 
		(138.206000,-151.747000) -- 
		(136.995000,-151.489000) -- 
		(136.597000,-151.404000) -- 
		(135.929000,-151.277000) -- 
		(135.527000,-151.188000) -- 
		(135.194000,-151.138000) -- 
		(134.895000,-151.118000) -- 
		(134.194000,-151.198000) -- 
		(133.882000,-151.256000) -- 
		(133.522000,-151.360000) -- 
		(132.977000,-151.555000) -- 
		(131.884000,-151.970000) -- 
		(131.151000,-152.214000) -- 
		(130.891000,-152.303000) -- 
		(130.268000,-152.514000) -- 
		(129.848000,-152.689000) -- 
		(129.480000,-152.842000) -- 
		(129.090000,-153.004000) -- 
		(128.927000,-153.071000) -- 
		(128.750000,-153.144000) -- 
		(128.034000,-153.410000) -- 
		(127.792000,-153.490000) -- 
		(127.664000,-153.531000) -- 
		(127.474000,-153.581000) -- 
		(127.135000,-153.669000) -- 
		(126.854000,-153.721000) -- 
		(126.506000,-153.786000) -- 
		(126.111000,-153.874000) -- 
		(125.790000,-153.973000) -- 
		(125.390000,-154.110000) -- 
		(125.056000,-154.257000) -- 
		(124.656000,-154.456000) -- 
		(124.372000,-154.635000) -- 
		(124.177000,-154.757000) -- 
		(124.075000,-154.830000) -- 
		(123.383000,-155.332000) -- 
		(123.344000,-155.362000) -- 
		(123.223000,-155.449000) -- 
		(123.183000,-155.478000) -- 
		(123.152000,-155.502000) -- 
		(123.055000,-155.571000) -- 
		(123.023000,-155.594000) -- 
		(122.958000,-155.643000) -- 
		(122.760000,-155.787000) -- 
		(122.694000,-155.833000) -- 
		(122.424000,-156.033000) -- 
		(122.201000,-156.198000) -- 
		(122.003000,-156.345000) -- 
		(121.639000,-156.561000) -- 
		(121.584000,-156.586000) -- 
		(121.357000,-156.681000) -- 
		(121.270000,-156.704000) -- 
		(120.945000,-156.790000) -- 
		(120.898000,-156.805000) -- 
		(120.663000,-156.878000) -- 
		(120.464000,-156.952000) -- 
		(120.345000,-156.996000) -- 
		(120.115000,-157.109000) -- 
		(119.847000,-157.281000) -- 
		(119.521000,-157.487000) -- 
		(119.316000,-157.622000) -- 
		(118.698000,-158.027000) -- 
		(118.492000,-158.160000) -- 
		(118.051000,-158.421000) -- 
		(117.909000,-158.504000) -- 
		(117.216000,-158.824000) -- 
		(116.636000,-159.012000) -- 
		(116.317000,-159.115000) -- 
		(116.153000,-159.183000) -- 
		(116.035000,-159.232000) -- 
		(115.898000,-159.289000) -- 
		(115.602000,-159.454000) -- 
		(115.348000,-159.623000) -- 
		(115.066000,-159.789000) -- 
		(114.395000,-160.124000) -- 
		(113.036000,-160.740000) -- 
		(112.291000,-161.070000) -- 
		(111.957000,-161.204000) -- 
		(111.866000,-161.240000) -- 
		(111.778000,-161.276000) -- 
		(111.724000,-161.297000) -- 
		(111.450000,-161.394000) -- 
		(111.208000,-161.449000) -- 
		(110.912000,-161.499000) -- 
		(110.176000,-161.530000) -- 
		(107.668000,-161.565000) -- 
		(107.371000,-161.590000) -- 
		(107.125000,-161.688000) -- 
		(106.609000,-161.894000) -- 
		(106.367000,-162.001000) -- 
		(106.239000,-162.057000) -- 
		(105.011000,-162.541000) -- 
		(104.389000,-162.826000) -- 
		(104.083000,-163.049000) -- 
		(103.907000,-163.189000) -- 
		(103.864000,-163.225000) -- 
		(103.802000,-163.274000) -- 
		(103.769000,-163.357000) -- 
		(103.747000,-163.408000) -- 
		(103.075000,-162.876000) -- 
		(103.074000,-162.875000) -- 
		(102.140000,-162.159000) -- 
		(101.847000,-161.933000) -- 
		(100.516000,-160.905000) -- 
		(100.116000,-160.615000) -- 
		(99.350000,-160.060000) -- 
		(97.982000,-159.029000) -- 
		(96.366000,-157.786000) -- 
		(96.093000,-157.577000) -- 
		(94.840000,-156.618000) -- 
		(94.178000,-156.119000) -- 
		(93.673000,-155.739000) -- 
		(93.355000,-155.498000) -- 
		(90.887000,-153.628000) -- 
		(90.064000,-153.003000) -- 
		(89.249000,-152.387000) -- 
		(87.116000,-150.775000) -- 
		(86.804000,-150.536000) -- 
		(85.992000,-149.913000) -- 
		(85.535000,-149.570000) -- 
		(85.094000,-149.239000) -- 
		(84.677000,-148.915000) -- 
		(84.179000,-148.515000) -- 
		(84.151000,-148.492000) -- 
		(83.755000,-148.129000) -- 
		(83.614000,-148.000000) -- 
		(83.451000,-147.850000) -- 
		(83.191000,-147.611000) -- 
		(83.154000,-147.577000) -- 
		(83.053000,-147.477000) -- 
		(83.017000,-147.442000) -- 
		(82.908000,-147.333000) -- 
		(82.871000,-147.296000) -- 
		(82.794000,-147.219000) -- 
		(82.390000,-146.781000) -- 
		(81.940000,-146.293000) -- 
		(81.046000,-145.151000) -- 
		(80.611000,-144.595000) -- 
		(80.308000,-144.212000) -- 
		(79.812000,-143.584000) -- 
		(79.412000,-143.052000) -- 
		(79.118000,-142.659000) -- 
		(78.979000,-142.467000) -- 
		(78.963000,-142.446000) -- 
		(78.541000,-141.905000) -- 
		(78.394000,-141.717000) -- 
		(78.271000,-141.560000) -- 
		(77.902000,-141.088000) -- 
		(77.779000,-140.930000) -- 
		(77.262000,-140.264000) -- 
		(77.256000,-140.257000) -- 
		(76.751000,-139.622000) -- 
		(76.429000,-139.225000) -- 
		(76.051000,-138.705000) -- 
		(75.684000,-138.238000) -- 
		(75.518000,-138.026000) -- 
		(75.179000,-137.550000) -- 
		(75.055000,-137.376000) -- 
		(75.001000,-137.301000) -- 
		(74.667000,-136.864000) -- 
		(74.536000,-136.694000) -- 
		(74.390000,-136.503000) -- 
		(74.249000,-136.323000) -- 
		(73.796000,-135.744000) -- 
		(73.538000,-135.414000) -- 
		(73.381000,-135.209000) -- 
		(73.093000,-134.836000) -- 
		(72.929000,-134.623000) -- 
		(72.437000,-133.984000) -- 
		(72.273000,-133.769000) -- 
		(72.156000,-133.617000) -- 
		(71.558000,-132.861000) -- 
		(70.667000,-131.734000) -- 
		(70.061000,-130.951000) -- 
		(69.433000,-130.120000) -- 
		(69.423000,-130.106000) -- 
		(68.770000,-129.275000) -- 
		(68.720000,-129.209000) -- 
		(68.656000,-129.125000) -- 
		(68.462000,-128.871000) -- 
		(68.397000,-128.786000) -- 
		(68.273000,-128.626000) -- 
		(67.918000,-128.170000) -- 
		(67.899000,-128.144000) -- 
		(67.776000,-127.981000) -- 
		(67.681000,-127.855000) -- 
		(67.609000,-127.759000) -- 
		(67.435000,-127.530000) -- 
		(67.391000,-127.475000) -- 
		(67.291000,-127.351000) -- 
		(66.865000,-126.821000) -- 
		(66.724000,-126.637000) -- 
		(65.742000,-125.354000) -- 
		(65.377000,-124.877000) -- 
		(65.066000,-124.549000) -- 
		(65.023000,-124.493000) -- 
		(64.644000,-124.007000) -- 
		(64.461000,-123.773000) -- 
		(64.311000,-123.580000) -- 
		(64.078000,-123.281000) -- 
		(63.859000,-123.000000) -- 
		(63.708000,-122.805000) -- 
		(63.652000,-122.733000) -- 
		(63.483000,-122.516000) -- 
		(63.426000,-122.443000) -- 
		(63.290000,-122.268000) -- 
		(62.884000,-121.747000) -- 
		(60.844000,-119.127000) -- 
		(57.056000,-114.264000) -- 
		(56.332000,-113.510000) -- 
		(55.759000,-112.858000) -- 
		(55.269000,-112.237000) -- 
		(55.106000,-112.030000) -- 
		(54.879000,-111.779000) -- 
		(54.886000,-111.766000) -- 
		(54.854000,-111.721000) -- 
		(54.788000,-111.627000) -- 
		(54.757000,-111.587000) -- 
		(53.436000,-109.886000) -- 
		(52.646000,-108.869000) -- 
		(52.586000,-108.792000) -- 
		(52.406000,-108.560000) -- 
		(52.346000,-108.482000) -- 
		(52.162000,-108.245000) -- 
		(52.008000,-108.046000) -- 
		(51.605000,-107.533000) -- 
		(51.590000,-107.514000) -- 
		(51.420000,-107.295000) -- 
		(51.326000,-107.175000) -- 
		(51.044000,-106.809000) -- 
		(50.949000,-106.687000) -- 
		(50.577000,-106.207000) -- 
		(50.047000,-105.523000) -- 
		(49.451000,-104.769000) -- 
		(49.335000,-104.621000) -- 
		(49.077000,-104.286000) -- 
		(48.631000,-103.708000) -- 
		(48.417000,-103.429000) -- 
		(47.287000,-101.972000) -- 
		(46.838000,-101.394000) -- 
		(46.676000,-101.186000) -- 
		(46.188000,-100.559000) -- 
		(46.025000,-100.350000) -- 
		(45.518000,-99.698000) -- 
		(43.994000,-97.742000) -- 
		(43.486000,-97.089000) -- 
		(43.410000,-96.991000) -- 
		(43.181000,-96.697000) -- 
		(43.104000,-96.599000) -- 
		(42.982000,-96.441000) -- 
		(42.615000,-95.966000) -- 
		(42.492000,-95.806000) -- 
		(42.554000,-95.749000) -- 
		(42.632000,-95.676000) -- 
		(42.740000,-95.576000) -- 
		(42.779000,-95.539000) -- 
		(42.804000,-95.521000) -- 
		(42.954000,-95.413000) -- 
		(43.432000,-95.316000) -- 
		(43.434000,-95.314000) -- 
		(43.642000,-95.342000) -- 
		(43.830000,-95.435000) -- 
		(44.407000,-96.062000) -- 
		(44.614000,-96.178000) -- 
		(44.900000,-96.260000) -- 
		(45.036000,-96.266000) -- 
		(45.108000,-96.281000) -- 
		(45.769000,-96.404000) -- 
		(46.017000,-96.391000) -- 
		(46.281000,-96.376000) -- 
		(46.755000,-96.430000) -- 
		(46.774000,-96.431000) -- 
		(47.000000,-96.462000) -- 
		(47.423000,-96.520000) -- 
		(48.138000,-96.658000) -- 
		(48.538000,-96.734000) -- 
		(49.615000,-96.883000) -- 
		(50.004000,-96.960000) -- 
		(50.224000,-97.032000) -- 
		(50.477000,-97.153000) -- 
		(51.268000,-97.400000) -- 
		(51.483000,-97.486000) -- 
		(52.053000,-97.714000) -- 
		(52.442000,-97.929000) -- 
		(52.520000,-97.985000) -- 
		(53.168000,-98.446000) -- 
		(53.810000,-98.946000) -- 
		(53.940000,-99.059000) -- 
		(54.076000,-99.178000) -- 
		(54.341000,-99.376000) -- 
		(54.471000,-99.497000) -- 
		(54.724000,-99.678000) -- 
		(54.964000,-99.827000) -- 
		(55.178000,-100.052000) -- 
		(55.340000,-100.267000) -- 
		(55.469000,-100.531000) -- 
		(55.586000,-100.938000) -- 
		(55.586000,-101.532000) -- 
		(55.624000,-101.702000) -- 
		(55.735000,-101.873000) -- 
		(55.806000,-101.955000) -- 
		(56.474000,-102.461000) -- 
		(56.727000,-102.599000) -- 
		(56.947000,-102.714000) -- 
		(57.024000,-102.741000) -- 
		(57.219000,-102.808000) -- 
		(57.238000,-102.814000) -- 
		(57.453000,-102.874000) -- 
		(57.985000,-102.962000) -- 
		(58.186000,-102.973000) -- 
		(58.886000,-103.100000) -- 
		(58.990000,-103.109000) -- 
		(59.087000,-103.117000) -- 
		(59.101000,-103.121000) -- 
		(59.191000,-103.144000) -- 
		(59.398000,-103.089000) -- 
		(59.421000,-103.077000) -- 
		(59.458000,-103.054000) -- 
		(59.516000,-103.020000) -- 
		(59.472000,-102.959000) -- 
		(59.455000,-102.926000) -- 
		(59.449000,-102.896000) -- 
		(59.441000,-102.856000) -- 
		(59.435000,-102.677000) -- 
		(59.415000,-102.500000) -- 
		(59.417000,-102.467000) -- 
		(59.421000,-102.322000) -- 
		(59.443000,-102.180000) -- 
		(59.529000,-101.963000) -- 
		(59.536000,-101.943000) -- 
		(59.592000,-101.697000) -- 
		(59.595000,-101.626000) -- 
		(59.580000,-101.483000) -- 
		(59.520000,-101.240000) -- 
		(59.487000,-101.151000) -- 
		(59.381000,-100.866000) -- 
		(59.248000,-100.511000) -- 
		(59.215000,-100.423000) -- 
		(58.921000,-99.791000) -- 
		(58.819000,-99.519000) -- 
		(58.791000,-99.415000) -- 
		(58.743000,-99.133000) -- 
		(58.743000,-99.062000) -- 
		(58.754000,-98.956000) -- 
		(58.771000,-98.849000) -- 
		(58.848000,-98.646000) -- 
		(58.902000,-98.560000) -- 
		(58.987000,-98.424000) -- 
		(59.011000,-98.395000) -- 
		(59.040000,-98.369000) -- 
		(59.142000,-98.254000) -- 
		(59.149000,-98.248000) -- 
		(59.332000,-98.110000) -- 
		(59.403000,-98.072000) -- 
		(59.591000,-97.988000) -- 
		(59.803000,-97.868000) -- 
		(60.039000,-97.787000) -- 
		(60.162000,-97.759000) -- 
		(60.245000,-97.745000) -- 
		(60.608000,-97.646000) -- 
		(60.762000,-97.586000) -- 
		(60.867000,-97.527000) -- 
		(60.900000,-97.503000) -- 
		(60.986000,-97.425000) -- 
		(61.062000,-97.340000) -- 
		(61.139000,-97.212000) -- 
		(61.412000,-96.730000) -- 
		(61.463000,-96.632000) -- 
		(61.472000,-96.597000) -- 
		(61.536000,-96.428000) -- 
		(61.646000,-96.085000) -- 
		(61.672000,-96.017000) -- 
		(61.704000,-95.914000) -- 
		(61.719000,-95.844000) -- 
		(61.741000,-95.703000) -- 
		(61.758000,-95.633000) -- 
		(61.776000,-95.527000) -- 
		(61.800000,-95.315000) -- 
		(61.856000,-95.071000) -- 
		(61.879000,-95.002000) -- 
		(61.985000,-94.807000) -- 
		(62.312000,-94.308000) -- 
		(62.365000,-94.210000) -- 
		(62.428000,-94.078000) -- 
		(62.446000,-94.008000) -- 
		(62.459000,-93.974000) -- 
		(62.485000,-93.870000) -- 
		(62.526000,-93.242000) -- 
		(62.522000,-93.206000) -- 
		(62.511000,-93.172000) -- 
		(62.501000,-93.101000) -- 
		(62.451000,-92.893000) -- 
		(62.415000,-92.791000) -- 
		(62.270000,-92.455000) -- 
		(62.252000,-92.423000) -- 
		(62.216000,-92.358000) -- 
		(62.038000,-92.060000) -- 
		(61.622000,-91.439000) -- 
		(61.491000,-91.217000) -- 
		(61.422000,-91.087000) -- 
		(61.413000,-91.053000) -- 
		(61.382000,-90.988000) -- 
		(61.332000,-90.816000) -- 
		(61.299000,-90.639000) -- 
		(61.292000,-90.462000) -- 
		(61.295000,-90.356000) -- 
		(61.337000,-90.181000) -- 
		(61.413000,-89.977000) -- 
		(61.521000,-89.782000) -- 
		(61.628000,-89.629000) -- 
		(61.678000,-89.571000) -- 
		(61.710000,-89.505000) -- 
		(61.853000,-89.246000) -- 
		(61.875000,-89.177000) -- 
		(61.920000,-88.894000) -- 
		(61.929000,-88.788000) -- 
		(61.927000,-88.717000) -- 
		(61.937000,-88.470000) -- 
		(61.932000,-88.435000) -- 
		(61.931000,-88.258000) -- 
		(61.904000,-87.766000) -- 
		(61.895000,-87.696000) -- 
		(61.877000,-87.414000) -- 
		(61.878000,-87.344000) -- 
		(61.871000,-87.238000) -- 
		(61.849000,-87.027000) -- 
		(61.771000,-86.499000) -- 
		(61.707000,-85.934000) -- 
		(61.696000,-85.864000) -- 
		(61.693000,-85.793000) -- 
		(61.671000,-85.617000) -- 
		(61.648000,-85.333000) -- 
		(61.652000,-85.156000) -- 
		(61.660000,-85.085000) -- 
		(61.680000,-85.015000) -- 
		(61.709000,-84.875000) -- 
		(61.731000,-84.795000) -- 
		(61.815000,-84.492000) -- 
		(61.826000,-84.448000) -- 
		(61.855000,-84.382000) -- 
		(62.083000,-83.664000) -- 
		(62.108000,-83.596000) -- 
		(62.178000,-83.353000) -- 
		(62.250000,-83.150000) -- 
		(62.278000,-83.083000) -- 
		(62.636000,-82.058000) -- 
		(62.666000,-81.882000) -- 
		(62.663000,-81.838000) -- 
		(62.708000,-81.800000) -- 
		(62.733000,-81.771000) -- 
		(62.753000,-81.739000) -- 
		(62.843000,-81.500000) -- 
		(62.960000,-81.180000) -- 
		(63.502000,-79.684000) -- 
		(63.703000,-79.253000) -- 
		(63.768000,-79.146000) -- 
		(63.779000,-79.126000) -- 
		(63.811000,-79.060000) -- 
		(63.887000,-78.933000) -- 
		(63.907000,-78.901000) -- 
		(63.954000,-78.842000) -- 
		(64.054000,-78.700000) -- 
		(64.112000,-78.618000) -- 
		(64.169000,-78.536000) -- 
		(64.648000,-77.914000) -- 
		(64.701000,-77.850000) -- 
		(65.217000,-77.204000) -- 
		(65.614000,-76.659000) -- 
		(65.674000,-76.565000) -- 
		(65.741000,-76.434000) -- 
		(65.902000,-76.063000) -- 
		(66.057000,-75.649000) -- 
		(66.359000,-74.834000) -- 
		(66.473000,-74.851000) -- 
		(68.179000,-75.106000) -- 
		(69.089000,-75.242000) -- 
		(69.279000,-79.837000) -- 
		(69.355000,-81.674000) -- 
		(69.440000,-81.678000) -- 
		(69.695000,-81.686000) -- 
		(69.779000,-81.688000) -- 
		(69.807000,-81.690000) -- 
		(69.891000,-81.695000) -- 
		(69.918000,-81.696000) -- 
		(69.949000,-81.714000) -- 
		(70.042000,-81.763000) -- 
		(70.073000,-81.779000) -- 
		(70.137000,-81.771000) -- 
		(70.233000,-81.757000) -- 
		(70.325000,-81.776000) -- 
		(70.387000,-81.788000) -- 
		(70.538000,-81.832000) -- 
		(71.989000,-81.848000) -- 
		(72.420000,-81.855000) -- 
		(73.594000,-81.867000) -- 
		(74.966000,-81.874000) -- 
		(75.744000,-81.870000) -- 
		(77.763000,-81.896000) -- 
		(78.413000,-81.888000) -- 
		(78.535000,-81.889000) -- 
		(79.380000,-81.888000) -- 
		(80.262000,-81.885000) -- 
		(80.572000,-81.859000) -- 
		(81.008000,-81.861000) -- 
		(82.316000,-81.864000) -- 
		(82.751000,-81.865000) -- 
		(82.752000,-81.952000) -- 
		(82.844000,-88.997000) -- 
		(84.215000,-89.028000) -- 
		(93.808000,-89.241000) -- 
		(93.944000,-89.176000) -- 
		(97.300000,-89.241000) -- 
		(97.371000,-89.242000) -- 
		(97.581000,-89.245000) -- 
		(97.651000,-89.245000) -- 
		(97.850000,-89.251000) -- 
		(98.431000,-89.262000) -- 
		(98.444000,-89.262000) -- 
		(98.528000,-89.262000) -- 
		(98.642000,-89.263000) -- 
		(98.758000,-89.261000) -- 
		(99.272000,-89.252000) -- 
		(101.160000,-89.215000) -- 
		(101.663000,-89.205000) -- 
		(101.789000,-89.203000) -- 
		(101.750000,-89.085000) -- 
		(101.419000,-87.599000) -- 
		(101.415000,-87.420000) -- 
		(101.404000,-87.030000) -- 
		(101.422000,-86.692000) -- 
		(101.543000,-86.348000) -- 
		(101.829000,-85.916000) -- 
		(102.186000,-85.455000) -- 
		(102.776000,-84.844000) -- 
		(103.570000,-83.837000) -- 
		(104.046000,-83.218000) -- 
		(104.262000,-82.917000) -- 
		(105.180000,-81.629000) -- 
		(105.315000,-81.425000) -- 
		(105.391000,-81.424000) -- 
		(105.467000,-81.424000) -- 
		(105.689000,-81.422000) -- 
		(106.809000,-81.413000) -- 
		(107.097000,-81.409000) -- 
		(107.182000,-81.408000) -- 
		(107.277000,-81.409000) -- 
		(107.411000,-81.410000) -- 
		(108.096000,-81.414000) -- 
		(108.324000,-81.415000) -- 
		(109.285000,-81.417000) -- 
		(112.168000,-81.420000) -- 
		(112.927000,-81.420000) -- 
		(113.028000,-81.420000) -- 
		(113.128000,-81.420000) -- 
		(113.238000,-81.420000) -- 
		(113.348000,-81.419000) -- 
		(113.528000,-81.419000) -- 
		(114.725000,-81.411000) -- 
		(114.933000,-81.409000) -- 
		(115.028000,-81.408000) -- 
		(115.123000,-81.408000) -- 
		(115.203000,-81.408000) -- 
		(115.286000,-81.409000) -- 
		(115.656000,-81.410000) -- 
		(117.254000,-81.414000) -- 
		(117.786000,-81.415000) -- 
		(118.115000,-81.453000) -- 
		(118.170000,-81.631000) -- 
		(118.627000,-83.116000) -- 
		(118.779000,-83.609000) -- 
		(122.392000,-83.283000) -- 
		(123.116000,-83.239000) -- 
		(123.722000,-83.203000) -- 
		(127.458000,-82.976000) -- 
		(138.554000,-82.016000) -- 
		(143.211000,-81.613000) -- 
		(143.349000,-81.863000) -- 
		(146.557000,-81.855000) -- 
		(156.181000,-81.830000) -- 
		(159.388000,-81.819000) -- 
		(161.325000,-81.824000) -- 
		(167.136000,-81.836000) -- 
		(168.745000,-81.839000) -- 
		(168.750000,-81.512000) -- 
		(168.814000,-75.641000) -- 
		(168.993000,-59.087000) -- 
		(169.005000,-58.023000) -- 
		(169.068000,-52.150000) -- 
		(169.085000,-51.683000) -- 
		(169.086000,-51.641000) -- 
		(167.729000,-51.556000) -- 
		(167.708000,-51.553000) -- 
		(167.479000,-51.434000) -- 
		(167.426000,-51.350000) -- 
		(167.428000,-51.261000) -- 
		(167.373000,-51.013000) -- 
		(167.367000,-50.994000) -- 
		(167.230000,-50.603000) -- 
		(167.107000,-50.212000) -- 
		(167.097000,-50.180000) -- 
		(166.997000,-49.960000) -- 
		(167.017000,-47.730000) -- 
		(167.052000,-43.877000) -- 
		(167.349000,-43.600000) -- 
		(167.359000,-43.591000) -- 
		(167.599000,-43.378000) -- 
		(167.803000,-43.194000) -- 
		(168.381000,-42.821000) -- 
		(168.479000,-42.757000) -- 
		(168.662000,-42.669000) -- 
		(168.900000,-42.555000) -- 
		(169.066000,-42.531000) -- 
		(169.309000,-42.565000) -- 
		(169.496000,-42.606000) -- 
		(169.697000,-42.616000) -- 
		(169.835000,-42.575000) -- 
		(170.056000,-42.462000) -- 
		(170.722000,-42.085000) -- 
		(170.857000,-41.960000) -- 
		(170.912000,-41.881000) -- 
		(170.930000,-41.784000) -- 
		(170.960000,-41.540000) -- 
		(170.745000,-40.339000) -- 
		(170.707000,-39.962000) -- 
		(170.723000,-39.843000) -- 
		(170.745000,-39.763000) -- 
		(171.909000,-38.080000) -- 
		(172.309000,-37.124000) -- 
		(173.714000,-37.103000) -- 
		(175.906000,-37.092000) -- 
		(176.107000,-37.092000) -- 
		(176.708000,-37.089000) -- 
		(176.908000,-37.088000) -- 
		(176.954000,-37.088000) -- 
		(177.091000,-37.088000) -- 
		(177.136000,-37.086000) -- 
		(178.762000,-37.078000) -- 
		(179.696000,-37.073000) -- 
		(179.873000,-37.072000) -- 
		(181.198000,-37.094000) -- 
		(182.743000,-37.141000) -- 
		(183.441000,-37.172000) -- 
		(183.635000,-37.185000) -- 
		(183.933000,-37.205000) -- 
		(183.934000,-37.206000) -- 
		(184.623000,-37.216000) -- 
		(184.787000,-37.218000) -- 
		(185.258000,-37.187000) -- 
		(185.311000,-37.330000) -- 
		(185.453000,-37.709000) -- 
		(185.875000,-38.831000) -- 
		(186.441000,-40.385000) -- 
		(186.592000,-40.821000) -- 
		(186.702000,-41.136000) -- 
		(186.737000,-41.242000) -- 
		(186.769000,-41.341000) -- 
		(186.838000,-41.537000) -- 
		(186.922000,-41.778000) -- 
		(187.231000,-42.630000) -- 
		(187.419000,-43.157000) -- 
		(187.771000,-44.137000) -- 
		(187.886000,-44.456000) -- 
		(188.228000,-45.411000) -- 
		(188.275000,-45.541000) -- 
		(188.276000,-45.542000) -- 
		(188.343000,-45.728000) -- 
		(188.385000,-45.841000) -- 
		(188.392000,-45.860000) -- 
		(188.536000,-46.252000) -- 
		(188.583000,-46.383000) -- 
		(188.606000,-46.449000) -- 
		(188.674000,-46.645000) -- 
		(188.696000,-46.709000) -- 
		(188.781000,-46.955000) -- 
		(188.845000,-47.139000) -- 
		(188.997000,-47.541000) -- 
		(189.046000,-47.682000) -- 
		(189.130000,-47.926000);
	\filldraw [draw=black, ultra thick, fill=gold]
		(204.191000,-282.799000) -- 
		(201.430000,-281.525000) -- 
		(199.916000,-280.313000) -- 
		(199.845000,-280.256000) -- 
		(198.192000,-279.817000) -- 
		(197.498000,-279.632000) -- 
		(196.934000,-279.482000) -- 
		(196.801000,-279.446000) -- 
		(192.769000,-278.477000) -- 
		(192.522000,-278.384000) -- 
		(190.853000,-277.758000) -- 
		(190.235000,-277.212000) -- 
		(189.437000,-276.281000) -- 
		(189.169000,-275.259000) -- 
		(189.147000,-274.638000) -- 
		(189.391000,-273.937000) -- 
		(189.680000,-273.631000) -- 
		(189.757000,-273.545000) -- 
		(190.052000,-273.212000) -- 
		(190.077000,-273.185000) -- 
		(190.078000,-273.184000) -- 
		(190.184000,-273.071000) -- 
		(190.409000,-272.816000) -- 
		(190.441000,-272.780000) -- 
		(190.795000,-272.376000) -- 
		(191.182000,-271.936000) -- 
		(191.558000,-271.508000) -- 
		(192.089000,-271.059000) -- 
		(192.189000,-270.975000) -- 
		(192.645000,-270.593000) -- 
		(194.168000,-269.783000) -- 
		(196.599000,-268.831000) -- 
		(197.792000,-268.263000) -- 
		(198.550000,-267.535000) -- 
		(198.973000,-266.662000) -- 
		(199.130000,-265.668000) -- 
		(199.156000,-265.104000) -- 
		(199.048000,-264.775000) -- 
		(199.012000,-264.664000) -- 
		(198.889000,-264.604000) -- 
		(198.481000,-264.307000) -- 
		(197.843000,-264.154000) -- 
		(197.795000,-264.159000) -- 
		(197.645000,-264.174000) -- 
		(197.094000,-264.228000) -- 
		(196.281000,-264.585000) -- 
		(195.213000,-265.446000) -- 
		(194.479000,-266.738000) -- 
		(193.811000,-267.780000) -- 
		(193.150000,-268.429000) -- 
		(191.705000,-269.465000) -- 
		(191.342000,-269.745000) -- 
		(191.296000,-269.780000) -- 
		(190.721000,-270.208000) -- 
		(190.159000,-270.652000) -- 
		(189.575000,-271.101000) -- 
		(189.070000,-271.425000) -- 
		(188.993000,-271.475000) -- 
		(188.739000,-271.683000) -- 
		(188.647000,-271.758000) -- 
		(188.157000,-272.051000) -- 
		(188.044000,-272.119000) -- 
		(187.317000,-272.545000) -- 
		(185.863000,-273.567000) -- 
		(182.165000,-276.642000) -- 
		(181.012000,-277.520000) -- 
		(179.868000,-277.993000) -- 
		(178.928000,-278.222000) -- 
		(177.923000,-278.314000) -- 
		(177.194000,-278.215000) -- 
		(177.136000,-278.207000) -- 
		(177.111000,-278.201000) -- 
		(176.871000,-278.090000) -- 
		(176.251000,-277.674000) -- 
		(176.129000,-277.596000) -- 
		(176.127000,-277.594000) -- 
		(175.703000,-277.318000) -- 
		(175.577000,-277.147000) -- 
		(175.457000,-276.983000) -- 
		(175.302000,-276.683000) -- 
		(175.298000,-276.658000) -- 
		(175.274000,-276.470000) -- 
		(175.192000,-276.050000) -- 
		(175.197000,-275.997000) -- 
		(175.459000,-275.305000) -- 
		(175.633000,-274.847000) -- 
		(175.876000,-274.207000) -- 
		(176.026000,-273.830000) -- 
		(176.118000,-273.658000) -- 
		(176.269000,-273.377000) -- 
		(176.796000,-272.380000) -- 
		(176.796000,-272.345000) -- 
		(177.421000,-271.291000) -- 
		(178.075000,-270.421000) -- 
		(179.295000,-268.981000) -- 
		(180.235000,-268.012000) -- 
		(180.427000,-267.813000) -- 
		(181.553000,-266.830000) -- 
		(181.738000,-266.570000) -- 
		(181.880000,-266.372000) -- 
		(181.977000,-266.238000) -- 
		(182.331000,-265.744000) -- 
		(182.784000,-265.122000) -- 
		(183.220000,-264.503000) -- 
		(183.608000,-263.965000) -- 
		(183.658000,-263.896000) -- 
		(183.891000,-263.570000) -- 
		(184.394000,-262.457000) -- 
		(184.904000,-261.170000) -- 
		(184.959000,-260.762000) -- 
		(184.960000,-260.761000) -- 
		(185.005000,-260.418000) -- 
		(184.936000,-259.943000) -- 
		(184.753000,-259.395000) -- 
		(184.534000,-258.777000) -- 
		(184.512000,-258.584000) -- 
		(184.468000,-258.185000) -- 
		(184.507000,-257.549000) -- 
		(184.511000,-257.486000) -- 
		(184.528000,-257.282000) -- 
		(184.849000,-256.560000) -- 
		(185.450000,-255.713000) -- 
		(185.808000,-255.088000) -- 
		(185.925000,-253.903000) -- 
		(185.593000,-252.913000) -- 
		(184.806000,-252.199000) -- 
		(183.761000,-251.620000) -- 
		(182.893000,-251.386000) -- 
		(181.500000,-251.245000) -- 
		(179.599000,-251.412000) -- 
		(177.948000,-251.679000) -- 
		(174.447000,-252.165000) -- 
		(174.240000,-252.205000) -- 
		(174.063000,-252.332000) -- 
		(173.202000,-252.940000) -- 
		(172.691000,-253.227000) -- 
		(172.195000,-253.350000) -- 
		(171.406000,-253.328000) -- 
		(170.634000,-253.468000) -- 
		(169.942000,-253.714000) -- 
		(169.143000,-254.326000) -- 
		(168.828000,-254.597000) -- 
		(168.772000,-254.645000) -- 
		(168.417000,-255.061000) -- 
		(167.771000,-255.828000) -- 
		(167.424000,-256.221000) -- 
		(167.107000,-256.848000) -- 
		(166.853000,-257.348000) -- 
		(166.819000,-257.384000) -- 
		(166.317000,-257.915000) -- 
		(164.992000,-258.992000) -- 
		(164.177000,-259.470000) -- 
		(163.298000,-259.852000) -- 
		(162.408000,-259.986000) -- 
		(161.498000,-260.077000) -- 
		(161.306000,-260.107000) -- 
		(160.646000,-260.080000) -- 
		(159.688000,-259.970000) -- 
		(153.678000,-259.297000) -- 
		(150.520000,-258.548000) -- 
		(148.032000,-257.314000) -- 
		(147.498000,-256.970000) -- 
		(146.878000,-256.374000) -- 
		(146.740000,-256.298000) -- 
		(146.166000,-255.458000) -- 
		(145.967000,-254.837000) -- 
		(145.994000,-254.433000) -- 
		(146.288000,-253.949000) -- 
		(146.358000,-253.821000) -- 
		(146.365000,-253.807000) -- 
		(146.699000,-253.734000) -- 
		(147.368000,-253.655000) -- 
		(148.118000,-253.845000) -- 
		(149.501000,-254.395000) -- 
		(150.832000,-255.039000) -- 
		(150.959000,-255.100000) -- 
		(151.253000,-255.242000) -- 
		(152.289000,-255.525000) -- 
		(152.402000,-255.556000) -- 
		(152.600000,-255.610000) -- 
		(153.578000,-255.599000) -- 
		(153.863000,-255.596000) -- 
		(155.491000,-255.798000) -- 
		(157.585000,-255.694000) -- 
		(160.235000,-255.323000) -- 
		(161.514000,-255.019000) -- 
		(161.995000,-254.679000) -- 
		(162.353000,-254.170000) -- 
		(162.569000,-253.782000) -- 
		(162.569000,-253.553000) -- 
		(162.567000,-253.490000) -- 
		(162.564000,-253.079000) -- 
		(162.448000,-252.480000) -- 
		(161.742000,-251.175000) -- 
		(161.593000,-250.834000) -- 
		(161.348000,-250.277000) -- 
		(160.983000,-248.961000) -- 
		(160.923000,-248.307000) -- 
		(160.932000,-248.130000) -- 
		(160.936000,-248.045000) -- 
		(161.157000,-247.153000) -- 
		(161.478000,-246.354000) -- 
		(161.919000,-245.785000) -- 
		(162.886000,-245.139000) -- 
		(164.154000,-244.796000) -- 
		(165.120000,-244.713000) -- 
		(165.384000,-244.618000) -- 
		(166.239000,-244.733000) -- 
		(167.205000,-244.568000) -- 
		(169.125000,-243.852000) -- 
		(170.042000,-243.267000) -- 
		(170.852000,-242.491000) -- 
		(171.307000,-242.052000) -- 
		(171.509000,-241.938000) -- 
		(171.948000,-241.219000) -- 
		(172.456000,-240.458000) -- 
		(172.747000,-239.674000) -- 
		(172.811000,-239.004000) -- 
		(172.713000,-238.367000) -- 
		(172.417000,-237.691000) -- 
		(171.835000,-236.788000) -- 
		(171.103000,-236.128000) -- 
		(169.867000,-235.252000) -- 
		(168.728000,-234.796000) -- 
		(167.789000,-234.590000) -- 
		(167.210000,-234.349000) -- 
		(166.830000,-234.132000) -- 
		(166.316000,-233.546000) -- 
		(165.845000,-233.215000) -- 
		(165.534000,-232.998000) -- 
		(164.300000,-232.482000) -- 
		(163.268000,-232.282000) -- 
		(163.235000,-232.281000) -- 
		(163.114000,-232.239000) -- 
		(162.302000,-231.965000) -- 
		(161.685000,-231.715000) -- 
		(161.138000,-231.618000) -- 
		(160.117000,-231.913000) -- 
		(159.205000,-232.400000) -- 
		(158.626000,-233.096000) -- 
		(158.411000,-233.837000) -- 
		(158.284000,-234.113000) -- 
		(158.283000,-234.114000) -- 
		(158.276000,-234.244000) -- 
		(158.148000,-235.260000) -- 
		(158.142000,-236.203000) -- 
		(158.358000,-237.855000) -- 
		(158.410000,-239.175000) -- 
		(158.161000,-240.466000) -- 
		(157.805000,-241.403000) -- 
		(157.344000,-242.138000) -- 
		(156.976000,-242.566000) -- 
		(156.761000,-242.746000) -- 
		(156.728000,-242.774000) -- 
		(156.404000,-243.038000) -- 
		(155.779000,-243.274000) -- 
		(155.769000,-243.297000) -- 
		(155.765000,-243.387000) -- 
		(154.991000,-243.546000) -- 
		(154.325000,-243.405000) -- 
		(154.290000,-243.399000) -- 
		(153.858000,-243.111000) -- 
		(153.704000,-242.882000) -- 
		(153.683000,-242.840000) -- 
		(153.486000,-242.567000) -- 
		(153.262000,-242.028000) -- 
		(153.149000,-241.171000) -- 
		(153.168000,-240.679000) -- 
		(153.197000,-240.452000) -- 
		(153.199000,-240.447000) -- 
		(153.254000,-240.130000) -- 
		(153.220000,-239.751000) -- 
		(153.237000,-239.380000) -- 
		(153.234000,-239.369000) -- 
		(153.110000,-238.717000) -- 
		(153.025000,-238.006000) -- 
		(153.017000,-237.964000) -- 
		(152.882000,-237.124000) -- 
		(152.554000,-235.697000) -- 
		(152.180000,-234.625000) -- 
		(151.625000,-233.333000) -- 
		(150.644000,-231.914000) -- 
		(149.387000,-230.472000) -- 
		(148.209000,-229.749000) -- 
		(147.346000,-229.438000) -- 
		(146.596000,-229.286000) -- 
		(145.923000,-229.199000) -- 
		(145.581000,-229.182000) -- 
		(145.563000,-229.182000) -- 
		(144.994000,-229.154000) -- 
		(143.466000,-229.254000) -- 
		(142.987000,-229.285000) -- 
		(141.881000,-229.524000) -- 
		(141.591000,-229.578000) -- 
		(141.544000,-229.586000) -- 
		(140.916000,-229.698000) -- 
		(139.970000,-230.002000) -- 
		(139.024000,-230.159000) -- 
		(138.055000,-230.041000) -- 
		(137.008000,-229.647000) -- 
		(136.197000,-228.888000) -- 
		(135.530000,-227.813000) -- 
		(135.284000,-226.624000) -- 
		(135.480000,-225.562000) -- 
		(136.096000,-224.605000) -- 
		(136.869000,-224.065000) -- 
		(137.196000,-223.927000) -- 
		(137.251000,-223.901000) -- 
		(138.537000,-223.303000) -- 
		(138.829000,-223.246000) -- 
		(139.660000,-223.085000) -- 
		(140.705000,-223.142000) -- 
		(140.844000,-223.149000) -- 
		(140.976000,-223.157000) -- 
		(142.638000,-223.710000) -- 
		(144.159000,-224.756000) -- 
		(148.069000,-227.540000) -- 
		(149.172000,-228.046000) -- 
		(149.883000,-228.230000) -- 
		(151.078000,-228.243000) -- 
		(152.509000,-227.795000) -- 
		(153.027000,-227.500000) -- 
		(153.071000,-227.468000) -- 
		(153.776000,-226.943000) -- 
		(154.829000,-225.864000) -- 
		(155.481000,-224.910000) -- 
		(155.803000,-224.122000) -- 
		(155.895000,-223.645000) -- 
		(155.800000,-223.146000) -- 
		(155.270000,-222.526000) -- 
		(154.655000,-222.275000) -- 
		(154.125000,-222.164000) -- 
		(153.518000,-222.075000) -- 
		(153.236000,-222.035000) -- 
		(151.878000,-222.075000) -- 
		(151.602000,-222.083000) -- 
		(150.273000,-221.903000) -- 
		(148.444000,-221.366000) -- 
		(148.353000,-221.340000) -- 
		(146.850000,-220.538000) -- 
		(145.488000,-219.480000) -- 
		(144.677000,-218.577000) -- 
		(144.217000,-217.507000) -- 
		(143.991000,-216.331000) -- 
		(143.910000,-215.321000) -- 
		(143.536000,-214.352000) -- 
		(143.424000,-214.209000) -- 
		(143.107000,-213.698000) -- 
		(143.172000,-213.036000) -- 
		(143.379000,-212.228000) -- 
		(143.936000,-211.283000) -- 
		(144.244000,-210.661000) -- 
		(144.300000,-210.480000) -- 
		(144.434000,-210.055000) -- 
		(144.588000,-209.216000) -- 
		(144.586000,-209.084000) -- 
		(144.585000,-209.053000) -- 
		(144.598000,-208.113000) -- 
		(144.646000,-206.272000) -- 
		(144.433000,-205.703000) -- 
		(144.415000,-205.656000) -- 
		(144.377000,-205.558000) -- 
		(143.915000,-204.809000) -- 
		(143.417000,-204.405000) -- 
		(142.868000,-204.178000) -- 
		(141.970000,-204.086000) -- 
		(141.311000,-204.210000) -- 
		(141.008000,-204.381000) -- 
		(140.994000,-204.384000) -- 
		(140.956000,-204.392000) -- 
		(140.978000,-204.338000) -- 
		(138.688000,-205.628000) -- 
		(138.682000,-205.632000) -- 
		(138.430000,-205.816000) -- 
		(137.785000,-206.450000) -- 
		(137.622000,-206.693000) -- 
		(137.312000,-206.903000) -- 
		(137.049000,-207.410000) -- 
		(136.908000,-207.958000) -- 
		(136.855000,-208.345000) -- 
		(136.846000,-208.418000) -- 
		(136.788000,-208.863000) -- 
		(136.785000,-208.884000) -- 
		(136.763000,-209.019000) -- 
		(136.745000,-209.173000) -- 
		(136.737000,-209.259000) -- 
		(136.712000,-209.528000) -- 
		(136.712000,-209.537000) -- 
		(136.673000,-209.931000) -- 
		(136.711000,-210.239000) -- 
		(136.714000,-210.263000) -- 
		(136.769000,-210.603000) -- 
		(136.779000,-210.710000) -- 
		(136.787000,-210.773000) -- 
		(136.850000,-210.974000) -- 
		(136.906000,-211.297000) -- 
		(136.960000,-211.602000) -- 
		(136.967000,-211.646000) -- 
		(137.023000,-211.945000) -- 
		(137.314000,-213.293000) -- 
		(137.303000,-214.797000) -- 
		(137.147000,-216.128000) -- 
		(136.887000,-217.018000) -- 
		(136.776000,-217.172000) -- 
		(136.735000,-217.227000) -- 
		(136.358000,-217.738000) -- 
		(135.575000,-218.347000) -- 
		(135.514000,-218.374000) -- 
		(134.629000,-218.771000) -- 
		(133.714000,-219.080000) -- 
		(133.686000,-219.089000) -- 
		(133.314000,-219.235000) -- 
		(132.045000,-219.497000) -- 
		(131.107000,-219.632000) -- 
		(130.065000,-219.731000) -- 
		(129.098000,-219.993000) -- 
		(128.189000,-220.296000) -- 
		(126.799000,-220.738000) -- 
		(125.814000,-221.064000) -- 
		(125.426000,-221.211000) -- 
		(124.594000,-221.642000) -- 
		(123.128000,-222.196000) -- 
		(122.408000,-222.312000) -- 
		(121.214000,-222.495000) -- 
		(120.399000,-222.498000) -- 
		(119.431000,-222.300000) -- 
		(118.614000,-222.030000) -- 
		(118.082000,-221.734000) -- 
		(117.598000,-221.484000) -- 
		(117.170000,-221.090000) -- 
		(117.119000,-221.022000) -- 
		(116.713000,-220.471000) -- 
		(116.444000,-219.690000) -- 
		(116.479000,-219.214000) -- 
		(116.581000,-218.657000) -- 
		(116.967000,-218.106000) -- 
		(117.749000,-217.304000) -- 
		(118.919000,-216.179000) -- 
		(120.034000,-215.383000) -- 
		(121.292000,-214.839000) -- 
		(122.806000,-213.969000) -- 
		(124.384000,-213.117000) -- 
		(125.546000,-212.330000) -- 
		(126.236000,-211.860000) -- 
		(126.830000,-211.228000) -- 
		(127.054000,-210.891000) -- 
		(127.080000,-210.851000) -- 
		(127.141000,-210.758000) -- 
		(127.357000,-210.360000) -- 
		(127.411000,-209.987000) -- 
		(127.343000,-209.517000) -- 
		(127.084000,-209.136000) -- 
		(126.785000,-208.926000) -- 
		(126.424000,-208.800000) -- 
		(125.880000,-208.752000) -- 
		(125.386000,-208.813000) -- 
		(125.365000,-208.820000) -- 
		(124.914000,-208.956000) -- 
		(123.933000,-209.558000) -- 
		(123.133000,-210.336000) -- 
		(122.813000,-210.725000) -- 
		(121.879000,-211.318000) -- 
		(120.446000,-212.295000) -- 
		(118.436000,-213.362000) -- 
		(117.368000,-213.819000) -- 
		(116.810000,-214.009000) -- 
		(116.445000,-214.114000) -- 
		(116.102000,-214.215000) -- 
		(115.356000,-214.311000) -- 
		(114.827000,-214.291000) -- 
		(114.525000,-214.154000) -- 
		(114.437000,-214.046000) -- 
		(114.057000,-213.580000) -- 
		(113.857000,-213.208000) -- 
		(113.823000,-213.142000) -- 
		(113.736000,-212.739000) -- 
		(113.753000,-212.353000) -- 
		(113.836000,-211.974000) -- 
		(114.004000,-211.667000) -- 
		(114.146000,-211.569000) -- 
		(114.505000,-211.302000) -- 
		(114.983000,-211.086000) -- 
		(115.130000,-211.019000) -- 
		(115.247000,-211.025000) -- 
		(115.697000,-210.806000) -- 
		(116.236000,-210.507000) -- 
		(116.859000,-210.028000) -- 
		(117.274000,-209.510000) -- 
		(117.518000,-209.017000) -- 
		(117.752000,-208.372000) -- 
		(117.870000,-207.208000) -- 
		(117.768000,-205.805000) -- 
		(117.183000,-204.322000) -- 
		(116.514000,-203.186000) -- 
		(115.762000,-202.543000) -- 
		(115.115000,-202.069000) -- 
		(113.919000,-201.629000) -- 
		(112.391000,-201.215000) -- 
		(111.546000,-201.066000) -- 
		(110.418000,-201.102000) -- 
		(109.489000,-201.194000) -- 
		(108.912000,-201.317000) -- 
		(107.073000,-201.486000) -- 
		(106.117000,-201.660000) -- 
		(102.903000,-201.753000) -- 
		(101.995000,-201.689000) -- 
		(100.067000,-201.553000) -- 
		(98.777000,-201.551000) -- 
		(96.413000,-201.147000) -- 
		(94.780000,-200.751000) -- 
		(93.592000,-200.606000) -- 
		(93.578000,-200.605000) -- 
		(92.995000,-200.530000) -- 
		(92.257000,-200.518000) -- 
		(91.669000,-200.590000) -- 
		(91.438000,-200.663000) -- 
		(91.373000,-200.684000) -- 
		(90.997000,-200.804000) -- 
		(90.277000,-200.976000) -- 
		(89.406000,-201.116000) -- 
		(88.431000,-201.241000) -- 
		(87.332000,-201.391000) -- 
		(86.641000,-201.483000) -- 
		(84.870000,-201.668000) -- 
		(83.513000,-201.455000) -- 
		(82.355000,-201.114000) -- 
		(81.357000,-200.745000) -- 
		(80.521000,-200.185000) -- 
		(79.694000,-199.639000) -- 
		(79.038000,-199.077000) -- 
		(78.474000,-198.199000) -- 
		(78.111000,-197.540000) -- 
		(77.851000,-196.660000) -- 
		(77.794000,-196.277000) -- 
		(77.752000,-195.990000) -- 
		(77.806000,-195.539000) -- 
		(77.925000,-195.187000) -- 
		(77.938000,-195.171000) -- 
		(78.059000,-194.988000) -- 
		(78.154000,-194.894000) -- 
		(78.239000,-194.810000) -- 
		(78.239000,-194.809000) -- 
		(78.498000,-194.552000) -- 
		(79.073000,-194.357000) -- 
		(79.374000,-194.326000) -- 
		(79.650000,-194.297000) -- 
		(80.303000,-194.351000) -- 
		(81.511000,-194.715000) -- 
		(81.587000,-194.737000) -- 
		(81.905000,-194.834000) -- 
		(83.758000,-195.164000) -- 
		(84.469000,-195.186000) -- 
		(84.820000,-194.983000) -- 
		(85.671000,-194.373000) -- 
		(86.086000,-193.749000) -- 
		(86.340000,-192.999000) -- 
		(86.327000,-192.231000) -- 
		(86.039000,-191.288000) -- 
		(85.866000,-190.359000) -- 
		(86.138000,-189.631000) -- 
		(86.506000,-188.993000) -- 
		(87.383000,-187.769000) -- 
		(88.318000,-186.700000) -- 
		(88.903000,-185.720000) -- 
		(89.428000,-184.242000) -- 
		(89.728000,-183.181000) -- 
		(89.849000,-182.829000) -- 
		(89.915000,-182.635000) -- 
		(90.037000,-182.276000) -- 
		(90.949000,-180.698000) -- 
		(91.712000,-179.792000) -- 
		(91.747000,-179.754000) -- 
		(91.842000,-179.649000) -- 
		(92.057000,-179.417000) -- 
		(92.127000,-179.340000) -- 
		(92.998000,-178.381000) -- 
		(94.392000,-176.714000) -- 
		(95.637000,-175.336000) -- 
		(96.138000,-174.845000) -- 
		(96.626000,-173.773000) -- 
		(96.757000,-172.834000) -- 
		(96.688000,-171.890000) -- 
		(96.424000,-171.288000) -- 
		(96.085000,-170.782000) -- 
		(95.406000,-170.385000) -- 
		(94.607000,-170.216000) -- 
		(93.966000,-170.258000) -- 
		(93.575000,-170.283000) -- 
		(92.405000,-170.779000) -- 
		(91.293000,-171.372000) -- 
		(89.959000,-172.329000) -- 
		(89.060000,-173.154000) -- 
		(88.452000,-173.855000) -- 
		(88.239000,-174.246000) -- 
		(87.821000,-174.845000) -- 
		(87.324000,-175.675000) -- 
		(86.904000,-176.678000) -- 
		(86.559000,-177.727000) -- 
		(86.206000,-179.117000) -- 
		(85.753000,-180.375000) -- 
		(85.741000,-180.408000) -- 
		(85.740000,-180.409000) -- 
		(85.607000,-180.775000) -- 
		(84.900000,-182.263000) -- 
		(84.430000,-183.323000) -- 
		(84.205000,-183.638000) -- 
		(83.459000,-184.684000) -- 
		(82.402000,-185.688000) -- 
		(81.445000,-186.457000) -- 
		(80.986000,-186.806000) -- 
		(79.468000,-187.677000) -- 
		(77.933000,-188.190000) -- 
		(76.657000,-188.493000) -- 
		(75.665000,-188.356000) -- 
		(75.527000,-188.336000) -- 
		(74.824000,-188.237000) -- 
		(74.384000,-188.085000) -- 
		(73.968000,-187.653000) -- 
		(73.504000,-187.158000) -- 
		(72.848000,-186.198000) -- 
		(72.511000,-185.627000) -- 
		(72.128000,-184.982000) -- 
		(71.552000,-183.941000) -- 
		(71.088000,-182.885000) -- 
		(70.800000,-181.926000) -- 
		(70.400000,-181.093000) -- 
		(70.080000,-180.213000) -- 
		(70.006000,-179.942000) -- 
		(69.808000,-179.221000) -- 
		(69.410000,-177.285000) -- 
		(69.400000,-177.219000) -- 
		(68.994000,-174.634000) -- 
		(68.991000,-174.327000) -- 
		(68.982000,-173.185000) -- 
		(68.938000,-171.801000) -- 
		(68.917000,-170.670000) -- 
		(68.896000,-169.538000) -- 
		(68.899000,-169.301000) -- 
		(68.913000,-168.213000) -- 
		(68.914000,-165.445000) -- 
		(69.123000,-163.813000) -- 
		(69.714000,-162.181000) -- 
		(70.483000,-160.437000) -- 
		(71.588000,-159.110000) -- 
		(72.837000,-157.751000) -- 
		(73.296000,-157.352000) -- 
		(73.924000,-156.761000) -- 
		(74.219000,-156.341000) -- 
		(74.220000,-156.340000) -- 
		(74.824000,-155.480000) -- 
		(74.886000,-155.363000) -- 
		(74.948000,-155.246000) -- 
		(75.279000,-154.659000) -- 
		(75.509000,-153.609000) -- 
		(75.439000,-152.851000) -- 
		(75.130000,-152.136000) -- 
		(74.824000,-151.746000) -- 
		(73.776000,-150.742000) -- 
		(72.912000,-150.358000) -- 
		(71.120000,-150.262000) -- 
		(69.376000,-150.310000) -- 
		(67.920000,-150.791000) -- 
		(66.624000,-151.478000) -- 
		(66.627000,-152.320000) -- 
		(66.633000,-154.322000) -- 
		(66.634000,-154.324000) -- 
		(66.634000,-154.344000) -- 
		(66.566000,-154.534000) -- 
		(65.825000,-156.375000) -- 
		(64.721000,-156.710000) -- 
		(63.728000,-157.495000) -- 
		(62.705000,-159.158000) -- 
		(61.760000,-161.142000) -- 
		(60.800000,-162.229000) -- 
		(58.288000,-165.749000) -- 
		(56.976000,-167.605000) -- 
		(55.776000,-169.077000) -- 
		(54.064000,-170.677000) -- 
		(52.256000,-171.829000) -- 
		(50.936000,-172.284000) -- 
		(50.726000,-172.262000) -- 
		(50.499000,-172.240000) -- 
		(50.503000,-172.074000) -- 
		(50.063000,-172.081000) -- 
		(49.464000,-172.149000) -- 
		(49.134000,-171.863000) -- 
		(49.310000,-171.803000) -- 
		(49.469000,-171.645000) -- 
		(49.739000,-170.546000) -- 
		(50.279000,-170.041000) -- 
		(50.441000,-169.766000) -- 
		(50.495000,-169.400000) -- 
		(50.306000,-169.171000) -- 
		(50.176000,-168.703000) -- 
		(49.981000,-168.002000) -- 
		(49.954000,-167.362000) -- 
		(50.467000,-166.215000) -- 
		(50.549000,-165.683000) -- 
		(50.790000,-163.717000) -- 
		(50.871000,-163.671000) -- 
		(50.871000,-163.144000) -- 
		(51.060000,-162.503000) -- 
		(51.411000,-161.976000) -- 
		(52.329000,-161.219000) -- 
		(53.004000,-160.509000) -- 
		(53.904000,-159.804000) -- 
		(54.599000,-159.153000) -- 
		(55.260000,-158.708000) -- 
		(56.274000,-158.270000) -- 
		(57.160000,-157.645000) -- 
		(57.942000,-157.551000) -- 
		(58.726000,-157.488000) -- 
		(59.669000,-157.313000) -- 
		(60.612000,-157.122000) -- 
		(61.956000,-156.739000) -- 
		(63.187000,-155.796000) -- 
		(64.017000,-154.774000) -- 
		(64.110000,-154.541000) -- 
		(64.167000,-154.359000) -- 
		(64.259000,-154.070000) -- 
		(63.632000,-153.718000) -- 
		(61.874000,-153.749000) -- 
		(61.777000,-153.750000) -- 
		(59.936000,-153.735000) -- 
		(59.567000,-153.645000) -- 
		(59.436000,-153.606000) -- 
		(59.379000,-153.590000) -- 
		(58.336000,-152.824000) -- 
		(56.624000,-151.544000) -- 
		(55.473000,-150.198000) -- 
		(53.905000,-148.855000) -- 
		(52.657000,-147.878000) -- 
		(52.020000,-147.403000) -- 
		(52.018000,-147.402000) -- 
		(51.899000,-147.313000) -- 
		(50.307000,-146.433000) -- 
		(48.951000,-145.889000) -- 
		(46.719000,-145.095000) -- 
		(44.703000,-144.566000) -- 
		(42.991000,-144.231000) -- 
		(40.959000,-143.942000) -- 
		(39.039000,-143.782000) -- 
		(37.199000,-143.654000) -- 
		(35.807000,-143.814000) -- 
		(34.079000,-144.150000) -- 
		(32.991000,-144.646000) -- 
		(32.239000,-145.334000) -- 
		(32.031000,-145.926000) -- 
		(31.951000,-146.982000) -- 
		(31.854000,-148.166000) -- 
		(31.818000,-149.426000) -- 
		(31.846000,-149.545000) -- 
		(32.015000,-150.295000) -- 
		(32.752000,-152.056000) -- 
		(34.304000,-153.784000) -- 
		(36.224000,-155.301000) -- 
		(37.648000,-156.901000) -- 
		(38.545000,-158.212000) -- 
		(38.929000,-159.684000) -- 
		(38.882000,-160.236000) -- 
		(38.859000,-160.501000) -- 
		(38.836000,-160.786000) -- 
		(38.673000,-161.269000) -- 
		(38.241000,-161.812000) -- 
		(37.649000,-162.133000) -- 
		(36.657000,-162.181000) -- 
		(33.969000,-162.068000) -- 
		(31.985000,-161.908000) -- 
		(30.145000,-161.556000) -- 
		(28.481000,-161.108000) -- 
		(26.833000,-160.580000) -- 
		(25.887000,-160.485000) -- 
		(24.671000,-160.181000) -- 
		(23.855000,-160.053000) -- 
		(22.495000,-159.797000) -- 
		(21.263000,-159.541000) -- 
		(20.271000,-159.333000) -- 
		(18.591000,-158.453000) -- 
		(18.350000,-158.242000) -- 
		(16.543000,-156.662000) -- 
		(14.431000,-154.502000) -- 
		(12.943000,-151.879000) -- 
		(12.805000,-151.377000) -- 
		(12.594000,-150.614000) -- 
		(12.384000,-149.848000) -- 
		(12.332000,-149.658000) -- 
		(12.279000,-149.466000) -- 
		(12.207000,-149.207000) -- 
		(12.351000,-145.879000) -- 
		(13.325000,-142.952000) -- 
		(13.677000,-142.506000) -- 
		(13.696000,-142.479000) -- 
		(14.845000,-141.017000) -- 
		(17.677000,-138.873000) -- 
		(20.525000,-137.575000) -- 
		(24.605000,-136.567000) -- 
		(24.765000,-136.559000) -- 
		(25.042000,-136.549000) -- 
		(26.144000,-136.505000) -- 
		(27.117000,-136.551000) -- 
		(29.534000,-136.215000) -- 
		(32.238000,-136.023000) -- 
		(34.975000,-135.878000) -- 
		(36.301000,-135.887000) -- 
		(36.576000,-135.882000) -- 
		(37.583000,-135.862000) -- 
		(39.535000,-135.318000) -- 
		(39.983000,-134.806000) -- 
		(40.079000,-134.294000) -- 
		(39.664000,-133.590000) -- 
		(38.704000,-133.206000) -- 
		(37.440000,-133.047000) -- 
		(35.851000,-133.171000) -- 
		(35.691000,-133.156000) -- 
		(35.527000,-133.140000) -- 
		(35.181000,-132.959000) -- 
		(34.606000,-132.491000) -- 
		(33.470000,-132.267000) -- 
		(33.314000,-132.283000) -- 
		(33.141000,-132.324000) -- 
		(32.801000,-132.548000) -- 
		(31.095000,-133.453000) -- 
		(30.838000,-133.506000) -- 
		(30.271000,-133.608000) -- 
		(29.453000,-133.593000) -- 
		(29.296000,-133.596000) -- 
		(28.187000,-133.613000) -- 
		(26.922000,-133.636000) -- 
		(26.875000,-133.638000) -- 
		(26.826000,-133.638000) -- 
		(26.031000,-133.655000) -- 
		(24.335000,-133.273000) -- 
		(24.232000,-133.250000) -- 
		(22.095000,-132.694000) -- 
		(16.591000,-132.535000) -- 
		(15.215000,-132.279000) -- 
		(13.919000,-131.830000) -- 
		(13.601000,-131.640000) -- 
		(12.017000,-130.741000) -- 
		(11.970000,-130.714000) -- 
		(11.137000,-130.242000) -- 
		(10.351000,-129.478000) -- 
		(9.397000,-128.683000) -- 
		(9.378000,-128.668000) -- 
		(8.911000,-128.278000) -- 
		(8.786000,-128.084000) -- 
		(8.770000,-128.058000) -- 
		(8.223000,-127.206000) -- 
		(8.431000,-126.439000) -- 
		(8.553000,-126.377000) -- 
		(9.247000,-126.022000) -- 
		(13.342000,-125.061000) -- 
		(15.341000,-124.309000) -- 
		(16.301000,-123.492000) -- 
		(16.797000,-121.732000) -- 
		(17.420000,-119.924000) -- 
		(18.359000,-118.575000) -- 
		(18.368000,-118.563000) -- 
		(18.412000,-118.500000) -- 
		(19.976000,-116.435000) -- 
		(20.328000,-116.733000) -- 
		(20.895000,-116.665000) -- 
		(21.814000,-117.144000) -- 
		(22.921000,-117.426000) -- 
		(24.052000,-117.715000) -- 
		(24.968000,-118.973000) -- 
		(24.655000,-122.265000) -- 
		(24.654000,-122.620000) -- 
		(24.372000,-124.534000) -- 
		(24.235000,-126.022000) -- 
		(24.232000,-126.054000) -- 
		(24.218000,-126.172000) -- 
		(24.621000,-127.405000) -- 
		(25.378000,-128.488000) -- 
		(26.638000,-129.108000) -- 
		(29.180000,-128.747000) -- 
		(33.289000,-127.383000) -- 
		(34.433000,-126.791000) -- 
		(35.218000,-126.312000) -- 
		(36.002000,-125.835000) -- 
		(36.031000,-125.818000) -- 
		(36.369000,-125.558000) -- 
		(38.341000,-123.518000) -- 
		(38.604000,-122.376000) -- 
		(38.100000,-121.329000) -- 
		(37.640000,-120.504000) -- 
		(35.898000,-118.723000) -- 
		(33.663000,-117.810000) -- 
		(30.929000,-116.436000) -- 
		(29.114000,-114.085000) -- 
		(27.998000,-111.967000) -- 
		(27.134000,-110.548000) -- 
		(25.951000,-109.324000) -- 
		(24.291000,-108.757000) -- 
		(21.423000,-108.155000) -- 
		(18.391000,-107.890000) -- 
		(18.145000,-107.873000) -- 
		(18.007000,-107.864000) -- 
		(16.985000,-107.799000) -- 
		(16.079000,-107.795000) -- 
		(15.821000,-107.719000) -- 
		(15.702000,-107.685000) -- 
		(15.294000,-107.612000) -- 
		(15.096000,-107.577000) -- 
		(14.915000,-107.489000) -- 
		(14.004000,-107.048000) -- 
		(13.288000,-106.701000) -- 
		(11.580000,-105.316000) -- 
		(10.549000,-103.958000) -- 
		(10.349000,-103.723000) -- 
		(9.410000,-102.471000) -- 
		(8.364000,-101.077000) -- 
		(9.001000,-98.448000) -- 
		(11.293000,-97.231000) -- 
		(13.063000,-95.092000) -- 
		(13.722000,-94.296000) -- 
		(15.334000,-92.349000) -- 
		(15.424000,-92.240000) -- 
		(16.291000,-91.192000) -- 
		(16.806000,-91.190000) -- 
		(17.062000,-91.190000) -- 
		(17.125000,-91.246000) -- 
		(17.139000,-91.190000) -- 
		(17.588000,-91.190000) -- 
		(17.761000,-91.190000) -- 
		(18.049000,-91.255000) -- 
		(18.559000,-91.256000) -- 
		(18.909000,-91.258000) -- 
		(19.200000,-91.185000) -- 
		(20.094000,-91.183000) -- 
		(20.204000,-91.183000) -- 
		(21.397000,-91.251000) -- 
		(21.914000,-91.214000) -- 
		(22.050000,-91.205000) -- 
		(22.187000,-91.197000) -- 
		(22.470000,-91.180000) -- 
		(22.564000,-91.179000) -- 
		(22.676000,-91.177000) -- 
		(22.789000,-91.175000) -- 
		(22.835000,-91.174000) -- 
		(23.030000,-91.174000) -- 
		(23.344000,-91.173000) -- 
		(23.616000,-91.172000) -- 
		(23.887000,-91.172000) -- 
		(28.010000,-91.163000) -- 
		(30.270000,-91.155000) -- 
		(30.688000,-91.154000) -- 
		(31.940000,-91.150000) -- 
		(32.178000,-91.149000) -- 
		(32.357000,-91.148000) -- 
		(33.207000,-91.145000) -- 
		(33.508000,-91.136000) -- 
		(33.540000,-91.134000) -- 
		(33.763000,-91.120000) -- 
		(33.986000,-91.100000) -- 
		(34.208000,-91.074000) -- 
		(34.620000,-91.013000) -- 
		(34.849000,-90.970000) -- 
		(35.029000,-90.929000) -- 
		(35.449000,-90.815000) -- 
		(35.597000,-90.771000) -- 
		(35.690000,-90.743000) -- 
		(35.929000,-90.665000) -- 
		(36.166000,-90.582000) -- 
		(36.845000,-90.327000) -- 
		(37.806000,-89.964000) -- 
		(37.923000,-89.926000) -- 
		(38.517000,-90.689000) -- 
		(40.296000,-92.978000) -- 
		(40.889000,-93.741000) -- 
		(41.135000,-94.058000) -- 
		(41.872000,-95.006000) -- 
		(42.117000,-95.320000) -- 
		(42.192000,-95.418000) -- 
		(42.417000,-95.710000) -- 
		(42.492000,-95.806000) -- 
		(42.615000,-95.966000) -- 
		(42.982000,-96.441000) -- 
		(43.104000,-96.599000) -- 
		(43.181000,-96.697000) -- 
		(43.410000,-96.991000) -- 
		(43.486000,-97.089000) -- 
		(43.994000,-97.742000) -- 
		(45.518000,-99.698000) -- 
		(46.025000,-100.350000) -- 
		(46.188000,-100.559000) -- 
		(46.676000,-101.186000) -- 
		(46.838000,-101.394000) -- 
		(47.287000,-101.972000) -- 
		(48.417000,-103.429000) -- 
		(48.631000,-103.708000) -- 
		(49.077000,-104.286000) -- 
		(49.335000,-104.621000) -- 
		(49.451000,-104.769000) -- 
		(50.047000,-105.523000) -- 
		(50.577000,-106.207000) -- 
		(50.949000,-106.687000) -- 
		(51.044000,-106.809000) -- 
		(51.326000,-107.175000) -- 
		(51.420000,-107.295000) -- 
		(51.590000,-107.514000) -- 
		(51.605000,-107.533000) -- 
		(52.008000,-108.046000) -- 
		(52.162000,-108.245000) -- 
		(52.346000,-108.482000) -- 
		(52.406000,-108.560000) -- 
		(52.586000,-108.792000) -- 
		(52.646000,-108.869000) -- 
		(53.436000,-109.886000) -- 
		(54.757000,-111.587000) -- 
		(54.788000,-111.627000) -- 
		(54.854000,-111.721000) -- 
		(54.886000,-111.766000) -- 
		(54.879000,-111.779000) -- 
		(55.106000,-112.030000) -- 
		(55.269000,-112.237000) -- 
		(55.759000,-112.858000) -- 
		(56.332000,-113.510000) -- 
		(57.056000,-114.264000) -- 
		(60.844000,-119.127000) -- 
		(62.884000,-121.747000) -- 
		(63.290000,-122.268000) -- 
		(63.426000,-122.443000) -- 
		(63.483000,-122.516000) -- 
		(63.652000,-122.733000) -- 
		(63.708000,-122.805000) -- 
		(63.859000,-123.000000) -- 
		(64.078000,-123.281000) -- 
		(64.311000,-123.580000) -- 
		(64.461000,-123.773000) -- 
		(64.644000,-124.007000) -- 
		(65.023000,-124.493000) -- 
		(65.066000,-124.549000) -- 
		(65.377000,-124.877000) -- 
		(65.742000,-125.354000) -- 
		(66.724000,-126.637000) -- 
		(66.865000,-126.821000) -- 
		(67.291000,-127.351000) -- 
		(67.391000,-127.475000) -- 
		(67.435000,-127.530000) -- 
		(67.609000,-127.759000) -- 
		(67.681000,-127.855000) -- 
		(67.776000,-127.981000) -- 
		(67.899000,-128.144000) -- 
		(67.918000,-128.170000) -- 
		(68.273000,-128.626000) -- 
		(68.397000,-128.786000) -- 
		(68.462000,-128.871000) -- 
		(68.656000,-129.125000) -- 
		(68.720000,-129.209000) -- 
		(68.770000,-129.275000) -- 
		(69.423000,-130.106000) -- 
		(69.433000,-130.120000) -- 
		(70.061000,-130.951000) -- 
		(70.667000,-131.734000) -- 
		(71.558000,-132.861000) -- 
		(72.156000,-133.617000) -- 
		(72.273000,-133.769000) -- 
		(72.437000,-133.984000) -- 
		(72.929000,-134.623000) -- 
		(73.093000,-134.836000) -- 
		(73.381000,-135.209000) -- 
		(73.538000,-135.414000) -- 
		(73.796000,-135.744000) -- 
		(74.249000,-136.323000) -- 
		(74.390000,-136.503000) -- 
		(74.536000,-136.694000) -- 
		(74.667000,-136.864000) -- 
		(75.001000,-137.301000) -- 
		(75.055000,-137.376000) -- 
		(75.179000,-137.550000) -- 
		(75.518000,-138.026000) -- 
		(75.684000,-138.238000) -- 
		(76.051000,-138.705000) -- 
		(76.429000,-139.225000) -- 
		(76.751000,-139.622000) -- 
		(77.256000,-140.257000) -- 
		(77.262000,-140.264000) -- 
		(77.779000,-140.930000) -- 
		(77.902000,-141.088000) -- 
		(78.271000,-141.560000) -- 
		(78.394000,-141.717000) -- 
		(78.541000,-141.905000) -- 
		(78.963000,-142.446000) -- 
		(78.979000,-142.467000) -- 
		(79.118000,-142.659000) -- 
		(79.412000,-143.052000) -- 
		(79.812000,-143.584000) -- 
		(80.308000,-144.212000) -- 
		(80.611000,-144.595000) -- 
		(81.046000,-145.151000) -- 
		(81.940000,-146.293000) -- 
		(82.390000,-146.781000) -- 
		(82.794000,-147.219000) -- 
		(82.871000,-147.296000) -- 
		(82.908000,-147.333000) -- 
		(83.017000,-147.442000) -- 
		(83.053000,-147.477000) -- 
		(83.154000,-147.577000) -- 
		(83.191000,-147.611000) -- 
		(83.451000,-147.850000) -- 
		(83.614000,-148.000000) -- 
		(83.755000,-148.129000) -- 
		(84.151000,-148.492000) -- 
		(84.179000,-148.515000) -- 
		(84.677000,-148.915000) -- 
		(85.094000,-149.239000) -- 
		(85.535000,-149.570000) -- 
		(85.992000,-149.913000) -- 
		(86.804000,-150.536000) -- 
		(87.116000,-150.775000) -- 
		(89.249000,-152.387000) -- 
		(90.064000,-153.003000) -- 
		(90.887000,-153.628000) -- 
		(93.355000,-155.498000) -- 
		(93.673000,-155.739000) -- 
		(94.178000,-156.119000) -- 
		(94.840000,-156.618000) -- 
		(96.093000,-157.577000) -- 
		(96.366000,-157.786000) -- 
		(97.982000,-159.029000) -- 
		(99.350000,-160.060000) -- 
		(100.116000,-160.615000) -- 
		(100.516000,-160.905000) -- 
		(101.847000,-161.933000) -- 
		(102.140000,-162.159000) -- 
		(103.074000,-162.875000) -- 
		(103.075000,-162.876000) -- 
		(103.747000,-163.408000) -- 
		(103.769000,-163.357000) -- 
		(103.802000,-163.274000) -- 
		(103.864000,-163.225000) -- 
		(103.907000,-163.189000) -- 
		(104.083000,-163.049000) -- 
		(104.389000,-162.826000) -- 
		(105.011000,-162.541000) -- 
		(106.239000,-162.057000) -- 
		(106.367000,-162.001000) -- 
		(106.609000,-161.894000) -- 
		(107.125000,-161.688000) -- 
		(107.371000,-161.590000) -- 
		(107.668000,-161.565000) -- 
		(110.176000,-161.530000) -- 
		(110.912000,-161.499000) -- 
		(111.208000,-161.449000) -- 
		(111.450000,-161.394000) -- 
		(111.724000,-161.297000) -- 
		(111.778000,-161.276000) -- 
		(111.866000,-161.240000) -- 
		(111.957000,-161.204000) -- 
		(112.291000,-161.070000) -- 
		(113.036000,-160.740000) -- 
		(114.395000,-160.124000) -- 
		(115.066000,-159.789000) -- 
		(115.348000,-159.623000) -- 
		(115.602000,-159.454000) -- 
		(115.898000,-159.289000) -- 
		(116.035000,-159.232000) -- 
		(116.153000,-159.183000) -- 
		(116.317000,-159.115000) -- 
		(116.636000,-159.012000) -- 
		(117.216000,-158.824000) -- 
		(117.909000,-158.504000) -- 
		(118.051000,-158.421000) -- 
		(118.492000,-158.160000) -- 
		(118.698000,-158.027000) -- 
		(119.316000,-157.622000) -- 
		(119.521000,-157.487000) -- 
		(119.847000,-157.281000) -- 
		(120.115000,-157.109000) -- 
		(120.345000,-156.996000) -- 
		(120.464000,-156.952000) -- 
		(120.663000,-156.878000) -- 
		(120.898000,-156.805000) -- 
		(120.945000,-156.790000) -- 
		(121.270000,-156.704000) -- 
		(121.357000,-156.681000) -- 
		(121.584000,-156.586000) -- 
		(121.639000,-156.561000) -- 
		(122.003000,-156.345000) -- 
		(122.201000,-156.198000) -- 
		(122.424000,-156.033000) -- 
		(122.694000,-155.833000) -- 
		(122.760000,-155.787000) -- 
		(122.958000,-155.643000) -- 
		(123.023000,-155.594000) -- 
		(123.055000,-155.571000) -- 
		(123.152000,-155.502000) -- 
		(123.183000,-155.478000) -- 
		(123.223000,-155.449000) -- 
		(123.344000,-155.362000) -- 
		(123.383000,-155.332000) -- 
		(124.075000,-154.830000) -- 
		(124.177000,-154.757000) -- 
		(124.372000,-154.635000) -- 
		(124.656000,-154.456000) -- 
		(125.056000,-154.257000) -- 
		(125.390000,-154.110000) -- 
		(125.790000,-153.973000) -- 
		(126.111000,-153.874000) -- 
		(126.506000,-153.786000) -- 
		(126.854000,-153.721000) -- 
		(127.135000,-153.669000) -- 
		(127.474000,-153.581000) -- 
		(127.664000,-153.531000) -- 
		(127.792000,-153.490000) -- 
		(128.034000,-153.410000) -- 
		(128.750000,-153.144000) -- 
		(128.927000,-153.071000) -- 
		(129.090000,-153.004000) -- 
		(129.480000,-152.842000) -- 
		(129.848000,-152.689000) -- 
		(130.268000,-152.514000) -- 
		(130.891000,-152.303000) -- 
		(131.151000,-152.214000) -- 
		(131.884000,-151.970000) -- 
		(132.977000,-151.555000) -- 
		(133.522000,-151.360000) -- 
		(133.882000,-151.256000) -- 
		(134.194000,-151.198000) -- 
		(134.895000,-151.118000) -- 
		(135.194000,-151.138000) -- 
		(135.527000,-151.188000) -- 
		(135.929000,-151.277000) -- 
		(136.597000,-151.404000) -- 
		(136.995000,-151.489000) -- 
		(138.206000,-151.747000) -- 
		(138.739000,-151.872000) -- 
		(139.054000,-151.945000) -- 
		(139.327000,-152.009000) -- 
		(139.455000,-152.045000) -- 
		(139.668000,-152.105000) -- 
		(140.272000,-152.274000) -- 
		(141.475000,-152.694000) -- 
		(141.503000,-152.703000) -- 
		(141.775000,-152.780000) -- 
		(142.023000,-152.818000) -- 
		(142.092000,-152.803000) -- 
		(142.417000,-152.723000) -- 
		(142.643000,-152.620000) -- 
		(143.058000,-152.352000) -- 
		(143.230000,-152.247000) -- 
		(144.428000,-151.511000) -- 
		(146.054000,-150.528000) -- 
		(146.507000,-150.255000) -- 
		(147.600000,-149.593000) -- 
		(147.660000,-149.557000) -- 
		(147.839000,-149.450000) -- 
		(147.898000,-149.413000) -- 
		(147.974000,-149.368000) -- 
		(148.077000,-149.305000) -- 
		(148.201000,-149.232000) -- 
		(148.276000,-149.185000) -- 
		(148.889000,-148.814000) -- 
		(149.622000,-148.368000) -- 
		(150.318000,-147.905000) -- 
		(150.380000,-147.853000) -- 
		(150.500000,-147.751000) -- 
		(150.652000,-147.594000) -- 
		(150.679000,-147.565000) -- 
		(150.812000,-147.385000) -- 
		(151.013000,-147.086000) -- 
		(151.065000,-147.009000) -- 
		(151.117000,-146.933000) -- 
		(151.118000,-146.930000) -- 
		(151.273000,-146.702000) -- 
		(151.324000,-146.625000) -- 
		(151.509000,-146.349000) -- 
		(151.579000,-146.243000) -- 
		(152.033000,-145.499000) -- 
		(152.117000,-145.360000) -- 
		(152.215000,-145.220000) -- 
		(152.306000,-145.090000) -- 
		(152.558000,-144.798000) -- 
		(152.952000,-144.443000) -- 
		(153.138000,-144.307000) -- 
		(153.806000,-143.933000) -- 
		(153.806000,-143.932000) -- 
		(153.892000,-143.888000) -- 
		(154.377000,-143.632000) -- 
		(155.095000,-143.259000) -- 
		(156.327000,-142.589000) -- 
		(157.365000,-142.049000) -- 
		(159.626000,-140.818000) -- 
		(159.804000,-140.720000) -- 
		(160.730000,-140.271000) -- 
		(160.744000,-140.264000) -- 
		(161.556000,-139.832000) -- 
		(162.663000,-139.242000) -- 
		(164.229000,-138.407000) -- 
		(165.218000,-137.890000) -- 
		(165.988000,-137.479000) -- 
		(166.322000,-137.299000) -- 
		(166.594000,-137.154000) -- 
		(167.099000,-136.897000) -- 
		(168.555000,-136.123000) -- 
		(169.488000,-135.623000) -- 
		(169.822000,-135.445000) -- 
		(170.681000,-134.987000) -- 
		(172.922000,-133.798000) -- 
		(173.687000,-133.391000) -- 
		(174.378000,-133.024000) -- 
		(174.930000,-132.736000) -- 
		(175.802000,-132.279000) -- 
		(176.586000,-131.868000) -- 
		(176.815000,-131.747000) -- 
		(177.136000,-131.577000) -- 
		(179.566000,-130.292000) -- 
		(179.863000,-130.137000) -- 
		(181.567000,-129.244000) -- 
		(182.500000,-128.746000) -- 
		(184.739000,-127.577000) -- 
		(187.180000,-126.281000) -- 
		(188.048000,-125.832000) -- 
		(188.723000,-125.482000) -- 
		(188.723000,-125.481000) -- 
		(190.079000,-124.758000) -- 
		(190.774000,-124.391000) -- 
		(190.817000,-124.369000) -- 
		(190.944000,-124.303000) -- 
		(190.986000,-124.279000) -- 
		(191.249000,-124.141000) -- 
		(191.561000,-123.976000) -- 
		(192.039000,-123.736000) -- 
		(192.094000,-123.708000) -- 
		(192.317000,-123.635000) -- 
		(192.441000,-123.655000) -- 
		(192.933000,-123.638000) -- 
		(194.068000,-123.678000) -- 
		(194.990000,-123.709000) -- 
		(195.850000,-123.729000) -- 
		(196.046000,-123.733000) -- 
		(197.257000,-123.766000) -- 
		(199.404000,-123.811000) -- 
		(199.593000,-123.823000) -- 
		(199.993000,-123.848000) -- 
		(200.761000,-123.860000) -- 
		(200.762000,-123.861000) -- 
		(201.822000,-123.882000) -- 
		(202.300000,-123.897000) -- 
		(202.301000,-123.898000) -- 
		(203.671000,-123.935000) -- 
		(204.972000,-123.968000) -- 
		(205.356000,-123.975000) -- 
		(205.512000,-123.978000) -- 
		(206.124000,-123.990000) -- 
		(206.125000,-123.991000) -- 
		(206.454000,-123.999000) -- 
		(207.332000,-124.017000) -- 
		(207.700000,-124.027000) -- 
		(209.988000,-124.088000) -- 
		(213.573000,-124.179000) -- 
		(214.524000,-124.203000) -- 
		(216.261000,-124.259000) -- 
		(220.834000,-124.376000) -- 
		(223.652000,-124.444000) -- 
		(224.328000,-124.462000) -- 
		(227.912000,-124.553000) -- 
		(228.313000,-124.565000) -- 
		(228.746000,-124.574000) -- 
		(229.513000,-124.594000) -- 
		(229.523000,-124.594000) -- 
		(229.913000,-124.604000) -- 
		(230.061000,-124.608000) -- 
		(230.505000,-124.619000) -- 
		(230.652000,-124.622000) -- 
		(233.387000,-124.690000) -- 
		(241.589000,-124.892000) -- 
		(242.176000,-124.906000) -- 
		(242.469000,-124.905000) -- 
		(242.736000,-124.876000) -- 
		(243.003000,-124.828000) -- 
		(243.335000,-124.714000) -- 
		(243.796000,-124.524000) -- 
		(244.219000,-124.331000) -- 
		(244.634000,-124.134000) -- 
		(244.908000,-124.002000) -- 
		(245.503000,-123.708000) -- 
		(246.977000,-123.020000) -- 
		(247.107000,-122.957000) -- 
		(247.664000,-122.686000) -- 
		(248.055000,-122.496000) -- 
		(248.228000,-122.411000) -- 
		(249.252000,-121.978000) -- 
		(249.333000,-121.943000) -- 
		(249.525000,-121.871000) -- 
		(249.661000,-121.832000) -- 
		(249.814000,-121.817000) -- 
		(250.179000,-121.829000) -- 
		(251.068000,-121.813000) -- 
		(251.495000,-121.825000) -- 
		(251.875000,-121.835000) -- 
		(253.318000,-122.075000) -- 
		(253.966000,-122.174000) -- 
		(254.075000,-122.183000) -- 
		(254.541000,-122.224000) -- 
		(255.065000,-122.244000) -- 
		(256.030000,-122.243000) -- 
		(256.814000,-122.256000) -- 
		(256.970000,-122.263000) -- 
		(257.794000,-122.295000) -- 
		(258.788000,-122.473000) -- 
		(258.789000,-128.918000) -- 
		(258.789000,-130.407000) -- 
		(258.789000,-131.536000) -- 
		(258.789000,-132.428000) -- 
		(258.788000,-138.460000) -- 
		(258.788000,-138.588000) -- 
		(258.788000,-139.569000) -- 
		(262.075000,-139.666000) -- 
		(263.960000,-139.722000) -- 
		(264.007000,-139.641000) -- 
		(264.024000,-139.622000) -- 
		(264.084000,-139.552000) -- 
		(265.727000,-139.573000) -- 
		(266.006000,-139.576000) -- 
		(266.309000,-139.581000) -- 
		(266.916000,-139.581000) -- 
		(270.510000,-139.588000) -- 
		(274.100000,-139.590000) -- 
		(275.653000,-139.592000) -- 
		(277.208000,-139.595000) -- 
		(281.971000,-139.598000) -- 
		(286.735000,-139.603000) -- 
		(286.882000,-139.603000) -- 
		(287.088000,-139.633000) -- 
		(288.960000,-139.613000) -- 
		(289.933000,-139.601000) -- 
		(291.533000,-139.590000) -- 
		(292.257000,-139.602000) -- 
		(292.842000,-139.601000) -- 
		(295.232000,-139.596000) -- 
		(300.035000,-139.586000) -- 
		(300.145000,-139.558000) -- 
		(300.657000,-139.587000) -- 
		(300.955000,-139.604000) -- 
		(301.736000,-139.627000) -- 
		(304.279000,-139.702000) -- 
		(305.843000,-139.877000) -- 
		(307.129000,-139.976000) -- 
		(307.991000,-140.041000) -- 
		(308.546000,-140.080000) -- 
		(308.556000,-140.080000) -- 
		(309.606000,-140.066000) -- 
		(315.512000,-139.987000) -- 
		(315.622000,-139.987000) -- 
		(315.798000,-139.977000) -- 
		(315.974000,-139.968000) -- 
		(316.215000,-139.956000) -- 
		(316.384000,-139.948000) -- 
		(316.553000,-139.939000) -- 
		(316.646000,-139.935000) -- 
		(317.775000,-139.878000) -- 
		(318.904000,-139.822000) -- 
		(321.044000,-139.827000) -- 
		(323.183000,-139.844000) -- 
		(324.829000,-139.858000) -- 
		(327.219000,-139.847000) -- 
		(327.174000,-139.891000) -- 
		(324.799000,-142.091000) -- 
		(322.907000,-142.981000) -- 
		(320.229000,-144.566000) -- 
		(318.243000,-146.061000) -- 
		(317.051000,-147.194000) -- 
		(315.236000,-149.568000) -- 
		(313.648000,-151.620000) -- 
		(311.388000,-153.559000) -- 
		(309.685000,-154.562000) -- 
		(308.133000,-155.274000) -- 
		(306.997000,-155.696000) -- 
		(303.862000,-155.661000) -- 
		(301.778000,-154.994000) -- 
		(299.873000,-154.020000) -- 
		(293.096000,-150.900000) -- 
		(290.329000,-149.970000) -- 
		(287.534000,-149.135000) -- 
		(283.953000,-148.601000) -- 
		(281.662000,-148.822000) -- 
		(280.109000,-149.349000) -- 
		(278.539000,-150.417000) -- 
		(276.620000,-152.260000) -- 
		(274.476000,-154.975000) -- 
		(272.776000,-157.899000) -- 
		(271.704000,-161.636000) -- 
		(271.255000,-164.791000) -- 
		(271.487000,-167.155000) -- 
		(272.087000,-168.614000) -- 
		(273.426000,-170.451000) -- 
		(275.418000,-172.739000) -- 
		(277.667000,-174.817000) -- 
		(278.583000,-175.407000) -- 
		(279.554000,-176.030000) -- 
		(280.820000,-176.604000) -- 
		(282.578000,-177.191000) -- 
		(286.026000,-178.036000) -- 
		(290.015000,-178.884000) -- 
		(293.273000,-179.584000) -- 
		(294.622000,-179.987000) -- 
		(295.821000,-180.618000) -- 
		(297.077000,-181.273000) -- 
		(298.280000,-182.259000) -- 
		(299.102000,-183.288000) -- 
		(299.640000,-184.366000) -- 
		(299.943000,-185.672000) -- 
		(300.200000,-187.212000) -- 
		(300.126000,-188.979000) -- 
		(299.817000,-190.959000) -- 
		(299.309000,-192.995000) -- 
		(298.493000,-194.284000) -- 
		(296.045000,-196.961000) -- 
		(293.961000,-198.547000) -- 
		(292.435000,-199.935000) -- 
		(291.533000,-201.272000) -- 
		(289.970000,-204.195000) -- 
		(289.158000,-206.072000) -- 
		(288.707000,-208.092000) -- 
		(288.506000,-209.175000) -- 
		(287.428000,-212.384000) -- 
		(286.923000,-213.388000) -- 
		(286.306000,-214.602000) -- 
		(285.735000,-215.719000) -- 
		(285.125000,-218.031000) -- 
		(284.373000,-221.683000) -- 
		(283.562000,-225.166000) -- 
		(283.084000,-226.616000) -- 
		(283.004000,-227.256000) -- 
		(283.003000,-227.256000) -- 
		(282.997000,-227.293000) -- 
		(282.988000,-227.362000) -- 
		(282.980000,-227.430000) -- 
		(282.916000,-227.574000) -- 
		(282.721000,-228.008000) -- 
		(282.631000,-228.156000) -- 
		(282.542000,-228.225000) -- 
		(282.404000,-228.332000) -- 
		(282.242000,-228.492000) -- 
		(281.782000,-229.080000) -- 
		(281.651000,-229.236000) -- 
		(281.476000,-229.445000) -- 
		(281.458000,-229.465000) -- 
		(281.069000,-229.888000) -- 
		(280.635000,-230.152000) -- 
		(280.169000,-230.339000) -- 
		(279.962000,-230.405000) -- 
		(279.735000,-230.499000) -- 
		(279.667000,-230.538000) -- 
		(279.599000,-230.576000) -- 
		(279.379000,-230.735000) -- 
		(279.249000,-230.861000) -- 
		(279.236000,-230.949000) -- 
		(279.262000,-231.103000) -- 
		(279.314000,-231.213000) -- 
		(279.573000,-231.412000) -- 
		(280.026000,-231.687000) -- 
		(280.311000,-231.874000) -- 
		(280.376000,-232.077000) -- 
		(280.337000,-232.374000) -- 
		(280.182000,-232.544000) -- 
		(279.793000,-232.709000) -- 
		(279.424000,-232.704000) -- 
		(278.938000,-232.649000) -- 
		(278.588000,-232.693000) -- 
		(278.433000,-232.770000) -- 
		(278.213000,-233.160000) -- 
		(278.096000,-233.782000) -- 
		(278.085000,-233.951000) -- 
		(278.038000,-234.612000) -- 
		(277.895000,-234.931000) -- 
		(277.804000,-235.200000) -- 
		(277.785000,-235.684000) -- 
		(277.785000,-236.075000) -- 
		(277.798000,-236.207000) -- 
		(277.895000,-236.361000) -- 
		(278.089000,-236.366000) -- 
		(278.160000,-236.465000) -- 
		(278.044000,-236.751000) -- 
		(278.018000,-236.889000) -- 
		(278.057000,-237.279000) -- 
		(278.018000,-237.769000) -- 
		(277.985000,-237.879000) -- 
		(277.707000,-238.385000) -- 
		(277.506000,-238.626000) -- 
		(277.305000,-238.775000) -- 
		(277.214000,-238.874000) -- 
		(277.001000,-239.000000) -- 
		(276.424000,-239.215000) -- 
		(275.861000,-239.380000) -- 
		(274.720000,-239.811000) -- 
		(274.274000,-239.979000) -- 
		(273.763000,-240.155000) -- 
		(273.478000,-240.342000) -- 
		(273.374000,-240.540000) -- 
		(273.340000,-240.658000) -- 
		(273.303000,-240.787000) -- 
		(273.381000,-241.062000) -- 
		(273.523000,-241.310000) -- 
		(273.588000,-241.728000) -- 
		(273.678000,-242.283000) -- 
		(273.788000,-242.696000) -- 
		(273.950000,-242.971000) -- 
		(274.047000,-243.108000) -- 
		(274.300000,-243.295000) -- 
		(274.572000,-243.422000) -- 
		(274.993000,-243.515000) -- 
		(275.213000,-243.620000) -- 
		(275.433000,-243.955000) -- 
		(275.433000,-244.197000) -- 
		(275.375000,-244.483000) -- 
		(275.187000,-244.780000) -- 
		(274.993000,-244.940000) -- 
		(274.461000,-245.198000) -- 
		(274.073000,-245.440000) -- 
		(273.736000,-245.819000) -- 
		(273.587000,-246.094000) -- 
		(273.594000,-246.358000) -- 
		(273.645000,-246.738000) -- 
		(273.891000,-247.057000) -- 
		(274.170000,-247.365000) -- 
		(274.254000,-247.458000) -- 
		(274.571000,-247.722000) -- 
		(274.720000,-247.953000) -- 
		(275.070000,-248.360000) -- 
		(275.264000,-248.487000) -- 
		(275.503000,-248.580000) -- 
		(275.873000,-248.553000) -- 
		(276.300000,-248.393000) -- 
		(276.656000,-248.102000) -- 
		(277.032000,-247.700000) -- 
		(277.498000,-247.343000) -- 
		(277.809000,-247.195000) -- 
		(278.029000,-247.189000) -- 
		(278.327000,-247.283000) -- 
		(278.476000,-247.486000) -- 
		(278.495000,-247.767000) -- 
		(278.482000,-248.119000) -- 
		(278.468000,-248.169000) -- 
		(278.359000,-248.520000) -- 
		(278.100000,-248.844000) -- 
		(277.770000,-249.191000) -- 
		(277.290000,-249.559000) -- 
		(276.598000,-249.840000) -- 
		(275.950000,-250.131000) -- 
		(275.704000,-250.362000) -- 
		(275.516000,-250.681000) -- 
		(275.497000,-250.978000) -- 
		(275.587000,-251.324000) -- 
		(275.782000,-251.522000) -- 
		(276.222000,-251.852000) -- 
		(276.841000,-252.014000) -- 
		(278.403000,-252.054000) -- 
		(278.593000,-252.329000) -- 
		(278.621000,-253.015000) -- 
		(278.811000,-253.333000) -- 
		(279.188000,-253.493000) -- 
		(280.724000,-253.670000) -- 
		(281.101000,-253.853000) -- 
		(281.695000,-254.422000) -- 
		(281.643000,-254.994000) -- 
		(281.537000,-255.154000) -- 
		(281.026000,-255.520000) -- 
		(280.273000,-256.188000) -- 
		(279.925000,-256.737000) -- 
		(279.684000,-257.264000) -- 
		(279.526000,-258.087000) -- 
		(279.418000,-258.224000) -- 
		(279.501000,-258.773000) -- 
		(280.014000,-259.320000) -- 
		(280.876000,-259.637000) -- 
		(281.255000,-260.025000) -- 
		(281.311000,-260.367000) -- 
		(281.230000,-260.617000) -- 
		(280.724000,-262.358000) -- 
		(280.673000,-263.227000) -- 
		(281.165000,-265.329000) -- 
		(281.570000,-265.853000) -- 
		(282.540000,-266.421000) -- 
		(283.107000,-266.282000) -- 
		(283.401000,-266.030000) -- 
		(284.312000,-264.747000) -- 
		(284.635000,-264.562000) -- 
		(286.033000,-264.489000) -- 
		(286.194000,-264.397000) -- 
		(286.946000,-263.412000) -- 
		(287.160000,-263.182000) -- 
		(287.617000,-263.089000) -- 
		(288.156000,-263.247000) -- 
		(289.424000,-264.272000) -- 
		(289.882000,-264.453000) -- 
		(290.404000,-264.685000) -- 
		(291.977000,-264.828000) -- 
		(292.552000,-265.233000) -- 
		(292.828000,-265.599000) -- 
		(292.862000,-266.174000) -- 
		(292.758000,-266.544000) -- 
		(292.774000,-267.810000) -- 
		(293.336000,-269.368000) -- 
		(294.164000,-270.693000) -- 
		(294.612000,-271.769000) -- 
		(294.648000,-272.574000) -- 
		(294.489000,-272.829000) -- 
		(294.168000,-273.110000) -- 
		(292.952000,-273.333000) -- 
		(292.332000,-273.663000) -- 
		(291.986000,-274.105000) -- 
		(291.908000,-274.254000) -- 
		(291.616000,-274.824000) -- 
		(291.295000,-277.380000) -- 
		(290.977000,-277.890000) -- 
		(290.968000,-277.896000) -- 
		(290.789000,-278.137000) -- 
		(290.286000,-278.622000) -- 
		(289.834000,-278.976000) -- 
		(289.627000,-279.180000) -- 
		(289.472000,-279.434000) -- 
		(289.402000,-279.802000) -- 
		(289.553000,-280.253000) -- 
		(289.819000,-280.599000) -- 
		(290.207000,-280.884000) -- 
		(290.720000,-281.278000) -- 
		(291.342000,-281.633000) -- 
		(291.699000,-281.731000) -- 
		(292.281000,-281.790000) -- 
		(293.474000,-281.675000) -- 
		(293.620000,-281.660000) -- 
		(293.912000,-281.693000) -- 
		(294.152000,-281.747000) -- 
		(294.443000,-281.916000) -- 
		(294.612000,-282.186000) -- 
		(294.736000,-282.597000) -- 
		(294.692000,-282.922000) -- 
		(294.453000,-283.335000) -- 
		(293.090000,-284.114000) -- 
		(292.683000,-284.418000) -- 
		(292.405000,-284.672000) -- 
		(292.226000,-285.073000) -- 
		(292.215000,-285.301000) -- 
		(292.212000,-285.359000) -- 
		(292.331000,-285.854000) -- 
		(292.597000,-286.299000) -- 
		(292.774000,-286.761000) -- 
		(292.885000,-287.447000) -- 
		(292.906000,-287.843000) -- 
		(292.816000,-288.311000) -- 
		(292.707000,-288.713000) -- 
		(292.470000,-289.505000) -- 
		(292.419000,-289.836000) -- 
		(292.446000,-290.276000) -- 
		(292.506000,-290.874000) -- 
		(292.694000,-291.463000) -- 
		(292.942000,-292.166000) -- 
		(293.093000,-292.770000) -- 
		(293.049000,-293.233000) -- 
		(292.824000,-293.722000) -- 
		(292.611000,-293.932000) -- 
		(292.242000,-294.070000) -- 
		(291.672000,-294.061000) -- 
		(291.155000,-294.034000) -- 
		(290.682000,-294.119000) -- 
		(290.198000,-294.389000) -- 
		(289.927000,-294.616000) -- 
		(289.863000,-295.017000) -- 
		(289.944000,-295.913000) -- 
		(289.944000,-296.458000) -- 
		(289.785000,-297.255000) -- 
		(289.799000,-297.624000) -- 
		(289.994000,-297.909000) -- 
		(290.324000,-298.040000) -- 
		(290.675000,-298.275000) -- 
		(291.128000,-298.511000) -- 
		(291.164000,-298.534000) -- 
		(291.504000,-298.747000) -- 
		(291.751000,-298.966000) -- 
		(291.854000,-299.224000) -- 
		(291.874000,-299.483000) -- 
		(291.759000,-299.631000) -- 
		(291.636000,-299.840000) -- 
		(291.409000,-300.100000) -- 
		(290.989000,-300.480000) -- 
		(290.816000,-300.740000) -- 
		(290.771000,-301.174000) -- 
		(290.675000,-301.570000) -- 
		(290.540000,-301.895000) -- 
		(290.211000,-302.154000) -- 
		(289.254000,-302.421000) -- 
		(287.851000,-302.925000) -- 
		(287.490000,-303.145000) -- 
		(287.473000,-303.289000) -- 
		(286.941000,-303.604000) -- 
		(286.082000,-304.276000) -- 
		(284.621000,-305.141000) -- 
		(284.162000,-305.484000) -- 
		(283.665000,-305.851000) -- 
		(282.799000,-307.244000) -- 
		(282.048000,-308.252000) -- 
		(281.483000,-309.173000) -- 
		(281.147000,-309.868000) -- 
		(280.657000,-310.576000) -- 
		(280.088000,-311.229000) -- 
		(279.516000,-311.637000) -- 
		(279.328000,-311.701000) -- 
		(279.031000,-311.769000) -- 
		(278.503000,-311.677000) -- 
		(278.009000,-311.304000) -- 
		(277.548000,-310.872000) -- 
		(276.996000,-310.492000) -- 
		(276.712000,-310.397000) -- 
		(276.498000,-310.407000) -- 
		(276.030000,-310.436000) -- 
		(275.850000,-310.513000) -- 
		(275.424000,-310.698000) -- 
		(274.485000,-311.068000) -- 
		(273.744000,-311.436000) -- 
		(273.243000,-311.952000) -- 
		(272.836000,-312.765000) -- 
		(272.804000,-312.803000) -- 
		(272.795000,-312.816000) -- 
		(272.608000,-313.065000) -- 
		(272.017000,-313.850000) -- 
		(271.915000,-313.985000) -- 
		(271.810000,-314.103000) -- 
		(271.492000,-314.125000) -- 
		(266.829000,-314.251000) -- 
		(264.781000,-314.132000) -- 
		(264.543000,-314.135000) -- 
		(264.543000,-314.136000) -- 
		(263.421000,-314.165000) -- 
		(263.118000,-314.137000) -- 
		(262.663000,-314.139000) -- 
		(259.814000,-314.126000) -- 
		(251.599000,-314.083000) -- 
		(251.066000,-314.081000) -- 
		(244.300000,-314.043000) -- 
		(244.261000,-314.043000) -- 
		(243.065000,-314.036000) -- 
		(242.067000,-314.030000) -- 
		(238.638000,-313.276000) -- 
		(238.572000,-313.261000) -- 
		(236.662000,-312.841000) -- 
		(233.021000,-312.858000) -- 
		(232.898000,-312.854000) -- 
		(232.280000,-312.835000) -- 
		(231.994000,-312.827000) -- 
		(231.306000,-312.824000) -- 
		(231.106000,-312.823000) -- 
		(230.550000,-312.819000) -- 
		(230.427000,-312.820000) -- 
		(229.838000,-312.816000) -- 
		(229.758000,-312.816000) -- 
		(229.387000,-312.814000) -- 
		(229.165000,-312.814000) -- 
		(228.751000,-312.814000) -- 
		(228.207000,-312.448000) -- 
		(227.864000,-312.209000) -- 
		(227.463000,-311.800000) -- 
		(227.223000,-311.479000) -- 
		(227.031000,-311.133000) -- 
		(226.828000,-310.739000) -- 
		(226.682000,-310.264000) -- 
		(226.678000,-309.788000) -- 
		(226.815000,-309.263000) -- 
		(227.076000,-308.825000) -- 
		(227.648000,-308.103000) -- 
		(227.742000,-307.978000) -- 
		(227.892000,-307.844000) -- 
		(227.933000,-307.807000) -- 
		(228.286000,-307.521000) -- 
		(228.691000,-307.211000) -- 
		(229.209000,-306.878000) -- 
		(229.745000,-306.509000) -- 
		(230.290000,-306.037000) -- 
		(230.883000,-305.686000) -- 
		(231.317000,-305.401000) -- 
		(231.645000,-305.140000) -- 
		(232.192000,-304.902000) -- 
		(232.692000,-304.664000) -- 
		(233.221000,-304.491000) -- 
		(234.004000,-304.171000) -- 
		(234.628000,-304.151000) -- 
		(235.131000,-304.308000) -- 
		(235.690000,-304.264000) -- 
		(236.124000,-304.092000) -- 
		(236.501000,-303.926000) -- 
		(237.000000,-303.568000) -- 
		(237.524000,-302.983000) -- 
		(237.843000,-302.497000) -- 
		(238.019000,-302.092000) -- 
		(238.119000,-301.656000) -- 
		(238.208000,-301.067000) -- 
		(238.129000,-300.518000) -- 
		(237.966000,-299.851000) -- 
		(237.897000,-299.556000) -- 
		(237.734000,-299.283000) -- 
		(237.486000,-299.043000) -- 
		(237.098000,-298.891000) -- 
		(236.681000,-298.861000) -- 
		(236.284000,-298.920000) -- 
		(235.811000,-299.060000) -- 
		(235.339000,-299.272000) -- 
		(234.962000,-299.476000) -- 
		(234.718000,-299.623000) -- 
		(234.596000,-299.849000) -- 
		(234.065000,-299.972000) -- 
		(233.396000,-300.356000) -- 
		(232.793000,-300.650000) -- 
		(232.069000,-301.002000) -- 
		(230.992000,-301.292000) -- 
		(230.162000,-301.475000) -- 
		(229.311000,-301.577000) -- 
		(228.753000,-301.573000) -- 
		(227.956000,-301.449000) -- 
		(227.141000,-301.213000) -- 
		(226.276000,-300.839000) -- 
		(225.316000,-300.265000) -- 
		(224.731000,-299.846000) -- 
		(224.159000,-299.288000) -- 
		(223.824000,-298.846000) -- 
		(223.089000,-297.858000) -- 
		(222.637000,-296.980000) -- 
		(222.101000,-296.055000) -- 
		(221.746000,-295.484000) -- 
		(221.593000,-295.267000) -- 
		(221.402000,-295.027000) -- 
		(221.250000,-294.858000) -- 
		(220.822000,-294.585000) -- 
		(220.004000,-294.154000) -- 
		(219.101000,-293.772000) -- 
		(217.413000,-293.364000) -- 
		(216.625000,-293.125000) -- 
		(215.773000,-293.033000) -- 
		(215.239000,-292.975000) -- 
		(214.417000,-292.873000) -- 
		(213.950000,-292.815000) -- 
		(213.665000,-293.030000) -- 
		(213.402000,-293.418000) -- 
		(213.112000,-294.002000) -- 
		(213.066000,-294.504000) -- 
		(213.068000,-294.917000) -- 
		(213.081000,-295.370000) -- 
		(213.256000,-296.016000) -- 
		(213.515000,-296.605000) -- 
		(213.882000,-297.479000) -- 
		(214.458000,-298.762000) -- 
		(215.109000,-299.836000) -- 
		(215.188000,-300.093000) -- 
		(215.540000,-300.348000) -- 
		(216.122000,-300.661000) -- 
		(216.854000,-300.931000) -- 
		(217.530000,-301.260000) -- 
		(218.111000,-301.555000) -- 
		(218.883000,-302.094000) -- 
		(219.388000,-302.496000) -- 
		(220.152000,-303.057000) -- 
		(220.647000,-303.362000) -- 
		(220.973000,-303.739000) -- 
		(221.460000,-304.247000) -- 
		(221.930000,-304.915000) -- 
		(222.314000,-305.455000) -- 
		(222.535000,-305.923000) -- 
		(222.548000,-306.358000) -- 
		(222.402000,-307.063000) -- 
		(221.901000,-307.527000) -- 
		(220.862000,-307.881000) -- 
		(219.657000,-307.927000) -- 
		(218.963000,-307.703000) -- 
		(218.003000,-307.393000) -- 
		(217.167000,-307.235000) -- 
		(216.323000,-307.118000) -- 
		(215.213000,-307.044000) -- 
		(213.751000,-306.898000) -- 
		(213.301000,-306.842000) -- 
		(213.185000,-306.829000) -- 
		(212.727000,-306.775000) -- 
		(212.167000,-306.718000) -- 
		(211.657000,-306.668000) -- 
		(211.104000,-306.636000) -- 
		(210.700000,-306.611000) -- 
		(210.562000,-306.581000) -- 
		(209.780000,-306.415000) -- 
		(209.321000,-305.927000) -- 
		(209.108000,-305.354000) -- 
		(209.140000,-304.773000) -- 
		(209.378000,-303.909000) -- 
		(209.512000,-303.052000) -- 
		(209.739000,-302.074000) -- 
		(209.819000,-301.517000) -- 
		(209.869000,-300.743000) -- 
		(209.770000,-300.267000) -- 
		(209.213000,-299.843000) -- 
		(207.560000,-295.973000) -- 
		(207.176000,-292.993000) -- 
		(207.134000,-292.666000) -- 
		(207.016000,-291.746000) -- 
		(207.071000,-291.608000) -- 
		(207.214000,-291.248000) -- 
		(207.357000,-290.890000) -- 
		(207.512000,-290.502000) -- 
		(208.716000,-287.504000) -- 
		(207.796000,-284.418000) -- 
		(206.853000,-283.978000) -- 
		(206.848000,-283.976000) -- 
		(205.513000,-283.387000) -- 
		(205.489000,-283.376000) -- 
		(204.191000,-282.799000);
	\filldraw [draw=black, ultra thick, fill=gold]
		(96.933000,27.305000) -- 
		(96.896000,27.197000) -- 
		(96.795000,26.902000) -- 
		(96.750000,26.798000) -- 
		(96.750000,26.797000) -- 
		(96.499000,26.220000) -- 
		(96.390000,26.000000) -- 
		(96.275000,25.767000) -- 
		(96.154000,25.522000) -- 
		(95.826000,24.835000) -- 
		(95.639000,24.448000) -- 
		(95.431000,24.068000) -- 
		(95.343000,23.907000) -- 
		(94.814000,23.128000) -- 
		(94.772000,23.078000) -- 
		(94.626000,22.902000) -- 
		(94.193000,22.389000) -- 
		(94.138000,22.319000) -- 
		(93.902000,22.086000) -- 
		(93.689000,21.877000) -- 
		(93.618000,21.808000) -- 
		(93.381000,21.578000) -- 
		(93.032000,21.242000) -- 
		(92.863000,21.009000) -- 
		(92.747000,20.822000) -- 
		(92.570000,20.544000) -- 
		(92.483000,20.377000) -- 
		(92.229000,19.895000) -- 
		(92.223000,19.879000) -- 
		(92.152000,19.706000) -- 
		(92.105000,19.590000) -- 
		(91.980000,19.189000) -- 
		(91.898000,18.929000) -- 
		(91.815000,18.577000) -- 
		(91.805000,18.464000) -- 
		(91.807000,18.241000) -- 
		(91.878000,17.677000) -- 
		(91.893000,17.585000) -- 
		(91.980000,17.048000) -- 
		(91.986000,17.009000) -- 
		(92.003000,16.893000) -- 
		(92.008000,16.856000) -- 
		(92.013000,16.826000) -- 
		(92.026000,16.737000) -- 
		(92.030000,16.708000) -- 
		(91.976000,16.679000) -- 
		(91.812000,16.592000) -- 
		(91.757000,16.565000) -- 
		(91.269000,16.308000) -- 
		(91.103000,16.223000) -- 
		(89.787000,15.571000) -- 
		(89.739000,15.548000) -- 
		(89.294000,15.322000) -- 
		(88.883000,15.113000) -- 
		(88.461000,14.898000) -- 
		(85.959000,13.626000) -- 
		(85.125000,13.203000) -- 
		(84.545000,12.902000) -- 
		(82.803000,11.999000) -- 
		(82.520000,11.853000) -- 
		(82.218000,11.707000) -- 
		(81.849000,11.528000) -- 
		(80.860000,11.125000) -- 
		(79.771000,10.696000) -- 
		(79.233000,10.528000) -- 
		(79.022000,10.467000) -- 
		(78.196000,10.229000) -- 
		(77.272000,10.007000) -- 
		(76.372000,9.861000) -- 
		(75.017000,9.616000) -- 
		(73.347000,9.289000) -- 
		(70.764000,8.875000) -- 
		(68.910000,8.571000) -- 
		(67.268000,8.303000) -- 
		(65.661000,8.021000) -- 
		(65.578000,8.007000) -- 
		(65.526000,7.998000) -- 
		(65.484000,8.197000) -- 
		(65.483000,8.233000) -- 
		(65.505000,8.374000) -- 
		(65.523000,8.624000) -- 
		(65.550000,8.806000) -- 
		(65.619000,8.996000) -- 
		(65.635000,9.029000) -- 
		(65.660000,9.058000) -- 
		(65.803000,9.190000) -- 
		(65.990000,9.384000) -- 
		(66.149000,9.504000) -- 
		(66.411000,9.686000) -- 
		(66.513000,9.749000) -- 
		(66.584000,9.788000) -- 
		(66.698000,9.858000) -- 
		(66.824000,9.935000) -- 
		(66.971000,10.007000) -- 
		(67.042000,10.047000) -- 
		(67.193000,10.113000) -- 
		(67.439000,10.197000) -- 
		(67.587000,10.248000) -- 
		(67.749000,10.294000) -- 
		(67.948000,10.356000) -- 
		(68.193000,10.419000) -- 
		(68.554000,10.525000) -- 
		(69.160000,10.695000) -- 
		(69.394000,10.751000) -- 
		(70.193000,10.943000) -- 
		(70.306000,10.970000) -- 
		(70.463000,11.021000) -- 
		(70.514000,11.035000) -- 
		(70.584000,11.053000) -- 
		(70.831000,11.108000) -- 
		(70.997000,11.132000) -- 
		(71.210000,11.163000) -- 
		(71.840000,11.267000) -- 
		(71.948000,11.297000) -- 
		(72.244000,11.379000) -- 
		(72.271000,11.389000) -- 
		(72.558000,11.494000) -- 
		(72.595000,11.513000) -- 
		(72.792000,11.610000) -- 
		(72.856000,11.657000) -- 
		(72.961000,11.770000) -- 
		(73.103000,11.990000) -- 
		(73.119000,12.023000) -- 
		(73.134000,12.094000) -- 
		(73.142000,12.146000) -- 
		(73.170000,12.343000) -- 
		(73.167000,12.550000) -- 
		(73.165000,12.664000) -- 
		(73.160000,12.735000) -- 
		(73.147000,12.806000) -- 
		(73.121000,12.911000) -- 
		(73.109000,12.951000) -- 
		(73.060000,13.119000) -- 
		(73.045000,13.152000) -- 
		(72.942000,13.326000) -- 
		(72.931000,13.344000) -- 
		(72.882000,13.402000) -- 
		(72.763000,13.505000) -- 
		(72.594000,13.615000) -- 
		(72.557000,13.633000) -- 
		(72.436000,13.667000) -- 
		(72.383000,13.676000) -- 
		(72.310000,13.688000) -- 
		(72.267000,13.687000) -- 
		(71.977000,13.624000) -- 
		(71.866000,13.571000) -- 
		(71.800000,13.526000) -- 
		(71.769000,13.500000) -- 
		(71.685000,13.409000) -- 
		(71.613000,13.330000) -- 
		(71.570000,13.269000) -- 
		(71.482000,13.143000) -- 
		(71.461000,13.114000) -- 
		(71.349000,12.963000) -- 
		(71.328000,12.942000) -- 
		(71.294000,12.908000) -- 
		(71.102000,12.767000) -- 
		(71.025000,12.735000) -- 
		(70.903000,12.703000) -- 
		(70.819000,12.691000) -- 
		(70.691000,12.692000) -- 
		(70.620000,12.695000) -- 
		(70.564000,12.698000) -- 
		(70.479000,12.708000) -- 
		(70.151000,12.783000) -- 
		(69.787000,12.881000) -- 
		(69.707000,12.906000) -- 
		(69.419000,12.967000) -- 
		(69.216000,13.019000) -- 
		(69.053000,13.065000) -- 
		(68.934000,13.103000) -- 
		(68.660000,13.197000) -- 
		(68.537000,13.240000) -- 
		(68.337000,13.301000) -- 
		(68.277000,13.323000) -- 
		(68.023000,13.415000) -- 
		(67.907000,13.462000) -- 
		(67.761000,13.530000) -- 
		(67.646000,13.583000) -- 
		(67.355000,13.734000) -- 
		(67.285000,13.774000) -- 
		(67.230000,13.811000) -- 
		(67.086000,13.909000) -- 
		(66.901000,14.061000) -- 
		(66.842000,14.109000) -- 
		(66.646000,14.298000) -- 
		(66.593000,14.354000) -- 
		(66.520000,14.441000) -- 
		(66.481000,14.511000) -- 
		(66.417000,14.627000) -- 
		(66.413000,14.635000) -- 
		(66.367000,14.773000) -- 
		(66.339000,14.840000) -- 
		(66.330000,14.875000) -- 
		(66.316000,14.981000) -- 
		(66.308000,15.158000) -- 
		(66.295000,15.300000) -- 
		(66.305000,15.514000) -- 
		(66.327000,15.656000) -- 
		(66.488000,16.063000) -- 
		(66.566000,16.191000) -- 
		(66.653000,16.314000) -- 
		(66.680000,16.376000) -- 
		(66.785000,16.572000) -- 
		(67.068000,17.012000) -- 
		(67.137000,17.143000) -- 
		(67.176000,17.282000) -- 
		(67.188000,17.353000) -- 
		(67.194000,17.460000) -- 
		(67.181000,17.531000) -- 
		(67.150000,17.635000) -- 
		(67.114000,17.699000) -- 
		(67.101000,17.733000) -- 
		(67.021000,17.859000) -- 
		(66.919000,17.974000) -- 
		(66.863000,18.027000) -- 
		(66.811000,18.083000) -- 
		(66.615000,18.270000) -- 
		(66.591000,18.299000) -- 
		(66.441000,18.425000) -- 
		(66.344000,18.495000) -- 
		(66.243000,18.561000) -- 
		(66.179000,18.608000) -- 
		(65.996000,18.757000) -- 
		(65.939000,18.809000) -- 
		(65.788000,18.933000) -- 
		(65.676000,19.038000) -- 
		(65.625000,19.095000) -- 
		(65.390000,19.393000) -- 
		(65.371000,19.425000) -- 
		(65.320000,19.643000) -- 
		(65.320000,19.715000) -- 
		(65.337000,19.812000) -- 
		(65.362000,19.962000) -- 
		(65.391000,20.029000) -- 
		(65.465000,20.158000) -- 
		(65.564000,20.274000) -- 
		(65.632000,20.316000) -- 
		(65.829000,20.454000) -- 
		(65.897000,20.496000) -- 
		(65.937000,20.510000) -- 
		(66.011000,20.545000) -- 
		(66.163000,20.608000) -- 
		(66.284000,20.641000) -- 
		(66.575000,20.713000) -- 
		(66.692000,20.742000) -- 
		(66.930000,20.822000) -- 
		(66.967000,20.839000) -- 
		(67.179000,20.961000) -- 
		(67.455000,21.185000) -- 
		(67.599000,21.303000) -- 
		(67.878000,21.537000) -- 
		(68.161000,21.805000) -- 
		(68.311000,21.976000) -- 
		(68.354000,22.038000) -- 
		(68.394000,22.139000) -- 
		(68.418000,22.244000) -- 
		(68.415000,22.279000) -- 
		(68.371000,22.454000) -- 
		(68.329000,22.555000) -- 
		(68.260000,22.685000) -- 
		(68.210000,22.743000) -- 
		(68.048000,22.951000) -- 
		(67.911000,23.087000) -- 
		(67.870000,23.121000) -- 
		(67.729000,23.236000) -- 
		(67.478000,23.476000) -- 
		(67.361000,23.625000) -- 
		(67.289000,23.754000) -- 
		(67.268000,23.785000) -- 
		(67.256000,23.820000) -- 
		(67.228000,23.996000) -- 
		(67.225000,24.104000) -- 
		(67.230000,24.140000) -- 
		(67.244000,24.246000) -- 
		(67.268000,24.314000) -- 
		(67.287000,24.384000) -- 
		(67.349000,24.554000) -- 
		(67.372000,24.601000) -- 
		(67.438000,24.735000) -- 
		(67.494000,24.849000) -- 
		(67.530000,24.914000) -- 
		(67.543000,24.948000) -- 
		(67.644000,25.105000) -- 
		(67.743000,25.206000) -- 
		(67.780000,25.241000) -- 
		(67.811000,25.267000) -- 
		(67.865000,25.321000) -- 
		(68.049000,25.469000) -- 
		(68.100000,25.525000) -- 
		(68.217000,25.674000) -- 
		(68.372000,25.843000) -- 
		(68.530000,26.055000) -- 
		(68.605000,26.182000) -- 
		(68.654000,26.281000) -- 
		(68.683000,26.348000) -- 
		(68.741000,26.450000) -- 
		(68.793000,26.541000) -- 
		(68.936000,26.718000) -- 
		(69.042000,26.824000) -- 
		(69.127000,26.908000) -- 
		(69.327000,27.093000) -- 
		(69.413000,27.172000) -- 
		(69.495000,27.254000) -- 
		(69.657000,27.462000) -- 
		(69.721000,27.554000) -- 
		(69.796000,27.681000) -- 
		(69.871000,27.850000) -- 
		(69.983000,28.192000) -- 
		(70.016000,28.258000) -- 
		(70.037000,28.289000) -- 
		(70.189000,28.414000) -- 
		(70.259000,28.454000) -- 
		(70.300000,28.462000) -- 
		(70.555000,28.470000) -- 
		(70.598000,28.469000) -- 
		(70.751000,28.444000) -- 
		(70.847000,28.429000) -- 
		(70.928000,28.410000) -- 
		(71.052000,28.390000) -- 
		(71.178000,28.376000) -- 
		(71.262000,28.381000) -- 
		(71.345000,28.391000) -- 
		(71.386000,28.401000) -- 
		(71.542000,28.458000) -- 
		(71.728000,28.543000) -- 
		(71.995000,28.720000) -- 
		(72.048000,28.775000) -- 
		(72.079000,28.800000) -- 
		(72.103000,28.829000) -- 
		(72.138000,28.894000) -- 
		(72.196000,29.029000) -- 
		(72.218000,29.098000) -- 
		(72.232000,29.168000) -- 
		(72.252000,29.417000) -- 
		(72.253000,29.524000) -- 
		(72.241000,29.810000) -- 
		(72.243000,29.953000) -- 
		(72.326000,30.267000) -- 
		(72.421000,30.466000) -- 
		(72.513000,30.628000) -- 
		(72.579000,30.720000) -- 
		(72.677000,30.835000) -- 
		(72.696000,30.868000) -- 
		(72.931000,31.202000) -- 
		(73.018000,31.363000) -- 
		(73.110000,31.601000) -- 
		(73.118000,31.636000) -- 
		(73.114000,31.780000) -- 
		(73.102000,31.886000) -- 
		(73.077000,31.991000) -- 
		(73.016000,32.163000) -- 
		(72.986000,32.229000) -- 
		(72.808000,32.475000) -- 
		(72.645000,32.640000) -- 
		(72.441000,32.824000) -- 
		(72.357000,32.905000) -- 
		(72.237000,33.008000) -- 
		(72.011000,33.221000) -- 
		(71.899000,33.374000) -- 
		(71.850000,33.432000) -- 
		(71.833000,33.465000) -- 
		(71.812000,33.535000) -- 
		(71.804000,33.713000) -- 
		(71.866000,33.845000) -- 
		(71.917000,33.903000) -- 
		(71.977000,33.953000) -- 
		(72.041000,33.999000) -- 
		(72.290000,34.137000) -- 
		(72.415000,34.235000) -- 
		(72.441000,34.262000) -- 
		(72.481000,34.325000) -- 
		(72.530000,34.607000) -- 
		(72.541000,34.714000) -- 
		(72.543000,34.821000) -- 
		(72.556000,34.963000) -- 
		(72.544000,35.029000) -- 
		(72.402000,35.163000) -- 
		(72.506000,35.217000) -- 
		(72.571000,35.250000) -- 
		(72.788000,35.534000) -- 
		(72.983000,35.967000) -- 
		(72.993000,36.146000) -- 
		(72.963000,36.296000) -- 
		(72.916000,36.330000) -- 
		(72.879000,36.437000) -- 
		(72.551000,36.786000) -- 
		(72.432000,37.332000) -- 
		(72.808000,37.897000) -- 
		(73.092000,38.149000) -- 
		(73.265000,38.355000) -- 
		(73.373000,38.536000) -- 
		(73.360000,38.706000) -- 
		(73.229000,38.884000) -- 
		(73.133000,39.003000) -- 
		(73.032000,39.127000) -- 
		(72.688000,39.398000) -- 
		(72.547000,39.562000) -- 
		(72.139000,40.322000) -- 
		(71.962000,40.893000) -- 
		(72.016000,41.044000) -- 
		(71.990000,41.162000) -- 
		(72.188000,41.385000) -- 
		(72.263000,41.504000) -- 
		(72.452000,41.571000) -- 
		(72.791000,41.610000) -- 
		(73.802000,41.410000) -- 
		(74.734000,41.241000) -- 
		(75.100000,41.191000) -- 
		(75.490000,41.177000) -- 
		(76.052000,41.220000) -- 
		(76.474000,41.309000) -- 
		(76.842000,41.507000) -- 
		(77.118000,42.180000) -- 
		(77.072000,42.321000) -- 
		(76.883000,42.660000) -- 
		(76.354000,43.236000) -- 
		(76.132000,43.634000) -- 
		(76.041000,44.033000) -- 
		(76.110000,44.223000) -- 
		(76.293000,44.403000) -- 
		(76.858000,44.709000) -- 
		(77.408000,44.731000) -- 
		(77.779000,44.735000) -- 
		(78.029000,44.682000) -- 
		(78.587000,44.518000) -- 
		(79.105000,44.457000) -- 
		(79.490000,44.482000) -- 
		(79.755000,44.647000) -- 
		(80.046000,45.011000) -- 
		(80.627000,45.956000) -- 
		(80.857000,46.635000) -- 
		(80.979000,46.825000) -- 
		(81.192000,46.929000) -- 
		(81.140000,46.994000) -- 
		(80.506000,47.588000) -- 
		(79.871000,49.775000) -- 
		(79.661000,53.368000) -- 
		(79.671000,56.019000) -- 
		(79.304000,58.925000) -- 
		(77.909000,61.800000) -- 
		(76.422000,63.844000) -- 
		(75.141000,65.312000) -- 
		(74.274000,65.231000) -- 
		(74.093000,65.311000) -- 
		(73.783000,65.326000) -- 
		(73.471000,65.390000) -- 
		(73.302000,65.709000) -- 
		(72.993000,66.443000) -- 
		(72.691000,66.922000) -- 
		(72.459000,67.098000) -- 
		(71.969000,67.177000) -- 
		(71.754000,67.353000) -- 
		(71.251000,69.806000) -- 
		(71.132000,70.119000) -- 
		(68.888000,75.993000) -- 
		(67.736000,77.903000) -- 
		(66.268000,80.335000) -- 
		(64.616000,83.072000) -- 
		(63.806000,83.854000) -- 
		(57.582000,89.866000) -- 
		(53.937000,92.184000) -- 
		(50.288000,94.505000) -- 
		(51.229000,95.785000) -- 
		(50.795000,96.017000) -- 
		(49.450000,96.655000) -- 
		(47.073000,97.473000) -- 
		(46.928000,97.626000) -- 
		(46.893000,97.795000) -- 
		(46.894000,97.974000) -- 
		(46.835000,99.184000) -- 
		(46.999000,100.809000) -- 
		(47.317000,101.789000) -- 
		(47.388000,102.010000) -- 
		(47.338000,102.248000) -- 
		(47.301000,102.236000) -- 
		(47.187000,102.203000) -- 
		(47.175000,102.201000) -- 
		(47.148000,102.195000) -- 
		(47.124000,102.250000) -- 
		(47.068000,102.334000) -- 
		(46.916000,102.564000) -- 
		(46.717000,102.631000) -- 
		(46.623000,102.663000) -- 
		(46.568000,102.694000) -- 
		(46.506000,102.729000) -- 
		(46.363000,102.910000) -- 
		(46.298000,103.169000) -- 
		(46.344000,103.213000) -- 
		(46.383000,103.218000) -- 
		(46.344000,103.306000) -- 
		(46.311000,103.358000) -- 
		(46.136000,103.301000) -- 
		(46.032000,103.378000) -- 
		(45.948000,103.394000) -- 
		(44.300000,102.906000) -- 
		(41.753000,102.153000) -- 
		(39.986000,101.630000) -- 
		(38.218000,101.108000) -- 
		(37.886000,101.012000) -- 
		(37.555000,100.916000) -- 
		(36.613000,100.619000) -- 
		(36.320000,100.614000) -- 
		(35.800000,100.713000) -- 
		(34.845000,100.895000) -- 
		(34.780000,100.928000) -- 
		(34.520000,101.258000) -- 
		(34.150000,101.511000) -- 
		(33.825000,101.676000) -- 
		(33.192000,101.896000) -- 
		(33.103000,103.135000) -- 
		(32.482000,105.422000) -- 
		(32.075000,106.254000) -- 
		(31.610000,107.204000) -- 
		(30.953000,109.199000) -- 
		(30.264000,112.577000) -- 
		(30.198000,112.760000) -- 
		(29.205000,115.485000) -- 
		(28.513000,117.039000) -- 
		(27.613000,119.062000) -- 
		(27.072000,122.314000) -- 
		(28.249000,125.254000) -- 
		(29.074000,126.566000) -- 
		(30.298000,127.156000) -- 
		(32.778000,128.754000) -- 
		(34.902000,130.027000) -- 
		(35.871000,130.595000) -- 
		(35.957000,130.645000) -- 
		(36.043000,130.695000) -- 
		(36.870000,131.180000) -- 
		(38.852000,132.655000) -- 
		(40.093000,133.713000) -- 
		(41.241000,135.303000) -- 
		(41.542000,136.196000) -- 
		(41.719000,136.721000) -- 
		(41.786000,137.074000) -- 
		(41.925000,137.811000) -- 
		(40.733000,138.643000) -- 
		(40.658000,138.696000) -- 
		(39.920000,139.426000) -- 
		(39.190000,140.213000) -- 
		(38.438000,141.257000) -- 
		(37.678000,142.156000) -- 
		(37.269000,142.666000) -- 
		(36.617000,143.430000) -- 
		(36.124000,144.476000) -- 
		(35.728000,145.372000) -- 
		(35.496000,146.277000) -- 
		(35.349000,147.254000) -- 
		(35.353000,148.210000) -- 
		(35.397000,148.483000) -- 
		(35.928000,150.975000) -- 
		(36.081000,151.696000) -- 
		(37.638000,156.074000) -- 
		(37.892000,156.789000) -- 
		(38.464000,157.354000) -- 
		(38.623000,158.071000) -- 
		(38.442000,158.846000) -- 
		(38.370000,159.153000) -- 
		(38.661000,160.355000) -- 
		(39.082000,161.215000) -- 
		(39.315000,161.430000) -- 
		(39.328000,161.469000) -- 
		(39.464000,161.634000) -- 
		(39.536000,161.771000) -- 
		(39.569000,162.057000) -- 
		(39.595000,162.206000) -- 
		(39.650000,162.320000) -- 
		(39.888000,162.821000) -- 
		(39.894000,162.997000) -- 
		(39.829000,163.256000) -- 
		(39.738000,163.382000) -- 
		(39.569000,163.635000) -- 
		(39.556000,163.916000) -- 
		(39.543000,164.878000) -- 
		(39.585000,165.099000) -- 
		(39.589000,165.126000) -- 
		(39.556000,165.230000) -- 
		(39.524000,165.714000) -- 
		(39.472000,165.934000) -- 
		(39.455000,166.061000) -- 
		(39.914000,166.127000) -- 
		(41.155000,166.313000) -- 
		(41.288000,166.330000) -- 
		(41.746000,166.396000) -- 
		(43.003000,166.569000) -- 
		(44.552000,166.773000) -- 
		(44.997000,166.805000) -- 
		(45.652000,166.842000) -- 
		(46.322000,166.875000) -- 
		(47.437000,166.914000) -- 
		(47.969000,166.919000) -- 
		(49.003000,166.930000) -- 
		(50.397000,166.965000) -- 
		(50.792000,166.974000) -- 
		(51.788000,166.997000) -- 
		(51.788000,166.998000) -- 
		(53.016000,167.014000) -- 
		(55.373000,167.069000) -- 
		(55.751000,167.078000) -- 
		(55.801000,167.094000) -- 
		(55.818000,167.098000) -- 
		(55.885000,167.080000) -- 
		(58.536000,167.125000) -- 
		(61.186000,167.154000) -- 
		(62.511000,167.188000) -- 
		(63.405000,167.214000) -- 
		(64.342000,167.246000) -- 
		(64.797000,167.261000) -- 
		(64.821000,167.262000) -- 
		(66.162000,167.284000) -- 
		(66.616000,167.293000) -- 
		(67.009000,167.300000) -- 
		(68.178000,167.323000) -- 
		(68.818000,167.339000) -- 
		(70.552000,167.383000) -- 
		(72.223000,167.413000) -- 
		(72.223000,167.414000) -- 
		(72.528000,167.419000) -- 
		(72.528000,167.420000) -- 
		(72.877000,167.429000) -- 
		(73.196000,167.436000) -- 
		(73.196000,167.437000) -- 
		(74.724000,167.476000) -- 
		(76.970000,167.550000) -- 
		(77.472000,167.544000) -- 
		(78.969000,167.433000) -- 
		(79.801000,167.321000) -- 
		(80.704000,167.199000) -- 
		(84.450000,166.567000) -- 
		(84.842000,166.509000) -- 
		(84.903000,166.500000) -- 
		(85.085000,166.473000) -- 
		(85.145000,166.465000) -- 
		(86.823000,166.216000) -- 
		(87.703000,166.081000) -- 
		(91.439000,165.511000) -- 
		(92.819000,165.275000) -- 
		(93.406000,165.186000) -- 
		(94.005000,165.095000) -- 
		(95.055000,164.917000) -- 
		(95.207000,164.897000) -- 
		(95.365000,164.877000) -- 
		(95.962000,164.803000) -- 
		(96.607000,164.768000) -- 
		(97.086000,164.804000) -- 
		(97.546000,164.863000) -- 
		(97.933000,164.935000) -- 
		(98.663000,165.073000) -- 
		(98.855000,165.122000) -- 
		(99.416000,165.265000) -- 
		(99.915000,165.347000) -- 
		(100.561000,165.473000) -- 
		(101.203000,165.510000) -- 
		(102.025000,165.501000) -- 
		(102.273000,165.481000) -- 
		(102.274000,165.480000) -- 
		(102.450000,165.467000) -- 
		(103.219000,165.317000) -- 
		(103.360000,165.290000) -- 
		(104.321000,165.066000) -- 
		(104.877000,164.952000) -- 
		(104.878000,164.951000) -- 
		(105.416000,164.841000) -- 
		(106.104000,164.680000) -- 
		(106.698000,164.511000) -- 
		(106.692000,164.560000) -- 
		(106.722000,164.979000) -- 
		(106.722000,164.980000) -- 
		(106.837000,166.493000) -- 
		(106.868000,166.914000) -- 
		(106.898000,167.501000) -- 
		(106.981000,168.687000) -- 
		(107.021000,169.364000) -- 
		(107.037000,169.890000) -- 
		(107.118000,171.037000) -- 
		(107.194000,172.445000) -- 
		(107.215000,172.840000) -- 
		(107.305000,173.289000) -- 
		(107.412000,173.652000) -- 
		(107.624000,174.168000) -- 
		(107.733000,174.334000) -- 
		(107.826000,174.475000) -- 
		(108.105000,174.812000) -- 
		(108.362000,175.053000) -- 
		(108.464000,175.157000) -- 
		(108.635000,175.332000) -- 
		(109.605000,176.115000) -- 
		(109.697000,176.191000) -- 
		(111.132000,177.434000) -- 
		(111.424000,177.728000) -- 
		(111.714000,178.163000) -- 
		(111.921000,178.682000) -- 
		(112.134000,179.250000) -- 
		(112.365000,179.919000) -- 
		(112.404000,180.031000) -- 
		(112.601000,180.506000) -- 
		(112.872000,181.113000) -- 
		(112.872000,181.114000) -- 
		(113.081000,181.626000) -- 
		(113.256000,182.028000) -- 
		(113.502000,182.498000) -- 
		(113.502000,182.499000) -- 
		(113.581000,182.613000) -- 
		(113.702000,182.792000) -- 
		(113.952000,183.110000) -- 
		(114.134000,183.321000) -- 
		(114.326000,183.493000) -- 
		(114.479000,183.630000) -- 
		(115.096000,184.070000) -- 
		(115.499000,184.325000) -- 
		(115.658000,184.425000) -- 
		(115.794000,184.512000) -- 
		(116.137000,184.721000) -- 
		(116.296000,184.819000) -- 
		(116.558000,184.939000) -- 
		(117.198000,185.133000) -- 
		(117.776000,185.177000) -- 
		(118.340000,185.203000) -- 
		(120.356000,185.221000) -- 
		(120.660000,185.222000) -- 
		(121.099000,185.227000) -- 
		(122.931000,185.254000) -- 
		(124.151000,185.248000) -- 
		(125.768000,185.288000) -- 
		(125.907000,185.315000) -- 
		(125.907000,185.316000) -- 
		(126.002000,185.335000) -- 
		(126.299000,185.439000) -- 
		(126.922000,185.690000) -- 
		(128.003000,186.256000) -- 
		(129.157000,186.931000) -- 
		(130.035000,187.462000) -- 
		(131.020000,188.011000) -- 
		(131.686000,188.414000) -- 
		(132.241000,188.719000) -- 
		(132.678000,188.996000) -- 
		(132.927000,189.144000) -- 
		(133.219000,189.319000) -- 
		(135.840000,190.842000) -- 
		(136.739000,191.383000) -- 
		(137.468000,191.822000) -- 
		(137.490000,191.833000) -- 
		(138.466000,192.383000) -- 
		(139.765000,193.145000) -- 
		(140.519000,193.589000) -- 
		(140.934000,193.832000) -- 
		(141.889000,194.436000) -- 
		(142.018000,194.519000) -- 
		(143.279000,195.225000) -- 
		(144.387000,195.874000) -- 
		(145.962000,196.795000) -- 
		(146.070000,196.854000) -- 
		(147.122000,197.447000) -- 
		(147.472000,197.648000) -- 
		(148.310000,198.151000) -- 
		(149.408000,198.813000) -- 
		(149.919000,199.158000) -- 
		(150.343000,199.523000) -- 
		(150.604000,199.917000) -- 
		(150.651000,199.990000) -- 
		(150.748000,200.182000) -- 
		(150.858000,200.460000) -- 
		(150.979000,200.816000) -- 
		(150.998000,200.998000) -- 
		(151.028000,201.298000) -- 
		(151.028000,201.548000) -- 
		(151.027000,201.732000) -- 
		(151.029000,202.057000) -- 
		(151.036000,203.039000) -- 
		(151.037000,203.367000) -- 
		(151.043000,203.861000) -- 
		(151.051000,204.414000) -- 
		(151.056000,204.767000) -- 
		(151.054000,205.660000) -- 
		(151.064000,207.341000) -- 
		(151.069000,208.020000) -- 
		(151.081000,209.886000) -- 
		(151.082000,210.059000) -- 
		(151.083000,210.739000) -- 
		(151.084000,211.121000) -- 
		(151.084000,211.253000) -- 
		(151.105000,212.016000) -- 
		(151.120000,212.269000) -- 
		(151.199000,212.644000) -- 
		(151.285000,213.056000) -- 
		(151.318000,213.217000) -- 
		(151.392000,213.769000) -- 
		(151.397000,214.063000) -- 
		(151.394000,214.313000) -- 
		(151.391000,214.482000) -- 
		(151.361000,214.733000) -- 
		(151.327000,215.013000) -- 
		(151.277000,215.313000) -- 
		(151.237000,215.549000) -- 
		(151.100000,216.102000) -- 
		(150.897000,216.593000) -- 
		(150.842000,216.697000) -- 
		(150.841000,216.698000) -- 
		(150.698000,216.968000) -- 
		(150.677000,217.006000) -- 
		(150.414000,217.378000) -- 
		(150.355000,217.445000) -- 
		(149.914000,217.947000) -- 
		(149.515000,218.403000) -- 
		(149.297000,218.650000) -- 
		(148.916000,219.083000) -- 
		(148.726000,219.335000) -- 
		(148.632000,219.492000) -- 
		(148.565000,219.603000) -- 
		(148.419000,219.945000) -- 
		(148.360000,220.103000) -- 
		(148.314000,220.227000) -- 
		(148.193000,220.623000) -- 
		(148.059000,221.237000) -- 
		(148.002000,221.497000) -- 
		(147.987000,221.666000) -- 
		(148.045000,222.007000) -- 
		(148.176000,222.473000) -- 
		(148.365000,222.831000) -- 
		(148.626000,223.290000) -- 
		(149.363000,224.399000) -- 
		(149.531000,224.654000) -- 
		(150.019000,225.376000) -- 
		(150.088000,225.492000) -- 
		(150.243000,225.754000) -- 
		(150.286000,225.846000) -- 
		(150.341000,225.970000) -- 
		(150.424000,226.152000) -- 
		(150.483000,226.338000) -- 
		(150.545000,226.585000) -- 
		(150.563000,226.853000) -- 
		(150.564000,226.950000) -- 
		(150.575000,228.449000) -- 
		(150.573000,229.590000) -- 
		(150.573000,229.997000) -- 
		(150.571000,230.624000) -- 
		(150.597000,231.013000) -- 
		(150.595000,231.331000) -- 
		(150.593000,231.551000) -- 
		(150.590000,232.286000) -- 
		(150.588000,232.606000) -- 
		(150.595000,234.087000) -- 
		(150.595000,234.259000) -- 
		(150.610000,238.535000) -- 
		(150.614000,240.018000) -- 
		(150.617000,240.757000) -- 
		(150.611000,241.451000) -- 
		(150.607000,241.775000) -- 
		(150.568000,243.047000) -- 
		(150.406000,245.679000) -- 
		(150.400000,245.747000) -- 
		(150.269000,247.175000) -- 
		(151.947000,246.238000) -- 
		(154.843000,244.620000) -- 
		(155.446000,244.312000) -- 
		(155.771000,244.201000) -- 
		(156.166000,244.090000) -- 
		(156.387000,244.045000) -- 
		(156.594000,244.013000) -- 
		(157.174000,243.996000) -- 
		(157.331000,243.993000) -- 
		(159.094000,244.021000) -- 
		(159.369000,244.019000) -- 
		(159.748000,244.015000) -- 
		(159.896000,244.014000) -- 
		(160.530000,244.027000) -- 
		(160.530000,244.028000) -- 
		(161.707000,244.052000) -- 
		(162.360000,244.068000) -- 
		(162.934000,244.081000) -- 
		(162.936000,244.080000) -- 
		(163.056000,244.076000) -- 
		(163.746000,244.084000) -- 
		(164.134000,244.098000) -- 
		(164.494000,244.141000) -- 
		(164.652000,244.192000) -- 
		(164.662000,244.196000) -- 
		(164.928000,244.339000) -- 
		(165.138000,244.497000) -- 
		(165.277000,244.625000) -- 
		(165.412000,244.780000) -- 
		(165.541000,245.055000) -- 
		(165.575000,245.197000) -- 
		(165.597000,245.471000) -- 
		(165.610000,245.743000) -- 
		(165.627000,246.118000) -- 
		(165.627000,246.119000) -- 
		(165.682000,247.237000) -- 
		(165.704000,247.680000) -- 
		(165.770000,248.137000) -- 
		(165.843000,248.446000) -- 
		(165.987000,248.932000) -- 
		(166.271000,249.840000) -- 
		(166.685000,251.170000) -- 
		(166.799000,251.519000) -- 
		(167.138000,252.567000) -- 
		(167.155000,252.621000) -- 
		(167.247000,252.918000) -- 
		(167.569000,253.963000) -- 
		(167.613000,254.097000) -- 
		(167.943000,255.128000) -- 
		(168.233000,255.813000) -- 
		(168.370000,256.029000) -- 
		(168.450000,256.154000) -- 
		(169.264000,257.384000) -- 
		(169.483000,257.716000) -- 
		(169.954000,258.408000) -- 
		(171.017000,259.991000) -- 
		(171.070000,260.071000) -- 
		(171.813000,261.196000) -- 
		(172.594000,262.331000) -- 
		(172.683000,262.479000) -- 
		(173.425000,263.570000) -- 
		(174.138000,264.663000) -- 
		(174.418000,265.058000) -- 
		(175.003000,265.891000) -- 
		(175.164000,266.095000) -- 
		(175.707000,266.577000) -- 
		(176.623000,267.202000) -- 
		(177.054000,267.527000) -- 
		(177.630000,267.964000) -- 
		(178.814000,268.888000) -- 
		(180.094000,269.817000) -- 
		(180.996000,270.515000) -- 
		(181.306000,270.757000) -- 
		(182.342000,271.466000) -- 
		(182.640000,271.670000) -- 
		(183.721000,272.323000) -- 
		(184.081000,272.488000) -- 
		(184.686000,272.675000) -- 
		(185.991000,273.023000) -- 
		(186.583000,273.180000) -- 
		(186.849000,273.252000) -- 
		(187.237000,273.348000) -- 
		(187.928000,273.469000) -- 
		(188.282000,273.494000) -- 
		(188.886000,273.482000) -- 
		(191.763000,273.471000) -- 
		(192.461000,273.468000) -- 
		(197.049000,273.498000) -- 
		(199.964000,273.491000) -- 
		(200.399000,273.499000) -- 
		(200.497000,273.502000) -- 
		(201.617000,273.502000) -- 
		(203.828000,273.503000) -- 
		(205.014000,273.523000) -- 
		(205.356000,273.515000) -- 
		(205.997000,273.501000) -- 
		(206.380000,273.507000) -- 
		(206.721000,273.516000) -- 
		(208.155000,273.513000) -- 
		(209.344000,273.511000) -- 
		(209.982000,273.508000) -- 
		(209.983000,273.507000) -- 
		(211.002000,273.502000) -- 
		(212.455000,273.511000) -- 
		(213.888000,273.521000) -- 
		(215.174000,273.529000) -- 
		(217.481000,273.521000) -- 
		(218.368000,273.453000) -- 
		(218.556000,273.429000) -- 
		(219.291000,273.247000) -- 
		(219.941000,273.032000) -- 
		(220.780000,272.696000) -- 
		(223.583000,271.405000) -- 
		(225.915000,270.351000) -- 
		(228.390000,269.231000) -- 
		(230.463000,268.271000) -- 
		(232.177000,267.497000) -- 
		(232.920000,267.132000) -- 
		(234.694000,266.330000) -- 
		(236.023000,265.743000) -- 
		(237.683000,264.991000) -- 
		(238.690000,264.517000) -- 
		(238.919000,264.409000) -- 
		(239.149000,264.301000) -- 
		(239.173000,264.289000) -- 
		(239.737000,264.040000) -- 
		(239.203000,263.303000) -- 
		(239.110000,263.026000) -- 
		(239.043000,262.590000) -- 
		(239.043000,262.264000) -- 
		(239.042000,262.036000) -- 
		(239.219000,261.952000) -- 
		(239.257000,261.941000) -- 
		(239.024000,261.575000) -- 
		(238.844000,261.360000) -- 
		(238.099000,260.662000) -- 
		(237.916000,260.544000) -- 
		(237.507000,260.466000) -- 
		(237.295000,260.510000) -- 
		(236.621000,260.731000) -- 
		(236.205000,260.899000) -- 
		(235.932000,261.054000) -- 
		(235.715000,261.242000) -- 
		(235.314000,261.614000) -- 
		(235.151000,261.785000) -- 
		(235.006000,261.892000) -- 
		(234.852000,261.926000) -- 
		(234.604000,261.900000) -- 
		(234.401000,261.812000) -- 
		(234.213000,261.620000) -- 
		(233.999000,261.255000) -- 
		(233.797000,260.876000) -- 
		(233.563000,260.485000) -- 
		(233.427000,260.353000) -- 
		(233.253000,260.269000) -- 
		(233.073000,260.271000) -- 
		(232.694000,260.355000) -- 
		(232.391000,260.436000) -- 
		(232.178000,260.474000) -- 
		(231.845000,260.499000) -- 
		(231.354000,260.569000) -- 
		(231.113000,260.591000) -- 
		(230.808000,260.723000) -- 
		(230.585000,260.879000) -- 
		(230.502000,260.988000) -- 
		(230.503000,261.066000) -- 
		(230.529000,261.178000) -- 
		(230.540000,261.291000) -- 
		(230.514000,261.375000) -- 
		(230.454000,261.450000) -- 
		(230.326000,261.549000) -- 
		(230.170000,261.652000) -- 
		(229.895000,261.775000) -- 
		(229.565000,261.858000) -- 
		(229.363000,261.888000) -- 
		(229.131000,261.954000) -- 
		(228.825000,261.906000) -- 
		(229.092000,261.574000) -- 
		(229.385000,261.338000) -- 
		(229.726000,261.166000) -- 
		(230.142000,261.018000) -- 
		(230.813000,260.659000) -- 
		(230.954000,260.569000) -- 
		(231.067000,260.287000) -- 
		(231.254000,260.028000) -- 
		(233.050000,259.169000) -- 
		(233.582000,259.076000) -- 
		(234.046000,259.000000) -- 
		(235.099000,258.814000) -- 
		(235.714000,258.520000) -- 
		(236.383000,258.055000) -- 
		(236.611000,257.965000) -- 
		(237.256000,257.848000) -- 
		(237.968000,257.449000) -- 
		(237.977000,257.126000) -- 
		(237.960000,256.721000) -- 
		(238.016000,256.446000) -- 
		(238.105000,256.314000) -- 
		(238.234000,256.236000) -- 
		(238.469000,256.261000) -- 
		(238.854000,256.487000) -- 
		(239.101000,256.639000) -- 
		(239.309000,256.652000) -- 
		(239.452000,256.624000) -- 
		(239.626000,256.561000) -- 
		(239.884000,256.405000) -- 
		(240.221000,256.272000) -- 
		(240.298000,256.231000) -- 
		(240.323000,256.106000) -- 
		(240.172000,255.969000) -- 
		(240.001000,255.763000) -- 
		(240.025000,255.089000) -- 
		(240.027000,255.056000) -- 
		(239.968000,254.567000) -- 
		(239.884000,254.302000) -- 
		(239.727000,254.028000) -- 
		(239.493000,253.797000) -- 
		(239.200000,253.544000) -- 
		(239.018000,253.373000) -- 
		(238.966000,253.241000) -- 
		(238.972000,253.022000) -- 
		(238.985000,252.923000) -- 
		(239.037000,252.714000) -- 
		(239.011000,252.615000) -- 
		(238.986000,252.405000) -- 
		(238.940000,252.169000) -- 
		(238.914000,251.779000) -- 
		(238.869000,251.405000) -- 
		(238.751000,251.015000) -- 
		(238.556000,250.751000) -- 
		(238.446000,250.646000) -- 
		(238.348000,250.601000) -- 
		(238.276000,250.607000) -- 
		(238.146000,250.597000) -- 
		(237.983000,250.646000) -- 
		(237.690000,250.828000) -- 
		(237.410000,251.048000) -- 
		(237.189000,251.157000) -- 
		(237.000000,251.184000) -- 
		(236.818000,251.146000) -- 
		(236.746000,251.046000) -- 
		(236.746000,250.855000) -- 
		(236.857000,250.685000) -- 
		(237.111000,250.476000) -- 
		(237.280000,250.217000) -- 
		(237.293000,250.013000) -- 
		(237.196000,249.733000) -- 
		(237.033000,249.326000) -- 
		(236.955000,248.907000) -- 
		(236.936000,248.771000) -- 
		(236.910000,248.254000) -- 
		(236.799000,247.842000) -- 
		(236.760000,247.611000) -- 
		(236.747000,247.176000) -- 
		(236.669000,246.967000) -- 
		(236.552000,246.720000) -- 
		(236.396000,246.604000) -- 
		(236.246000,246.577000) -- 
		(236.025000,246.621000) -- 
		(235.829000,246.736000) -- 
		(235.667000,246.846000) -- 
		(235.562000,246.852000) -- 
		(235.439000,246.785000) -- 
		(235.367000,246.636000) -- 
		(235.432000,246.433000) -- 
		(235.621000,245.949000) -- 
		(235.615000,245.746000) -- 
		(235.517000,245.437000) -- 
		(235.504000,245.423000) -- 
		(235.348000,245.250000) -- 
		(235.091000,245.139000) -- 
		(234.825000,245.140000) -- 
		(234.528000,245.225000) -- 
		(234.210000,245.342000) -- 
		(234.050000,245.361000) -- 
		(233.890000,245.336000) -- 
		(233.659000,245.163000) -- 
		(233.422000,245.000000) -- 
		(233.227000,244.915000) -- 
		(233.049000,244.739000) -- 
		(232.945000,244.597000) -- 
		(232.882000,244.492000) -- 
		(232.701000,244.423000) -- 
		(232.365000,244.324000) -- 
		(231.997000,244.294000) -- 
		(231.705000,244.313000) -- 
		(231.539000,244.357000) -- 
		(231.300000,244.369000) -- 
		(230.756000,244.218000) -- 
		(230.486000,244.101000) -- 
		(230.242000,243.944000) -- 
		(230.206000,243.832000) -- 
		(230.296000,243.701000) -- 
		(230.382000,243.627000) -- 
		(230.530000,243.426000) -- 
		(230.553000,243.273000) -- 
		(230.496000,243.168000) -- 
		(230.348000,243.075000) -- 
		(230.073000,243.010000) -- 
		(229.937000,242.944000) -- 
		(229.869000,242.879000) -- 
		(229.799000,242.769000) -- 
		(229.779000,242.499000) -- 
		(229.748000,242.331000) -- 
		(229.646000,242.203000) -- 
		(229.470000,242.174000) -- 
		(229.244000,242.163000) -- 
		(228.962000,242.035000) -- 
		(228.804000,241.881000) -- 
		(228.722000,241.765000) -- 
		(228.561000,241.506000) -- 
		(228.428000,241.248000) -- 
		(228.343000,241.005000) -- 
		(228.216000,240.755000) -- 
		(227.882000,240.481000) -- 
		(227.774000,240.438000) -- 
		(227.589000,240.472000) -- 
		(227.235000,240.711000) -- 
		(227.043000,240.796000) -- 
		(226.917000,240.827000) -- 
		(226.870000,240.783000) -- 
		(226.807000,240.661000) -- 
		(226.834000,240.396000) -- 
		(226.928000,240.139000) -- 
		(227.006000,239.862000) -- 
		(227.028000,239.393000) -- 
		(226.986000,239.036000) -- 
		(226.885000,238.745000) -- 
		(226.781000,238.557000) -- 
		(226.611000,238.448000) -- 
		(226.544000,238.428000) -- 
		(226.263000,238.382000) -- 
		(226.128000,238.273000) -- 
		(225.960000,238.013000) -- 
		(225.845000,237.937000) -- 
		(225.768000,237.800000) -- 
		(225.782000,237.627000) -- 
		(226.189000,237.689000) -- 
		(226.262000,237.331000) -- 
		(226.253000,237.042000) -- 
		(226.168000,236.786000) -- 
		(225.928000,236.367000) -- 
		(225.843000,236.151000) -- 
		(225.848000,235.963000) -- 
		(225.907000,235.740000) -- 
		(226.058000,235.338000) -- 
		(226.116000,235.037000) -- 
		(226.115000,234.714000) -- 
		(226.048000,234.380000) -- 
		(226.015000,234.052000) -- 
		(226.046000,233.718000) -- 
		(226.111000,233.400000) -- 
		(226.203000,233.172000) -- 
		(226.282000,233.109000) -- 
		(226.401000,233.121000) -- 
		(226.751000,233.321000) -- 
		(226.948000,233.274000) -- 
		(227.099000,233.130000) -- 
		(227.408000,232.649000) -- 
		(227.533000,232.566000) -- 
		(227.652000,232.616000) -- 
		(227.645000,232.839000) -- 
		(227.765000,232.939000) -- 
		(228.022000,233.049000) -- 
		(228.305000,233.082000) -- 
		(228.510000,233.024000) -- 
		(228.832000,232.880000) -- 
		(228.971000,232.774000) -- 
		(229.207000,232.689000) -- 
		(229.366000,232.521000) -- 
		(229.596000,232.203000) -- 
		(229.582000,232.065000) -- 
		(229.542000,232.014000) -- 
		(229.436000,231.987000) -- 
		(229.238000,231.877000) -- 
		(228.901000,231.738000) -- 
		(228.683000,231.565000) -- 
		(228.611000,231.394000) -- 
		(228.590000,231.166000) -- 
		(228.524000,231.086000) -- 
		(228.339000,230.981000) -- 
		(228.207000,230.848000) -- 
		(228.133000,230.649000) -- 
		(228.073000,230.187000) -- 
		(228.144000,229.508000) -- 
		(228.249000,229.334000) -- 
		(228.440000,229.305000) -- 
		(228.572000,229.322000) -- 
		(228.756000,229.411000) -- 
		(228.855000,229.410000) -- 
		(228.954000,229.266000) -- 
		(229.026000,229.237000) -- 
		(229.145000,229.292000) -- 
		(229.179000,229.420000) -- 
		(229.264000,229.477000) -- 
		(229.337000,229.442000) -- 
		(229.343000,229.276000) -- 
		(229.277000,229.131000) -- 
		(229.276000,228.931000) -- 
		(229.348000,228.786000) -- 
		(229.611000,228.651000) -- 
		(229.723000,228.522000) -- 
		(229.745000,228.461000) -- 
		(229.814000,228.269000) -- 
		(229.898000,228.070000) -- 
		(230.035000,227.873000) -- 
		(230.257000,227.763000) -- 
		(230.348000,227.681000) -- 
		(230.406000,227.516000) -- 
		(230.537000,227.335000) -- 
		(230.823000,227.147000) -- 
		(231.051000,226.956000) -- 
		(231.220000,226.752000) -- 
		(231.311000,226.593000) -- 
		(231.390000,226.307000) -- 
		(231.494000,226.142000) -- 
		(231.592000,226.026000) -- 
		(231.728000,225.971000) -- 
		(231.871000,225.987000) -- 
		(232.132000,226.097000) -- 
		(232.386000,226.092000) -- 
		(232.633000,225.987000) -- 
		(232.783000,225.818000) -- 
		(232.802000,225.620000) -- 
		(232.627000,225.092000) -- 
		(232.575000,224.492000) -- 
		(232.471000,224.377000) -- 
		(232.243000,224.344000) -- 
		(232.015000,224.366000) -- 
		(231.904000,224.448000) -- 
		(231.781000,224.498000) -- 
		(231.566000,224.498000) -- 
		(231.468000,224.448000) -- 
		(231.469000,224.354000) -- 
		(231.527000,224.234000) -- 
		(231.651000,224.124000) -- 
		(231.983000,223.915000) -- 
		(232.048000,223.815000) -- 
		(232.107000,223.530000) -- 
		(232.191000,223.420000) -- 
		(232.491000,223.255000) -- 
		(232.504000,223.173000) -- 
		(232.452000,223.057000) -- 
		(232.237000,222.886000) -- 
		(232.198000,222.672000) -- 
		(232.263000,222.474000) -- 
		(232.511000,222.143000) -- 
		(232.713000,221.918000) -- 
		(232.849000,221.859000) -- 
		(232.999000,221.842000) -- 
		(233.012000,221.918000) -- 
		(232.979000,222.145000) -- 
		(232.986000,222.331000) -- 
		(233.051000,222.408000) -- 
		(233.448000,222.425000) -- 
		(233.480000,222.491000) -- 
		(233.454000,222.705000) -- 
		(233.506000,222.815000) -- 
		(233.715000,222.996000) -- 
		(233.832000,222.996000) -- 
		(233.897000,222.903000) -- 
		(234.014000,222.782000) -- 
		(234.196000,222.717000) -- 
		(234.359000,222.765000) -- 
		(234.554000,222.805000) -- 
		(234.697000,222.776000) -- 
		(234.854000,222.629000) -- 
		(234.938000,222.402000) -- 
		(234.978000,222.050000) -- 
		(235.004000,221.711000) -- 
		(234.939000,221.562000) -- 
		(234.919000,221.414000) -- 
		(234.724000,221.227000) -- 
		(234.717000,221.161000) -- 
		(234.739000,221.113000) -- 
		(234.763000,221.062000) -- 
		(235.075000,221.106000) -- 
		(235.101000,221.040000) -- 
		(234.893000,221.001000) -- 
		(234.639000,220.913000) -- 
		(234.405000,220.578000) -- 
		(234.288000,220.263000) -- 
		(234.178000,220.131000) -- 
		(234.054000,220.121000) -- 
		(233.689000,220.010000) -- 
		(233.468000,219.900000) -- 
		(233.374000,219.819000) -- 
		(233.306000,219.758000) -- 
		(233.214000,219.610000) -- 
		(233.195000,218.807000) -- 
		(233.137000,218.510000) -- 
		(233.039000,218.378000) -- 
		(232.863000,218.306000) -- 
		(232.466000,218.312000) -- 
		(232.297000,218.234000) -- 
		(232.134000,218.059000) -- 
		(232.004000,218.015000) -- 
		(231.887000,218.085000) -- 
		(231.881000,218.257000) -- 
		(231.822000,218.366000) -- 
		(231.724000,218.420000) -- 
		(231.601000,218.405000) -- 
		(231.432000,218.228000) -- 
		(231.406000,217.975000) -- 
		(231.432000,217.750000) -- 
		(231.608000,217.558000) -- 
		(231.861000,217.333000) -- 
		(231.881000,217.139000) -- 
		(231.888000,216.876000) -- 
		(231.985000,216.765000) -- 
		(232.141000,216.700000) -- 
		(232.330000,216.683000) -- 
		(232.441000,216.601000) -- 
		(232.434000,216.453000) -- 
		(232.064000,216.315000) -- 
		(231.758000,216.255000) -- 
		(231.517000,216.271000) -- 
		(231.204000,216.387000) -- 
		(230.944000,216.530000) -- 
		(230.742000,216.517000) -- 
		(230.684000,216.486000) -- 
		(230.580000,216.370000) -- 
		(230.554000,216.160000) -- 
		(230.463000,216.050000) -- 
		(230.183000,216.018000) -- 
		(229.922000,216.039000) -- 
		(229.792000,216.001000) -- 
		(229.701000,215.891000) -- 
		(229.649000,215.754000) -- 
		(229.643000,215.473000) -- 
		(229.680000,215.403000) -- 
		(229.910000,214.968000) -- 
		(230.007000,214.808000) -- 
		(230.235000,214.726000) -- 
		(230.437000,214.687000) -- 
		(230.645000,214.632000) -- 
		(230.808000,214.516000) -- 
		(230.847000,214.390000) -- 
		(230.860000,214.017000) -- 
		(230.893000,213.890000) -- 
		(231.082000,213.604000) -- 
		(231.173000,213.362000) -- 
		(231.193000,213.064000) -- 
		(231.173000,212.807000) -- 
		(231.088000,212.653000) -- 
		(230.874000,212.597000) -- 
		(230.483000,212.532000) -- 
		(230.262000,212.416000) -- 
		(230.217000,212.327000) -- 
		(230.249000,212.224000) -- 
		(230.425000,212.026000) -- 
		(230.431000,211.882000) -- 
		(230.424000,211.861000) -- 
		(230.334000,211.581000) -- 
		(230.299000,211.400000) -- 
		(239.903000,211.221000) -- 
		(244.737000,211.131000) -- 
		(244.918000,211.127000) -- 
		(255.257000,211.166000) -- 
		(255.129000,200.144000) -- 
		(255.088000,198.242000) -- 
		(255.087000,198.241000) -- 
		(255.067000,191.432000) -- 
		(255.023000,177.103000) -- 
		(254.979000,162.775000) -- 
		(254.979000,162.716000) -- 
		(254.979000,162.657000) -- 
		(255.010000,153.718000) -- 
		(255.010000,153.715000) -- 
		(254.978000,146.188000) -- 
		(254.947000,138.657000) -- 
		(254.926000,133.211000) -- 
		(254.905000,127.764000) -- 
		(254.882000,123.500000) -- 
		(254.881000,123.499000) -- 
		(254.804000,109.250000) -- 
		(254.683000,94.006000) -- 
		(254.344000,94.144000) -- 
		(254.028000,94.252000) -- 
		(253.754000,94.317000) -- 
		(253.483000,94.361000) -- 
		(253.220000,94.385000) -- 
		(252.152000,94.412000) -- 
		(250.994000,94.423000) -- 
		(250.528000,94.427000) -- 
		(250.527000,94.428000) -- 
		(248.506000,94.460000) -- 
		(246.892000,94.454000) -- 
		(242.657000,94.502000) -- 
		(241.476000,94.507000) -- 
		(240.458000,94.526000) -- 
		(239.754000,94.530000) -- 
		(238.383000,94.538000) -- 
		(237.125000,94.533000) -- 
		(236.236000,94.521000) -- 
		(236.007000,94.509000) -- 
		(235.814000,94.498000) -- 
		(232.805000,94.216000) -- 
		(231.698000,94.122000) -- 
		(229.725000,93.929000) -- 
		(229.002000,93.848000) -- 
		(228.458000,93.812000) -- 
		(227.745000,93.749000) -- 
		(227.320000,93.714000) -- 
		(226.519000,93.680000) -- 
		(226.071000,93.692000) -- 
		(224.852000,93.754000) -- 
		(224.153000,93.798000) -- 
		(223.959000,93.810000) -- 
		(223.374000,93.846000) -- 
		(223.290000,93.852000) -- 
		(223.181000,93.874000) -- 
		(222.536000,94.021000) -- 
		(222.391000,94.055000) -- 
		(221.570000,94.246000) -- 
		(221.441000,94.275000) -- 
		(221.440000,94.276000) -- 
		(220.619000,94.468000) -- 
		(220.601000,94.471000) -- 
		(220.359000,94.518000) -- 
		(220.022000,94.569000) -- 
		(219.947000,94.572000) -- 
		(219.716000,94.582000) -- 
		(218.467000,94.605000) -- 
		(217.399000,94.606000) -- 
		(215.788000,94.607000) -- 
		(214.217000,94.616000) -- 
		(213.673000,94.619000) -- 
		(212.706000,94.625000) -- 
		(212.040000,94.638000) -- 
		(211.495000,94.649000) -- 
		(211.428000,94.650000) -- 
		(211.227000,94.654000) -- 
		(211.160000,94.656000) -- 
		(210.902000,94.661000) -- 
		(210.128000,94.682000) -- 
		(209.869000,94.690000) -- 
		(209.612000,94.696000) -- 
		(208.841000,94.717000) -- 
		(208.835000,94.718000) -- 
		(208.583000,94.721000) -- 
		(208.126000,94.726000) -- 
		(206.753000,94.744000) -- 
		(206.295000,94.750000) -- 
		(205.950000,94.753000) -- 
		(205.261000,94.760000) -- 
		(204.913000,94.768000) -- 
		(204.567000,94.777000) -- 
		(204.013000,94.790000) -- 
		(203.538000,94.803000) -- 
		(202.348000,94.810000) -- 
		(201.793000,94.815000) -- 
		(200.315000,94.821000) -- 
		(199.907000,94.829000) -- 
		(199.261000,94.842000) -- 
		(197.738000,94.859000) -- 
		(196.171000,94.890000) -- 
		(194.740000,94.950000) -- 
		(194.247000,94.952000) -- 
		(192.360000,94.963000) -- 
		(192.099000,94.964000) -- 
		(190.884000,95.012000) -- 
		(190.769000,95.017000) -- 
		(189.469000,95.049000) -- 
		(188.112000,95.096000) -- 
		(186.791000,95.098000) -- 
		(186.451000,95.103000) -- 
		(184.973000,95.126000) -- 
		(184.972000,94.018000) -- 
		(184.968000,91.747000) -- 
		(184.963000,91.023000) -- 
		(184.967000,90.696000) -- 
		(184.978000,89.589000) -- 
		(184.975000,89.032000) -- 
		(184.969000,88.178000) -- 
		(184.969000,87.361000) -- 
		(184.969000,86.804000) -- 
		(184.970000,86.422000) -- 
		(184.970000,85.275000) -- 
		(184.970000,84.895000) -- 
		(184.548000,84.591000) -- 
		(184.096000,84.224000) -- 
		(183.765000,84.023000) -- 
		(183.624000,83.970000) -- 
		(183.505000,83.950000) -- 
		(183.329000,83.947000) -- 
		(182.792000,83.993000) -- 
		(182.378000,84.017000) -- 
		(181.782000,84.034000) -- 
		(181.217000,84.034000) -- 
		(180.311000,84.047000) -- 
		(179.578000,84.047000) -- 
		(178.274000,84.048000) -- 
		(178.081000,84.024000) -- 
		(177.962000,83.953000) -- 
		(177.702000,83.758000) -- 
		(177.457000,83.531000) -- 
		(177.089000,83.229000) -- 
		(176.835000,83.005000) -- 
		(176.451000,82.689000) -- 
		(174.977000,81.417000) -- 
		(173.721000,80.310000) -- 
		(173.190000,79.880000) -- 
		(171.828000,78.803000) -- 
		(171.309000,78.369000) -- 
		(170.905000,77.973000) -- 
		(170.288000,77.318000) -- 
		(170.008000,77.052000) -- 
		(169.481000,76.524000) -- 
		(169.425000,76.399000) -- 
		(169.412000,76.212000) -- 
		(169.415000,75.695000) -- 
		(169.402000,75.419000) -- 
		(169.425000,75.004000) -- 
		(169.412000,73.340000) -- 
		(169.380000,72.386000) -- 
		(169.337000,71.131000) -- 
		(169.327000,69.577000) -- 
		(169.358000,67.582000) -- 
		(169.405000,67.417000) -- 
		(169.459000,67.287000) -- 
		(169.587000,67.109000) -- 
		(169.810000,66.834000) -- 
		(169.428000,66.723000) -- 
		(169.107000,66.601000) -- 
		(168.973000,66.503000) -- 
		(168.798000,66.334000) -- 
		(168.620000,66.064000) -- 
		(168.449000,65.827000) -- 
		(168.315000,65.667000) -- 
		(168.098000,65.510000) -- 
		(167.979000,65.468000) -- 
		(167.720000,65.426000) -- 
		(167.684000,65.421000) -- 
		(166.960000,65.441000) -- 
		(165.503000,65.467000) -- 
		(164.441000,65.511000) -- 
		(162.866000,65.498000) -- 
		(161.919000,65.520000) -- 
		(159.865000,65.527000) -- 
		(159.828000,65.524000) -- 
		(159.416000,65.512000) -- 
		(159.037000,65.485000) -- 
		(158.620000,65.343000) -- 
		(158.341000,65.229000) -- 
		(158.119000,65.160000) -- 
		(157.943000,65.149000) -- 
		(157.820000,65.168000) -- 
		(157.720000,65.194000) -- 
		(157.285000,65.349000) -- 
		(157.265000,65.323000) -- 
		(157.161000,65.103000) -- 
		(156.985000,64.878000) -- 
		(156.927000,64.762000) -- 
		(156.894000,64.592000) -- 
		(156.855000,64.487000) -- 
		(156.771000,64.394000) -- 
		(156.751000,64.251000) -- 
		(156.784000,64.202000) -- 
		(156.855000,64.125000) -- 
		(156.901000,64.036000) -- 
		(156.868000,63.899000) -- 
		(156.693000,63.751000) -- 
		(156.446000,63.624000) -- 
		(156.304000,63.478000) -- 
		(156.192000,63.366000) -- 
		(156.179000,63.223000) -- 
		(156.127000,62.909000) -- 
		(155.971000,62.403000) -- 
		(155.471000,61.623000) -- 
		(155.289000,61.496000) -- 
		(154.983000,61.436000) -- 
		(154.678000,61.452000) -- 
		(154.360000,61.419000) -- 
		(154.113000,61.359000) -- 
		(153.892000,61.216000) -- 
		(153.781000,61.078000) -- 
		(153.762000,60.886000) -- 
		(153.794000,60.721000) -- 
		(153.866000,60.628000) -- 
		(153.898000,60.534000) -- 
		(153.736000,60.452000) -- 
		(153.671000,60.446000) -- 
		(153.521000,60.380000) -- 
		(153.450000,60.331000) -- 
		(153.417000,60.210000) -- 
		(153.443000,60.072000) -- 
		(153.437000,59.940000) -- 
		(153.365000,59.781000) -- 
		(153.248000,59.605000) -- 
		(153.112000,59.473000) -- 
		(153.008000,59.291000) -- 
		(152.924000,59.110000) -- 
		(152.885000,59.082000) -- 
		(152.709000,59.060000) -- 
		(152.618000,59.011000) -- 
		(152.560000,58.841000) -- 
		(152.527000,58.813000) -- 
		(152.382000,58.790000) -- 
		(152.079000,58.747000) -- 
		(151.949000,58.648000) -- 
		(151.644000,58.577000) -- 
		(151.494000,58.494000) -- 
		(151.423000,58.384000) -- 
		(151.390000,58.032000) -- 
		(151.130000,58.104000) -- 
		(150.994000,58.076000) -- 
		(150.870000,57.994000) -- 
		(150.773000,57.900000) -- 
		(150.682000,57.862000) -- 
		(150.558000,57.911000) -- 
		(150.448000,58.071000) -- 
		(150.357000,58.263000) -- 
		(150.253000,58.346000) -- 
		(150.143000,58.351000) -- 
		(150.013000,58.335000) -- 
		(149.883000,58.252000) -- 
		(149.759000,58.082000) -- 
		(149.603000,57.906000) -- 
		(149.480000,57.845000) -- 
		(149.396000,57.840000) -- 
		(149.318000,57.895000) -- 
		(149.272000,57.955000) -- 
		(149.266000,58.071000) -- 
		(149.220000,58.236000) -- 
		(149.116000,58.406000) -- 
		(148.954000,58.483000) -- 
		(148.869000,58.494000) -- 
		(148.720000,58.467000) -- 
		(148.512000,58.335000) -- 
		(148.363000,58.269000) -- 
		(148.135000,58.275000) -- 
		(148.064000,58.263000) -- 
		(148.025000,58.209000) -- 
		(148.005000,58.126000) -- 
		(148.012000,58.044000) -- 
		(148.044000,57.934000) -- 
		(147.921000,57.818000) -- 
		(147.836000,57.791000) -- 
		(147.570000,57.785000) -- 
		(147.362000,57.829000) -- 
		(147.265000,57.829000) -- 
		(147.187000,57.758000) -- 
		(147.148000,57.626000) -- 
		(147.161000,57.521000) -- 
		(147.200000,57.455000) -- 
		(147.297000,57.411000) -- 
		(147.420000,57.395000) -- 
		(147.570000,57.422000) -- 
		(147.667000,57.428000) -- 
		(147.732000,57.373000) -- 
		(147.758000,57.186000) -- 
		(147.765000,56.823000) -- 
		(147.706000,56.685000) -- 
		(147.635000,56.603000) -- 
		(147.563000,56.592000) -- 
		(147.479000,56.603000) -- 
		(147.355000,56.658000) -- 
		(147.239000,56.735000) -- 
		(147.109000,56.779000) -- 
		(147.044000,56.757000) -- 
		(146.979000,56.625000) -- 
		(146.985000,56.526000) -- 
		(147.044000,56.443000) -- 
		(147.109000,56.394000) -- 
		(147.277000,56.372000) -- 
		(147.433000,56.267000) -- 
		(147.531000,56.152000) -- 
		(147.544000,56.009000) -- 
		(147.518000,55.899000) -- 
		(147.440000,55.849000) -- 
		(147.323000,55.866000) -- 
		(147.160000,55.959000) -- 
		(146.953000,56.053000) -- 
		(146.790000,56.113000) -- 
		(146.673000,56.113000) -- 
		(146.563000,56.047000) -- 
		(146.621000,55.729000) -- 
		(146.725000,55.613000) -- 
		(146.836000,55.448000) -- 
		(146.810000,55.360000) -- 
		(146.706000,55.333000) -- 
		(146.576000,55.327000) -- 
		(146.517000,55.272000) -- 
		(146.491000,55.168000) -- 
		(146.426000,54.997000) -- 
		(146.329000,54.937000) -- 
		(146.192000,54.909000) -- 
		(146.088000,54.942000) -- 
		(146.017000,54.997000) -- 
		(145.933000,55.107000) -- 
		(145.848000,55.278000) -- 
		(145.738000,55.410000) -- 
		(145.735000,55.412000) -- 
		(145.601000,55.503000) -- 
		(145.400000,55.608000) -- 
		(145.296000,55.619000) -- 
		(145.107000,55.569000) -- 
		(144.965000,55.448000) -- 
		(144.770000,55.404000) -- 
		(144.516000,55.333000) -- 
		(144.477000,55.278000) -- 
		(144.380000,55.063000) -- 
		(144.302000,54.970000) -- 
		(144.224000,54.942000) -- 
		(144.042000,54.970000) -- 
		(143.750000,55.080000) -- 
		(143.516000,55.140000) -- 
		(143.308000,55.140000) -- 
		(143.165000,55.096000) -- 
		(143.119000,55.052000) -- 
		(143.061000,54.931000) -- 
		(143.035000,54.821000) -- 
		(142.925000,54.750000) -- 
		(142.769000,54.711000) -- 
		(142.697000,54.788000) -- 
		(142.645000,54.871000) -- 
		(142.639000,55.063000) -- 
		(142.606000,55.228000) -- 
		(142.541000,55.322000) -- 
		(142.418000,55.437000) -- 
		(142.437000,55.652000) -- 
		(142.327000,55.784000) -- 
		(142.171000,55.839000) -- 
		(142.035000,55.855000) -- 
		(141.892000,55.839000) -- 
		(141.677000,55.784000) -- 
		(141.515000,55.811000) -- 
		(141.385000,55.905000) -- 
		(141.372000,55.949000) -- 
		(141.456000,56.042000) -- 
		(141.586000,56.136000) -- 
		(141.599000,56.229000) -- 
		(141.515000,56.339000) -- 
		(141.047000,56.328000) -- 
		(140.917000,56.235000) -- 
		(140.826000,56.229000) -- 
		(140.716000,56.235000) -- 
		(140.553000,56.020000) -- 
		(140.436000,55.971000) -- 
		(140.332000,55.965000) -- 
		(140.079000,55.844000) -- 
		(139.982000,55.850000) -- 
		(139.826000,55.943000) -- 
		(139.689000,55.943000) -- 
		(139.618000,55.883000) -- 
		(139.618000,55.751000) -- 
		(139.722000,55.641000) -- 
		(139.787000,55.547000) -- 
		(139.774000,55.470000) -- 
		(139.689000,55.415000) -- 
		(139.293000,55.465000) -- 
		(139.189000,55.470000) -- 
		(139.163000,55.360000) -- 
		(139.234000,55.256000) -- 
		(139.345000,55.173000) -- 
		(139.572000,55.047000) -- 
		(139.637000,54.981000) -- 
		(139.598000,54.909000) -- 
		(139.507000,54.854000) -- 
		(138.825000,54.645000) -- 
		(138.650000,54.541000) -- 
		(138.507000,54.420000) -- 
		(138.468000,54.431000) -- 
		(138.403000,54.541000) -- 
		(138.396000,54.805000) -- 
		(138.364000,54.893000) -- 
		(138.305000,54.942000) -- 
		(138.208000,54.942000) -- 
		(138.072000,54.898000) -- 
		(137.903000,54.904000) -- 
		(137.779000,54.882000) -- 
		(137.753000,54.810000) -- 
		(137.766000,54.711000) -- 
		(137.721000,54.634000) -- 
		(137.584000,54.579000) -- 
		(137.526000,54.607000) -- 
		(137.383000,54.739000) -- 
		(137.253000,54.739000) -- 
		(137.117000,54.634000) -- 
		(136.928000,54.464000) -- 
		(136.837000,54.425000) -- 
		(136.766000,54.453000) -- 
		(136.766000,54.513000) -- 
		(136.824000,54.766000) -- 
		(136.837000,54.992000) -- 
		(136.720000,55.085000) -- 
		(136.629000,55.074000) -- 
		(136.525000,55.008000) -- 
		(136.233000,54.909000) -- 
		(136.103000,54.821000) -- 
		(136.103000,54.733000) -- 
		(136.175000,54.475000) -- 
		(136.162000,54.343000) -- 
		(136.064000,54.239000) -- 
		(135.967000,54.211000) -- 
		(135.830000,54.211000) -- 
		(135.674000,54.277000) -- 
		(135.603000,54.277000) -- 
		(135.479000,54.200000) -- 
		(135.337000,53.953000) -- 
		(135.285000,53.936000) -- 
		(135.135000,54.002000) -- 
		(135.005000,54.085000) -- 
		(134.895000,54.101000) -- 
		(134.752000,54.084000) -- 
		(134.687000,54.052000) -- 
		(134.661000,53.947000) -- 
		(134.680000,53.848000) -- 
		(134.765000,53.782000) -- 
		(134.953000,53.727000) -- 
		(134.862000,53.546000) -- 
		(134.901000,53.408000) -- 
		(134.895000,53.386000) -- 
		(134.778000,53.392000) -- 
		(134.615000,53.441000) -- 
		(134.472000,53.408000) -- 
		(134.466000,53.265000) -- 
		(134.485000,53.106000) -- 
		(134.538000,52.930000) -- 
		(134.447000,52.831000) -- 
		(134.252000,52.770000) -- 
		(134.141000,52.787000) -- 
		(133.849000,52.880000) -- 
		(133.719000,52.864000) -- 
		(133.596000,52.792000) -- 
		(133.459000,52.655000) -- 
		(133.375000,52.622000) -- 
		(133.342000,52.649000) -- 
		(133.394000,52.710000) -- 
		(133.401000,52.781000) -- 
		(133.375000,52.814000) -- 
		(133.290000,52.842000) -- 
		(133.030000,52.803000) -- 
		(132.985000,52.825000) -- 
		(132.965000,53.062000) -- 
		(132.835000,53.128000) -- 
		(132.491000,53.106000) -- 
		(132.316000,53.073000) -- 
		(132.218000,53.018000) -- 
		(132.199000,52.847000) -- 
		(132.446000,52.693000) -- 
		(132.459000,52.633000) -- 
		(132.452000,52.589000) -- 
		(132.290000,52.583000) -- 
		(132.056000,52.644000) -- 
		(131.828000,52.677000) -- 
		(131.725000,52.644000) -- 
		(131.666000,52.572000) -- 
		(131.712000,52.462000) -- 
		(131.848000,52.380000) -- 
		(131.965000,52.325000) -- 
		(132.036000,52.308000) -- 
		(132.368000,52.319000) -- 
		(132.485000,52.198000) -- 
		(132.459000,52.127000) -- 
		(132.374000,52.077000) -- 
		(132.192000,52.039000) -- 
		(131.705000,52.044000) -- 
		(131.543000,51.995000) -- 
		(131.367000,51.989000) -- 
		(131.257000,51.846000) -- 
		(131.302000,51.742000) -- 
		(131.400000,51.687000) -- 
		(131.595000,51.681000) -- 
		(131.660000,51.648000) -- 
		(131.751000,51.362000) -- 
		(131.855000,51.148000) -- 
		(131.881000,50.983000) -- 
		(131.952000,50.895000) -- 
		(132.050000,50.884000) -- 
		(132.153000,50.846000) -- 
		(132.264000,50.741000) -- 
		(132.413000,50.626000) -- 
		(132.491000,50.703000) -- 
		(132.686000,50.780000) -- 
		(132.751000,50.741000) -- 
		(132.810000,50.659000) -- 
		(132.784000,50.582000) -- 
		(132.686000,50.466000) -- 
		(132.543000,50.257000) -- 
		(132.550000,50.158000) -- 
		(132.589000,50.076000) -- 
		(132.712000,50.015000) -- 
		(132.810000,49.988000) -- 
		(132.920000,49.933000) -- 
		(132.862000,49.790000) -- 
		(132.615000,49.619000) -- 
		(132.556000,49.542000) -- 
		(132.563000,49.454000) -- 
		(132.660000,49.350000) -- 
		(132.966000,49.267000) -- 
		(132.985000,49.201000) -- 
		(132.979000,49.157000) -- 
		(132.894000,49.097000) -- 
		(132.725000,49.075000) -- 
		(132.660000,49.036000) -- 
		(132.680000,48.959000) -- 
		(132.738000,48.877000) -- 
		(132.725000,48.789000) -- 
		(132.699000,48.761000) -- 
		(132.602000,48.761000) -- 
		(132.478000,48.844000) -- 
		(132.251000,49.031000) -- 
		(132.225000,49.003000) -- 
		(132.238000,48.904000) -- 
		(132.355000,48.706000) -- 
		(132.550000,48.437000) -- 
		(132.621000,48.272000) -- 
		(132.667000,48.201000) -- 
		(132.628000,48.047000) -- 
		(132.556000,48.036000) -- 
		(132.167000,48.074000) -- 
		(132.076000,48.074000) -- 
		(132.056000,48.019000) -- 
		(132.186000,47.750000) -- 
		(132.173000,47.618000) -- 
		(132.141000,47.546000) -- 
		(132.043000,47.480000) -- 
		(131.926000,47.431000) -- 
		(131.868000,47.304000) -- 
		(131.910000,47.250000) -- 
		(131.965000,47.178000) -- 
		(132.069000,47.101000) -- 
		(132.147000,47.007000) -- 
		(132.141000,46.925000) -- 
		(132.069000,46.864000) -- 
		(131.517000,46.914000) -- 
		(131.459000,46.886000) -- 
		(131.491000,46.633000) -- 
		(131.472000,46.551000) -- 
		(131.394000,46.534000) -- 
		(131.101000,46.578000) -- 
		(130.679000,46.622000) -- 
		(130.582000,46.347000) -- 
		(130.562000,46.226000) -- 
		(130.497000,46.149000) -- 
		(130.361000,46.122000) -- 
		(130.166000,46.155000) -- 
		(129.906000,46.265000) -- 
		(129.705000,46.336000) -- 
		(129.594000,46.364000) -- 
		(129.529000,46.347000) -- 
		(129.503000,46.281000) -- 
		(129.568000,46.144000) -- 
		(129.679000,46.034000) -- 
		(129.841000,45.973000) -- 
		(129.984000,45.880000) -- 
		(130.043000,45.770000) -- 
		(130.069000,45.676000) -- 
		(130.004000,45.621000) -- 
		(129.802000,45.588000) -- 
		(129.607000,45.594000) -- 
		(129.556000,45.566000) -- 
		(129.530000,45.500000) -- 
		(129.478000,45.451000) -- 
		(129.289000,45.385000) -- 
		(129.205000,45.335000) -- 
		(129.062000,45.198000) -- 
		(128.990000,45.104000) -- 
		(128.847000,45.104000) -- 
		(128.718000,45.154000) -- 
		(128.672000,45.154000) -- 
		(128.666000,45.071000) -- 
		(128.750000,44.906000) -- 
		(128.770000,44.538000) -- 
		(128.750000,43.928000) -- 
		(128.841000,43.499000) -- 
		(128.828000,43.202000) -- 
		(128.770000,43.108000) -- 
		(128.763000,43.037000) -- 
		(128.971000,42.762000) -- 
		(128.984000,42.679000) -- 
		(128.939000,42.624000) -- 
		(128.822000,42.591000) -- 
		(128.659000,42.564000) -- 
		(128.452000,42.630000) -- 
		(128.283000,42.652000) -- 
		(128.120000,42.652000) -- 
		(128.049000,42.624000) -- 
		(128.003000,42.283000) -- 
		(128.094000,42.168000) -- 
		(128.283000,42.058000) -- 
		(128.361000,41.981000) -- 
		(128.354000,41.926000) -- 
		(128.198000,41.766000) -- 
		(127.984000,41.508000) -- 
		(128.010000,41.425000) -- 
		(128.094000,41.359000) -- 
		(128.016000,41.244000) -- 
		(128.010000,41.117000) -- 
		(128.068000,40.969000) -- 
		(128.055000,40.887000) -- 
		(128.023000,40.832000) -- 
		(127.893000,40.826000) -- 
		(127.724000,40.870000) -- 
		(127.367000,41.090000) -- 
		(127.174000,41.154000) -- 
		(127.088000,41.183000) -- 
		(126.808000,41.178000) -- 
		(126.613000,41.134000) -- 
		(126.445000,40.996000) -- 
		(126.354000,40.870000) -- 
		(126.276000,40.826000) -- 
		(126.204000,40.809000) -- 
		(126.035000,40.815000) -- 
		(125.808000,40.864000) -- 
		(125.600000,40.820000) -- 
		(125.451000,40.749000) -- 
		(125.360000,40.672000) -- 
		(125.321000,40.584000) -- 
		(125.347000,40.512000) -- 
		(125.418000,40.446000) -- 
		(125.620000,40.375000) -- 
		(125.698000,40.281000) -- 
		(125.685000,40.226000) -- 
		(125.652000,40.177000) -- 
		(125.425000,40.034000) -- 
		(125.308000,39.984000) -- 
		(125.217000,39.907000) -- 
		(125.061000,39.852000) -- 
		(124.769000,39.775000) -- 
		(124.503000,39.687000) -- 
		(124.405000,39.627000) -- 
		(124.379000,39.550000) -- 
		(124.392000,39.429000) -- 
		(124.496000,39.308000) -- 
		(124.516000,39.278000) -- 
		(124.542000,39.237000) -- 
		(124.646000,39.022000) -- 
		(124.749000,38.923000) -- 
		(124.918000,38.835000) -- 
		(124.964000,38.747000) -- 
		(124.866000,38.423000) -- 
		(124.695000,38.017000) -- 
		(124.639000,37.884000) -- 
		(124.575000,37.707000) -- 
		(124.516000,37.548000) -- 
		(124.308000,37.009000) -- 
		(124.228000,36.773000) -- 
		(124.120000,36.459000) -- 
		(124.105000,36.416000) -- 
		(124.059000,36.290000) -- 
		(124.043000,36.248000) -- 
		(123.808000,36.262000) -- 
		(123.496000,36.252000) -- 
		(122.720000,36.229000) -- 
		(121.852000,36.222000) -- 
		(121.304000,36.219000) -- 
		(120.399000,36.249000) -- 
		(119.804000,36.296000) -- 
		(119.632000,36.326000) -- 
		(119.631000,36.327000) -- 
		(119.160000,36.379000) -- 
		(118.747000,36.371000) -- 
		(118.717000,36.368000) -- 
		(116.899000,36.248000) -- 
		(116.048000,36.192000) -- 
		(115.822000,36.191000) -- 
		(115.248000,36.188000) -- 
		(115.173000,36.186000) -- 
		(114.335000,36.168000) -- 
		(113.242000,36.173000) -- 
		(112.659000,36.173000) -- 
		(112.330000,36.177000) -- 
		(111.704000,36.172000) -- 
		(110.935000,36.165000) -- 
		(110.897000,36.165000) -- 
		(110.017000,36.167000) -- 
		(108.339000,36.142000) -- 
		(107.245000,36.153000) -- 
		(106.688000,36.160000) -- 
		(103.961000,36.143000) -- 
		(103.426000,36.142000) -- 
		(102.947000,36.149000) -- 
		(102.867000,36.158000) -- 
		(102.674000,35.489000) -- 
		(102.510000,35.040000) -- 
		(102.419000,34.791000) -- 
		(102.053000,34.123000) -- 
		(102.017000,34.076000) -- 
		(101.940000,33.972000) -- 
		(101.810000,33.795000) -- 
		(101.641000,33.569000) -- 
		(101.307000,33.170000) -- 
		(101.191000,33.031000) -- 
		(101.005000,32.809000) -- 
		(100.535000,32.367000) -- 
		(100.535000,32.366000) -- 
		(100.271000,32.118000) -- 
		(100.077000,31.978000) -- 
		(99.448000,31.524000) -- 
		(99.225000,31.385000) -- 
		(99.225000,31.384000) -- 
		(98.731000,31.085000) -- 
		(98.440000,30.907000) -- 
		(98.268000,30.775000) -- 
		(98.171000,30.702000) -- 
		(97.976000,30.487000) -- 
		(97.976000,30.486000) -- 
		(97.840000,30.336000) -- 
		(97.700000,30.086000) -- 
		(97.569000,29.823000) -- 
		(97.537000,29.719000) -- 
		(97.537000,29.718000) -- 
		(97.302000,28.956000) -- 
		(97.302000,28.955000) -- 
		(97.275000,28.868000) -- 
		(97.235000,28.738000) -- 
		(97.061000,27.953000) -- 
		(97.033000,27.830000) -- 
		(96.933000,27.305000);
	\filldraw [draw=black, ultra thick, fill=lightgreen]
		(150.488000,7.633000) -- 
		(150.583000,7.657000) -- 
		(150.786000,7.709000) -- 
		(151.264000,7.828000) -- 
		(151.493000,7.886000) -- 
		(152.700000,8.174000) -- 
		(152.942000,8.232000) -- 
		(153.179000,8.288000) -- 
		(153.465000,8.356000) -- 
		(153.656000,8.402000) -- 
		(154.031000,8.490000) -- 
		(154.322000,8.557000) -- 
		(154.607000,8.623000) -- 
		(154.966000,8.705000) -- 
		(155.260000,8.774000) -- 
		(156.043000,8.954000) -- 
		(156.401000,9.038000) -- 
		(156.693000,9.105000) -- 
		(157.570000,9.308000) -- 
		(157.861000,9.376000) -- 
		(158.212000,9.457000) -- 
		(159.265000,9.700000) -- 
		(159.384000,9.728000) -- 
		(159.615000,9.782000) -- 
		(159.835000,9.833000) -- 
		(160.594000,10.010000) -- 
		(160.759000,10.048000) -- 
		(161.185000,10.148000) -- 
		(161.349000,10.186000) -- 
		(162.039000,10.347000) -- 
		(162.153000,10.371000) -- 
		(162.494000,10.443000) -- 
		(162.607000,10.468000) -- 
		(162.946000,10.524000) -- 
		(163.354000,10.572000) -- 
		(164.785000,10.746000) -- 
		(165.600000,10.841000) -- 
		(166.348000,10.930000) -- 
		(167.329000,11.046000) -- 
		(168.526000,11.191000) -- 
		(169.866000,11.339000) -- 
		(170.381000,11.400000) -- 
		(170.771000,11.457000) -- 
		(171.622000,11.582000) -- 
		(172.260000,11.676000) -- 
		(172.658000,11.735000) -- 
		(173.063000,11.776000) -- 
		(173.673000,11.838000) -- 
		(174.432000,11.935000) -- 
		(175.118000,12.023000) -- 
		(175.474000,12.058000) -- 
		(176.279000,12.142000) -- 
		(176.577000,12.179000) -- 
		(177.334000,12.274000) -- 
		(177.470000,12.288000) -- 
		(177.767000,12.322000) -- 
		(178.025000,12.350000) -- 
		(178.128000,12.362000) -- 
		(179.042000,12.470000) -- 
		(179.209000,12.494000) -- 
		(179.298000,12.508000) -- 
		(179.569000,12.537000) -- 
		(180.439000,12.627000) -- 
		(181.629000,12.755000) -- 
		(181.661000,12.758000) -- 
		(182.568000,12.855000) -- 
		(183.046000,12.907000) -- 
		(183.291000,12.934000) -- 
		(183.914000,13.005000) -- 
		(184.111000,13.027000) -- 
		(184.588000,13.082000) -- 
		(184.699000,13.099000) -- 
		(184.894000,13.130000) -- 
		(184.894000,13.084000) -- 
		(184.892000,12.948000) -- 
		(184.891000,12.903000) -- 
		(184.872000,11.715000) -- 
		(184.870000,11.602000) -- 
		(184.907000,10.499000) -- 
		(184.887000,9.225000) -- 
		(184.893000,8.439000) -- 
		(184.896000,8.151000) -- 
		(184.901000,7.521000) -- 
		(184.898000,6.964000) -- 
		(184.895000,6.327000) -- 
		(184.891000,5.814000) -- 
		(184.891000,5.813000) -- 
		(184.886000,4.418000) -- 
		(184.883000,3.782000) -- 
		(184.885000,3.087000) -- 
		(184.886000,3.002000) -- 
		(184.886000,2.753000) -- 
		(184.888000,2.396000) -- 
		(184.862000,1.989000) -- 
		(184.867000,1.769000) -- 
		(184.873000,1.566000) -- 
		(184.867000,1.144000) -- 
		(184.862000,0.866000) -- 
		(184.871000,-0.122000) -- 
		(184.874000,-0.543000) -- 
		(184.816000,-0.586000) -- 
		(184.706000,-0.669000) -- 
		(184.058000,-1.149000) -- 
		(183.199000,-1.784000) -- 
		(181.895000,-2.716000) -- 
		(181.569000,-2.911000) -- 
		(180.693000,-3.427000) -- 
		(180.596000,-3.485000) -- 
		(180.305000,-3.651000) -- 
		(180.207000,-3.707000) -- 
		(179.131000,-4.327000) -- 
		(177.963000,-4.895000) -- 
		(177.383000,-5.176000) -- 
		(175.825000,-5.817000) -- 
		(174.169000,-6.384000) -- 
		(173.264000,-6.665000) -- 
		(171.442000,-7.154000) -- 
		(170.748000,-7.287000) -- 
		(169.537000,-7.516000) -- 
		(168.477000,-7.683000) -- 
		(168.243000,-7.708000) -- 
		(167.161000,-7.826000) -- 
		(166.729000,-7.853000) -- 
		(165.138000,-7.984000) -- 
		(163.254000,-8.027000) -- 
		(162.529000,-8.021000) -- 
		(162.755000,-8.297000) -- 
		(162.882000,-8.453000) -- 
		(162.964000,-8.553000) -- 
		(163.770000,-9.458000) -- 
		(163.975000,-9.719000) -- 
		(164.319000,-10.155000) -- 
		(164.405000,-10.265000) -- 
		(165.181000,-11.159000) -- 
		(165.338000,-11.344000) -- 
		(165.346000,-11.354000) -- 
		(165.368000,-11.380000) -- 
		(165.375000,-11.388000) -- 
		(165.578000,-11.631000) -- 
		(165.730000,-11.813000) -- 
		(166.189000,-12.354000) -- 
		(166.328000,-12.518000) -- 
		(166.394000,-12.593000) -- 
		(166.369000,-12.604000) -- 
		(165.718000,-12.871000) -- 
		(166.866000,-13.140000) -- 
		(166.939000,-13.216000) -- 
		(167.356000,-13.748000) -- 
		(167.798000,-14.252000) -- 
		(167.990000,-14.480000) -- 
		(168.403000,-14.967000) -- 
		(168.758000,-15.384000) -- 
		(168.928000,-15.582000) -- 
		(169.701000,-16.489000) -- 
		(169.878000,-16.651000) -- 
		(170.063000,-16.781000) -- 
		(170.166000,-16.823000) -- 
		(170.360000,-16.902000) -- 
		(170.411000,-16.914000) -- 
		(170.504000,-16.934000) -- 
		(170.543000,-16.942000) -- 
		(170.841000,-17.007000) -- 
		(171.207000,-17.024000) -- 
		(171.562000,-17.017000) -- 
		(171.570000,-17.441000) -- 
		(171.553000,-18.339000) -- 
		(171.554000,-18.716000) -- 
		(171.555000,-19.140000) -- 
		(171.555000,-19.159000) -- 
		(171.329000,-19.356000) -- 
		(171.101000,-19.470000) -- 
		(170.775000,-19.500000) -- 
		(170.131000,-19.558000) -- 
		(169.813000,-19.540000) -- 
		(169.681000,-19.482000) -- 
		(169.628000,-19.430000) -- 
		(169.400000,-19.417000) -- 
		(169.014000,-19.434000) -- 
		(168.280000,-19.480000) -- 
		(168.111000,-19.479000) -- 
		(167.944000,-19.478000) -- 
		(167.330000,-19.627000) -- 
		(167.237000,-19.671000) -- 
		(167.235000,-19.639000) -- 
		(167.229000,-19.536000) -- 
		(167.227000,-19.501000) -- 
		(165.974000,-19.560000) -- 
		(163.778000,-19.662000) -- 
		(161.424000,-19.376000) -- 
		(160.450000,-19.304000) -- 
		(159.761000,-19.253000) -- 
		(158.069000,-18.823000) -- 
		(149.809000,-16.720000) -- 
		(141.672000,-16.922000) -- 
		(137.457000,-16.551000) -- 
		(132.443000,-16.605000) -- 
		(132.118000,-16.608000) -- 
		(130.637000,-16.639000) -- 
		(121.400000,-16.827000) -- 
		(120.552000,-16.839000) -- 
		(120.042000,-16.850000) -- 
		(115.966000,-16.930000) -- 
		(114.607000,-16.956000) -- 
		(113.969000,-16.969000) -- 
		(112.519000,-16.998000) -- 
		(106.252000,-17.123000) -- 
		(104.163000,-17.164000) -- 
		(101.652000,-17.240000) -- 
		(101.421000,-17.247000) -- 
		(101.413000,-17.248000) -- 
		(99.794000,-17.367000) -- 
		(96.245000,-17.935000) -- 
		(93.256000,-18.478000) -- 
		(91.920000,-18.720000) -- 
		(90.654000,-18.887000) -- 
		(90.537000,-18.903000) -- 
		(90.520000,-18.829000) -- 
		(90.515000,-18.811000) -- 
		(90.507000,-18.772000) -- 
		(90.477000,-18.628000) -- 
		(90.472000,-18.603000) -- 
		(90.471000,-18.601000) -- 
		(90.460000,-18.527000) -- 
		(90.442000,-18.415000) -- 
		(90.425000,-18.303000) -- 
		(90.387000,-18.091000) -- 
		(90.303000,-17.814000) -- 
		(90.228000,-17.608000) -- 
		(90.158000,-17.479000) -- 
		(89.848000,-17.095000) -- 
		(89.507000,-16.685000) -- 
		(89.259000,-16.439000) -- 
		(89.208000,-16.392000) -- 
		(89.001000,-16.199000) -- 
		(88.834000,-16.037000) -- 
		(88.730000,-15.881000) -- 
		(88.719000,-15.859000) -- 
		(88.626000,-15.685000) -- 
		(88.615000,-15.651000) -- 
		(88.595000,-15.619000) -- 
		(88.567000,-15.592000) -- 
		(88.535000,-15.568000) -- 
		(88.527000,-15.533000) -- 
		(88.511000,-15.500000) -- 
		(88.250000,-15.091000) -- 
		(88.082000,-14.803000) -- 
		(88.031000,-14.705000) -- 
		(88.019000,-14.670000) -- 
		(87.886000,-14.369000) -- 
		(87.795000,-14.207000) -- 
		(87.626000,-13.958000) -- 
		(87.423000,-13.691000) -- 
		(87.198000,-13.380000) -- 
		(86.948000,-12.965000) -- 
		(86.786000,-12.755000) -- 
		(86.734000,-12.699000) -- 
		(86.602000,-12.534000) -- 
		(86.448000,-12.343000) -- 
		(86.183000,-12.116000) -- 
		(86.060000,-12.011000) -- 
		(85.947000,-11.898000) -- 
		(85.839000,-11.791000) -- 
		(85.634000,-11.564000) -- 
		(85.608000,-11.535000) -- 
		(85.411000,-11.259000) -- 
		(85.364000,-11.209000) -- 
		(85.408000,-11.061000) -- 
		(85.407000,-11.025000) -- 
		(85.390000,-10.955000) -- 
		(85.377000,-10.921000) -- 
		(85.353000,-10.894000) -- 
		(85.312000,-10.890000) -- 
		(85.269000,-10.897000) -- 
		(85.156000,-10.946000) -- 
		(85.114000,-10.953000) -- 
		(84.999000,-11.000000) -- 
		(84.916000,-10.991000) -- 
		(84.707000,-10.959000) -- 
		(84.580000,-10.948000) -- 
		(84.451000,-10.930000) -- 
		(83.526000,-10.799000) -- 
		(83.059000,-10.758000) -- 
		(83.032000,-10.752000) -- 
		(82.977000,-10.739000) -- 
		(82.869000,-10.729000) -- 
		(82.765000,-10.719000) -- 
		(82.658000,-10.725000) -- 
		(82.468000,-10.748000) -- 
		(82.383000,-10.742000) -- 
		(82.299000,-10.729000) -- 
		(82.219000,-10.704000) -- 
		(82.140000,-10.675000) -- 
		(81.906000,-10.588000) -- 
		(81.870000,-10.569000) -- 
		(81.749000,-10.493000) -- 
		(81.700000,-10.461000) -- 
		(81.634000,-10.408000) -- 
		(81.577000,-10.362000) -- 
		(81.489000,-10.283000) -- 
		(81.436000,-10.227000) -- 
		(81.338000,-10.110000) -- 
		(81.294000,-10.062000) -- 
		(81.261000,-10.025000) -- 
		(81.153000,-9.916000) -- 
		(80.974000,-9.764000) -- 
		(80.952000,-9.734000) -- 
		(80.876000,-9.648000) -- 
		(80.792000,-9.568000) -- 
		(80.632000,-9.400000) -- 
		(80.603000,-9.375000) -- 
		(80.563000,-9.363000) -- 
		(80.551000,-9.357000) -- 
		(80.526000,-9.345000) -- 
		(80.392000,-9.256000) -- 
		(80.363000,-9.230000) -- 
		(80.315000,-9.171000) -- 
		(80.235000,-9.095000) -- 
		(80.146000,-9.009000) -- 
		(80.047000,-8.892000) -- 
		(80.104000,-8.720000) -- 
		(80.108000,-8.704000) -- 
		(80.120000,-8.650000) -- 
		(80.123000,-8.614000) -- 
		(80.103000,-8.292000) -- 
		(80.102000,-8.283000) -- 
		(80.086000,-8.134000) -- 
		(80.071000,-7.985000) -- 
		(80.058000,-7.864000) -- 
		(80.020000,-7.580000) -- 
		(79.942000,-7.265000) -- 
		(79.911000,-7.197000) -- 
		(79.891000,-7.166000) -- 
		(79.927000,-7.053000) -- 
		(79.986000,-6.910000) -- 
		(80.025000,-6.816000) -- 
		(80.156000,-6.551000) -- 
		(80.264000,-6.357000) -- 
		(80.348000,-6.174000) -- 
		(80.417000,-6.023000) -- 
		(80.578000,-5.731000) -- 
		(80.659000,-5.528000) -- 
		(80.721000,-5.394000) -- 
		(80.930000,-5.073000) -- 
		(80.966000,-5.017000) -- 
		(81.272000,-4.630000) -- 
		(81.359000,-4.530000) -- 
		(81.522000,-4.341000) -- 
		(81.665000,-4.164000) -- 
		(81.896000,-3.821000) -- 
		(81.911000,-3.787000) -- 
		(81.948000,-3.723000) -- 
		(81.978000,-3.657000) -- 
		(81.997000,-3.625000) -- 
		(82.089000,-3.349000) -- 
		(82.122000,-3.102000) -- 
		(82.137000,-3.031000) -- 
		(82.145000,-2.961000) -- 
		(82.135000,-2.854000) -- 
		(82.138000,-2.639000) -- 
		(82.147000,-2.604000) -- 
		(82.125000,-2.574000) -- 
		(82.050000,-2.427000) -- 
		(81.990000,-2.311000) -- 
		(81.902000,-2.109000) -- 
		(81.742000,-1.778000) -- 
		(81.533000,-1.426000) -- 
		(81.384000,-1.131000) -- 
		(81.195000,-0.814000) -- 
		(81.194000,-0.812000) -- 
		(81.055000,-0.513000) -- 
		(80.927000,-0.328000) -- 
		(80.836000,-0.207000) -- 
		(80.796000,-0.144000) -- 
		(80.670000,0.000000) -- 
		(80.583000,0.078000) -- 
		(80.329000,0.268000) -- 
		(80.195000,0.357000) -- 
		(80.091000,0.420000) -- 
		(79.913000,0.516000) -- 
		(79.809000,0.577000) -- 
		(79.515000,0.785000) -- 
		(79.485000,0.810000) -- 
		(79.352000,0.946000) -- 
		(79.321000,0.971000) -- 
		(79.242000,1.054000) -- 
		(79.099000,1.230000) -- 
		(78.803000,1.621000) -- 
		(78.670000,1.768000) -- 
		(78.393000,2.076000) -- 
		(78.321000,2.115000) -- 
		(78.171000,2.182000) -- 
		(77.897000,2.276000) -- 
		(77.842000,2.303000) -- 
		(77.787000,2.331000) -- 
		(77.712000,2.363000) -- 
		(77.201000,2.683000) -- 
		(77.068000,2.771000) -- 
		(76.973000,2.843000) -- 
		(76.871000,2.956000) -- 
		(76.781000,3.032000) -- 
		(76.666000,3.138000) -- 
		(76.442000,3.300000) -- 
		(76.407000,3.328000) -- 
		(76.381000,3.349000) -- 
		(76.326000,3.404000) -- 
		(76.081000,3.695000) -- 
		(75.994000,3.774000) -- 
		(75.901000,3.848000) -- 
		(75.804000,3.918000) -- 
		(75.768000,3.937000) -- 
		(75.692000,3.969000) -- 
		(75.657000,3.990000) -- 
		(75.598000,4.042000) -- 
		(75.578000,4.073000) -- 
		(75.559000,4.088000) -- 
		(75.558000,4.089000) -- 
		(75.516000,4.122000) -- 
		(75.418000,4.190000) -- 
		(75.337000,4.210000) -- 
		(75.331000,4.213000) -- 
		(75.184000,4.275000) -- 
		(75.144000,4.288000) -- 
		(74.902000,4.435000) -- 
		(74.828000,4.472000) -- 
		(74.751000,4.501000) -- 
		(74.067000,4.708000) -- 
		(73.428000,4.912000) -- 
		(72.940000,5.039000) -- 
		(72.656000,5.117000) -- 
		(72.215000,5.255000) -- 
		(72.134000,5.276000) -- 
		(71.635000,5.368000) -- 
		(71.464000,5.366000) -- 
		(71.296000,5.342000) -- 
		(71.172000,5.313000) -- 
		(70.849000,5.223000) -- 
		(70.433000,5.038000) -- 
		(70.181000,4.904000) -- 
		(69.899000,4.743000) -- 
		(69.831000,4.700000) -- 
		(69.581000,4.505000) -- 
		(69.544000,4.487000) -- 
		(69.479000,4.441000) -- 
		(69.309000,4.333000) -- 
		(69.236000,4.296000) -- 
		(69.197000,4.281000) -- 
		(69.036000,4.232000) -- 
		(68.952000,4.218000) -- 
		(68.910000,4.217000) -- 
		(68.783000,4.234000) -- 
		(68.742000,4.244000) -- 
		(68.547000,4.317000) -- 
		(68.300000,4.458000) -- 
		(68.134000,4.570000) -- 
		(67.806000,4.851000) -- 
		(67.620000,5.047000) -- 
		(67.494000,5.191000) -- 
		(67.415000,5.318000) -- 
		(67.356000,5.490000) -- 
		(67.284000,5.657000) -- 
		(67.048000,5.997000) -- 
		(67.010000,6.061000) -- 
		(66.926000,6.226000) -- 
		(66.854000,6.314000) -- 
		(66.690000,6.478000) -- 
		(66.595000,6.597000) -- 
		(66.490000,6.709000) -- 
		(66.417000,6.797000) -- 
		(66.251000,6.960000) -- 
		(66.177000,7.047000) -- 
		(65.929000,7.292000) -- 
		(65.877000,7.348000) -- 
		(65.847000,7.374000) -- 
		(65.667000,7.573000) -- 
		(65.650000,7.606000) -- 
		(65.536000,7.951000) -- 
		(65.526000,7.998000) -- 
		(65.578000,8.007000) -- 
		(65.661000,8.021000) -- 
		(67.268000,8.303000) -- 
		(68.910000,8.571000) -- 
		(70.764000,8.875000) -- 
		(73.347000,9.289000) -- 
		(75.017000,9.616000) -- 
		(76.372000,9.861000) -- 
		(77.272000,10.007000) -- 
		(78.196000,10.229000) -- 
		(79.022000,10.467000) -- 
		(79.233000,10.528000) -- 
		(79.771000,10.696000) -- 
		(80.860000,11.125000) -- 
		(81.849000,11.528000) -- 
		(82.218000,11.707000) -- 
		(82.520000,11.853000) -- 
		(82.803000,11.999000) -- 
		(84.545000,12.902000) -- 
		(85.125000,13.203000) -- 
		(85.959000,13.626000) -- 
		(88.461000,14.898000) -- 
		(88.883000,15.113000) -- 
		(89.294000,15.322000) -- 
		(89.739000,15.548000) -- 
		(89.787000,15.571000) -- 
		(91.103000,16.223000) -- 
		(91.269000,16.308000) -- 
		(91.757000,16.565000) -- 
		(91.812000,16.592000) -- 
		(91.976000,16.679000) -- 
		(92.030000,16.708000) -- 
		(92.026000,16.737000) -- 
		(92.013000,16.826000) -- 
		(92.008000,16.856000) -- 
		(92.003000,16.893000) -- 
		(91.986000,17.009000) -- 
		(91.980000,17.048000) -- 
		(91.893000,17.585000) -- 
		(91.878000,17.677000) -- 
		(91.807000,18.241000) -- 
		(91.805000,18.464000) -- 
		(91.815000,18.577000) -- 
		(91.898000,18.929000) -- 
		(91.980000,19.189000) -- 
		(92.105000,19.590000) -- 
		(92.152000,19.706000) -- 
		(92.223000,19.879000) -- 
		(92.229000,19.895000) -- 
		(92.483000,20.377000) -- 
		(92.570000,20.544000) -- 
		(92.747000,20.822000) -- 
		(92.863000,21.009000) -- 
		(93.032000,21.242000) -- 
		(93.381000,21.578000) -- 
		(93.618000,21.808000) -- 
		(93.689000,21.877000) -- 
		(93.902000,22.086000) -- 
		(94.138000,22.319000) -- 
		(94.193000,22.389000) -- 
		(94.626000,22.902000) -- 
		(94.772000,23.078000) -- 
		(94.814000,23.128000) -- 
		(95.343000,23.907000) -- 
		(95.431000,24.068000) -- 
		(95.639000,24.448000) -- 
		(95.826000,24.835000) -- 
		(96.154000,25.522000) -- 
		(96.275000,25.767000) -- 
		(96.390000,26.000000) -- 
		(96.499000,26.220000) -- 
		(96.750000,26.797000) -- 
		(96.750000,26.798000) -- 
		(96.795000,26.902000) -- 
		(96.896000,27.197000) -- 
		(96.933000,27.305000) -- 
		(97.033000,27.830000) -- 
		(97.061000,27.953000) -- 
		(97.235000,28.738000) -- 
		(97.275000,28.868000) -- 
		(97.302000,28.955000) -- 
		(97.302000,28.956000) -- 
		(97.537000,29.718000) -- 
		(97.537000,29.719000) -- 
		(97.569000,29.823000) -- 
		(97.700000,30.086000) -- 
		(97.840000,30.336000) -- 
		(97.976000,30.486000) -- 
		(97.976000,30.487000) -- 
		(98.171000,30.702000) -- 
		(98.268000,30.775000) -- 
		(98.440000,30.907000) -- 
		(98.731000,31.085000) -- 
		(99.225000,31.384000) -- 
		(99.225000,31.385000) -- 
		(99.448000,31.524000) -- 
		(100.077000,31.978000) -- 
		(100.271000,32.118000) -- 
		(100.535000,32.366000) -- 
		(100.535000,32.367000) -- 
		(101.005000,32.809000) -- 
		(101.191000,33.031000) -- 
		(101.307000,33.170000) -- 
		(101.641000,33.569000) -- 
		(101.810000,33.795000) -- 
		(101.940000,33.972000) -- 
		(102.017000,34.076000) -- 
		(102.053000,34.123000) -- 
		(102.419000,34.791000) -- 
		(102.510000,35.040000) -- 
		(102.674000,35.489000) -- 
		(102.867000,36.158000) -- 
		(102.947000,36.149000) -- 
		(103.426000,36.142000) -- 
		(103.961000,36.143000) -- 
		(106.688000,36.160000) -- 
		(107.245000,36.153000) -- 
		(108.339000,36.142000) -- 
		(110.017000,36.167000) -- 
		(110.897000,36.165000) -- 
		(110.935000,36.165000) -- 
		(111.704000,36.172000) -- 
		(112.330000,36.177000) -- 
		(112.659000,36.173000) -- 
		(113.242000,36.173000) -- 
		(114.335000,36.168000) -- 
		(115.173000,36.186000) -- 
		(115.248000,36.188000) -- 
		(115.822000,36.191000) -- 
		(116.048000,36.192000) -- 
		(116.899000,36.248000) -- 
		(118.717000,36.368000) -- 
		(118.747000,36.371000) -- 
		(119.160000,36.379000) -- 
		(119.631000,36.327000) -- 
		(119.632000,36.326000) -- 
		(119.804000,36.296000) -- 
		(120.399000,36.249000) -- 
		(121.304000,36.219000) -- 
		(121.852000,36.222000) -- 
		(122.720000,36.229000) -- 
		(123.496000,36.252000) -- 
		(123.808000,36.262000) -- 
		(124.043000,36.248000) -- 
		(124.593000,36.214000) -- 
		(125.230000,36.175000) -- 
		(126.026000,36.182000) -- 
		(126.242000,36.190000) -- 
		(126.791000,36.211000) -- 
		(127.045000,36.219000) -- 
		(127.312000,36.231000) -- 
		(127.805000,36.225000) -- 
		(128.058000,36.223000) -- 
		(128.110000,35.968000) -- 
		(128.246000,35.469000) -- 
		(128.298000,35.182000) -- 
		(128.422000,34.485000) -- 
		(128.535000,33.799000) -- 
		(128.680000,32.766000) -- 
		(128.699000,32.479000) -- 
		(128.678000,32.008000) -- 
		(128.668000,31.797000) -- 
		(128.651000,30.940000) -- 
		(128.650000,30.874000) -- 
		(128.646000,30.678000) -- 
		(128.644000,30.613000) -- 
		(128.644000,30.551000) -- 
		(128.640000,30.367000) -- 
		(128.638000,30.306000) -- 
		(128.633000,30.096000) -- 
		(128.633000,30.095000) -- 
		(128.629000,29.943000) -- 
		(128.599000,28.860000) -- 
		(128.588000,28.499000) -- 
		(128.577000,28.037000) -- 
		(128.575000,27.909000) -- 
		(128.563000,27.362000) -- 
		(128.547000,26.901000) -- 
		(128.539000,26.506000) -- 
		(128.524000,26.141000) -- 
		(128.520000,26.047000) -- 
		(128.514000,25.757000) -- 
		(128.509000,25.553000) -- 
		(128.507000,25.443000) -- 
		(128.501000,25.121000) -- 
		(128.498000,25.013000) -- 
		(128.497000,24.908000) -- 
		(128.493000,24.593000) -- 
		(128.491000,24.490000) -- 
		(128.488000,24.281000) -- 
		(128.482000,23.893000) -- 
		(128.463000,22.859000) -- 
		(128.449000,22.105000) -- 
		(128.437000,21.509000) -- 
		(128.436000,21.410000) -- 
		(128.429000,20.603000) -- 
		(128.421000,19.712000) -- 
		(128.395000,18.354000) -- 
		(128.402000,17.889000) -- 
		(128.414000,16.984000) -- 
		(128.405000,16.209000) -- 
		(128.401000,15.971000) -- 
		(128.401000,15.970000) -- 
		(128.403000,15.571000) -- 
		(128.405000,15.186000) -- 
		(128.406000,15.019000) -- 
		(128.407000,14.846000) -- 
		(128.407000,14.673000) -- 
		(128.403000,14.105000) -- 
		(128.403000,14.104000) -- 
		(128.402000,14.002000) -- 
		(128.402000,13.885000) -- 
		(128.402000,13.111000) -- 
		(128.403000,12.989000) -- 
		(128.403000,12.623000) -- 
		(128.403000,12.501000) -- 
		(128.414000,12.210000) -- 
		(128.444000,11.341000) -- 
		(128.454000,11.052000) -- 
		(128.452000,10.865000) -- 
		(128.448000,10.568000) -- 
		(128.445000,10.310000) -- 
		(128.442000,10.125000) -- 
		(128.432000,9.299000) -- 
		(128.409000,7.467000) -- 
		(128.408000,7.395000) -- 
		(128.412000,7.201000) -- 
		(128.438000,6.821000) -- 
		(128.494000,5.998000) -- 
		(129.234000,6.046000) -- 
		(129.610000,6.070000) -- 
		(130.563000,6.143000) -- 
		(130.979000,6.175000) -- 
		(131.195000,6.179000) -- 
		(131.195000,6.180000) -- 
		(131.923000,6.199000) -- 
		(131.923000,6.200000) -- 
		(132.315000,6.211000) -- 
		(132.454000,6.213000) -- 
		(132.777000,6.224000) -- 
		(132.868000,6.219000) -- 
		(133.006000,6.212000) -- 
		(133.094000,6.207000) -- 
		(133.241000,6.200000) -- 
		(133.358000,6.193000) -- 
		(133.445000,6.189000) -- 
		(133.580000,6.179000) -- 
		(133.984000,6.154000) -- 
		(134.118000,6.148000) -- 
		(134.422000,6.126000) -- 
		(135.274000,6.071000) -- 
		(135.331000,6.066000) -- 
		(135.633000,6.042000) -- 
		(135.873000,6.022000) -- 
		(136.815000,5.981000) -- 
		(136.984000,5.955000) -- 
		(137.045000,5.945000) -- 
		(137.194000,5.931000) -- 
		(138.287000,5.856000) -- 
		(139.290000,5.917000) -- 
		(139.738000,5.958000) -- 
		(140.105000,5.998000) -- 
		(140.195000,6.011000) -- 
		(141.302000,6.175000) -- 
		(141.851000,6.255000) -- 
		(142.776000,6.394000) -- 
		(143.465000,6.480000) -- 
		(143.499000,6.483000) -- 
		(144.049000,6.553000) -- 
		(144.179000,6.569000) -- 
		(144.415000,6.599000) -- 
		(144.564000,6.624000) -- 
		(144.692000,6.648000) -- 
		(145.025000,6.706000) -- 
		(145.097000,6.719000) -- 
		(145.483000,6.766000) -- 
		(145.726000,6.797000) -- 
		(146.030000,6.842000) -- 
		(146.364000,6.893000) -- 
		(146.617000,6.928000) -- 
		(147.376000,7.034000) -- 
		(147.628000,7.070000) -- 
		(147.726000,7.080000) -- 
		(147.968000,7.111000) -- 
		(148.019000,7.119000) -- 
		(148.115000,7.136000) -- 
		(148.151000,7.142000) -- 
		(148.258000,7.160000) -- 
		(148.293000,7.167000) -- 
		(148.422000,7.189000) -- 
		(148.589000,7.220000) -- 
		(148.798000,7.259000) -- 
		(149.476000,7.384000) -- 
		(149.542000,7.398000) -- 
		(149.768000,7.454000) -- 
		(149.972000,7.505000) -- 
		(150.488000,7.633000);
	\filldraw [draw=black, ultra thick, fill=blue]
		(190.239000,-52.177000) -- 
		(190.232000,-51.881000) -- 
		(190.145000,-51.016000) -- 
		(190.088000,-50.577000) -- 
		(190.087000,-50.576000) -- 
		(190.042000,-50.436000) -- 
		(189.872000,-49.992000) -- 
		(189.130000,-47.926000) -- 
		(189.046000,-47.682000) -- 
		(188.997000,-47.541000) -- 
		(188.845000,-47.139000) -- 
		(188.781000,-46.955000) -- 
		(188.696000,-46.709000) -- 
		(188.674000,-46.645000) -- 
		(188.606000,-46.449000) -- 
		(188.583000,-46.383000) -- 
		(188.536000,-46.252000) -- 
		(188.392000,-45.860000) -- 
		(188.385000,-45.841000) -- 
		(188.343000,-45.728000) -- 
		(188.276000,-45.542000) -- 
		(188.275000,-45.541000) -- 
		(188.228000,-45.411000) -- 
		(187.886000,-44.456000) -- 
		(187.771000,-44.137000) -- 
		(187.419000,-43.157000) -- 
		(187.231000,-42.630000) -- 
		(186.922000,-41.778000) -- 
		(186.838000,-41.537000) -- 
		(186.769000,-41.341000) -- 
		(186.737000,-41.242000) -- 
		(186.702000,-41.136000) -- 
		(186.592000,-40.821000) -- 
		(186.441000,-40.385000) -- 
		(185.875000,-38.831000) -- 
		(185.453000,-37.709000) -- 
		(185.311000,-37.330000) -- 
		(185.258000,-37.187000) -- 
		(184.787000,-37.218000) -- 
		(184.623000,-37.216000) -- 
		(183.934000,-37.206000) -- 
		(183.933000,-37.205000) -- 
		(183.635000,-37.185000) -- 
		(183.441000,-37.172000) -- 
		(182.743000,-37.141000) -- 
		(181.198000,-37.094000) -- 
		(179.873000,-37.072000) -- 
		(179.696000,-37.073000) -- 
		(178.762000,-37.078000) -- 
		(177.136000,-37.086000) -- 
		(177.091000,-37.088000) -- 
		(176.954000,-37.088000) -- 
		(176.908000,-37.088000) -- 
		(176.708000,-37.089000) -- 
		(176.107000,-37.092000) -- 
		(175.906000,-37.092000) -- 
		(173.714000,-37.103000) -- 
		(172.309000,-37.124000) -- 
		(171.909000,-38.080000) -- 
		(170.745000,-39.763000) -- 
		(170.723000,-39.843000) -- 
		(170.707000,-39.962000) -- 
		(170.745000,-40.339000) -- 
		(170.960000,-41.540000) -- 
		(170.930000,-41.784000) -- 
		(170.912000,-41.881000) -- 
		(170.857000,-41.960000) -- 
		(170.722000,-42.085000) -- 
		(170.056000,-42.462000) -- 
		(169.835000,-42.575000) -- 
		(169.697000,-42.616000) -- 
		(169.496000,-42.606000) -- 
		(169.309000,-42.565000) -- 
		(169.066000,-42.531000) -- 
		(168.900000,-42.555000) -- 
		(168.662000,-42.669000) -- 
		(168.479000,-42.757000) -- 
		(168.381000,-42.821000) -- 
		(167.803000,-43.194000) -- 
		(167.599000,-43.378000) -- 
		(167.359000,-43.591000) -- 
		(167.349000,-43.600000) -- 
		(167.052000,-43.877000) -- 
		(167.129000,-38.722000) -- 
		(167.133000,-38.499000) -- 
		(167.145000,-37.826000) -- 
		(167.148000,-37.602000) -- 
		(167.186000,-35.223000) -- 
		(167.237000,-32.009000) -- 
		(167.300000,-28.085000) -- 
		(167.337000,-25.704000) -- 
		(167.319000,-24.597000) -- 
		(167.318000,-24.498000) -- 
		(167.258000,-20.878000) -- 
		(167.237000,-19.671000) -- 
		(167.330000,-19.627000) -- 
		(167.944000,-19.478000) -- 
		(168.111000,-19.479000) -- 
		(168.280000,-19.480000) -- 
		(169.014000,-19.434000) -- 
		(169.400000,-19.417000) -- 
		(169.628000,-19.430000) -- 
		(169.681000,-19.482000) -- 
		(169.813000,-19.540000) -- 
		(170.131000,-19.558000) -- 
		(170.775000,-19.500000) -- 
		(171.101000,-19.470000) -- 
		(171.329000,-19.356000) -- 
		(171.555000,-19.159000) -- 
		(171.555000,-19.140000) -- 
		(171.554000,-18.716000) -- 
		(171.553000,-18.339000) -- 
		(171.570000,-17.441000) -- 
		(171.562000,-17.017000) -- 
		(171.207000,-17.024000) -- 
		(170.841000,-17.007000) -- 
		(170.543000,-16.942000) -- 
		(170.504000,-16.934000) -- 
		(170.411000,-16.914000) -- 
		(170.360000,-16.902000) -- 
		(170.166000,-16.823000) -- 
		(170.063000,-16.781000) -- 
		(169.878000,-16.651000) -- 
		(169.701000,-16.489000) -- 
		(168.928000,-15.582000) -- 
		(168.758000,-15.384000) -- 
		(168.403000,-14.967000) -- 
		(167.990000,-14.480000) -- 
		(167.798000,-14.252000) -- 
		(167.356000,-13.748000) -- 
		(166.939000,-13.216000) -- 
		(166.866000,-13.140000) -- 
		(165.718000,-12.871000) -- 
		(166.369000,-12.604000) -- 
		(166.394000,-12.593000) -- 
		(166.328000,-12.518000) -- 
		(166.189000,-12.354000) -- 
		(165.730000,-11.813000) -- 
		(165.578000,-11.631000) -- 
		(165.375000,-11.388000) -- 
		(165.368000,-11.380000) -- 
		(165.346000,-11.354000) -- 
		(165.338000,-11.344000) -- 
		(165.181000,-11.159000) -- 
		(164.405000,-10.265000) -- 
		(164.319000,-10.155000) -- 
		(163.975000,-9.719000) -- 
		(163.770000,-9.458000) -- 
		(162.964000,-8.553000) -- 
		(162.882000,-8.453000) -- 
		(162.755000,-8.297000) -- 
		(162.529000,-8.021000) -- 
		(163.254000,-8.027000) -- 
		(165.138000,-7.984000) -- 
		(166.729000,-7.853000) -- 
		(167.161000,-7.826000) -- 
		(168.243000,-7.708000) -- 
		(168.477000,-7.683000) -- 
		(169.537000,-7.516000) -- 
		(170.748000,-7.287000) -- 
		(171.442000,-7.154000) -- 
		(173.264000,-6.665000) -- 
		(174.169000,-6.384000) -- 
		(175.825000,-5.817000) -- 
		(177.383000,-5.176000) -- 
		(177.963000,-4.895000) -- 
		(179.131000,-4.327000) -- 
		(180.207000,-3.707000) -- 
		(180.305000,-3.651000) -- 
		(180.596000,-3.485000) -- 
		(180.693000,-3.427000) -- 
		(181.569000,-2.911000) -- 
		(181.895000,-2.716000) -- 
		(183.199000,-1.784000) -- 
		(184.058000,-1.149000) -- 
		(184.706000,-0.669000) -- 
		(184.816000,-0.586000) -- 
		(184.874000,-0.543000) -- 
		(184.871000,-0.122000) -- 
		(184.862000,0.866000) -- 
		(184.867000,1.144000) -- 
		(184.873000,1.566000) -- 
		(184.867000,1.769000) -- 
		(184.862000,1.989000) -- 
		(184.888000,2.396000) -- 
		(184.886000,2.753000) -- 
		(184.886000,3.002000) -- 
		(184.885000,3.087000) -- 
		(184.883000,3.782000) -- 
		(184.886000,4.418000) -- 
		(184.891000,5.813000) -- 
		(184.891000,5.814000) -- 
		(184.895000,6.327000) -- 
		(184.898000,6.964000) -- 
		(184.901000,7.521000) -- 
		(184.896000,8.151000) -- 
		(184.893000,8.439000) -- 
		(184.887000,9.225000) -- 
		(184.907000,10.499000) -- 
		(184.870000,11.602000) -- 
		(184.872000,11.715000) -- 
		(184.891000,12.903000) -- 
		(184.892000,12.948000) -- 
		(184.894000,13.084000) -- 
		(184.894000,13.130000) -- 
		(184.699000,13.099000) -- 
		(184.588000,13.082000) -- 
		(184.111000,13.027000) -- 
		(183.914000,13.005000) -- 
		(183.291000,12.934000) -- 
		(183.046000,12.907000) -- 
		(182.568000,12.855000) -- 
		(181.661000,12.758000) -- 
		(181.629000,12.755000) -- 
		(180.439000,12.627000) -- 
		(179.569000,12.537000) -- 
		(179.298000,12.508000) -- 
		(179.209000,12.494000) -- 
		(179.042000,12.470000) -- 
		(178.128000,12.362000) -- 
		(178.025000,12.350000) -- 
		(177.767000,12.322000) -- 
		(177.470000,12.288000) -- 
		(177.334000,12.274000) -- 
		(176.577000,12.179000) -- 
		(176.279000,12.142000) -- 
		(175.474000,12.058000) -- 
		(175.118000,12.023000) -- 
		(174.432000,11.935000) -- 
		(173.673000,11.838000) -- 
		(173.063000,11.776000) -- 
		(172.658000,11.735000) -- 
		(172.260000,11.676000) -- 
		(171.622000,11.582000) -- 
		(170.771000,11.457000) -- 
		(170.381000,11.400000) -- 
		(169.866000,11.339000) -- 
		(168.526000,11.191000) -- 
		(167.329000,11.046000) -- 
		(166.348000,10.930000) -- 
		(165.600000,10.841000) -- 
		(164.785000,10.746000) -- 
		(163.354000,10.572000) -- 
		(162.946000,10.524000) -- 
		(162.607000,10.468000) -- 
		(162.494000,10.443000) -- 
		(162.153000,10.371000) -- 
		(162.039000,10.347000) -- 
		(161.349000,10.186000) -- 
		(161.185000,10.148000) -- 
		(160.759000,10.048000) -- 
		(160.594000,10.010000) -- 
		(159.835000,9.833000) -- 
		(159.615000,9.782000) -- 
		(159.384000,9.728000) -- 
		(159.265000,9.700000) -- 
		(158.212000,9.457000) -- 
		(157.861000,9.376000) -- 
		(157.570000,9.308000) -- 
		(156.693000,9.105000) -- 
		(156.401000,9.038000) -- 
		(156.043000,8.954000) -- 
		(155.260000,8.774000) -- 
		(154.966000,8.705000) -- 
		(154.607000,8.623000) -- 
		(154.322000,8.557000) -- 
		(154.031000,8.490000) -- 
		(153.656000,8.402000) -- 
		(153.465000,8.356000) -- 
		(153.179000,8.288000) -- 
		(152.942000,8.232000) -- 
		(152.700000,8.174000) -- 
		(151.493000,7.886000) -- 
		(151.264000,7.828000) -- 
		(150.786000,7.709000) -- 
		(150.583000,7.657000) -- 
		(150.488000,7.633000) -- 
		(149.972000,7.505000) -- 
		(149.768000,7.454000) -- 
		(149.542000,7.398000) -- 
		(149.476000,7.384000) -- 
		(148.798000,7.259000) -- 
		(148.589000,7.220000) -- 
		(148.422000,7.189000) -- 
		(148.293000,7.167000) -- 
		(148.258000,7.160000) -- 
		(148.151000,7.142000) -- 
		(148.115000,7.136000) -- 
		(148.019000,7.119000) -- 
		(147.968000,7.111000) -- 
		(147.726000,7.080000) -- 
		(147.628000,7.070000) -- 
		(147.376000,7.034000) -- 
		(146.617000,6.928000) -- 
		(146.364000,6.893000) -- 
		(146.030000,6.842000) -- 
		(145.726000,6.797000) -- 
		(145.483000,6.766000) -- 
		(145.097000,6.719000) -- 
		(145.025000,6.706000) -- 
		(144.692000,6.648000) -- 
		(144.564000,6.624000) -- 
		(144.415000,6.599000) -- 
		(144.179000,6.569000) -- 
		(144.049000,6.553000) -- 
		(143.499000,6.483000) -- 
		(143.465000,6.480000) -- 
		(142.776000,6.394000) -- 
		(141.851000,6.255000) -- 
		(141.302000,6.175000) -- 
		(140.195000,6.011000) -- 
		(140.105000,5.998000) -- 
		(139.738000,5.958000) -- 
		(139.290000,5.917000) -- 
		(138.287000,5.856000) -- 
		(137.194000,5.931000) -- 
		(137.045000,5.945000) -- 
		(136.984000,5.955000) -- 
		(136.815000,5.981000) -- 
		(135.873000,6.022000) -- 
		(135.633000,6.042000) -- 
		(135.331000,6.066000) -- 
		(135.274000,6.071000) -- 
		(134.422000,6.126000) -- 
		(134.118000,6.148000) -- 
		(133.984000,6.154000) -- 
		(133.580000,6.179000) -- 
		(133.445000,6.189000) -- 
		(133.358000,6.193000) -- 
		(133.241000,6.200000) -- 
		(133.094000,6.207000) -- 
		(133.006000,6.212000) -- 
		(132.868000,6.219000) -- 
		(132.777000,6.224000) -- 
		(132.454000,6.213000) -- 
		(132.315000,6.211000) -- 
		(131.923000,6.200000) -- 
		(131.923000,6.199000) -- 
		(131.195000,6.180000) -- 
		(131.195000,6.179000) -- 
		(130.979000,6.175000) -- 
		(130.563000,6.143000) -- 
		(129.610000,6.070000) -- 
		(129.234000,6.046000) -- 
		(128.494000,5.998000) -- 
		(128.438000,6.821000) -- 
		(128.412000,7.201000) -- 
		(128.408000,7.395000) -- 
		(128.409000,7.467000) -- 
		(128.432000,9.299000) -- 
		(128.442000,10.125000) -- 
		(128.445000,10.310000) -- 
		(128.448000,10.568000) -- 
		(128.452000,10.865000) -- 
		(128.454000,11.052000) -- 
		(128.444000,11.341000) -- 
		(128.414000,12.210000) -- 
		(128.403000,12.501000) -- 
		(128.403000,12.623000) -- 
		(128.403000,12.989000) -- 
		(128.402000,13.111000) -- 
		(128.402000,13.885000) -- 
		(128.402000,14.002000) -- 
		(128.403000,14.104000) -- 
		(128.403000,14.105000) -- 
		(128.407000,14.673000) -- 
		(128.407000,14.846000) -- 
		(128.406000,15.019000) -- 
		(128.405000,15.186000) -- 
		(128.403000,15.571000) -- 
		(128.401000,15.970000) -- 
		(128.401000,15.971000) -- 
		(128.405000,16.209000) -- 
		(128.414000,16.984000) -- 
		(128.402000,17.889000) -- 
		(128.395000,18.354000) -- 
		(128.421000,19.712000) -- 
		(128.429000,20.603000) -- 
		(128.436000,21.410000) -- 
		(128.437000,21.509000) -- 
		(128.449000,22.105000) -- 
		(128.463000,22.859000) -- 
		(128.482000,23.893000) -- 
		(128.488000,24.281000) -- 
		(128.491000,24.490000) -- 
		(128.493000,24.593000) -- 
		(128.497000,24.908000) -- 
		(128.498000,25.013000) -- 
		(128.501000,25.121000) -- 
		(128.507000,25.443000) -- 
		(128.509000,25.553000) -- 
		(128.514000,25.757000) -- 
		(128.520000,26.047000) -- 
		(128.524000,26.141000) -- 
		(128.539000,26.506000) -- 
		(128.547000,26.901000) -- 
		(128.563000,27.362000) -- 
		(128.575000,27.909000) -- 
		(128.577000,28.037000) -- 
		(128.588000,28.499000) -- 
		(128.599000,28.860000) -- 
		(128.629000,29.943000) -- 
		(128.633000,30.095000) -- 
		(128.633000,30.096000) -- 
		(128.638000,30.306000) -- 
		(128.640000,30.367000) -- 
		(128.644000,30.551000) -- 
		(128.644000,30.613000) -- 
		(128.646000,30.678000) -- 
		(128.650000,30.874000) -- 
		(128.651000,30.940000) -- 
		(128.668000,31.797000) -- 
		(128.678000,32.008000) -- 
		(128.699000,32.479000) -- 
		(128.680000,32.766000) -- 
		(128.535000,33.799000) -- 
		(128.422000,34.485000) -- 
		(128.298000,35.182000) -- 
		(128.246000,35.469000) -- 
		(128.110000,35.968000) -- 
		(128.058000,36.223000) -- 
		(127.805000,36.225000) -- 
		(127.312000,36.231000) -- 
		(127.045000,36.219000) -- 
		(126.791000,36.211000) -- 
		(126.242000,36.190000) -- 
		(126.026000,36.182000) -- 
		(125.230000,36.175000) -- 
		(124.593000,36.214000) -- 
		(124.043000,36.248000) -- 
		(124.059000,36.290000) -- 
		(124.105000,36.416000) -- 
		(124.120000,36.459000) -- 
		(124.228000,36.773000) -- 
		(124.308000,37.009000) -- 
		(124.516000,37.548000) -- 
		(124.575000,37.707000) -- 
		(124.639000,37.884000) -- 
		(124.695000,38.017000) -- 
		(124.866000,38.423000) -- 
		(124.964000,38.747000) -- 
		(124.918000,38.835000) -- 
		(124.749000,38.923000) -- 
		(124.646000,39.022000) -- 
		(124.542000,39.237000) -- 
		(124.516000,39.278000) -- 
		(124.496000,39.308000) -- 
		(124.392000,39.429000) -- 
		(124.379000,39.550000) -- 
		(124.405000,39.627000) -- 
		(124.503000,39.687000) -- 
		(124.769000,39.775000) -- 
		(125.061000,39.852000) -- 
		(125.217000,39.907000) -- 
		(125.308000,39.984000) -- 
		(125.425000,40.034000) -- 
		(125.652000,40.177000) -- 
		(125.685000,40.226000) -- 
		(125.698000,40.281000) -- 
		(125.620000,40.375000) -- 
		(125.418000,40.446000) -- 
		(125.347000,40.512000) -- 
		(125.321000,40.584000) -- 
		(125.360000,40.672000) -- 
		(125.451000,40.749000) -- 
		(125.600000,40.820000) -- 
		(125.808000,40.864000) -- 
		(126.035000,40.815000) -- 
		(126.204000,40.809000) -- 
		(126.276000,40.826000) -- 
		(126.354000,40.870000) -- 
		(126.445000,40.996000) -- 
		(126.613000,41.134000) -- 
		(126.808000,41.178000) -- 
		(127.088000,41.183000) -- 
		(127.174000,41.154000) -- 
		(127.367000,41.090000) -- 
		(127.724000,40.870000) -- 
		(127.893000,40.826000) -- 
		(128.023000,40.832000) -- 
		(128.055000,40.887000) -- 
		(128.068000,40.969000) -- 
		(128.010000,41.117000) -- 
		(128.016000,41.244000) -- 
		(128.094000,41.359000) -- 
		(128.010000,41.425000) -- 
		(127.984000,41.508000) -- 
		(128.198000,41.766000) -- 
		(128.354000,41.926000) -- 
		(128.361000,41.981000) -- 
		(128.283000,42.058000) -- 
		(128.094000,42.168000) -- 
		(128.003000,42.283000) -- 
		(128.049000,42.624000) -- 
		(128.120000,42.652000) -- 
		(128.283000,42.652000) -- 
		(128.452000,42.630000) -- 
		(128.659000,42.564000) -- 
		(128.822000,42.591000) -- 
		(128.939000,42.624000) -- 
		(128.984000,42.679000) -- 
		(128.971000,42.762000) -- 
		(128.763000,43.037000) -- 
		(128.770000,43.108000) -- 
		(128.828000,43.202000) -- 
		(128.841000,43.499000) -- 
		(128.750000,43.928000) -- 
		(128.770000,44.538000) -- 
		(128.750000,44.906000) -- 
		(128.666000,45.071000) -- 
		(128.672000,45.154000) -- 
		(128.718000,45.154000) -- 
		(128.847000,45.104000) -- 
		(128.990000,45.104000) -- 
		(129.062000,45.198000) -- 
		(129.205000,45.335000) -- 
		(129.289000,45.385000) -- 
		(129.478000,45.451000) -- 
		(129.530000,45.500000) -- 
		(129.556000,45.566000) -- 
		(129.607000,45.594000) -- 
		(129.802000,45.588000) -- 
		(130.004000,45.621000) -- 
		(130.069000,45.676000) -- 
		(130.043000,45.770000) -- 
		(129.984000,45.880000) -- 
		(129.841000,45.973000) -- 
		(129.679000,46.034000) -- 
		(129.568000,46.144000) -- 
		(129.503000,46.281000) -- 
		(129.529000,46.347000) -- 
		(129.594000,46.364000) -- 
		(129.705000,46.336000) -- 
		(129.906000,46.265000) -- 
		(130.166000,46.155000) -- 
		(130.361000,46.122000) -- 
		(130.497000,46.149000) -- 
		(130.562000,46.226000) -- 
		(130.582000,46.347000) -- 
		(130.679000,46.622000) -- 
		(131.101000,46.578000) -- 
		(131.394000,46.534000) -- 
		(131.472000,46.551000) -- 
		(131.491000,46.633000) -- 
		(131.459000,46.886000) -- 
		(131.517000,46.914000) -- 
		(132.069000,46.864000) -- 
		(132.141000,46.925000) -- 
		(132.147000,47.007000) -- 
		(132.069000,47.101000) -- 
		(131.965000,47.178000) -- 
		(131.910000,47.250000) -- 
		(131.868000,47.304000) -- 
		(131.926000,47.431000) -- 
		(132.043000,47.480000) -- 
		(132.141000,47.546000) -- 
		(132.173000,47.618000) -- 
		(132.186000,47.750000) -- 
		(132.056000,48.019000) -- 
		(132.076000,48.074000) -- 
		(132.167000,48.074000) -- 
		(132.556000,48.036000) -- 
		(132.628000,48.047000) -- 
		(132.667000,48.201000) -- 
		(132.621000,48.272000) -- 
		(132.550000,48.437000) -- 
		(132.355000,48.706000) -- 
		(132.238000,48.904000) -- 
		(132.225000,49.003000) -- 
		(132.251000,49.031000) -- 
		(132.478000,48.844000) -- 
		(132.602000,48.761000) -- 
		(132.699000,48.761000) -- 
		(132.725000,48.789000) -- 
		(132.738000,48.877000) -- 
		(132.680000,48.959000) -- 
		(132.660000,49.036000) -- 
		(132.725000,49.075000) -- 
		(132.894000,49.097000) -- 
		(132.979000,49.157000) -- 
		(132.985000,49.201000) -- 
		(132.966000,49.267000) -- 
		(132.660000,49.350000) -- 
		(132.563000,49.454000) -- 
		(132.556000,49.542000) -- 
		(132.615000,49.619000) -- 
		(132.862000,49.790000) -- 
		(132.920000,49.933000) -- 
		(132.810000,49.988000) -- 
		(132.712000,50.015000) -- 
		(132.589000,50.076000) -- 
		(132.550000,50.158000) -- 
		(132.543000,50.257000) -- 
		(132.686000,50.466000) -- 
		(132.784000,50.582000) -- 
		(132.810000,50.659000) -- 
		(132.751000,50.741000) -- 
		(132.686000,50.780000) -- 
		(132.491000,50.703000) -- 
		(132.413000,50.626000) -- 
		(132.264000,50.741000) -- 
		(132.153000,50.846000) -- 
		(132.050000,50.884000) -- 
		(131.952000,50.895000) -- 
		(131.881000,50.983000) -- 
		(131.855000,51.148000) -- 
		(131.751000,51.362000) -- 
		(131.660000,51.648000) -- 
		(131.595000,51.681000) -- 
		(131.400000,51.687000) -- 
		(131.302000,51.742000) -- 
		(131.257000,51.846000) -- 
		(131.367000,51.989000) -- 
		(131.543000,51.995000) -- 
		(131.705000,52.044000) -- 
		(132.192000,52.039000) -- 
		(132.374000,52.077000) -- 
		(132.459000,52.127000) -- 
		(132.485000,52.198000) -- 
		(132.368000,52.319000) -- 
		(132.036000,52.308000) -- 
		(131.965000,52.325000) -- 
		(131.848000,52.380000) -- 
		(131.712000,52.462000) -- 
		(131.666000,52.572000) -- 
		(131.725000,52.644000) -- 
		(131.828000,52.677000) -- 
		(132.056000,52.644000) -- 
		(132.290000,52.583000) -- 
		(132.452000,52.589000) -- 
		(132.459000,52.633000) -- 
		(132.446000,52.693000) -- 
		(132.199000,52.847000) -- 
		(132.218000,53.018000) -- 
		(132.316000,53.073000) -- 
		(132.491000,53.106000) -- 
		(132.835000,53.128000) -- 
		(132.965000,53.062000) -- 
		(132.985000,52.825000) -- 
		(133.030000,52.803000) -- 
		(133.290000,52.842000) -- 
		(133.375000,52.814000) -- 
		(133.401000,52.781000) -- 
		(133.394000,52.710000) -- 
		(133.342000,52.649000) -- 
		(133.375000,52.622000) -- 
		(133.459000,52.655000) -- 
		(133.596000,52.792000) -- 
		(133.719000,52.864000) -- 
		(133.849000,52.880000) -- 
		(134.141000,52.787000) -- 
		(134.252000,52.770000) -- 
		(134.447000,52.831000) -- 
		(134.538000,52.930000) -- 
		(134.485000,53.106000) -- 
		(134.466000,53.265000) -- 
		(134.472000,53.408000) -- 
		(134.615000,53.441000) -- 
		(134.778000,53.392000) -- 
		(134.895000,53.386000) -- 
		(134.901000,53.408000) -- 
		(134.862000,53.546000) -- 
		(134.953000,53.727000) -- 
		(134.765000,53.782000) -- 
		(134.680000,53.848000) -- 
		(134.661000,53.947000) -- 
		(134.687000,54.052000) -- 
		(134.752000,54.084000) -- 
		(134.895000,54.101000) -- 
		(135.005000,54.085000) -- 
		(135.135000,54.002000) -- 
		(135.285000,53.936000) -- 
		(135.337000,53.953000) -- 
		(135.479000,54.200000) -- 
		(135.603000,54.277000) -- 
		(135.674000,54.277000) -- 
		(135.830000,54.211000) -- 
		(135.967000,54.211000) -- 
		(136.064000,54.239000) -- 
		(136.162000,54.343000) -- 
		(136.175000,54.475000) -- 
		(136.103000,54.733000) -- 
		(136.103000,54.821000) -- 
		(136.233000,54.909000) -- 
		(136.525000,55.008000) -- 
		(136.629000,55.074000) -- 
		(136.720000,55.085000) -- 
		(136.837000,54.992000) -- 
		(136.824000,54.766000) -- 
		(136.766000,54.513000) -- 
		(136.766000,54.453000) -- 
		(136.837000,54.425000) -- 
		(136.928000,54.464000) -- 
		(137.117000,54.634000) -- 
		(137.253000,54.739000) -- 
		(137.383000,54.739000) -- 
		(137.526000,54.607000) -- 
		(137.584000,54.579000) -- 
		(137.721000,54.634000) -- 
		(137.766000,54.711000) -- 
		(137.753000,54.810000) -- 
		(137.779000,54.882000) -- 
		(137.903000,54.904000) -- 
		(138.072000,54.898000) -- 
		(138.208000,54.942000) -- 
		(138.305000,54.942000) -- 
		(138.364000,54.893000) -- 
		(138.396000,54.805000) -- 
		(138.403000,54.541000) -- 
		(138.468000,54.431000) -- 
		(138.507000,54.420000) -- 
		(138.650000,54.541000) -- 
		(138.825000,54.645000) -- 
		(139.507000,54.854000) -- 
		(139.598000,54.909000) -- 
		(139.637000,54.981000) -- 
		(139.572000,55.047000) -- 
		(139.345000,55.173000) -- 
		(139.234000,55.256000) -- 
		(139.163000,55.360000) -- 
		(139.189000,55.470000) -- 
		(139.293000,55.465000) -- 
		(139.689000,55.415000) -- 
		(139.774000,55.470000) -- 
		(139.787000,55.547000) -- 
		(139.722000,55.641000) -- 
		(139.618000,55.751000) -- 
		(139.618000,55.883000) -- 
		(139.689000,55.943000) -- 
		(139.826000,55.943000) -- 
		(139.982000,55.850000) -- 
		(140.079000,55.844000) -- 
		(140.332000,55.965000) -- 
		(140.436000,55.971000) -- 
		(140.553000,56.020000) -- 
		(140.716000,56.235000) -- 
		(140.826000,56.229000) -- 
		(140.917000,56.235000) -- 
		(141.047000,56.328000) -- 
		(141.515000,56.339000) -- 
		(141.599000,56.229000) -- 
		(141.586000,56.136000) -- 
		(141.456000,56.042000) -- 
		(141.372000,55.949000) -- 
		(141.385000,55.905000) -- 
		(141.515000,55.811000) -- 
		(141.677000,55.784000) -- 
		(141.892000,55.839000) -- 
		(142.035000,55.855000) -- 
		(142.171000,55.839000) -- 
		(142.327000,55.784000) -- 
		(142.437000,55.652000) -- 
		(142.418000,55.437000) -- 
		(142.541000,55.322000) -- 
		(142.606000,55.228000) -- 
		(142.639000,55.063000) -- 
		(142.645000,54.871000) -- 
		(142.697000,54.788000) -- 
		(142.769000,54.711000) -- 
		(142.925000,54.750000) -- 
		(143.035000,54.821000) -- 
		(143.061000,54.931000) -- 
		(143.119000,55.052000) -- 
		(143.165000,55.096000) -- 
		(143.308000,55.140000) -- 
		(143.516000,55.140000) -- 
		(143.750000,55.080000) -- 
		(144.042000,54.970000) -- 
		(144.224000,54.942000) -- 
		(144.302000,54.970000) -- 
		(144.380000,55.063000) -- 
		(144.477000,55.278000) -- 
		(144.516000,55.333000) -- 
		(144.770000,55.404000) -- 
		(144.965000,55.448000) -- 
		(145.107000,55.569000) -- 
		(145.296000,55.619000) -- 
		(145.400000,55.608000) -- 
		(145.601000,55.503000) -- 
		(145.735000,55.412000) -- 
		(145.738000,55.410000) -- 
		(145.848000,55.278000) -- 
		(145.933000,55.107000) -- 
		(146.017000,54.997000) -- 
		(146.088000,54.942000) -- 
		(146.192000,54.909000) -- 
		(146.329000,54.937000) -- 
		(146.426000,54.997000) -- 
		(146.491000,55.168000) -- 
		(146.517000,55.272000) -- 
		(146.576000,55.327000) -- 
		(146.706000,55.333000) -- 
		(146.810000,55.360000) -- 
		(146.836000,55.448000) -- 
		(146.725000,55.613000) -- 
		(146.621000,55.729000) -- 
		(146.563000,56.047000) -- 
		(146.673000,56.113000) -- 
		(146.790000,56.113000) -- 
		(146.953000,56.053000) -- 
		(147.160000,55.959000) -- 
		(147.323000,55.866000) -- 
		(147.440000,55.849000) -- 
		(147.518000,55.899000) -- 
		(147.544000,56.009000) -- 
		(147.531000,56.152000) -- 
		(147.433000,56.267000) -- 
		(147.277000,56.372000) -- 
		(147.109000,56.394000) -- 
		(147.044000,56.443000) -- 
		(146.985000,56.526000) -- 
		(146.979000,56.625000) -- 
		(147.044000,56.757000) -- 
		(147.109000,56.779000) -- 
		(147.239000,56.735000) -- 
		(147.355000,56.658000) -- 
		(147.479000,56.603000) -- 
		(147.563000,56.592000) -- 
		(147.635000,56.603000) -- 
		(147.706000,56.685000) -- 
		(147.765000,56.823000) -- 
		(147.758000,57.186000) -- 
		(147.732000,57.373000) -- 
		(147.667000,57.428000) -- 
		(147.570000,57.422000) -- 
		(147.420000,57.395000) -- 
		(147.297000,57.411000) -- 
		(147.200000,57.455000) -- 
		(147.161000,57.521000) -- 
		(147.148000,57.626000) -- 
		(147.187000,57.758000) -- 
		(147.265000,57.829000) -- 
		(147.362000,57.829000) -- 
		(147.570000,57.785000) -- 
		(147.836000,57.791000) -- 
		(147.921000,57.818000) -- 
		(148.044000,57.934000) -- 
		(148.012000,58.044000) -- 
		(148.005000,58.126000) -- 
		(148.025000,58.209000) -- 
		(148.064000,58.263000) -- 
		(148.135000,58.275000) -- 
		(148.363000,58.269000) -- 
		(148.512000,58.335000) -- 
		(148.720000,58.467000) -- 
		(148.869000,58.494000) -- 
		(148.954000,58.483000) -- 
		(149.116000,58.406000) -- 
		(149.220000,58.236000) -- 
		(149.266000,58.071000) -- 
		(149.272000,57.955000) -- 
		(149.318000,57.895000) -- 
		(149.396000,57.840000) -- 
		(149.480000,57.845000) -- 
		(149.603000,57.906000) -- 
		(149.759000,58.082000) -- 
		(149.883000,58.252000) -- 
		(150.013000,58.335000) -- 
		(150.143000,58.351000) -- 
		(150.253000,58.346000) -- 
		(150.357000,58.263000) -- 
		(150.448000,58.071000) -- 
		(150.558000,57.911000) -- 
		(150.682000,57.862000) -- 
		(150.773000,57.900000) -- 
		(150.870000,57.994000) -- 
		(150.994000,58.076000) -- 
		(151.130000,58.104000) -- 
		(151.390000,58.032000) -- 
		(151.423000,58.384000) -- 
		(151.494000,58.494000) -- 
		(151.644000,58.577000) -- 
		(151.949000,58.648000) -- 
		(152.079000,58.747000) -- 
		(152.382000,58.790000) -- 
		(152.527000,58.813000) -- 
		(152.560000,58.841000) -- 
		(152.618000,59.011000) -- 
		(152.709000,59.060000) -- 
		(152.885000,59.082000) -- 
		(152.924000,59.110000) -- 
		(153.008000,59.291000) -- 
		(153.112000,59.473000) -- 
		(153.248000,59.605000) -- 
		(153.365000,59.781000) -- 
		(153.437000,59.940000) -- 
		(153.443000,60.072000) -- 
		(153.417000,60.210000) -- 
		(153.450000,60.331000) -- 
		(153.521000,60.380000) -- 
		(153.671000,60.446000) -- 
		(153.736000,60.452000) -- 
		(153.898000,60.534000) -- 
		(153.866000,60.628000) -- 
		(153.794000,60.721000) -- 
		(153.762000,60.886000) -- 
		(153.781000,61.078000) -- 
		(153.892000,61.216000) -- 
		(154.113000,61.359000) -- 
		(154.360000,61.419000) -- 
		(154.678000,61.452000) -- 
		(154.983000,61.436000) -- 
		(155.289000,61.496000) -- 
		(155.471000,61.623000) -- 
		(155.971000,62.403000) -- 
		(156.127000,62.909000) -- 
		(156.179000,63.223000) -- 
		(156.192000,63.366000) -- 
		(156.304000,63.478000) -- 
		(156.446000,63.624000) -- 
		(156.693000,63.751000) -- 
		(156.868000,63.899000) -- 
		(156.901000,64.036000) -- 
		(156.855000,64.125000) -- 
		(156.784000,64.202000) -- 
		(156.751000,64.251000) -- 
		(156.771000,64.394000) -- 
		(156.855000,64.487000) -- 
		(156.894000,64.592000) -- 
		(156.927000,64.762000) -- 
		(156.985000,64.878000) -- 
		(157.161000,65.103000) -- 
		(157.265000,65.323000) -- 
		(157.285000,65.349000) -- 
		(157.720000,65.194000) -- 
		(157.820000,65.168000) -- 
		(157.943000,65.149000) -- 
		(158.119000,65.160000) -- 
		(158.341000,65.229000) -- 
		(158.620000,65.343000) -- 
		(159.037000,65.485000) -- 
		(159.416000,65.512000) -- 
		(159.828000,65.524000) -- 
		(159.865000,65.527000) -- 
		(161.919000,65.520000) -- 
		(162.866000,65.498000) -- 
		(164.441000,65.511000) -- 
		(165.503000,65.467000) -- 
		(166.960000,65.441000) -- 
		(167.684000,65.421000) -- 
		(167.720000,65.426000) -- 
		(167.979000,65.468000) -- 
		(168.098000,65.510000) -- 
		(168.315000,65.667000) -- 
		(168.449000,65.827000) -- 
		(168.620000,66.064000) -- 
		(168.798000,66.334000) -- 
		(168.973000,66.503000) -- 
		(169.107000,66.601000) -- 
		(169.428000,66.723000) -- 
		(169.810000,66.834000) -- 
		(169.587000,67.109000) -- 
		(169.459000,67.287000) -- 
		(169.405000,67.417000) -- 
		(169.358000,67.582000) -- 
		(169.327000,69.577000) -- 
		(169.337000,71.131000) -- 
		(169.380000,72.386000) -- 
		(169.412000,73.340000) -- 
		(169.425000,75.004000) -- 
		(169.402000,75.419000) -- 
		(169.415000,75.695000) -- 
		(169.412000,76.212000) -- 
		(169.425000,76.399000) -- 
		(169.481000,76.524000) -- 
		(170.008000,77.052000) -- 
		(170.288000,77.318000) -- 
		(170.905000,77.973000) -- 
		(171.309000,78.369000) -- 
		(171.828000,78.803000) -- 
		(173.190000,79.880000) -- 
		(173.721000,80.310000) -- 
		(174.977000,81.417000) -- 
		(176.451000,82.689000) -- 
		(176.835000,83.005000) -- 
		(177.089000,83.229000) -- 
		(177.457000,83.531000) -- 
		(177.702000,83.758000) -- 
		(177.962000,83.953000) -- 
		(178.081000,84.024000) -- 
		(178.274000,84.048000) -- 
		(179.578000,84.047000) -- 
		(180.311000,84.047000) -- 
		(181.217000,84.034000) -- 
		(181.782000,84.034000) -- 
		(182.378000,84.017000) -- 
		(182.792000,83.993000) -- 
		(183.329000,83.947000) -- 
		(183.505000,83.950000) -- 
		(183.624000,83.970000) -- 
		(183.765000,84.023000) -- 
		(184.096000,84.224000) -- 
		(184.548000,84.591000) -- 
		(184.970000,84.895000) -- 
		(184.970000,85.275000) -- 
		(184.970000,86.422000) -- 
		(184.969000,86.804000) -- 
		(184.969000,87.361000) -- 
		(184.969000,88.178000) -- 
		(184.975000,89.032000) -- 
		(184.978000,89.589000) -- 
		(184.967000,90.696000) -- 
		(184.963000,91.023000) -- 
		(184.968000,91.747000) -- 
		(184.972000,94.018000) -- 
		(184.973000,95.126000) -- 
		(186.451000,95.103000) -- 
		(186.791000,95.098000) -- 
		(188.112000,95.096000) -- 
		(189.469000,95.049000) -- 
		(190.769000,95.017000) -- 
		(190.884000,95.012000) -- 
		(192.099000,94.964000) -- 
		(192.360000,94.963000) -- 
		(194.247000,94.952000) -- 
		(194.740000,94.950000) -- 
		(196.171000,94.890000) -- 
		(197.738000,94.859000) -- 
		(199.261000,94.842000) -- 
		(199.907000,94.829000) -- 
		(200.315000,94.821000) -- 
		(201.793000,94.815000) -- 
		(202.348000,94.810000) -- 
		(203.538000,94.803000) -- 
		(204.013000,94.790000) -- 
		(204.567000,94.777000) -- 
		(204.913000,94.768000) -- 
		(205.261000,94.760000) -- 
		(205.950000,94.753000) -- 
		(206.295000,94.750000) -- 
		(206.753000,94.744000) -- 
		(208.126000,94.726000) -- 
		(208.583000,94.721000) -- 
		(208.835000,94.718000) -- 
		(208.841000,94.717000) -- 
		(209.612000,94.696000) -- 
		(209.869000,94.690000) -- 
		(210.128000,94.682000) -- 
		(210.902000,94.661000) -- 
		(211.160000,94.656000) -- 
		(211.227000,94.654000) -- 
		(211.428000,94.650000) -- 
		(211.495000,94.649000) -- 
		(212.040000,94.638000) -- 
		(212.706000,94.625000) -- 
		(213.673000,94.619000) -- 
		(214.217000,94.616000) -- 
		(215.788000,94.607000) -- 
		(217.399000,94.606000) -- 
		(218.467000,94.605000) -- 
		(219.716000,94.582000) -- 
		(219.947000,94.572000) -- 
		(220.022000,94.569000) -- 
		(220.359000,94.518000) -- 
		(220.601000,94.471000) -- 
		(220.619000,94.468000) -- 
		(221.440000,94.276000) -- 
		(221.441000,94.275000) -- 
		(221.570000,94.246000) -- 
		(222.391000,94.055000) -- 
		(222.536000,94.021000) -- 
		(223.181000,93.874000) -- 
		(223.290000,93.852000) -- 
		(223.374000,93.846000) -- 
		(223.959000,93.810000) -- 
		(224.153000,93.798000) -- 
		(224.852000,93.754000) -- 
		(226.071000,93.692000) -- 
		(226.519000,93.680000) -- 
		(227.320000,93.714000) -- 
		(227.745000,93.749000) -- 
		(228.458000,93.812000) -- 
		(229.002000,93.848000) -- 
		(229.725000,93.929000) -- 
		(231.698000,94.122000) -- 
		(232.805000,94.216000) -- 
		(235.814000,94.498000) -- 
		(236.007000,94.509000) -- 
		(236.236000,94.521000) -- 
		(237.125000,94.533000) -- 
		(238.383000,94.538000) -- 
		(239.754000,94.530000) -- 
		(240.458000,94.526000) -- 
		(241.476000,94.507000) -- 
		(242.657000,94.502000) -- 
		(246.892000,94.454000) -- 
		(248.506000,94.460000) -- 
		(250.527000,94.428000) -- 
		(250.528000,94.427000) -- 
		(250.994000,94.423000) -- 
		(252.152000,94.412000) -- 
		(253.220000,94.385000) -- 
		(253.483000,94.361000) -- 
		(253.754000,94.317000) -- 
		(254.028000,94.252000) -- 
		(254.344000,94.144000) -- 
		(254.683000,94.006000) -- 
		(254.400000,63.825000) -- 
		(254.380000,60.806000) -- 
		(254.343000,57.777000) -- 
		(254.134000,48.126000) -- 
		(253.926000,38.481000) -- 
		(253.925000,38.475000) -- 
		(253.844000,32.198000) -- 
		(253.876000,31.046000) -- 
		(253.900000,30.247000) -- 
		(253.924000,29.451000) -- 
		(253.948000,28.648000) -- 
		(253.972000,27.846000) -- 
		(253.962000,27.659000) -- 
		(253.956000,27.558000) -- 
		(253.966000,27.418000) -- 
		(253.972000,27.334000) -- 
		(253.972000,21.495000) -- 
		(253.973000,21.446000) -- 
		(253.982000,21.322000) -- 
		(253.989000,21.255000) -- 
		(253.978000,21.105000) -- 
		(253.972000,21.030000) -- 
		(253.951000,17.161000) -- 
		(253.946000,16.167000) -- 
		(253.940000,15.174000) -- 
		(253.941000,14.278000) -- 
		(253.942000,14.263000) -- 
		(253.973000,14.022000) -- 
		(253.980000,13.867000) -- 
		(254.008000,13.288000) -- 
		(254.036000,12.710000) -- 
		(253.986000,10.566000) -- 
		(253.936000,8.421000) -- 
		(253.928000,8.007000) -- 
		(253.920000,7.592000) -- 
		(253.895000,5.596000) -- 
		(253.894000,5.595000) -- 
		(253.892000,5.462000) -- 
		(253.940000,1.929000) -- 
		(253.944000,0.308000) -- 
		(253.949000,-1.312000) -- 
		(253.950000,-1.544000) -- 
		(253.950000,-1.775000) -- 
		(253.950000,-2.155000) -- 
		(253.951000,-2.535000) -- 
		(253.953000,-2.877000) -- 
		(253.954000,-3.219000) -- 
		(253.954000,-3.401000) -- 
		(253.955000,-3.583000) -- 
		(253.956000,-3.657000) -- 
		(253.958000,-4.646000) -- 
		(253.961000,-5.636000) -- 
		(253.962000,-5.762000) -- 
		(253.962000,-5.887000) -- 
		(253.962000,-5.984000) -- 
		(253.962000,-6.080000) -- 
		(253.964000,-6.621000) -- 
		(253.965000,-7.164000) -- 
		(253.972000,-9.595000) -- 
		(253.958000,-10.197000) -- 
		(253.956000,-10.245000) -- 
		(253.955000,-10.329000) -- 
		(253.952000,-10.462000) -- 
		(253.903000,-12.457000) -- 
		(253.854000,-14.451000) -- 
		(253.853000,-14.477000) -- 
		(253.844000,-14.777000) -- 
		(253.852000,-14.827000) -- 
		(253.859000,-14.868000) -- 
		(253.876000,-14.969000) -- 
		(253.870000,-15.339000) -- 
		(253.748000,-27.961000) -- 
		(253.740000,-29.462000) -- 
		(253.733000,-30.964000) -- 
		(253.704000,-37.057000) -- 
		(253.674000,-43.150000) -- 
		(253.671000,-43.755000) -- 
		(253.668000,-44.360000) -- 
		(253.600000,-49.848000) -- 
		(253.502000,-52.028000) -- 
		(258.775000,-52.165000) -- 
		(258.776000,-53.550000) -- 
		(258.774000,-54.934000) -- 
		(258.776000,-60.618000) -- 
		(258.777000,-66.299000) -- 
		(257.655000,-65.657000) -- 
		(257.355000,-65.510000) -- 
		(257.160000,-65.408000) -- 
		(256.953000,-65.300000) -- 
		(256.478000,-65.078000) -- 
		(255.859000,-64.849000) -- 
		(255.045000,-64.493000) -- 
		(254.676000,-64.357000) -- 
		(254.614000,-64.341000) -- 
		(254.452000,-64.297000) -- 
		(254.136000,-64.234000) -- 
		(253.768000,-64.185000) -- 
		(253.508000,-64.150000) -- 
		(253.001000,-64.099000) -- 
		(252.270000,-63.953000) -- 
		(252.003000,-63.857000) -- 
		(251.894000,-63.811000) -- 
		(251.879000,-63.804000) -- 
		(251.641000,-63.702000) -- 
		(250.230000,-63.007000) -- 
		(249.682000,-62.734000) -- 
		(248.411000,-62.117000) -- 
		(248.376000,-62.099000) -- 
		(247.346000,-61.560000) -- 
		(246.754000,-61.279000) -- 
		(244.181000,-59.994000) -- 
		(243.578000,-59.679000) -- 
		(242.837000,-59.324000) -- 
		(242.338000,-59.083000) -- 
		(241.544000,-58.683000) -- 
		(240.979000,-58.420000) -- 
		(240.725000,-58.302000) -- 
		(240.218000,-58.043000) -- 
		(239.930000,-57.897000) -- 
		(238.679000,-57.253000) -- 
		(238.083000,-56.963000) -- 
		(237.611000,-56.744000) -- 
		(237.215000,-56.550000) -- 
		(237.001000,-56.417000) -- 
		(236.827000,-56.256000) -- 
		(236.683000,-56.123000) -- 
		(236.492000,-55.884000) -- 
		(236.283000,-55.586000) -- 
		(236.094000,-55.342000) -- 
		(235.905000,-55.100000) -- 
		(235.435000,-54.531000) -- 
		(235.088000,-54.111000) -- 
		(234.897000,-53.942000) -- 
		(234.438000,-53.614000) -- 
		(234.052000,-53.359000) -- 
		(233.785000,-53.218000) -- 
		(233.406000,-53.025000) -- 
		(233.405000,-53.024000) -- 
		(233.155000,-52.895000) -- 
		(232.876000,-52.780000) -- 
		(232.473000,-52.613000) -- 
		(231.899000,-52.406000) -- 
		(231.762000,-52.358000) -- 
		(231.448000,-52.243000) -- 
		(231.351000,-52.205000) -- 
		(231.214000,-52.151000) -- 
		(230.898000,-52.158000) -- 
		(230.480000,-52.132000) -- 
		(229.776000,-52.115000) -- 
		(229.105000,-52.087000) -- 
		(228.753000,-52.122000) -- 
		(228.693000,-52.131000) -- 
		(228.112000,-52.213000) -- 
		(227.488000,-52.266000) -- 
		(227.395000,-52.267000) -- 
		(226.794000,-52.277000) -- 
		(226.518000,-52.261000) -- 
		(225.932000,-52.197000) -- 
		(225.514000,-52.177000) -- 
		(225.399000,-52.171000) -- 
		(225.192000,-52.160000) -- 
		(224.534000,-52.157000) -- 
		(224.230000,-52.142000) -- 
		(223.728000,-52.116000) -- 
		(223.639000,-52.110000) -- 
		(222.993000,-52.068000) -- 
		(221.974000,-52.012000) -- 
		(221.135000,-52.009000) -- 
		(220.939000,-52.008000) -- 
		(220.129000,-52.014000) -- 
		(219.035000,-52.036000) -- 
		(218.611000,-52.064000) -- 
		(218.413000,-52.077000) -- 
		(218.226000,-52.089000) -- 
		(218.092000,-52.099000) -- 
		(218.003000,-52.104000) -- 
		(216.530000,-52.116000) -- 
		(216.401000,-52.116000) -- 
		(216.009000,-52.091000) -- 
		(214.651000,-52.005000) -- 
		(214.152000,-51.995000) -- 
		(211.392000,-51.932000) -- 
		(210.361000,-51.946000) -- 
		(209.250000,-51.957000) -- 
		(208.571000,-51.953000) -- 
		(208.340000,-51.951000) -- 
		(207.106000,-51.950000) -- 
		(207.105000,-51.949000) -- 
		(206.710000,-51.949000) -- 
		(206.291000,-51.950000) -- 
		(206.111000,-51.950000) -- 
		(205.032000,-51.939000) -- 
		(204.612000,-51.934000) -- 
		(204.151000,-51.945000) -- 
		(203.654000,-51.956000) -- 
		(203.457000,-51.970000) -- 
		(203.223000,-51.986000) -- 
		(202.776000,-52.074000) -- 
		(202.694000,-52.088000) -- 
		(202.328000,-52.184000) -- 
		(202.081000,-52.249000) -- 
		(201.947000,-52.284000) -- 
		(201.947000,-52.285000) -- 
		(201.556000,-52.387000) -- 
		(201.528000,-52.394000) -- 
		(200.646000,-52.604000) -- 
		(199.899000,-52.782000) -- 
		(199.228000,-52.952000) -- 
		(198.988000,-53.012000) -- 
		(198.453000,-53.144000) -- 
		(197.866000,-53.290000) -- 
		(197.134000,-53.461000) -- 
		(195.244000,-53.899000) -- 
		(194.471000,-53.908000) -- 
		(193.333000,-53.793000) -- 
		(193.126000,-53.738000) -- 
		(192.268000,-53.508000) -- 
		(191.820000,-53.366000) -- 
		(191.777000,-53.352000) -- 
		(191.620000,-53.327000) -- 
		(191.513000,-53.327000) -- 
		(190.579000,-53.327000) -- 
		(190.267000,-53.327000) -- 
		(190.239000,-52.177000);
	\filldraw [draw=black, ultra thick, fill=gold]
		(-57.346000,152.830000) -- 
		(-55.912000,153.684000) -- 
		(-52.763000,155.559000) -- 
		(-51.611000,156.246000) -- 
		(-50.178000,157.102000) -- 
		(-48.024000,158.440000) -- 
		(-41.565000,162.455000) -- 
		(-40.222000,163.291000) -- 
		(-39.334000,163.637000) -- 
		(-38.963000,163.661000) -- 
		(-38.917000,163.664000) -- 
		(-38.240000,164.489000) -- 
		(-38.005000,164.778000) -- 
		(-37.885000,164.578000) -- 
		(-37.772000,164.508000) -- 
		(-37.716000,164.474000) -- 
		(-37.534000,164.375000) -- 
		(-37.417000,164.172000) -- 
		(-37.339000,164.117000) -- 
		(-37.079000,164.018000) -- 
		(-36.994000,163.930000) -- 
		(-36.959000,163.842000) -- 
		(-36.864000,163.605000) -- 
		(-36.776000,163.539000) -- 
		(-36.532000,163.358000) -- 
		(-36.447000,163.314000) -- 
		(-36.304000,163.100000) -- 
		(-36.122000,163.001000) -- 
		(-36.063000,162.902000) -- 
		(-36.063000,162.865000) -- 
		(-36.063000,162.682000) -- 
		(-36.076000,162.572000) -- 
		(-36.219000,162.418000) -- 
		(-36.440000,162.236000) -- 
		(-36.466000,162.099000) -- 
		(-36.466000,161.901000) -- 
		(-36.511000,161.676000) -- 
		(-36.479000,161.483000) -- 
		(-36.407000,161.351000) -- 
		(-36.264000,161.192000) -- 
		(-36.225000,160.994000) -- 
		(-36.192000,160.856000) -- 
		(-36.069000,160.746000) -- 
		(-35.737000,160.571000) -- 
		(-35.581000,160.516000) -- 
		(-35.502000,160.496000) -- 
		(-35.353000,160.461000) -- 
		(-34.911000,160.219000) -- 
		(-34.605000,160.032000) -- 
		(-34.514000,159.933000) -- 
		(-34.488000,159.823000) -- 
		(-34.539000,159.548000) -- 
		(-34.546000,159.437000) -- 
		(-34.559000,159.235000) -- 
		(-34.396000,158.735000) -- 
		(-34.142000,158.432000) -- 
		(-34.103000,158.317000) -- 
		(-34.129000,158.207000) -- 
		(-34.370000,157.954000) -- 
		(-34.441000,157.767000) -- 
		(-34.439000,157.760000) -- 
		(-34.402000,157.629000) -- 
		(-34.298000,157.470000) -- 
		(-34.135000,157.277000) -- 
		(-34.089000,157.241000) -- 
		(-33.759000,156.986000) -- 
		(-33.686000,156.931000) -- 
		(-33.595000,156.810000) -- 
		(-33.614000,156.563000) -- 
		(-33.575000,156.464000) -- 
		(-33.484000,156.376000) -- 
		(-33.217000,156.343000) -- 
		(-33.016000,156.332000) -- 
		(-32.971000,156.301000) -- 
		(-32.912000,156.261000) -- 
		(-32.775000,156.096000) -- 
		(-32.651000,156.057000) -- 
		(-32.539000,156.051000) -- 
		(-32.203000,156.035000) -- 
		(-32.092000,156.030000) -- 
		(-32.008000,155.981000) -- 
		(-31.988000,155.926000) -- 
		(-31.999000,155.851000) -- 
		(-32.027000,155.667000) -- 
		(-31.968000,155.497000) -- 
		(-31.897000,155.392000) -- 
		(-31.734000,155.348000) -- 
		(-31.701000,155.345000) -- 
		(-31.552000,155.332000) -- 
		(-31.481000,155.291000) -- 
		(-31.397000,155.279000) -- 
		(-31.270000,155.280000) -- 
		(-31.056000,155.290000) -- 
		(-31.013000,155.289000) -- 
		(-30.972000,155.282000) -- 
		(-30.866000,155.220000) -- 
		(-30.829000,155.202000) -- 
		(-30.807000,155.173000) -- 
		(-30.772000,155.107000) -- 
		(-30.751000,155.076000) -- 
		(-30.725000,155.048000) -- 
		(-30.607000,154.943000) -- 
		(-30.535000,154.904000) -- 
		(-30.082000,154.707000) -- 
		(-29.978000,154.644000) -- 
		(-29.882000,154.573000) -- 
		(-29.806000,154.487000) -- 
		(-29.747000,154.391000) -- 
		(-29.608000,154.053000) -- 
		(-29.581000,154.025000) -- 
		(-29.436000,153.949000) -- 
		(-29.357000,153.922000) -- 
		(-29.196000,153.899000) -- 
		(-29.104000,153.886000) -- 
		(-29.022000,153.866000) -- 
		(-28.895000,153.867000) -- 
		(-28.839000,153.876000) -- 
		(-28.605000,153.826000) -- 
		(-28.501000,153.755000) -- 
		(-28.429000,153.595000) -- 
		(-28.377000,153.430000) -- 
		(-28.260000,153.381000) -- 
		(-27.948000,153.392000) -- 
		(-27.688000,153.343000) -- 
		(-27.577000,153.244000) -- 
		(-27.499000,153.079000) -- 
		(-27.480000,152.826000) -- 
		(-27.447000,152.771000) -- 
		(-27.252000,152.710000) -- 
		(-27.063000,152.639000) -- 
		(-27.031000,152.524000) -- 
		(-27.063000,152.447000) -- 
		(-27.232000,152.221000) -- 
		(-27.226000,152.133000) -- 
		(-27.167000,151.968000) -- 
		(-27.102000,151.633000) -- 
		(-27.141000,150.989000) -- 
		(-27.056000,150.764000) -- 
		(-26.939000,150.555000) -- 
		(-26.939000,150.242000) -- 
		(-27.081000,149.769000) -- 
		(-27.075000,149.488000) -- 
		(-26.990000,149.307000) -- 
		(-27.061000,148.823000) -- 
		(-26.977000,148.548000) -- 
		(-26.931000,148.383000) -- 
		(-26.911000,147.800000) -- 
		(-26.983000,147.360000) -- 
		(-26.918000,147.294000) -- 
		(-26.749000,147.212000) -- 
		(-26.723000,147.184000) -- 
		(-26.638000,147.096000) -- 
		(-26.553000,146.943000) -- 
		(-26.540000,146.800000) -- 
		(-26.618000,146.552000) -- 
		(-26.722000,146.409000) -- 
		(-27.112000,146.140000) -- 
		(-27.866000,145.749000) -- 
		(-28.146000,145.457000) -- 
		(-28.290000,145.228000) -- 
		(-28.354000,145.127000) -- 
		(-28.425000,144.841000) -- 
		(-28.464000,144.314000) -- 
		(-28.594000,144.170000) -- 
		(-28.696000,144.127000) -- 
		(-28.809000,144.082000) -- 
		(-28.971000,143.961000) -- 
		(-29.049000,143.829000) -- 
		(-29.108000,143.582000) -- 
		(-29.094000,143.356000) -- 
		(-28.886000,143.170000) -- 
		(-28.815000,142.884000) -- 
		(-28.853000,142.774000) -- 
		(-29.081000,142.515000) -- 
		(-29.322000,142.411000) -- 
		(-29.523000,142.383000) -- 
		(-29.653000,142.422000) -- 
		(-29.777000,142.465000) -- 
		(-29.978000,142.454000) -- 
		(-30.134000,142.515000) -- 
		(-30.323000,142.669000) -- 
		(-30.414000,142.702000) -- 
		(-30.473000,142.669000) -- 
		(-30.531000,142.504000) -- 
		(-30.551000,142.439000) -- 
		(-30.628000,142.201000) -- 
		(-30.713000,142.075000) -- 
		(-31.285000,141.695000) -- 
		(-31.402000,141.635000) -- 
		(-31.423000,141.618000) -- 
		(-31.695000,141.392000) -- 
		(-31.779000,141.304000) -- 
		(-32.091000,141.183000) -- 
		(-32.332000,141.183000) -- 
		(-32.559000,141.238000) -- 
		(-32.657000,141.222000) -- 
		(-32.878000,141.046000) -- 
		(-33.242000,140.979000) -- 
		(-33.333000,140.897000) -- 
		(-33.344000,140.845000) -- 
		(-33.352000,140.809000) -- 
		(-33.216000,140.606000) -- 
		(-33.079000,140.512000) -- 
		(-32.982000,140.353000) -- 
		(-32.994000,140.182000) -- 
		(-33.072000,139.990000) -- 
		(-33.209000,139.819000) -- 
		(-33.352000,139.517000) -- 
		(-33.371000,139.434000) -- 
		(-33.469000,139.412000) -- 
		(-33.631000,139.445000) -- 
		(-33.722000,139.401000) -- 
		(-33.560000,139.071000) -- 
		(-33.670000,138.802000) -- 
		(-33.605000,138.670000) -- 
		(-33.507000,138.593000) -- 
		(-33.338000,138.593000) -- 
		(-33.182000,138.648000) -- 
		(-33.117000,138.626000) -- 
		(-33.072000,138.489000) -- 
		(-33.111000,138.291000) -- 
		(-33.260000,138.098000) -- 
		(-33.507000,137.834000) -- 
		(-33.533000,137.724000) -- 
		(-33.494000,137.642000) -- 
		(-33.331000,137.537000) -- 
		(-33.136000,137.515000) -- 
		(-32.856000,137.510000) -- 
		(-32.700000,137.461000) -- 
		(-32.564000,137.334000) -- 
		(-32.479000,137.147000) -- 
		(-32.466000,136.982000) -- 
		(-32.479000,136.817000) -- 
		(-32.453000,136.740000) -- 
		(-32.368000,136.652000) -- 
		(-32.271000,136.641000) -- 
		(-32.154000,136.691000) -- 
		(-32.069000,136.735000) -- 
		(-32.020000,136.723000) -- 
		(-31.972000,136.713000) -- 
		(-31.900000,136.652000) -- 
		(-31.848000,136.191000) -- 
		(-31.705000,136.081000) -- 
		(-31.425000,135.965000) -- 
		(-31.347000,135.872000) -- 
		(-31.328000,135.679000) -- 
		(-31.360000,135.569000) -- 
		(-31.490000,135.437000) -- 
		(-31.581000,135.355000) -- 
		(-31.548000,135.217000) -- 
		(-31.444000,135.140000) -- 
		(-31.333000,135.128000) -- 
		(-31.249000,135.119000) -- 
		(-31.113000,134.998000) -- 
		(-30.918000,134.937000) -- 
		(-30.716000,134.965000) -- 
		(-30.521000,135.025000) -- 
		(-30.319000,135.080000) -- 
		(-30.040000,135.048000) -- 
		(-29.955000,134.960000) -- 
		(-29.871000,134.740000) -- 
		(-29.844000,134.542000) -- 
		(-29.844000,134.524000) -- 
		(-29.864000,134.014000) -- 
		(-29.688000,133.706000) -- 
		(-29.571000,133.651000) -- 
		(-29.337000,133.679000) -- 
		(-29.155000,133.811000) -- 
		(-29.090000,133.871000) -- 
		(-28.921000,133.899000) -- 
		(-28.752000,133.805000) -- 
		(-28.596000,133.607000) -- 
		(-28.231000,133.113000) -- 
		(-28.114000,132.898000) -- 
		(-28.010000,132.788000) -- 
		(-27.906000,132.607000) -- 
		(-27.763000,132.530000) -- 
		(-27.561000,132.502000) -- 
		(-27.295000,132.530000) -- 
		(-27.139000,132.492000) -- 
		(-27.002000,132.404000) -- 
		(-26.937000,132.272000) -- 
		(-26.950000,132.008000) -- 
		(-26.930000,131.892000) -- 
		(-26.839000,131.810000) -- 
		(-26.579000,131.744000) -- 
		(-26.410000,131.766000) -- 
		(-26.267000,131.837000) -- 
		(-26.065000,131.948000) -- 
		(-26.050000,131.950000) -- 
		(-25.936000,131.981000) -- 
		(-25.838000,131.964000) -- 
		(-25.662000,131.832000) -- 
		(-25.526000,131.816000) -- 
		(-25.370000,131.860000) -- 
		(-25.279000,131.942000) -- 
		(-25.194000,132.129000) -- 
		(-25.103000,132.173000) -- 
		(-24.947000,132.140000) -- 
		(-24.869000,132.085000) -- 
		(-24.791000,131.898000) -- 
		(-24.804000,131.640000) -- 
		(-24.797000,131.578000) -- 
		(-24.977000,131.237000) -- 
		(-25.519000,130.217000) -- 
		(-25.700000,129.877000) -- 
		(-25.781000,129.922000) -- 
		(-25.901000,129.949000) -- 
		(-26.008000,129.929000) -- 
		(-26.067000,129.875000) -- 
		(-26.242000,129.682000) -- 
		(-26.275000,129.648000) -- 
		(-26.621000,129.477000) -- 
		(-26.810000,129.408000) -- 
		(-27.120000,129.389000) -- 
		(-28.168000,129.398000) -- 
		(-28.833000,129.405000) -- 
		(-28.876000,129.401000) -- 
		(-28.984000,129.394000) -- 
		(-29.004000,129.381000) -- 
		(-29.043000,129.362000) -- 
		(-29.052000,129.312000) -- 
		(-29.006000,128.991000) -- 
		(-29.005000,128.841000) -- 
		(-29.020000,128.763000) -- 
		(-29.071000,128.663000) -- 
		(-29.203000,128.544000) -- 
		(-29.376000,128.483000) -- 
		(-29.908000,128.441000) -- 
		(-30.244000,128.458000) -- 
		(-30.995000,128.500000) -- 
		(-31.836000,128.556000) -- 
		(-32.515000,128.608000) -- 
		(-33.591000,128.715000) -- 
		(-33.643000,128.686000) -- 
		(-34.080000,128.377000) -- 
		(-34.569000,128.013000) -- 
		(-34.671000,127.965000) -- 
		(-34.788000,127.933000) -- 
		(-35.472000,127.927000) -- 
		(-35.836000,127.963000) -- 
		(-37.776000,128.155000) -- 
		(-37.678000,127.991000) -- 
		(-37.633000,127.914000) -- 
		(-37.441000,127.472000) -- 
		(-37.438000,127.463000) -- 
		(-37.370000,127.296000) -- 
		(-37.643000,127.203000) -- 
		(-37.851000,127.209000) -- 
		(-38.033000,127.341000) -- 
		(-38.319000,127.561000) -- 
		(-38.475000,127.632000) -- 
		(-38.596000,127.657000) -- 
		(-38.735000,127.687000) -- 
		(-39.171000,127.742000) -- 
		(-39.373000,127.736000) -- 
		(-39.542000,127.637000) -- 
		(-40.205000,127.114000) -- 
		(-40.452000,126.993000) -- 
		(-40.588000,126.911000) -- 
		(-40.757000,126.729000) -- 
		(-41.166000,126.405000) -- 
		(-41.407000,126.025000) -- 
		(-41.504000,125.948000) -- 
		(-42.032000,125.891000) -- 
		(-43.403000,125.749000) -- 
		(-43.579000,126.065000) -- 
		(-44.109000,127.013000) -- 
		(-44.286000,127.329000) -- 
		(-44.399000,127.532000) -- 
		(-44.434000,127.594000) -- 
		(-44.822000,128.289000) -- 
		(-44.823000,128.290000) -- 
		(-44.881000,128.393000) -- 
		(-45.030000,128.661000) -- 
		(-45.062000,128.718000) -- 
		(-45.211000,128.724000) -- 
		(-45.856000,128.750000) -- 
		(-45.956000,128.755000) -- 
		(-46.071000,128.760000) -- 
		(-46.072000,128.738000) -- 
		(-46.079000,128.673000) -- 
		(-46.082000,128.652000) -- 
		(-46.095000,128.524000) -- 
		(-46.119000,128.274000) -- 
		(-46.131000,128.164000) -- 
		(-46.078000,127.938000) -- 
		(-46.083000,127.615000) -- 
		(-46.090000,127.143000) -- 
		(-46.092000,127.052000) -- 
		(-46.113000,126.765000) -- 
		(-46.140000,126.396000) -- 
		(-46.348000,124.669000) -- 
		(-46.590000,122.676000) -- 
		(-46.666000,121.951000) -- 
		(-47.107000,118.388000) -- 
		(-47.160000,117.969000) -- 
		(-47.331000,116.524000) -- 
		(-47.357000,116.294000) -- 
		(-47.404000,115.878000) -- 
		(-47.461000,115.396000) -- 
		(-47.555000,114.637000) -- 
		(-47.607000,114.224000) -- 
		(-47.646000,113.902000) -- 
		(-47.767000,112.936000) -- 
		(-47.782000,112.822000) -- 
		(-47.806000,112.615000) -- 
		(-47.954000,112.593000) -- 
		(-48.041000,112.645000) -- 
		(-48.091000,112.650000) -- 
		(-48.244000,112.669000) -- 
		(-48.295000,112.676000) -- 
		(-48.988000,112.761000) -- 
		(-49.761000,112.857000) -- 
		(-50.000000,112.857000) -- 
		(-50.311000,112.891000) -- 
		(-50.850000,112.929000) -- 
		(-51.074000,112.931000) -- 
		(-51.302000,112.935000) -- 
		(-51.671000,112.985000) -- 
		(-51.762000,113.024000) -- 
		(-51.725000,113.083000) -- 
		(-51.617000,113.266000) -- 
		(-51.581000,113.328000) -- 
		(-51.274000,114.264000) -- 
		(-50.434000,116.607000) -- 
		(-50.162000,117.371000) -- 
		(-49.716000,119.442000) -- 
		(-49.700000,120.524000) -- 
		(-50.179000,121.624000) -- 
		(-51.343000,123.346000) -- 
		(-52.400000,124.543000) -- 
		(-53.155000,125.884000) -- 
		(-53.176000,125.921000) -- 
		(-53.402000,127.155000) -- 
		(-53.312000,127.843000) -- 
		(-53.082000,128.301000) -- 
		(-52.922000,128.752000) -- 
		(-52.941000,129.246000) -- 
		(-52.934000,129.341000) -- 
		(-52.922000,129.548000) -- 
		(-52.921000,129.629000) -- 
		(-52.921000,129.725000) -- 
		(-52.921000,129.828000) -- 
		(-52.921000,129.889000) -- 
		(-52.934000,130.071000) -- 
		(-52.941000,130.141000) -- 
		(-52.952000,130.247000) -- 
		(-52.986000,130.576000) -- 
		(-52.986000,130.829000) -- 
		(-53.031000,131.286000) -- 
		(-53.096000,131.627000) -- 
		(-53.109000,131.726000) -- 
		(-53.186000,132.089000) -- 
		(-53.209000,132.170000) -- 
		(-53.225000,132.226000) -- 
		(-53.264000,132.342000) -- 
		(-53.329000,132.562000) -- 
		(-53.407000,132.672000) -- 
		(-53.485000,132.870000) -- 
		(-53.589000,133.117000) -- 
		(-53.725000,133.387000) -- 
		(-53.829000,133.508000) -- 
		(-53.894000,133.667000) -- 
		(-54.057000,133.898000) -- 
		(-54.141000,133.992000) -- 
		(-54.330000,134.151000) -- 
		(-54.667000,134.679000) -- 
		(-54.836000,134.888000) -- 
		(-55.168000,135.218000) -- 
		(-55.356000,135.290000) -- 
		(-55.852000,135.276000) -- 
		(-56.292000,135.263000) -- 
		(-56.377000,135.252000) -- 
		(-56.605000,135.186000) -- 
		(-56.936000,134.983000) -- 
		(-57.195000,134.811000) -- 
		(-57.203000,134.807000) -- 
		(-57.372000,134.659000) -- 
		(-57.619000,134.543000) -- 
		(-57.671000,134.472000) -- 
		(-57.756000,134.378000) -- 
		(-57.925000,134.417000) -- 
		(-58.062000,134.378000) -- 
		(-58.133000,134.395000) -- 
		(-58.413000,134.412000) -- 
		(-58.419000,134.544000) -- 
		(-58.452000,134.593000) -- 
		(-58.772000,134.767000) -- 
		(-58.378000,135.116000) -- 
		(-58.026000,135.240000) -- 
		(-57.148000,135.553000) -- 
		(-56.779000,135.718000) -- 
		(-56.625000,135.830000) -- 
		(-56.034000,136.841000) -- 
		(-55.643000,137.512000) -- 
		(-55.572000,137.613000) -- 
		(-55.529000,137.674000) -- 
		(-55.399000,137.859000) -- 
		(-55.357000,137.921000) -- 
		(-55.419000,137.903000) -- 
		(-55.601000,137.853000) -- 
		(-55.609000,137.851000) -- 
		(-55.674000,137.843000) -- 
		(-55.725000,137.924000) -- 
		(-55.803000,138.150000) -- 
		(-55.926000,138.458000) -- 
		(-56.102000,139.073000) -- 
		(-56.186000,139.222000) -- 
		(-56.335000,139.508000) -- 
		(-56.433000,139.816000) -- 
		(-56.491000,139.909000) -- 
		(-56.556000,140.047000) -- 
		(-56.706000,140.245000) -- 
		(-56.744000,140.382000) -- 
		(-56.822000,140.575000) -- 
		(-56.887000,140.661000) -- 
		(-56.913000,140.696000) -- 
		(-56.959000,140.833000) -- 
		(-57.115000,141.158000) -- 
		(-57.206000,141.229000) -- 
		(-57.323000,141.444000) -- 
		(-57.336000,141.532000) -- 
		(-57.433000,141.807000) -- 
		(-57.569000,142.005000) -- 
		(-57.699000,142.181000) -- 
		(-57.855000,142.330000) -- 
		(-57.985000,142.385000) -- 
		(-58.408000,142.649000) -- 
		(-58.752000,142.847000) -- 
		(-58.999000,143.188000) -- 
		(-59.220000,143.336000) -- 
		(-59.415000,143.540000) -- 
		(-59.454000,143.589000) -- 
		(-59.656000,143.749000) -- 
		(-59.747000,143.864000) -- 
		(-59.883000,143.991000) -- 
		(-60.189000,144.370000) -- 
		(-60.266000,144.623000) -- 
		(-60.286000,144.794000) -- 
		(-60.364000,145.091000) -- 
		(-60.507000,145.283000) -- 
		(-60.741000,145.564000) -- 
		(-61.046000,145.800000) -- 
		(-61.033000,145.976000) -- 
		(-61.091000,146.295000) -- 
		(-61.072000,146.565000) -- 
		(-61.084000,147.142000) -- 
		(-61.006000,147.434000) -- 
		(-60.863000,147.769000) -- 
		(-60.798000,147.961000) -- 
		(-60.655000,148.121000) -- 
		(-60.603000,148.181000) -- 
		(-60.544000,148.242000) -- 
		(-60.414000,148.484000) -- 
		(-60.355000,148.665000) -- 
		(-60.297000,148.978000) -- 
		(-60.335000,149.341000) -- 
		(-60.303000,149.550000) -- 
		(-60.348000,149.721000) -- 
		(-60.439000,149.974000) -- 
		(-60.556000,150.436000) -- 
		(-60.517000,150.535000) -- 
		(-60.361000,150.656000) -- 
		(-60.341000,150.892000) -- 
		(-60.354000,150.996000) -- 
		(-59.756000,151.368000) -- 
		(-57.962000,152.486000) -- 
		(-57.957000,152.491000) -- 
		(-57.346000,152.830000);
	\filldraw [draw=black, ultra thick, fill=lightgreen]
		(-68.109000,100.219000) -- 
		(-67.511000,100.621000) -- 
		(-67.397000,100.764000) -- 
		(-65.852000,102.696000) -- 
		(-66.594000,104.539000) -- 
		(-68.058000,105.915000) -- 
		(-69.698000,107.401000) -- 
		(-70.899000,109.060000) -- 
		(-71.037000,109.727000) -- 
		(-71.045000,109.767000) -- 
		(-71.168000,110.390000) -- 
		(-71.203000,110.569000) -- 
		(-71.837000,111.378000) -- 
		(-71.904000,111.426000) -- 
		(-71.958000,111.465000) -- 
		(-73.378000,112.374000) -- 
		(-75.681000,114.496000) -- 
		(-76.596000,116.583000) -- 
		(-76.517000,117.211000) -- 
		(-76.360000,118.574000) -- 
		(-75.361000,121.083000) -- 
		(-74.498000,123.032000) -- 
		(-74.928000,124.142000) -- 
		(-76.268000,124.156000) -- 
		(-79.278000,122.796000) -- 
		(-79.500000,122.696000) -- 
		(-81.180000,120.627000) -- 
		(-83.634000,119.197000) -- 
		(-86.098000,118.679000) -- 
		(-86.589000,118.814000) -- 
		(-86.773000,118.863000) -- 
		(-87.211000,118.983000) -- 
		(-87.695000,120.083000) -- 
		(-88.178000,122.218000) -- 
		(-88.440000,123.363000) -- 
		(-88.324000,128.275000) -- 
		(-86.770000,137.008000) -- 
		(-86.725000,137.330000) -- 
		(-86.246000,137.343000) -- 
		(-85.921000,137.412000) -- 
		(-84.042000,137.499000) -- 
		(-84.047000,137.457000) -- 
		(-84.057000,137.388000) -- 
		(-84.035000,137.334000) -- 
		(-84.020000,137.297000) -- 
		(-83.956000,137.220000) -- 
		(-83.864000,137.207000) -- 
		(-83.812000,137.201000) -- 
		(-83.485000,137.206000) -- 
		(-83.307000,137.248000) -- 
		(-83.293000,137.254000) -- 
		(-83.120000,137.337000) -- 
		(-81.892000,137.348000) -- 
		(-81.225000,137.347000) -- 
		(-79.527000,137.345000) -- 
		(-78.622000,137.355000) -- 
		(-76.478000,137.395000) -- 
		(-75.708000,137.416000) -- 
		(-75.557000,137.384000) -- 
		(-75.547000,137.379000) -- 
		(-75.077000,137.198000) -- 
		(-74.605000,137.052000) -- 
		(-74.070000,136.965000) -- 
		(-73.719000,136.926000) -- 
		(-73.122000,136.943000) -- 
		(-71.570000,136.993000) -- 
		(-71.332000,137.019000) -- 
		(-71.084000,137.047000) -- 
		(-70.901000,137.105000) -- 
		(-70.754000,137.153000) -- 
		(-70.727000,137.161000) -- 
		(-70.649000,137.185000) -- 
		(-70.624000,137.195000) -- 
		(-69.681000,137.501000) -- 
		(-69.166000,137.689000) -- 
		(-68.911000,137.747000) -- 
		(-68.580000,137.803000) -- 
		(-67.707000,137.803000) -- 
		(-66.807000,137.805000) -- 
		(-64.265000,137.821000) -- 
		(-62.062000,137.803000) -- 
		(-61.853000,137.801000) -- 
		(-60.489000,137.815000) -- 
		(-58.680000,137.801000) -- 
		(-56.370000,137.787000) -- 
		(-55.920000,137.808000) -- 
		(-55.674000,137.843000) -- 
		(-55.609000,137.851000) -- 
		(-55.601000,137.853000) -- 
		(-55.419000,137.903000) -- 
		(-55.357000,137.921000) -- 
		(-55.399000,137.859000) -- 
		(-55.529000,137.674000) -- 
		(-55.572000,137.613000) -- 
		(-55.643000,137.512000) -- 
		(-56.034000,136.841000) -- 
		(-56.625000,135.830000) -- 
		(-56.779000,135.718000) -- 
		(-57.148000,135.553000) -- 
		(-58.026000,135.240000) -- 
		(-58.378000,135.116000) -- 
		(-58.772000,134.767000) -- 
		(-58.452000,134.593000) -- 
		(-58.419000,134.544000) -- 
		(-58.413000,134.412000) -- 
		(-58.133000,134.395000) -- 
		(-58.062000,134.378000) -- 
		(-57.925000,134.417000) -- 
		(-57.756000,134.378000) -- 
		(-57.671000,134.472000) -- 
		(-57.619000,134.543000) -- 
		(-57.372000,134.659000) -- 
		(-57.203000,134.807000) -- 
		(-57.195000,134.811000) -- 
		(-56.936000,134.983000) -- 
		(-56.605000,135.186000) -- 
		(-56.377000,135.252000) -- 
		(-56.292000,135.263000) -- 
		(-55.852000,135.276000) -- 
		(-55.356000,135.290000) -- 
		(-55.168000,135.218000) -- 
		(-54.836000,134.888000) -- 
		(-54.667000,134.679000) -- 
		(-54.330000,134.151000) -- 
		(-54.141000,133.992000) -- 
		(-54.057000,133.898000) -- 
		(-53.894000,133.667000) -- 
		(-53.829000,133.508000) -- 
		(-53.725000,133.387000) -- 
		(-53.589000,133.117000) -- 
		(-53.485000,132.870000) -- 
		(-53.407000,132.672000) -- 
		(-53.329000,132.562000) -- 
		(-53.264000,132.342000) -- 
		(-53.225000,132.226000) -- 
		(-53.209000,132.170000) -- 
		(-53.186000,132.089000) -- 
		(-53.109000,131.726000) -- 
		(-53.096000,131.627000) -- 
		(-53.031000,131.286000) -- 
		(-52.986000,130.829000) -- 
		(-52.986000,130.576000) -- 
		(-52.952000,130.247000) -- 
		(-52.941000,130.141000) -- 
		(-52.934000,130.071000) -- 
		(-52.921000,129.889000) -- 
		(-52.921000,129.828000) -- 
		(-52.921000,129.725000) -- 
		(-52.921000,129.629000) -- 
		(-52.922000,129.548000) -- 
		(-52.934000,129.341000) -- 
		(-52.941000,129.246000) -- 
		(-52.922000,128.752000) -- 
		(-53.082000,128.301000) -- 
		(-53.312000,127.843000) -- 
		(-53.402000,127.155000) -- 
		(-53.176000,125.921000) -- 
		(-53.155000,125.884000) -- 
		(-52.400000,124.543000) -- 
		(-51.343000,123.346000) -- 
		(-50.179000,121.624000) -- 
		(-49.700000,120.524000) -- 
		(-49.716000,119.442000) -- 
		(-50.162000,117.371000) -- 
		(-50.434000,116.607000) -- 
		(-51.274000,114.264000) -- 
		(-51.581000,113.328000) -- 
		(-51.617000,113.266000) -- 
		(-51.725000,113.083000) -- 
		(-51.762000,113.024000) -- 
		(-51.671000,112.985000) -- 
		(-51.302000,112.935000) -- 
		(-51.074000,112.931000) -- 
		(-50.850000,112.929000) -- 
		(-50.311000,112.891000) -- 
		(-50.000000,112.857000) -- 
		(-49.761000,112.857000) -- 
		(-48.988000,112.761000) -- 
		(-48.295000,112.676000) -- 
		(-48.244000,112.669000) -- 
		(-48.091000,112.650000) -- 
		(-48.041000,112.645000) -- 
		(-47.954000,112.593000) -- 
		(-47.806000,112.615000) -- 
		(-47.782000,112.822000) -- 
		(-47.767000,112.936000) -- 
		(-47.646000,113.902000) -- 
		(-47.607000,114.224000) -- 
		(-47.555000,114.637000) -- 
		(-47.461000,115.396000) -- 
		(-47.404000,115.878000) -- 
		(-47.357000,116.294000) -- 
		(-47.331000,116.524000) -- 
		(-47.160000,117.969000) -- 
		(-47.107000,118.388000) -- 
		(-46.666000,121.951000) -- 
		(-46.590000,122.676000) -- 
		(-46.348000,124.669000) -- 
		(-46.140000,126.396000) -- 
		(-46.113000,126.765000) -- 
		(-46.092000,127.052000) -- 
		(-46.090000,127.143000) -- 
		(-46.083000,127.615000) -- 
		(-46.078000,127.938000) -- 
		(-46.131000,128.164000) -- 
		(-46.119000,128.274000) -- 
		(-46.095000,128.524000) -- 
		(-46.082000,128.652000) -- 
		(-46.079000,128.673000) -- 
		(-46.072000,128.738000) -- 
		(-46.071000,128.760000) -- 
		(-45.956000,128.755000) -- 
		(-45.856000,128.750000) -- 
		(-45.211000,128.724000) -- 
		(-45.062000,128.718000) -- 
		(-45.030000,128.661000) -- 
		(-44.881000,128.393000) -- 
		(-44.823000,128.290000) -- 
		(-44.822000,128.289000) -- 
		(-44.434000,127.594000) -- 
		(-44.399000,127.532000) -- 
		(-44.286000,127.329000) -- 
		(-44.109000,127.013000) -- 
		(-43.579000,126.065000) -- 
		(-43.403000,125.749000) -- 
		(-42.032000,125.891000) -- 
		(-41.504000,125.948000) -- 
		(-41.407000,126.025000) -- 
		(-41.166000,126.405000) -- 
		(-40.757000,126.729000) -- 
		(-40.588000,126.911000) -- 
		(-40.452000,126.993000) -- 
		(-40.205000,127.114000) -- 
		(-39.542000,127.637000) -- 
		(-39.373000,127.736000) -- 
		(-39.171000,127.742000) -- 
		(-38.735000,127.687000) -- 
		(-38.596000,127.657000) -- 
		(-38.475000,127.632000) -- 
		(-38.319000,127.561000) -- 
		(-38.033000,127.341000) -- 
		(-37.851000,127.209000) -- 
		(-37.643000,127.203000) -- 
		(-37.370000,127.296000) -- 
		(-37.438000,127.463000) -- 
		(-37.441000,127.472000) -- 
		(-37.633000,127.914000) -- 
		(-37.678000,127.991000) -- 
		(-37.776000,128.155000) -- 
		(-35.836000,127.963000) -- 
		(-35.472000,127.927000) -- 
		(-34.788000,127.933000) -- 
		(-34.671000,127.965000) -- 
		(-34.569000,128.013000) -- 
		(-34.080000,128.377000) -- 
		(-33.643000,128.686000) -- 
		(-33.591000,128.715000) -- 
		(-32.515000,128.608000) -- 
		(-31.836000,128.556000) -- 
		(-30.995000,128.500000) -- 
		(-30.244000,128.458000) -- 
		(-29.908000,128.441000) -- 
		(-29.376000,128.483000) -- 
		(-29.203000,128.544000) -- 
		(-29.071000,128.663000) -- 
		(-29.020000,128.763000) -- 
		(-29.005000,128.841000) -- 
		(-29.006000,128.991000) -- 
		(-29.052000,129.312000) -- 
		(-29.043000,129.362000) -- 
		(-29.004000,129.381000) -- 
		(-28.984000,129.394000) -- 
		(-28.876000,129.401000) -- 
		(-28.833000,129.405000) -- 
		(-28.168000,129.398000) -- 
		(-27.120000,129.389000) -- 
		(-26.810000,129.408000) -- 
		(-26.621000,129.477000) -- 
		(-26.275000,129.648000) -- 
		(-26.242000,129.682000) -- 
		(-26.067000,129.875000) -- 
		(-26.008000,129.929000) -- 
		(-25.901000,129.949000) -- 
		(-25.781000,129.922000) -- 
		(-25.700000,129.877000) -- 
		(-25.519000,130.217000) -- 
		(-24.977000,131.237000) -- 
		(-24.797000,131.578000) -- 
		(-24.804000,131.640000) -- 
		(-24.791000,131.898000) -- 
		(-24.869000,132.085000) -- 
		(-24.947000,132.140000) -- 
		(-25.103000,132.173000) -- 
		(-25.194000,132.129000) -- 
		(-25.279000,131.942000) -- 
		(-25.370000,131.860000) -- 
		(-25.526000,131.816000) -- 
		(-25.662000,131.832000) -- 
		(-25.838000,131.964000) -- 
		(-25.936000,131.981000) -- 
		(-26.050000,131.950000) -- 
		(-26.065000,131.948000) -- 
		(-26.267000,131.837000) -- 
		(-26.410000,131.766000) -- 
		(-26.579000,131.744000) -- 
		(-26.839000,131.810000) -- 
		(-26.930000,131.892000) -- 
		(-26.950000,132.008000) -- 
		(-26.937000,132.272000) -- 
		(-27.002000,132.404000) -- 
		(-27.139000,132.492000) -- 
		(-27.295000,132.530000) -- 
		(-27.561000,132.502000) -- 
		(-27.763000,132.530000) -- 
		(-27.906000,132.607000) -- 
		(-28.010000,132.788000) -- 
		(-28.114000,132.898000) -- 
		(-28.231000,133.113000) -- 
		(-28.596000,133.607000) -- 
		(-28.752000,133.805000) -- 
		(-28.921000,133.899000) -- 
		(-29.090000,133.871000) -- 
		(-29.155000,133.811000) -- 
		(-29.337000,133.679000) -- 
		(-29.571000,133.651000) -- 
		(-29.688000,133.706000) -- 
		(-29.864000,134.014000) -- 
		(-29.844000,134.524000) -- 
		(-29.844000,134.542000) -- 
		(-29.871000,134.740000) -- 
		(-29.955000,134.960000) -- 
		(-30.040000,135.048000) -- 
		(-30.319000,135.080000) -- 
		(-30.521000,135.025000) -- 
		(-30.716000,134.965000) -- 
		(-30.918000,134.937000) -- 
		(-31.113000,134.998000) -- 
		(-31.249000,135.119000) -- 
		(-31.333000,135.128000) -- 
		(-31.444000,135.140000) -- 
		(-31.548000,135.217000) -- 
		(-31.581000,135.355000) -- 
		(-31.490000,135.437000) -- 
		(-31.360000,135.569000) -- 
		(-31.328000,135.679000) -- 
		(-31.347000,135.872000) -- 
		(-31.425000,135.965000) -- 
		(-31.705000,136.081000) -- 
		(-31.848000,136.191000) -- 
		(-31.900000,136.652000) -- 
		(-31.972000,136.713000) -- 
		(-32.020000,136.723000) -- 
		(-32.069000,136.735000) -- 
		(-32.154000,136.691000) -- 
		(-32.271000,136.641000) -- 
		(-32.368000,136.652000) -- 
		(-32.453000,136.740000) -- 
		(-32.479000,136.817000) -- 
		(-32.466000,136.982000) -- 
		(-32.479000,137.147000) -- 
		(-32.564000,137.334000) -- 
		(-32.700000,137.461000) -- 
		(-32.856000,137.510000) -- 
		(-33.136000,137.515000) -- 
		(-33.331000,137.537000) -- 
		(-33.494000,137.642000) -- 
		(-33.533000,137.724000) -- 
		(-33.507000,137.834000) -- 
		(-33.260000,138.098000) -- 
		(-33.111000,138.291000) -- 
		(-33.072000,138.489000) -- 
		(-33.117000,138.626000) -- 
		(-33.182000,138.648000) -- 
		(-33.338000,138.593000) -- 
		(-33.507000,138.593000) -- 
		(-33.605000,138.670000) -- 
		(-33.670000,138.802000) -- 
		(-33.560000,139.071000) -- 
		(-33.722000,139.401000) -- 
		(-33.631000,139.445000) -- 
		(-33.469000,139.412000) -- 
		(-33.371000,139.434000) -- 
		(-33.352000,139.517000) -- 
		(-33.209000,139.819000) -- 
		(-33.072000,139.990000) -- 
		(-32.994000,140.182000) -- 
		(-32.982000,140.353000) -- 
		(-33.079000,140.512000) -- 
		(-33.216000,140.606000) -- 
		(-33.352000,140.809000) -- 
		(-33.344000,140.845000) -- 
		(-33.333000,140.897000) -- 
		(-33.242000,140.979000) -- 
		(-32.878000,141.046000) -- 
		(-32.657000,141.222000) -- 
		(-32.559000,141.238000) -- 
		(-32.332000,141.183000) -- 
		(-32.091000,141.183000) -- 
		(-31.779000,141.304000) -- 
		(-31.695000,141.392000) -- 
		(-31.423000,141.618000) -- 
		(-31.402000,141.635000) -- 
		(-31.285000,141.695000) -- 
		(-30.713000,142.075000) -- 
		(-30.628000,142.201000) -- 
		(-30.551000,142.439000) -- 
		(-30.531000,142.504000) -- 
		(-30.473000,142.669000) -- 
		(-30.414000,142.702000) -- 
		(-30.323000,142.669000) -- 
		(-30.134000,142.515000) -- 
		(-29.978000,142.454000) -- 
		(-29.777000,142.465000) -- 
		(-29.653000,142.422000) -- 
		(-29.523000,142.383000) -- 
		(-29.322000,142.411000) -- 
		(-29.081000,142.515000) -- 
		(-28.853000,142.774000) -- 
		(-28.815000,142.884000) -- 
		(-28.886000,143.170000) -- 
		(-29.094000,143.356000) -- 
		(-29.108000,143.582000) -- 
		(-29.049000,143.829000) -- 
		(-28.971000,143.961000) -- 
		(-28.809000,144.082000) -- 
		(-28.696000,144.127000) -- 
		(-28.594000,144.170000) -- 
		(-28.464000,144.314000) -- 
		(-28.425000,144.841000) -- 
		(-28.354000,145.127000) -- 
		(-28.290000,145.228000) -- 
		(-28.061000,145.177000) -- 
		(-27.374000,145.025000) -- 
		(-27.145000,144.975000) -- 
		(-26.664000,144.867000) -- 
		(-25.876000,144.692000) -- 
		(-25.391000,144.600000) -- 
		(-25.216000,144.577000) -- 
		(-25.082000,144.560000) -- 
		(-24.838000,144.562000) -- 
		(-24.728000,144.585000) -- 
		(-24.546000,144.624000) -- 
		(-24.010000,144.795000) -- 
		(-22.849000,145.168000) -- 
		(-22.429000,145.316000) -- 
		(-21.875000,145.497000) -- 
		(-21.495000,145.622000) -- 
		(-21.164000,145.730000) -- 
		(-20.796000,145.850000) -- 
		(-19.695000,146.213000) -- 
		(-19.585000,146.250000) -- 
		(-19.328000,146.335000) -- 
		(-18.937000,146.461000) -- 
		(-18.036000,146.755000) -- 
		(-17.766000,146.841000) -- 
		(-17.376000,146.966000) -- 
		(-17.150000,147.037000) -- 
		(-17.122000,147.046000) -- 
		(-17.122000,147.047000) -- 
		(-16.763000,147.162000) -- 
		(-16.475000,147.260000) -- 
		(-16.252000,147.339000) -- 
		(-15.931000,147.450000) -- 
		(-15.440000,147.621000) -- 
		(-14.968000,147.775000) -- 
		(-14.646000,147.882000) -- 
		(-13.166000,148.368000) -- 
		(-12.849000,148.473000) -- 
		(-10.925000,149.118000) -- 
		(-8.583000,149.885000) -- 
		(-8.157000,150.024000) -- 
		(-8.157000,150.025000) -- 
		(-7.803000,150.142000) -- 
		(-7.803000,150.143000) -- 
		(-7.461000,150.256000) -- 
		(-7.058000,150.390000) -- 
		(-6.043000,150.726000) -- 
		(-5.666000,150.851000) -- 
		(-5.597000,150.874000) -- 
		(-4.127000,151.355000) -- 
		(-3.985000,151.399000) -- 
		(-3.471000,151.561000) -- 
		(-3.036000,151.659000) -- 
		(-2.657000,151.709000) -- 
		(-2.270000,151.746000) -- 
		(-1.722000,151.760000) -- 
		(-1.508000,151.766000) -- 
		(-0.770000,151.762000) -- 
		(0.491000,151.755000) -- 
		(1.277000,151.760000) -- 
		(1.644000,151.764000) -- 
		(2.068000,151.833000) -- 
		(2.721000,151.983000) -- 
		(3.007000,152.076000) -- 
		(3.238000,152.151000) -- 
		(3.976000,152.479000) -- 
		(4.630000,152.861000) -- 
		(4.650000,152.872000) -- 
		(6.059000,153.749000) -- 
		(6.769000,154.201000) -- 
		(7.525000,154.643000) -- 
		(9.327000,155.753000) -- 
		(10.885000,156.714000) -- 
		(11.010000,156.791000) -- 
		(11.135000,156.868000) -- 
		(11.196000,156.905000) -- 
		(11.336000,156.992000) -- 
		(11.477000,157.079000) -- 
		(12.498000,157.708000) -- 
		(12.941000,158.040000) -- 
		(13.794000,158.557000) -- 
		(14.507000,159.008000) -- 
		(14.935000,159.261000) -- 
		(15.651000,159.709000) -- 
		(16.020000,159.940000) -- 
		(16.203000,160.031000) -- 
		(16.248000,160.054000) -- 
		(16.566000,160.194000) -- 
		(17.294000,160.471000) -- 
		(17.813000,160.644000) -- 
		(17.999000,160.696000) -- 
		(18.613000,160.871000) -- 
		(18.966000,160.969000) -- 
		(20.023000,161.264000) -- 
		(20.375000,161.363000) -- 
		(20.860000,161.506000) -- 
		(21.220000,161.596000) -- 
		(22.001000,161.793000) -- 
		(23.754000,162.291000) -- 
		(23.931000,162.342000) -- 
		(24.410000,162.496000) -- 
		(24.592000,162.542000) -- 
		(24.646000,162.556000) -- 
		(25.052000,162.657000) -- 
		(25.158000,162.686000) -- 
		(25.414000,162.761000) -- 
		(25.414000,162.762000) -- 
		(25.922000,162.913000) -- 
		(26.420000,163.046000) -- 
		(26.532000,163.076000) -- 
		(26.878000,163.168000) -- 
		(27.602000,163.377000) -- 
		(27.843000,163.440000) -- 
		(28.007000,163.482000) -- 
		(28.007000,163.483000) -- 
		(28.837000,163.703000) -- 
		(28.966000,163.737000) -- 
		(29.413000,163.856000) -- 
		(30.059000,164.026000) -- 
		(30.741000,164.230000) -- 
		(30.741000,164.231000) -- 
		(31.903000,164.570000) -- 
		(32.991000,164.843000) -- 
		(34.176000,165.190000) -- 
		(34.176000,165.191000) -- 
		(34.275000,165.215000) -- 
		(34.275000,165.216000) -- 
		(34.731000,165.330000) -- 
		(35.232000,165.427000) -- 
		(35.652000,165.509000) -- 
		(35.921000,165.547000) -- 
		(36.357000,165.601000) -- 
		(36.357000,165.602000) -- 
		(37.371000,165.749000) -- 
		(37.788000,165.812000) -- 
		(37.969000,165.840000) -- 
		(37.969000,165.841000) -- 
		(38.025000,165.850000) -- 
		(39.039000,165.998000) -- 
		(39.455000,166.061000) -- 
		(39.472000,165.934000) -- 
		(39.524000,165.714000) -- 
		(39.556000,165.230000) -- 
		(39.589000,165.126000) -- 
		(39.585000,165.099000) -- 
		(39.543000,164.878000) -- 
		(39.556000,163.916000) -- 
		(39.569000,163.635000) -- 
		(39.738000,163.382000) -- 
		(39.829000,163.256000) -- 
		(39.894000,162.997000) -- 
		(39.888000,162.821000) -- 
		(39.650000,162.320000) -- 
		(39.595000,162.206000) -- 
		(39.569000,162.057000) -- 
		(39.536000,161.771000) -- 
		(39.464000,161.634000) -- 
		(39.328000,161.469000) -- 
		(39.315000,161.430000) -- 
		(39.082000,161.215000) -- 
		(38.661000,160.355000) -- 
		(38.370000,159.153000) -- 
		(38.442000,158.846000) -- 
		(38.623000,158.071000) -- 
		(38.464000,157.354000) -- 
		(37.892000,156.789000) -- 
		(37.638000,156.074000) -- 
		(36.081000,151.696000) -- 
		(35.928000,150.975000) -- 
		(35.397000,148.483000) -- 
		(35.353000,148.210000) -- 
		(35.349000,147.254000) -- 
		(35.496000,146.277000) -- 
		(35.728000,145.372000) -- 
		(36.124000,144.476000) -- 
		(36.617000,143.430000) -- 
		(37.269000,142.666000) -- 
		(37.678000,142.156000) -- 
		(38.438000,141.257000) -- 
		(39.190000,140.213000) -- 
		(39.920000,139.426000) -- 
		(40.658000,138.696000) -- 
		(40.733000,138.643000) -- 
		(41.925000,137.811000) -- 
		(41.786000,137.074000) -- 
		(41.719000,136.721000) -- 
		(41.542000,136.196000) -- 
		(41.241000,135.303000) -- 
		(40.093000,133.713000) -- 
		(38.852000,132.655000) -- 
		(36.870000,131.180000) -- 
		(36.043000,130.695000) -- 
		(35.957000,130.645000) -- 
		(35.871000,130.595000) -- 
		(34.902000,130.027000) -- 
		(32.778000,128.754000) -- 
		(30.298000,127.156000) -- 
		(29.074000,126.566000) -- 
		(28.249000,125.254000) -- 
		(27.072000,122.314000) -- 
		(27.613000,119.062000) -- 
		(28.513000,117.039000) -- 
		(29.205000,115.485000) -- 
		(30.198000,112.760000) -- 
		(30.264000,112.577000) -- 
		(30.953000,109.199000) -- 
		(31.610000,107.204000) -- 
		(32.075000,106.254000) -- 
		(32.482000,105.422000) -- 
		(33.103000,103.135000) -- 
		(33.192000,101.896000) -- 
		(33.825000,101.676000) -- 
		(34.150000,101.511000) -- 
		(34.520000,101.258000) -- 
		(34.780000,100.928000) -- 
		(34.845000,100.895000) -- 
		(35.800000,100.713000) -- 
		(36.320000,100.614000) -- 
		(36.613000,100.619000) -- 
		(37.555000,100.916000) -- 
		(37.886000,101.012000) -- 
		(38.218000,101.108000) -- 
		(39.986000,101.630000) -- 
		(41.753000,102.153000) -- 
		(44.300000,102.906000) -- 
		(45.948000,103.394000) -- 
		(46.032000,103.378000) -- 
		(46.136000,103.301000) -- 
		(46.311000,103.358000) -- 
		(46.344000,103.306000) -- 
		(46.383000,103.218000) -- 
		(46.344000,103.213000) -- 
		(46.298000,103.169000) -- 
		(46.363000,102.910000) -- 
		(46.506000,102.729000) -- 
		(46.568000,102.694000) -- 
		(46.623000,102.663000) -- 
		(46.717000,102.631000) -- 
		(46.916000,102.564000) -- 
		(47.068000,102.334000) -- 
		(47.124000,102.250000) -- 
		(47.148000,102.195000) -- 
		(47.175000,102.201000) -- 
		(47.187000,102.203000) -- 
		(47.301000,102.236000) -- 
		(47.338000,102.248000) -- 
		(47.388000,102.010000) -- 
		(47.317000,101.789000) -- 
		(46.999000,100.809000) -- 
		(46.835000,99.184000) -- 
		(46.894000,97.974000) -- 
		(46.893000,97.795000) -- 
		(46.928000,97.626000) -- 
		(47.073000,97.473000) -- 
		(49.450000,96.655000) -- 
		(50.795000,96.017000) -- 
		(51.229000,95.785000) -- 
		(50.288000,94.505000) -- 
		(53.937000,92.184000) -- 
		(57.582000,89.866000) -- 
		(63.806000,83.854000) -- 
		(64.616000,83.072000) -- 
		(66.268000,80.335000) -- 
		(67.736000,77.903000) -- 
		(68.888000,75.993000) -- 
		(71.132000,70.119000) -- 
		(71.251000,69.806000) -- 
		(71.754000,67.353000) -- 
		(71.969000,67.177000) -- 
		(72.459000,67.098000) -- 
		(72.691000,66.922000) -- 
		(72.993000,66.443000) -- 
		(73.302000,65.709000) -- 
		(73.471000,65.390000) -- 
		(73.783000,65.326000) -- 
		(74.093000,65.311000) -- 
		(74.274000,65.231000) -- 
		(75.141000,65.312000) -- 
		(76.422000,63.844000) -- 
		(77.909000,61.800000) -- 
		(79.304000,58.925000) -- 
		(79.671000,56.019000) -- 
		(79.661000,53.368000) -- 
		(79.871000,49.775000) -- 
		(80.506000,47.588000) -- 
		(81.140000,46.994000) -- 
		(81.192000,46.929000) -- 
		(80.979000,46.825000) -- 
		(80.857000,46.635000) -- 
		(80.627000,45.956000) -- 
		(80.046000,45.011000) -- 
		(79.755000,44.647000) -- 
		(79.490000,44.482000) -- 
		(79.105000,44.457000) -- 
		(78.587000,44.518000) -- 
		(78.029000,44.682000) -- 
		(77.779000,44.735000) -- 
		(77.408000,44.731000) -- 
		(76.858000,44.709000) -- 
		(76.293000,44.403000) -- 
		(76.110000,44.223000) -- 
		(76.041000,44.033000) -- 
		(76.132000,43.634000) -- 
		(76.354000,43.236000) -- 
		(76.883000,42.660000) -- 
		(77.072000,42.321000) -- 
		(77.118000,42.180000) -- 
		(76.842000,41.507000) -- 
		(76.474000,41.309000) -- 
		(76.052000,41.220000) -- 
		(75.490000,41.177000) -- 
		(75.100000,41.191000) -- 
		(74.734000,41.241000) -- 
		(73.802000,41.410000) -- 
		(72.791000,41.610000) -- 
		(72.452000,41.571000) -- 
		(72.263000,41.504000) -- 
		(72.188000,41.385000) -- 
		(71.990000,41.162000) -- 
		(72.016000,41.044000) -- 
		(71.962000,40.893000) -- 
		(72.139000,40.322000) -- 
		(72.547000,39.562000) -- 
		(72.688000,39.398000) -- 
		(73.032000,39.127000) -- 
		(73.133000,39.003000) -- 
		(73.229000,38.884000) -- 
		(73.360000,38.706000) -- 
		(73.373000,38.536000) -- 
		(73.265000,38.355000) -- 
		(73.092000,38.149000) -- 
		(72.808000,37.897000) -- 
		(72.432000,37.332000) -- 
		(72.551000,36.786000) -- 
		(72.879000,36.437000) -- 
		(72.916000,36.330000) -- 
		(72.963000,36.296000) -- 
		(72.993000,36.146000) -- 
		(72.983000,35.967000) -- 
		(72.788000,35.534000) -- 
		(72.571000,35.250000) -- 
		(72.506000,35.217000) -- 
		(72.402000,35.163000) -- 
		(72.313000,35.165000) -- 
		(71.948000,35.437000) -- 
		(71.748000,35.592000) -- 
		(71.688000,35.848000) -- 
		(71.646000,36.315000) -- 
		(71.547000,36.640000) -- 
		(71.456000,36.829000) -- 
		(71.474000,36.903000) -- 
		(71.384000,36.953000) -- 
		(71.280000,37.074000) -- 
		(71.234000,37.035000) -- 
		(71.222000,37.121000) -- 
		(71.176000,37.453000) -- 
		(71.169000,37.489000) -- 
		(71.175000,37.523000) -- 
		(71.171000,37.558000) -- 
		(71.159000,37.592000) -- 
		(71.141000,37.624000) -- 
		(71.118000,37.654000) -- 
		(71.084000,37.712000) -- 
		(71.069000,37.752000) -- 
		(71.046000,37.782000) -- 
		(71.017000,37.808000) -- 
		(70.951000,37.855000) -- 
		(70.913000,37.860000) -- 
		(70.830000,37.845000) -- 
		(70.702000,37.834000) -- 
		(70.660000,37.822000) -- 
		(70.604000,37.871000) -- 
		(70.507000,37.989000) -- 
		(70.479000,38.016000) -- 
		(70.446000,38.039000) -- 
		(70.294000,38.103000) -- 
		(70.209000,38.106000) -- 
		(70.081000,38.104000) -- 
		(69.955000,38.087000) -- 
		(69.832000,38.058000) -- 
		(69.678000,37.997000) -- 
		(69.355000,37.822000) -- 
		(69.280000,37.787000) -- 
		(68.934000,37.651000) -- 
		(68.500000,37.500000) -- 
		(68.312000,37.416000) -- 
		(68.003000,37.296000) -- 
		(67.924000,37.269000) -- 
		(67.600000,37.179000) -- 
		(67.097000,37.005000) -- 
		(66.776000,36.883000) -- 
		(66.741000,36.863000) -- 
		(66.651000,36.795000) -- 
		(66.589000,36.738000) -- 
		(66.519000,36.648000) -- 
		(66.479000,36.545000) -- 
		(66.458000,36.471000) -- 
		(66.467000,36.435000) -- 
		(66.542000,36.231000) -- 
		(66.576000,36.165000) -- 
		(66.604000,36.138000) -- 
		(66.739000,36.050000) -- 
		(66.809000,36.010000) -- 
		(67.038000,35.914000) -- 
		(67.118000,35.889000) -- 
		(67.280000,35.847000) -- 
		(67.440000,35.800000) -- 
		(67.563000,35.774000) -- 
		(67.647000,35.762000) -- 
		(67.770000,35.738000) -- 
		(67.810000,35.727000) -- 
		(68.103000,35.685000) -- 
		(68.223000,35.653000) -- 
		(68.373000,35.584000) -- 
		(68.405000,35.560000) -- 
		(68.455000,35.502000) -- 
		(68.532000,35.376000) -- 
		(68.581000,35.277000) -- 
		(68.591000,35.242000) -- 
		(68.594000,35.171000) -- 
		(68.582000,35.064000) -- 
		(68.557000,34.995000) -- 
		(68.485000,34.866000) -- 
		(68.416000,34.775000) -- 
		(68.362000,34.721000) -- 
		(68.328000,34.699000) -- 
		(68.179000,34.630000) -- 
		(68.101000,34.601000) -- 
		(67.723000,34.551000) -- 
		(67.638000,34.556000) -- 
		(67.517000,34.592000) -- 
		(67.440000,34.621000) -- 
		(67.331000,34.677000) -- 
		(67.263000,34.720000) -- 
		(67.088000,34.820000) -- 
		(66.646000,35.154000) -- 
		(66.282000,35.458000) -- 
		(66.254000,35.485000) -- 
		(65.889000,35.728000) -- 
		(65.820000,35.770000) -- 
		(65.415000,36.089000) -- 
		(65.328000,36.168000) -- 
		(65.199000,36.308000) -- 
		(65.170000,36.334000) -- 
		(65.129000,36.397000) -- 
		(65.001000,36.659000) -- 
		(64.994000,36.695000) -- 
		(64.989000,36.837000) -- 
		(65.002000,36.978000) -- 
		(65.014000,37.012000) -- 
		(65.044000,37.152000) -- 
		(65.186000,37.487000) -- 
		(65.183000,37.593000) -- 
		(65.177000,37.628000) -- 
		(65.161000,37.661000) -- 
		(65.114000,37.720000) -- 
		(65.060000,37.776000) -- 
		(64.988000,37.814000) -- 
		(64.823000,37.851000) -- 
		(64.781000,37.857000) -- 
		(64.696000,37.850000) -- 
		(64.530000,37.815000) -- 
		(64.413000,37.771000) -- 
		(64.209000,37.642000) -- 
		(64.130000,37.558000) -- 
		(64.040000,37.437000) -- 
		(63.998000,37.388000) -- 
		(63.941000,37.321000) -- 
		(63.920000,37.290000) -- 
		(63.825000,37.091000) -- 
		(63.802000,37.023000) -- 
		(63.729000,36.744000) -- 
		(63.702000,36.603000) -- 
		(63.661000,36.213000) -- 
		(63.622000,35.894000) -- 
		(63.606000,35.576000) -- 
		(63.603000,35.434000) -- 
		(63.608000,35.319000) -- 
		(63.617000,35.135000) -- 
		(63.617000,35.115000) -- 
		(63.648000,34.974000) -- 
		(63.679000,34.872000) -- 
		(63.705000,34.805000) -- 
		(63.743000,34.741000) -- 
		(63.806000,34.623000) -- 
		(63.892000,34.461000) -- 
		(63.906000,34.434000) -- 
		(64.062000,34.220000) -- 
		(64.087000,34.191000) -- 
		(64.201000,34.085000) -- 
		(64.352000,33.958000) -- 
		(64.688000,33.634000) -- 
		(64.774000,33.556000) -- 
		(64.898000,33.412000) -- 
		(64.951000,33.357000) -- 
		(65.128000,33.157000) -- 
		(65.223000,33.039000) -- 
		(65.322000,32.880000) -- 
		(65.394000,32.751000) -- 
		(65.424000,32.646000) -- 
		(65.434000,32.597000) -- 
		(65.446000,32.541000) -- 
		(65.445000,32.505000) -- 
		(65.431000,32.434000) -- 
		(65.404000,32.366000) -- 
		(65.385000,32.334000) -- 
		(65.340000,32.274000) -- 
		(65.285000,32.218000) -- 
		(65.255000,32.193000) -- 
		(65.184000,32.154000) -- 
		(65.121000,32.105000) -- 
		(65.087000,32.084000) -- 
		(64.893000,32.008000) -- 
		(64.463000,31.924000) -- 
		(64.310000,31.902000) -- 
		(64.014000,31.885000) -- 
		(63.978000,31.887000) -- 
		(63.972000,31.888000) -- 
		(63.948000,31.891000) -- 
		(63.923000,31.895000) -- 
		(63.680000,31.928000) -- 
		(63.555000,31.949000) -- 
		(63.475000,31.972000) -- 
		(63.280000,32.045000) -- 
		(63.245000,32.065000) -- 
		(63.080000,32.178000) -- 
		(62.938000,32.310000) -- 
		(62.890000,32.368000) -- 
		(62.837000,32.424000) -- 
		(62.630000,32.695000) -- 
		(62.504000,32.839000) -- 
		(62.446000,32.934000) -- 
		(62.317000,33.119000) -- 
		(62.166000,33.452000) -- 
		(62.130000,33.516000) -- 
		(62.088000,33.579000) -- 
		(62.069000,33.628000) -- 
		(62.007000,33.722000) -- 
		(61.979000,33.749000) -- 
		(61.937000,33.811000) -- 
		(61.853000,33.892000) -- 
		(61.717000,33.978000) -- 
		(61.603000,34.028000) -- 
		(61.562000,34.039000) -- 
		(61.477000,34.037000) -- 
		(61.311000,34.007000) -- 
		(61.190000,33.970000) -- 
		(60.995000,33.898000) -- 
		(60.888000,33.839000) -- 
		(60.823000,33.793000) -- 
		(60.765000,33.741000) -- 
		(60.632000,33.601000) -- 
		(60.612000,33.569000) -- 
		(60.527000,33.368000) -- 
		(60.453000,33.125000) -- 
		(60.441000,33.055000) -- 
		(60.431000,32.913000) -- 
		(60.412000,32.735000) -- 
		(60.414000,32.700000) -- 
		(60.466000,32.527000) -- 
		(60.513000,32.427000) -- 
		(60.627000,32.276000) -- 
		(60.754000,32.090000) -- 
		(60.877000,31.944000) -- 
		(61.048000,31.698000) -- 
		(61.146000,31.581000) -- 
		(61.199000,31.526000) -- 
		(61.256000,31.475000) -- 
		(61.452000,31.239000) -- 
		(61.497000,31.179000) -- 
		(61.576000,31.054000) -- 
		(61.725000,30.686000) -- 
		(61.761000,30.583000) -- 
		(61.808000,30.409000) -- 
		(61.822000,30.304000) -- 
		(61.821000,30.268000) -- 
		(61.832000,30.198000) -- 
		(61.825000,30.057000) -- 
		(61.824000,29.916000) -- 
		(61.818000,29.881000) -- 
		(61.796000,29.812000) -- 
		(61.789000,29.777000) -- 
		(61.716000,29.572000) -- 
		(61.674000,29.510000) -- 
		(61.613000,29.460000) -- 
		(61.544000,29.419000) -- 
		(61.504000,29.405000) -- 
		(61.432000,29.369000) -- 
		(61.356000,29.337000) -- 
		(61.275000,29.313000) -- 
		(61.151000,29.295000) -- 
		(61.092000,29.292000) -- 
		(60.924000,29.308000) -- 
		(60.800000,29.331000) -- 
		(60.721000,29.358000) -- 
		(60.607000,29.405000) -- 
		(60.468000,29.484000) -- 
		(60.385000,29.566000) -- 
		(60.314000,29.653000) -- 
		(60.255000,29.747000) -- 
		(60.233000,29.816000) -- 
		(60.206000,29.883000) -- 
		(60.159000,30.092000) -- 
		(60.141000,30.233000) -- 
		(60.128000,30.267000) -- 
		(60.116000,30.337000) -- 
		(60.089000,30.441000) -- 
		(59.954000,30.888000) -- 
		(59.938000,30.957000) -- 
		(59.906000,31.060000) -- 
		(59.749000,31.503000) -- 
		(59.717000,31.606000) -- 
		(59.710000,31.641000) -- 
		(59.684000,31.708000) -- 
		(59.542000,32.083000) -- 
		(59.493000,32.182000) -- 
		(59.410000,32.307000) -- 
		(59.182000,32.609000) -- 
		(59.125000,32.662000) -- 
		(58.962000,32.827000) -- 
		(58.700000,33.109000) -- 
		(58.617000,33.190000) -- 
		(58.552000,33.236000) -- 
		(58.306000,33.375000) -- 
		(58.177000,33.468000) -- 
		(58.138000,33.483000) -- 
		(58.068000,33.523000) -- 
		(57.881000,33.607000) -- 
		(57.771000,33.662000) -- 
		(57.737000,33.683000) -- 
		(57.588000,33.752000) -- 
		(57.357000,33.845000) -- 
		(57.121000,33.926000) -- 
		(56.998000,33.954000) -- 
		(56.830000,33.978000) -- 
		(56.787000,33.979000) -- 
		(56.624000,33.938000) -- 
		(56.508000,33.893000) -- 
		(56.439000,33.852000) -- 
		(56.312000,33.757000) -- 
		(56.243000,33.715000) -- 
		(56.162000,33.632000) -- 
		(56.048000,33.527000) -- 
		(55.889000,33.359000) -- 
		(55.747000,33.181000) -- 
		(55.522000,32.836000) -- 
		(55.467000,32.740000) -- 
		(55.336000,32.439000) -- 
		(55.310000,32.371000) -- 
		(55.249000,32.163000) -- 
		(55.247000,32.127000) -- 
		(55.221000,32.059000) -- 
		(55.094000,32.065000) -- 
		(55.052000,32.073000) -- 
		(54.817000,32.157000) -- 
		(54.453000,32.255000) -- 
		(54.371000,32.274000) -- 
		(54.118000,32.311000) -- 
		(54.076000,32.310000) -- 
		(53.907000,32.319000) -- 
		(53.696000,32.339000) -- 
		(53.444000,32.374000) -- 
		(53.361000,32.390000) -- 
		(53.126000,32.467000) -- 
		(52.938000,32.544000) -- 
		(52.762000,32.645000) -- 
		(52.734000,32.672000) -- 
		(52.671000,32.718000) -- 
		(52.616000,32.770000) -- 
		(52.466000,32.938000) -- 
		(52.448000,32.971000) -- 
		(52.402000,33.029000) -- 
		(52.333000,33.159000) -- 
		(52.325000,33.194000) -- 
		(52.300000,33.260000) -- 
		(52.283000,33.328000) -- 
		(52.269000,33.362000) -- 
		(52.249000,33.394000) -- 
		(52.182000,33.563000) -- 
		(52.140000,33.626000) -- 
		(52.126000,33.641000) -- 
		(52.114000,33.654000) -- 
		(52.078000,33.647000) -- 
		(51.928000,33.577000) -- 
		(51.895000,33.555000) -- 
		(51.761000,33.417000) -- 
		(51.741000,33.386000) -- 
		(51.600000,33.239000) -- 
		(51.526000,33.163000) -- 
		(51.479000,33.104000) -- 
		(51.426000,33.048000) -- 
		(51.265000,32.931000) -- 
		(51.225000,32.940000) -- 
		(51.186000,32.954000) -- 
		(51.079000,33.013000) -- 
		(51.045000,33.036000) -- 
		(50.949000,33.153000) -- 
		(50.850000,33.311000) -- 
		(50.723000,33.537000) -- 
		(50.699000,33.606000) -- 
		(50.670000,33.710000) -- 
		(50.639000,33.851000) -- 
		(50.613000,34.136000) -- 
		(50.578000,34.299000) -- 
		(50.559000,34.381000) -- 
		(50.458000,34.654000) -- 
		(50.395000,34.862000) -- 
		(50.385000,34.962000) -- 
		(50.493000,35.050000) -- 
		(50.509000,35.065000) -- 
		(50.681000,35.245000) -- 
		(50.777000,35.363000) -- 
		(50.834000,35.422000) -- 
		(50.884000,35.475000) -- 
		(50.942000,35.542000) -- 
		(51.034000,35.649000) -- 
		(51.228000,35.928000) -- 
		(51.277000,36.028000) -- 
		(51.366000,36.229000) -- 
		(51.408000,36.405000) -- 
		(51.474000,36.722000) -- 
		(51.512000,36.935000) -- 
		(51.520000,37.078000) -- 
		(51.522000,37.293000) -- 
		(51.531000,37.543000) -- 
		(51.516000,37.758000) -- 
		(51.467000,38.005000) -- 
		(51.310000,38.600000) -- 
		(51.118000,39.074000) -- 
		(51.086000,39.179000) -- 
		(51.072000,39.212000) -- 
		(51.049000,39.244000) -- 
		(51.014000,39.309000) -- 
		(50.884000,39.495000) -- 
		(50.674000,39.765000) -- 
		(50.335000,40.222000) -- 
		(50.095000,40.518000) -- 
		(50.026000,40.609000) -- 
		(49.767000,40.894000) -- 
		(49.606000,41.106000) -- 
		(49.386000,41.325000) -- 
		(49.177000,41.505000) -- 
		(48.972000,41.688000) -- 
		(48.751000,41.907000) -- 
		(48.403000,42.223000) -- 
		(48.032000,42.519000) -- 
		(47.867000,42.633000) -- 
		(47.795000,42.672000) -- 
		(47.646000,42.743000) -- 
		(47.543000,42.807000) -- 
		(47.387000,42.929000) -- 
		(47.253000,43.019000) -- 
		(46.981000,43.246000) -- 
		(46.947000,43.268000) -- 
		(46.659000,43.423000) -- 
		(46.545000,43.472000) -- 
		(46.362000,43.566000) -- 
		(46.053000,43.688000) -- 
		(45.562000,43.813000) -- 
		(45.311000,43.861000) -- 
		(45.141000,43.873000) -- 
		(44.884000,43.873000) -- 
		(44.715000,43.855000) -- 
		(44.631000,43.840000) -- 
		(44.093000,43.722000) -- 
		(43.773000,43.620000) -- 
		(43.537000,43.535000) -- 
		(43.460000,43.504000) -- 
		(43.236000,43.398000) -- 
		(43.167000,43.356000) -- 
		(42.981000,43.208000) -- 
		(42.968000,43.148000) -- 
		(42.947000,43.118000) -- 
		(42.868000,43.033000) -- 
		(42.727000,42.898000) -- 
		(42.603000,42.710000) -- 
		(42.567000,42.607000) -- 
		(42.548000,42.576000) -- 
		(42.449000,42.506000) -- 
		(42.429000,42.477000) -- 
		(42.327000,42.166000) -- 
		(42.240000,41.853000) -- 
		(42.229000,41.782000) -- 
		(42.203000,41.676000) -- 
		(42.144000,41.504000) -- 
		(42.098000,41.404000) -- 
		(42.009000,41.164000) -- 
		(41.820000,40.688000) -- 
		(41.689000,40.385000) -- 
		(41.547000,39.936000) -- 
		(41.185000,38.905000) -- 
		(40.958000,38.365000) -- 
		(40.804000,38.031000) -- 
		(40.765000,37.967000) -- 
		(40.590000,37.764000) -- 
		(40.321000,37.440000) -- 
		(40.251000,37.350000) -- 
		(40.045000,37.121000) -- 
		(39.679000,36.771000) -- 
		(39.492000,36.623000) -- 
		(39.390000,36.558000) -- 
		(39.318000,36.519000) -- 
		(38.796000,36.274000) -- 
		(38.569000,36.173000) -- 
		(38.332000,36.091000) -- 
		(38.048000,36.012000) -- 
		(37.964000,35.997000) -- 
		(37.884000,35.994000) -- 
		(37.676000,35.989000) -- 
		(37.505000,35.998000) -- 
		(37.379000,36.018000) -- 
		(37.296000,36.037000) -- 
		(37.010000,36.113000) -- 
		(36.932000,36.141000) -- 
		(36.596000,36.299000) -- 
		(36.506000,36.350000) -- 
		(36.063000,36.598000) -- 
		(35.644000,36.905000) -- 
		(35.491000,37.029000) -- 
		(35.165000,37.262000) -- 
		(35.009000,37.385000) -- 
		(34.860000,37.513000) -- 
		(34.514000,37.780000) -- 
		(34.394000,37.888000) -- 
		(33.832000,38.416000) -- 
		(33.075000,39.113000) -- 
		(32.672000,39.481000) -- 
		(32.276000,39.833000) -- 
		(32.182000,39.913000) -- 
		(31.633000,40.363000) -- 
		(31.568000,40.410000) -- 
		(31.460000,40.468000) -- 
		(31.237000,40.575000) -- 
		(31.078000,40.625000) -- 
		(30.913000,40.661000) -- 
		(30.832000,40.684000) -- 
		(30.793000,40.699000) -- 
		(30.712000,40.721000) -- 
		(30.505000,40.767000) -- 
		(30.380000,40.790000) -- 
		(30.252000,40.800000) -- 
		(30.081000,40.804000) -- 
		(29.783000,40.793000) -- 
		(29.057000,40.741000) -- 
		(29.023000,40.736000) -- 
		(28.720000,40.694000) -- 
		(28.557000,40.650000) -- 
		(28.519000,40.635000) -- 
		(28.372000,40.562000) -- 
		(27.819000,40.368000) -- 
		(27.632000,40.280000) -- 
		(27.528000,40.218000) -- 
		(27.302000,40.114000) -- 
		(27.067000,39.959000) -- 
		(26.973000,39.885000) -- 
		(26.946000,39.858000) -- 
		(26.672000,39.537000) -- 
		(26.655000,39.504000) -- 
		(26.559000,39.267000) -- 
		(26.530000,39.162000) -- 
		(26.534000,38.928000) -- 
		(26.531000,38.892000) -- 
		(26.469000,38.647000) -- 
		(26.449000,38.541000) -- 
		(26.437000,38.290000) -- 
		(26.414000,38.004000) -- 
		(26.406000,37.574000) -- 
		(26.437000,37.253000) -- 
		(26.407000,37.003000) -- 
		(26.397000,36.753000) -- 
		(26.413000,36.467000) -- 
		(26.417000,36.145000) -- 
		(26.381000,35.788000) -- 
		(26.369000,35.752000) -- 
		(26.388000,35.723000) -- 
		(26.439000,35.665000) -- 
		(26.473000,35.641000) -- 
		(26.482000,35.610000) -- 
		(26.469000,35.574000) -- 
		(26.458000,35.400000) -- 
		(26.454000,35.326000) -- 
		(26.449000,35.233000) -- 
		(26.428000,35.163000) -- 
		(26.396000,35.138000) -- 
		(26.384000,35.108000) -- 
		(26.437000,34.862000) -- 
		(26.498000,34.473000) -- 
		(26.473000,34.007000) -- 
		(26.472000,33.900000) -- 
		(26.380000,33.574000) -- 
		(26.304000,33.310000) -- 
		(26.228000,33.143000) -- 
		(26.146000,32.939000) -- 
		(26.098000,32.839000) -- 
		(26.060000,32.736000) -- 
		(25.999000,32.603000) -- 
		(25.933000,32.432000) -- 
		(25.750000,31.798000) -- 
		(25.651000,31.459000) -- 
		(25.562000,31.109000) -- 
		(25.563000,31.073000) -- 
		(25.589000,30.896000) -- 
		(25.580000,30.646000) -- 
		(25.574000,30.610000) -- 
		(25.541000,30.496000) -- 
		(25.513000,30.402000) -- 
		(25.473000,30.300000) -- 
		(25.456000,30.229000) -- 
		(25.413000,30.091000) -- 
		(25.280000,29.789000) -- 
		(25.014000,29.340000) -- 
		(24.928000,29.176000) -- 
		(24.777000,28.766000) -- 
		(24.665000,28.421000) -- 
		(24.550000,28.115000) -- 
		(24.442000,27.845000) -- 
		(24.395000,27.708000) -- 
		(24.367000,27.641000) -- 
		(24.321000,27.503000) -- 
		(24.293000,27.436000) -- 
		(24.142000,26.992000) -- 
		(24.114000,26.925000) -- 
		(23.989000,26.475000) -- 
		(23.969000,26.405000) -- 
		(23.804000,26.005000) -- 
		(23.770000,25.942000) -- 
		(23.726000,25.843000) -- 
		(23.614000,25.388000) -- 
		(23.560000,25.201000) -- 
		(23.546000,25.152000) -- 
		(23.522000,25.076000) -- 
		(23.420000,24.919000) -- 
		(23.376000,24.859000) -- 
		(23.311000,24.747000) -- 
		(23.308000,24.731000) -- 
		(23.292000,24.655000) -- 
		(23.240000,24.433000) -- 
		(23.223000,24.359000) -- 
		(23.187000,24.202000) -- 
		(23.038000,23.691000) -- 
		(22.946000,23.342000) -- 
		(23.036000,22.639000) -- 
		(23.067000,22.496000) -- 
		(23.133000,22.194000) -- 
		(23.179000,21.510000) -- 
		(23.189000,20.329000) -- 
		(23.129000,20.005000) -- 
		(23.142000,18.742000) -- 
		(23.135000,18.702000) -- 
		(23.143000,18.636000) -- 
		(23.152000,17.811000) -- 
		(23.096000,17.175000) -- 
		(22.932000,16.827000) -- 
		(22.483000,16.541000) -- 
		(21.768000,16.224000) -- 
		(20.631000,15.786000) -- 
		(18.959000,15.322000) -- 
		(18.786000,15.306000) -- 
		(18.647000,15.294000) -- 
		(18.594000,15.295000) -- 
		(18.536000,15.346000) -- 
		(18.515000,15.377000) -- 
		(18.485000,15.401000) -- 
		(18.252000,15.557000) -- 
		(17.846000,15.875000) -- 
		(17.790000,15.928000) -- 
		(17.642000,16.056000) -- 
		(17.453000,16.245000) -- 
		(17.354000,16.401000) -- 
		(17.306000,16.459000) -- 
		(17.131000,16.700000) -- 
		(16.985000,16.873000) -- 
		(16.950000,16.937000) -- 
		(16.850000,17.091000) -- 
		(16.759000,17.261000) -- 
		(16.694000,17.383000) -- 
		(16.653000,17.444000) -- 
		(16.599000,17.541000) -- 
		(16.499000,17.737000) -- 
		(16.391000,17.970000) -- 
		(16.230000,18.377000) -- 
		(16.211000,18.447000) -- 
		(16.164000,18.584000) -- 
		(16.139000,18.688000) -- 
		(16.135000,18.724000) -- 
		(16.052000,19.108000) -- 
		(15.998000,19.533000) -- 
		(15.951000,20.321000) -- 
		(15.937000,20.893000) -- 
		(15.885000,21.428000) -- 
		(15.863000,21.858000) -- 
		(15.783000,22.458000) -- 
		(15.757000,22.598000) -- 
		(15.745000,22.740000) -- 
		(15.714000,22.988000) -- 
		(15.699000,23.167000) -- 
		(15.682000,23.309000) -- 
		(15.686000,23.630000) -- 
		(15.720000,23.985000) -- 
		(15.724000,24.199000) -- 
		(15.715000,24.341000) -- 
		(15.707000,24.625000) -- 
		(15.696000,24.802000) -- 
		(15.676000,25.334000) -- 
		(15.652000,25.584000) -- 
		(15.628000,25.761000) -- 
		(15.584000,26.154000) -- 
		(15.565000,26.579000) -- 
		(15.579000,26.612000) -- 
		(15.596000,26.716000) -- 
		(15.589000,26.903000) -- 
		(15.602000,28.041000) -- 
		(15.635000,28.239000) -- 
		(15.693000,28.437000) -- 
		(15.862000,28.712000) -- 
		(16.258000,29.080000) -- 
		(16.641000,29.416000) -- 
		(16.706000,29.520000) -- 
		(17.012000,29.877000) -- 
		(17.239000,30.163000) -- 
		(17.453000,30.389000) -- 
		(17.680000,30.614000) -- 
		(17.869000,30.933000) -- 
		(18.148000,31.329000) -- 
		(18.434000,31.791000) -- 
		(18.590000,32.160000) -- 
		(18.681000,32.324000) -- 
		(18.830000,32.665000) -- 
		(18.993000,32.918000) -- 
		(19.149000,33.215000) -- 
		(19.168000,33.331000) -- 
		(19.136000,33.732000) -- 
		(19.006000,34.249000) -- 
		(18.752000,34.585000) -- 
		(18.584000,34.733000) -- 
		(18.441000,34.838000) -- 
		(18.135000,35.035000) -- 
		(18.070000,35.057000) -- 
		(18.006000,35.123000) -- 
		(17.720000,35.261000) -- 
		(17.596000,35.305000) -- 
		(17.291000,35.398000) -- 
		(16.960000,35.475000) -- 
		(16.551000,35.497000) -- 
		(16.154000,35.470000) -- 
		(15.696000,35.352000) -- 
		(15.570000,35.321000) -- 
		(15.180000,35.135000) -- 
		(14.797000,34.865000) -- 
		(14.544000,34.733000) -- 
		(13.751000,34.431000) -- 
		(13.368000,34.299000) -- 
		(13.331000,34.277000) -- 
		(12.900000,34.024000) -- 
		(12.550000,33.901000) -- 
		(12.166000,33.765000) -- 
		(11.920000,33.666000) -- 
		(11.686000,33.617000) -- 
		(11.549000,33.540000) -- 
		(11.238000,33.424000) -- 
		(10.134000,33.171000) -- 
		(8.854000,33.028000) -- 
		(8.458000,33.016000) -- 
		(8.107000,33.006000) -- 
		(7.529000,33.056000) -- 
		(7.295000,33.061000) -- 
		(6.568000,33.171000) -- 
		(6.165000,33.243000) -- 
		(5.555000,33.320000) -- 
		(4.243000,33.528000) -- 
		(3.892000,33.561000) -- 
		(3.067000,33.726000) -- 
		(2.197000,33.979000) -- 
		(1.027000,34.430000) -- 
		(0.820000,34.496000) -- 
		(-0.122000,34.925000) -- 
		(-0.265000,35.002000) -- 
		(-1.006000,35.376000) -- 
		(-1.428000,35.639000) -- 
		(-1.662000,35.799000) -- 
		(-1.863000,35.953000) -- 
		(-1.922000,35.980000) -- 
		(-2.168000,36.145000) -- 
		(-2.539000,36.338000) -- 
		(-2.974000,36.646000) -- 
		(-3.366000,36.881000) -- 
		(-3.403000,36.904000) -- 
		(-3.598000,37.041000) -- 
		(-3.851000,37.234000) -- 
		(-4.137000,37.415000) -- 
		(-4.741000,37.850000) -- 
		(-5.423000,38.553000) -- 
		(-5.553000,38.806000) -- 
		(-5.702000,39.169000) -- 
		(-5.715000,39.197000) -- 
		(-5.742000,39.532000) -- 
		(-5.657000,39.906000) -- 
		(-5.591000,40.017000) -- 
		(-5.501000,40.170000) -- 
		(-5.222000,40.395000) -- 
		(-4.871000,40.588000) -- 
		(-4.729000,40.676000) -- 
		(-3.832000,41.308000) -- 
		(-3.598000,41.534000) -- 
		(-3.332000,41.913000) -- 
		(-3.267000,42.083000) -- 
		(-3.196000,42.271000) -- 
		(-2.923000,42.766000) -- 
		(-2.624000,43.140000) -- 
		(-2.358000,43.371000) -- 
		(-2.098000,43.541000) -- 
		(-1.780000,43.761000) -- 
		(-1.468000,43.998000) -- 
		(-1.338000,44.097000) -- 
		(-1.195000,44.306000) -- 
		(-1.079000,44.515000) -- 
		(-0.975000,44.806000) -- 
		(-0.890000,44.966000) -- 
		(-0.748000,45.142000) -- 
		(-0.540000,45.351000) -- 
		(-0.312000,45.620000) -- 
		(-0.092000,45.813000) -- 
		(0.071000,45.967000) -- 
		(0.545000,46.379000) -- 
		(0.681000,46.511000) -- 
		(1.143000,46.824000) -- 
		(1.604000,47.149000) -- 
		(1.812000,47.303000) -- 
		(2.123000,47.506000) -- 
		(2.292000,47.660000) -- 
		(2.513000,47.820000) -- 
		(2.747000,48.166000) -- 
		(2.974000,48.348000) -- 
		(3.423000,48.591000) -- 
		(3.630000,48.705000) -- 
		(4.111000,48.991000) -- 
		(4.124000,49.019000) -- 
		(4.637000,49.354000) -- 
		(5.196000,49.613000) -- 
		(5.436000,49.745000) -- 
		(6.242000,49.987000) -- 
		(6.268000,50.036000) -- 
		(6.429000,50.088000) -- 
		(6.469000,50.102000) -- 
		(6.742000,50.245000) -- 
		(6.872000,50.339000) -- 
		(6.987000,50.454000) -- 
		(7.008000,50.476000) -- 
		(7.073000,50.509000) -- 
		(7.171000,50.438000) -- 
		(7.417000,50.652000) -- 
		(7.516000,50.712000) -- 
		(7.678000,50.874000) -- 
		(7.726000,50.931000) -- 
		(7.810000,51.010000) -- 
		(7.928000,51.111000) -- 
		(8.087000,51.231000) -- 
		(8.230000,51.308000) -- 
		(8.387000,51.366000) -- 
		(8.470000,51.382000) -- 
		(8.577000,51.408000) -- 
		(8.715000,51.445000) -- 
		(8.942000,51.545000) -- 
		(9.014000,51.583000) -- 
		(9.223000,51.706000) -- 
		(9.323000,51.773000) -- 
		(9.354000,51.798000) -- 
		(9.422000,51.841000) -- 
		(9.542000,51.943000) -- 
		(9.788000,52.189000) -- 
		(9.940000,52.361000) -- 
		(10.032000,52.482000) -- 
		(10.071000,52.545000) -- 
		(10.119000,52.603000) -- 
		(10.244000,52.701000) -- 
		(10.287000,52.762000) -- 
		(10.303000,52.795000) -- 
		(10.323000,52.865000) -- 
		(10.398000,53.069000) -- 
		(10.412000,53.140000) -- 
		(10.423000,53.174000) -- 
		(10.467000,53.275000) -- 
		(10.487000,53.307000) -- 
		(10.531000,53.408000) -- 
		(10.597000,53.578000) -- 
		(10.650000,53.787000) -- 
		(10.679000,53.923000) -- 
		(10.718000,54.102000) -- 
		(10.793000,54.490000) -- 
		(10.797000,54.667000) -- 
		(10.780000,54.737000) -- 
		(10.721000,54.871000) -- 
		(10.595000,55.014000) -- 
		(10.418000,55.165000) -- 
		(10.347000,55.206000) -- 
		(10.233000,55.254000) -- 
		(10.077000,55.286000) -- 
		(10.028000,55.298000) -- 
		(9.943000,55.309000) -- 
		(9.688000,55.330000) -- 
		(9.603000,55.332000) -- 
		(9.519000,55.328000) -- 
		(9.093000,55.356000) -- 
		(9.051000,55.362000) -- 
		(8.928000,55.389000) -- 
		(8.893000,55.408000) -- 
		(8.814000,55.436000) -- 
		(8.463000,55.480000) -- 
		(8.359000,55.502000) -- 
		(8.067000,55.684000) -- 
		(7.846000,55.876000) -- 
		(7.703000,55.959000) -- 
		(7.365000,56.091000) -- 
		(7.273000,56.110000) -- 
		(7.047000,56.157000) -- 
		(6.734000,56.175000) -- 
		(6.001000,56.222000) -- 
		(4.422000,56.250000) -- 
		(4.143000,56.239000) -- 
		(2.856000,56.327000) -- 
		(2.642000,56.338000) -- 
		(2.135000,56.415000) -- 
		(1.882000,56.442000) -- 
		(1.347000,56.568000) -- 
		(1.089000,56.629000) -- 
		(0.186000,56.821000) -- 
		(0.128000,56.871000) -- 
		(-0.217000,56.953000) -- 
		(-0.399000,56.986000) -- 
		(-0.737000,57.118000) -- 
		(-1.412000,57.335000) -- 
		(-2.088000,57.552000) -- 
		(-2.517000,57.657000) -- 
		(-2.829000,57.739000) -- 
		(-3.021000,57.793000) -- 
		(-3.023000,57.794000) -- 
		(-4.095000,57.953000) -- 
		(-4.466000,57.992000) -- 
		(-5.076000,57.992000) -- 
		(-5.336000,57.975000) -- 
		(-5.694000,57.931000) -- 
		(-6.317000,57.783000) -- 
		(-6.869000,57.601000) -- 
		(-7.350000,57.420000) -- 
		(-7.850000,57.090000) -- 
		(-8.039000,56.683000) -- 
		(-8.078000,56.578000) -- 
		(-8.142000,56.358000) -- 
		(-8.194000,55.687000) -- 
		(-8.285000,55.412000) -- 
		(-8.324000,55.159000) -- 
		(-8.383000,54.956000) -- 
		(-8.525000,54.631000) -- 
		(-8.597000,54.527000) -- 
		(-8.620000,54.497000) -- 
		(-8.653000,54.476000) -- 
		(-8.668000,54.443000) -- 
		(-8.784000,54.294000) -- 
		(-8.804000,54.262000) -- 
		(-8.918000,54.132000) -- 
		(-9.134000,53.888000) -- 
		(-9.216000,53.764000) -- 
		(-9.233000,53.731000) -- 
		(-9.253000,53.662000) -- 
		(-9.258000,53.627000) -- 
		(-9.297000,53.489000) -- 
		(-9.302000,53.280000) -- 
		(-9.308000,53.210000) -- 
		(-9.293000,52.893000) -- 
		(-9.248000,52.683000) -- 
		(-9.247000,52.647000) -- 
		(-9.259000,52.543000) -- 
		(-9.296000,52.334000) -- 
		(-9.335000,52.159000) -- 
		(-9.334000,52.016000) -- 
		(-9.317000,51.519000) -- 
		(-9.313000,51.413000) -- 
		(-9.306000,51.198000) -- 
		(-9.337000,51.169000) -- 
		(-9.310000,51.129000) -- 
		(-9.310000,51.118000) -- 
		(-9.315000,51.014000) -- 
		(-9.323000,50.815000) -- 
		(-9.292000,50.786000) -- 
		(-9.289000,50.760000) -- 
		(-9.349000,50.709000) -- 
		(-9.439000,50.361000) -- 
		(-9.536000,50.164000) -- 
		(-9.721000,49.932000) -- 
		(-9.796000,49.839000) -- 
		(-9.960000,49.675000) -- 
		(-10.236000,49.506000) -- 
		(-10.606000,49.328000) -- 
		(-10.833000,49.227000) -- 
		(-11.144000,49.107000) -- 
		(-11.304000,49.057000) -- 
		(-11.636000,48.986000) -- 
		(-11.678000,48.980000) -- 
		(-11.977000,48.961000) -- 
		(-12.958000,49.022000) -- 
		(-12.999000,49.029000) -- 
		(-13.040000,49.041000) -- 
		(-13.119000,49.052000) -- 
		(-13.208000,49.067000) -- 
		(-13.857000,49.251000) -- 
		(-13.895000,49.269000) -- 
		(-14.014000,49.417000) -- 
		(-14.044000,49.443000) -- 
		(-14.082000,49.460000) -- 
		(-14.121000,49.471000) -- 
		(-14.218000,49.509000) -- 
		(-14.352000,49.560000) -- 
		(-14.426000,49.595000) -- 
		(-14.468000,49.603000) -- 
		(-14.553000,49.610000) -- 
		(-14.681000,49.608000) -- 
		(-14.741000,49.643000) -- 
		(-14.785000,49.669000) -- 
		(-15.008000,49.774000) -- 
		(-15.116000,49.833000) -- 
		(-15.484000,50.016000) -- 
		(-15.517000,50.039000) -- 
		(-15.607000,50.115000) -- 
		(-15.754000,50.291000) -- 
		(-15.799000,50.352000) -- 
		(-15.900000,50.509000) -- 
		(-15.998000,50.651000) -- 
		(-16.006000,50.663000) -- 
		(-16.065000,50.758000) -- 
		(-16.081000,50.791000) -- 
		(-16.103000,50.821000) -- 
		(-16.137000,50.842000) -- 
		(-16.178000,50.858000) -- 
		(-16.170000,50.913000) -- 
		(-16.230000,51.117000) -- 
		(-16.255000,51.204000) -- 
		(-16.300000,51.265000) -- 
		(-16.298000,51.297000) -- 
		(-16.289000,51.332000) -- 
		(-16.273000,51.365000) -- 
		(-16.276000,51.398000) -- 
		(-16.335000,51.493000) -- 
		(-16.383000,51.593000) -- 
		(-16.395000,51.627000) -- 
		(-16.408000,51.697000) -- 
		(-16.427000,51.767000) -- 
		(-16.492000,51.937000) -- 
		(-16.655000,52.230000) -- 
		(-16.878000,52.730000) -- 
		(-16.952000,52.922000) -- 
		(-16.970000,52.968000) -- 
		(-17.005000,53.049000) -- 
		(-17.043000,53.135000) -- 
		(-17.124000,53.337000) -- 
		(-17.267000,53.633000) -- 
		(-17.534000,53.999000) -- 
		(-17.551000,54.031000) -- 
		(-17.561000,54.066000) -- 
		(-17.589000,54.279000) -- 
		(-17.623000,54.344000) -- 
		(-17.635000,54.378000) -- 
		(-17.694000,54.473000) -- 
		(-17.780000,54.636000) -- 
		(-18.030000,54.967000) -- 
		(-18.125000,55.086000) -- 
		(-18.395000,55.362000) -- 
		(-18.571000,55.519000) -- 
		(-18.819000,55.713000) -- 
		(-18.995000,55.869000) -- 
		(-19.059000,55.916000) -- 
		(-19.094000,55.937000) -- 
		(-19.366000,56.160000) -- 
		(-19.413000,56.220000) -- 
		(-19.473000,56.270000) -- 
		(-19.554000,56.353000) -- 
		(-19.585000,56.377000) -- 
		(-19.764000,56.474000) -- 
		(-19.900000,56.561000) -- 
		(-20.046000,56.636000) -- 
		(-20.606000,56.897000) -- 
		(-20.880000,56.995000) -- 
		(-20.933000,57.011000) -- 
		(-21.237000,57.112000) -- 
		(-21.314000,57.141000) -- 
		(-21.424000,57.194000) -- 
		(-21.531000,57.253000) -- 
		(-21.687000,57.310000) -- 
		(-21.888000,57.369000) -- 
		(-21.970000,57.388000) -- 
		(-22.388000,57.464000) -- 
		(-22.473000,57.465000) -- 
		(-22.558000,57.456000) -- 
		(-22.643000,57.454000) -- 
		(-22.728000,57.449000) -- 
		(-22.832000,57.476000) -- 
		(-22.844000,57.387000) -- 
		(-22.876000,57.100000) -- 
		(-22.885000,56.837000) -- 
		(-22.887000,56.790000) -- 
		(-22.857000,54.912000) -- 
		(-22.847000,54.271000) -- 
		(-22.835000,53.665000) -- 
		(-22.815000,52.655000) -- 
		(-22.802000,51.852000) -- 
		(-22.798000,51.525000) -- 
		(-22.810000,51.387000) -- 
		(-22.827000,51.318000) -- 
		(-22.862000,51.260000) -- 
		(-22.902000,51.199000) -- 
		(-22.934000,51.179000) -- 
		(-23.062000,51.154000) -- 
		(-23.143000,51.144000) -- 
		(-23.305000,51.136000) -- 
		(-23.483000,51.135000) -- 
		(-23.610000,51.137000) -- 
		(-24.742000,51.141000) -- 
		(-25.472000,51.137000) -- 
		(-26.135000,51.134000) -- 
		(-26.612000,51.132000) -- 
		(-27.681000,51.127000) -- 
		(-28.047000,51.126000) -- 
		(-28.525000,51.125000) -- 
		(-28.962000,51.126000) -- 
		(-30.273000,51.129000) -- 
		(-30.710000,51.132000) -- 
		(-31.335000,51.132000) -- 
		(-33.211000,51.139000) -- 
		(-33.618000,51.141000) -- 
		(-33.835000,51.110000) -- 
		(-34.012000,51.104000) -- 
		(-34.057000,51.105000) -- 
		(-34.057000,51.104000) -- 
		(-34.546000,51.094000) -- 
		(-34.725000,51.091000) -- 
		(-34.802000,51.086000) -- 
		(-34.839000,51.084000) -- 
		(-35.035000,51.072000) -- 
		(-35.113000,51.068000) -- 
		(-35.170000,51.064000) -- 
		(-35.344000,51.053000) -- 
		(-35.402000,51.051000) -- 
		(-35.558000,51.041000) -- 
		(-35.994000,51.014000) -- 
		(-36.029000,51.014000) -- 
		(-36.186000,51.022000) -- 
		(-36.333000,51.028000) -- 
		(-36.599000,51.039000) -- 
		(-36.777000,51.024000) -- 
		(-36.925000,51.012000) -- 
		(-37.318000,51.734000) -- 
		(-38.500000,53.903000) -- 
		(-38.894000,54.626000) -- 
		(-39.076000,54.653000) -- 
		(-39.692000,54.816000) -- 
		(-40.165000,54.990000) -- 
		(-40.511000,55.110000) -- 
		(-41.139000,55.328000) -- 
		(-41.670000,55.489000) -- 
		(-42.014000,55.542000) -- 
		(-42.544000,55.523000) -- 
		(-42.783000,55.462000) -- 
		(-43.451000,55.185000) -- 
		(-43.829000,55.103000) -- 
		(-44.197000,55.050000) -- 
		(-44.620000,55.026000) -- 
		(-44.909000,54.991000) -- 
		(-45.227000,54.884000) -- 
		(-45.383000,54.761000) -- 
		(-45.558000,54.625000) -- 
		(-45.915000,54.361000) -- 
		(-45.993000,54.205000) -- 
		(-45.991000,53.952000) -- 
		(-45.937000,53.404000) -- 
		(-45.930000,53.357000) -- 
		(-45.986000,53.365000) -- 
		(-47.346000,53.525000) -- 
		(-48.946000,53.572000) -- 
		(-50.210000,53.588000) -- 
		(-51.586000,53.636000) -- 
		(-52.178000,53.668000) -- 
		(-53.650000,53.683000) -- 
		(-55.788000,54.027000) -- 
		(-57.827000,54.355000) -- 
		(-58.012000,54.445000) -- 
		(-58.202000,54.537000) -- 
		(-59.300000,55.441000) -- 
		(-59.537000,55.677000) -- 
		(-59.971000,56.107000) -- 
		(-59.652000,57.383000) -- 
		(-58.747000,59.156000) -- 
		(-56.861000,61.323000) -- 
		(-55.525000,62.913000) -- 
		(-54.375000,64.338000) -- 
		(-54.356000,64.362000) -- 
		(-54.279000,64.829000) -- 
		(-53.015000,67.113000) -- 
		(-51.515000,70.858000) -- 
		(-51.238000,73.482000) -- 
		(-51.237000,75.147000) -- 
		(-51.812000,77.462000) -- 
		(-51.808000,78.055000) -- 
		(-51.213000,79.999000) -- 
		(-51.324000,80.237000) -- 
		(-53.128000,80.410000) -- 
		(-53.527000,81.356000) -- 
		(-55.170000,84.422000) -- 
		(-57.655000,87.960000) -- 
		(-60.094000,90.705000) -- 
		(-61.311000,91.891000) -- 
		(-62.200000,91.098000) -- 
		(-63.037000,91.264000) -- 
		(-63.624000,91.388000) -- 
		(-64.948000,90.991000) -- 
		(-66.063000,89.957000) -- 
		(-66.400000,89.435000) -- 
		(-67.192000,88.263000) -- 
		(-67.413000,88.117000) -- 
		(-67.445000,88.095000) -- 
		(-67.917000,87.766000) -- 
		(-67.971000,87.730000) -- 
		(-67.971000,87.729000) -- 
		(-68.203000,87.576000) -- 
		(-68.279000,87.526000) -- 
		(-68.435000,87.422000) -- 
		(-68.711000,87.240000) -- 
		(-70.873000,86.510000) -- 
		(-74.594000,86.587000) -- 
		(-76.999000,87.204000) -- 
		(-77.441000,87.393000) -- 
		(-77.484000,87.413000) -- 
		(-77.884000,87.591000) -- 
		(-77.995000,87.630000) -- 
		(-79.856000,88.286000) -- 
		(-81.818000,89.063000) -- 
		(-83.896000,90.439000) -- 
		(-85.338000,91.992000) -- 
		(-85.883000,93.608000) -- 
		(-85.606000,94.424000) -- 
		(-84.851000,95.127000) -- 
		(-84.696000,95.276000) -- 
		(-84.667000,95.304000) -- 
		(-83.075000,96.304000) -- 
		(-82.338000,97.126000) -- 
		(-82.116000,97.481000) -- 
		(-82.075000,97.547000) -- 
		(-81.380000,98.650000) -- 
		(-81.248000,98.853000) -- 
		(-80.000000,99.577000) -- 
		(-79.342000,99.655000) -- 
		(-78.659000,99.740000) -- 
		(-76.731000,99.324000) -- 
		(-74.485000,98.818000) -- 
		(-71.452000,98.938000) -- 
		(-70.375000,99.188000) -- 
		(-70.269000,99.213000) -- 
		(-69.243000,99.448000) -- 
		(-68.226000,100.140000) -- 
		(-68.168000,100.180000) -- 
		(-68.109000,100.219000);
	\filldraw [draw=black, ultra thick, fill=blue]
		(23.560000,25.201000) -- 
		(23.614000,25.388000) -- 
		(23.726000,25.843000) -- 
		(23.770000,25.942000) -- 
		(23.804000,26.005000) -- 
		(23.969000,26.405000) -- 
		(23.989000,26.475000) -- 
		(24.114000,26.925000) -- 
		(24.142000,26.992000) -- 
		(24.293000,27.436000) -- 
		(24.321000,27.503000) -- 
		(24.367000,27.641000) -- 
		(24.395000,27.708000) -- 
		(24.442000,27.845000) -- 
		(24.550000,28.115000) -- 
		(24.665000,28.421000) -- 
		(24.777000,28.766000) -- 
		(24.928000,29.176000) -- 
		(25.014000,29.340000) -- 
		(25.280000,29.789000) -- 
		(25.413000,30.091000) -- 
		(25.456000,30.229000) -- 
		(25.473000,30.300000) -- 
		(25.513000,30.402000) -- 
		(25.541000,30.496000) -- 
		(25.574000,30.610000) -- 
		(25.580000,30.646000) -- 
		(25.589000,30.896000) -- 
		(25.563000,31.073000) -- 
		(25.562000,31.109000) -- 
		(25.651000,31.459000) -- 
		(25.750000,31.798000) -- 
		(25.933000,32.432000) -- 
		(25.999000,32.603000) -- 
		(26.060000,32.736000) -- 
		(26.098000,32.839000) -- 
		(26.146000,32.939000) -- 
		(26.228000,33.143000) -- 
		(26.304000,33.310000) -- 
		(26.380000,33.574000) -- 
		(26.472000,33.900000) -- 
		(26.473000,34.007000) -- 
		(26.498000,34.473000) -- 
		(26.437000,34.862000) -- 
		(26.384000,35.108000) -- 
		(26.396000,35.138000) -- 
		(26.428000,35.163000) -- 
		(26.449000,35.233000) -- 
		(26.454000,35.326000) -- 
		(26.458000,35.400000) -- 
		(26.469000,35.574000) -- 
		(26.482000,35.610000) -- 
		(26.473000,35.641000) -- 
		(26.439000,35.665000) -- 
		(26.388000,35.723000) -- 
		(26.369000,35.752000) -- 
		(26.381000,35.788000) -- 
		(26.417000,36.145000) -- 
		(26.413000,36.467000) -- 
		(26.397000,36.753000) -- 
		(26.407000,37.003000) -- 
		(26.437000,37.253000) -- 
		(26.406000,37.574000) -- 
		(26.414000,38.004000) -- 
		(26.437000,38.290000) -- 
		(26.449000,38.541000) -- 
		(26.469000,38.647000) -- 
		(26.531000,38.892000) -- 
		(26.534000,38.928000) -- 
		(26.530000,39.162000) -- 
		(26.559000,39.267000) -- 
		(26.655000,39.504000) -- 
		(26.672000,39.537000) -- 
		(26.946000,39.858000) -- 
		(26.973000,39.885000) -- 
		(27.067000,39.959000) -- 
		(27.302000,40.114000) -- 
		(27.528000,40.218000) -- 
		(27.632000,40.280000) -- 
		(27.819000,40.368000) -- 
		(28.372000,40.562000) -- 
		(28.519000,40.635000) -- 
		(28.557000,40.650000) -- 
		(28.720000,40.694000) -- 
		(29.023000,40.736000) -- 
		(29.057000,40.741000) -- 
		(29.783000,40.793000) -- 
		(30.081000,40.804000) -- 
		(30.252000,40.800000) -- 
		(30.380000,40.790000) -- 
		(30.505000,40.767000) -- 
		(30.712000,40.721000) -- 
		(30.793000,40.699000) -- 
		(30.832000,40.684000) -- 
		(30.913000,40.661000) -- 
		(31.078000,40.625000) -- 
		(31.237000,40.575000) -- 
		(31.460000,40.468000) -- 
		(31.568000,40.410000) -- 
		(31.633000,40.363000) -- 
		(32.182000,39.913000) -- 
		(32.276000,39.833000) -- 
		(32.672000,39.481000) -- 
		(33.075000,39.113000) -- 
		(33.832000,38.416000) -- 
		(34.394000,37.888000) -- 
		(34.514000,37.780000) -- 
		(34.860000,37.513000) -- 
		(35.009000,37.385000) -- 
		(35.165000,37.262000) -- 
		(35.491000,37.029000) -- 
		(35.644000,36.905000) -- 
		(36.063000,36.598000) -- 
		(36.506000,36.350000) -- 
		(36.596000,36.299000) -- 
		(36.932000,36.141000) -- 
		(37.010000,36.113000) -- 
		(37.296000,36.037000) -- 
		(37.379000,36.018000) -- 
		(37.505000,35.998000) -- 
		(37.676000,35.989000) -- 
		(37.884000,35.994000) -- 
		(37.964000,35.997000) -- 
		(38.048000,36.012000) -- 
		(38.332000,36.091000) -- 
		(38.569000,36.173000) -- 
		(38.796000,36.274000) -- 
		(39.318000,36.519000) -- 
		(39.390000,36.558000) -- 
		(39.492000,36.623000) -- 
		(39.679000,36.771000) -- 
		(40.045000,37.121000) -- 
		(40.251000,37.350000) -- 
		(40.321000,37.440000) -- 
		(40.590000,37.764000) -- 
		(40.765000,37.967000) -- 
		(40.804000,38.031000) -- 
		(40.958000,38.365000) -- 
		(41.185000,38.905000) -- 
		(41.547000,39.936000) -- 
		(41.689000,40.385000) -- 
		(41.820000,40.688000) -- 
		(42.009000,41.164000) -- 
		(42.098000,41.404000) -- 
		(42.144000,41.504000) -- 
		(42.203000,41.676000) -- 
		(42.229000,41.782000) -- 
		(42.240000,41.853000) -- 
		(42.327000,42.166000) -- 
		(42.429000,42.477000) -- 
		(42.449000,42.506000) -- 
		(42.548000,42.576000) -- 
		(42.567000,42.607000) -- 
		(42.603000,42.710000) -- 
		(42.727000,42.898000) -- 
		(42.868000,43.033000) -- 
		(42.947000,43.118000) -- 
		(42.968000,43.148000) -- 
		(42.981000,43.208000) -- 
		(43.167000,43.356000) -- 
		(43.236000,43.398000) -- 
		(43.460000,43.504000) -- 
		(43.537000,43.535000) -- 
		(43.773000,43.620000) -- 
		(44.093000,43.722000) -- 
		(44.631000,43.840000) -- 
		(44.715000,43.855000) -- 
		(44.884000,43.873000) -- 
		(45.141000,43.873000) -- 
		(45.311000,43.861000) -- 
		(45.562000,43.813000) -- 
		(46.053000,43.688000) -- 
		(46.362000,43.566000) -- 
		(46.545000,43.472000) -- 
		(46.659000,43.423000) -- 
		(46.947000,43.268000) -- 
		(46.981000,43.246000) -- 
		(47.253000,43.019000) -- 
		(47.387000,42.929000) -- 
		(47.543000,42.807000) -- 
		(47.646000,42.743000) -- 
		(47.795000,42.672000) -- 
		(47.867000,42.633000) -- 
		(48.032000,42.519000) -- 
		(48.403000,42.223000) -- 
		(48.751000,41.907000) -- 
		(48.972000,41.688000) -- 
		(49.177000,41.505000) -- 
		(49.386000,41.325000) -- 
		(49.606000,41.106000) -- 
		(49.767000,40.894000) -- 
		(50.026000,40.609000) -- 
		(50.095000,40.518000) -- 
		(50.335000,40.222000) -- 
		(50.674000,39.765000) -- 
		(50.884000,39.495000) -- 
		(51.014000,39.309000) -- 
		(51.049000,39.244000) -- 
		(51.072000,39.212000) -- 
		(51.086000,39.179000) -- 
		(51.118000,39.074000) -- 
		(51.310000,38.600000) -- 
		(51.467000,38.005000) -- 
		(51.516000,37.758000) -- 
		(51.531000,37.543000) -- 
		(51.522000,37.293000) -- 
		(51.520000,37.078000) -- 
		(51.512000,36.935000) -- 
		(51.474000,36.722000) -- 
		(51.408000,36.405000) -- 
		(51.366000,36.229000) -- 
		(51.277000,36.028000) -- 
		(51.228000,35.928000) -- 
		(51.034000,35.649000) -- 
		(50.942000,35.542000) -- 
		(50.884000,35.475000) -- 
		(50.834000,35.422000) -- 
		(50.777000,35.363000) -- 
		(50.681000,35.245000) -- 
		(50.509000,35.065000) -- 
		(50.493000,35.050000) -- 
		(50.385000,34.962000) -- 
		(50.395000,34.862000) -- 
		(50.458000,34.654000) -- 
		(50.559000,34.381000) -- 
		(50.578000,34.299000) -- 
		(50.613000,34.136000) -- 
		(50.639000,33.851000) -- 
		(50.670000,33.710000) -- 
		(50.699000,33.606000) -- 
		(50.723000,33.537000) -- 
		(50.850000,33.311000) -- 
		(50.949000,33.153000) -- 
		(51.045000,33.036000) -- 
		(51.079000,33.013000) -- 
		(51.186000,32.954000) -- 
		(51.225000,32.940000) -- 
		(51.265000,32.931000) -- 
		(51.426000,33.048000) -- 
		(51.479000,33.104000) -- 
		(51.526000,33.163000) -- 
		(51.600000,33.239000) -- 
		(51.741000,33.386000) -- 
		(51.761000,33.417000) -- 
		(51.895000,33.555000) -- 
		(51.928000,33.577000) -- 
		(52.078000,33.647000) -- 
		(52.114000,33.654000) -- 
		(52.126000,33.641000) -- 
		(52.140000,33.626000) -- 
		(52.182000,33.563000) -- 
		(52.249000,33.394000) -- 
		(52.269000,33.362000) -- 
		(52.283000,33.328000) -- 
		(52.300000,33.260000) -- 
		(52.325000,33.194000) -- 
		(52.333000,33.159000) -- 
		(52.402000,33.029000) -- 
		(52.448000,32.971000) -- 
		(52.466000,32.938000) -- 
		(52.616000,32.770000) -- 
		(52.671000,32.718000) -- 
		(52.734000,32.672000) -- 
		(52.762000,32.645000) -- 
		(52.938000,32.544000) -- 
		(53.126000,32.467000) -- 
		(53.361000,32.390000) -- 
		(53.444000,32.374000) -- 
		(53.696000,32.339000) -- 
		(53.907000,32.319000) -- 
		(54.076000,32.310000) -- 
		(54.118000,32.311000) -- 
		(54.371000,32.274000) -- 
		(54.453000,32.255000) -- 
		(54.817000,32.157000) -- 
		(55.052000,32.073000) -- 
		(55.094000,32.065000) -- 
		(55.221000,32.059000) -- 
		(55.247000,32.127000) -- 
		(55.249000,32.163000) -- 
		(55.310000,32.371000) -- 
		(55.336000,32.439000) -- 
		(55.467000,32.740000) -- 
		(55.522000,32.836000) -- 
		(55.747000,33.181000) -- 
		(55.889000,33.359000) -- 
		(56.048000,33.527000) -- 
		(56.162000,33.632000) -- 
		(56.243000,33.715000) -- 
		(56.312000,33.757000) -- 
		(56.439000,33.852000) -- 
		(56.508000,33.893000) -- 
		(56.624000,33.938000) -- 
		(56.787000,33.979000) -- 
		(56.830000,33.978000) -- 
		(56.998000,33.954000) -- 
		(57.121000,33.926000) -- 
		(57.357000,33.845000) -- 
		(57.588000,33.752000) -- 
		(57.737000,33.683000) -- 
		(57.771000,33.662000) -- 
		(57.881000,33.607000) -- 
		(58.068000,33.523000) -- 
		(58.138000,33.483000) -- 
		(58.177000,33.468000) -- 
		(58.306000,33.375000) -- 
		(58.552000,33.236000) -- 
		(58.617000,33.190000) -- 
		(58.700000,33.109000) -- 
		(58.962000,32.827000) -- 
		(59.125000,32.662000) -- 
		(59.182000,32.609000) -- 
		(59.410000,32.307000) -- 
		(59.493000,32.182000) -- 
		(59.542000,32.083000) -- 
		(59.684000,31.708000) -- 
		(59.710000,31.641000) -- 
		(59.717000,31.606000) -- 
		(59.749000,31.503000) -- 
		(59.906000,31.060000) -- 
		(59.938000,30.957000) -- 
		(59.954000,30.888000) -- 
		(60.089000,30.441000) -- 
		(60.116000,30.337000) -- 
		(60.128000,30.267000) -- 
		(60.141000,30.233000) -- 
		(60.159000,30.092000) -- 
		(60.206000,29.883000) -- 
		(60.233000,29.816000) -- 
		(60.255000,29.747000) -- 
		(60.314000,29.653000) -- 
		(60.385000,29.566000) -- 
		(60.468000,29.484000) -- 
		(60.607000,29.405000) -- 
		(60.721000,29.358000) -- 
		(60.800000,29.331000) -- 
		(60.924000,29.308000) -- 
		(61.092000,29.292000) -- 
		(61.151000,29.295000) -- 
		(61.275000,29.313000) -- 
		(61.356000,29.337000) -- 
		(61.432000,29.369000) -- 
		(61.504000,29.405000) -- 
		(61.544000,29.419000) -- 
		(61.613000,29.460000) -- 
		(61.674000,29.510000) -- 
		(61.716000,29.572000) -- 
		(61.789000,29.777000) -- 
		(61.796000,29.812000) -- 
		(61.818000,29.881000) -- 
		(61.824000,29.916000) -- 
		(61.825000,30.057000) -- 
		(61.832000,30.198000) -- 
		(61.821000,30.268000) -- 
		(61.822000,30.304000) -- 
		(61.808000,30.409000) -- 
		(61.761000,30.583000) -- 
		(61.725000,30.686000) -- 
		(61.576000,31.054000) -- 
		(61.497000,31.179000) -- 
		(61.452000,31.239000) -- 
		(61.256000,31.475000) -- 
		(61.199000,31.526000) -- 
		(61.146000,31.581000) -- 
		(61.048000,31.698000) -- 
		(60.877000,31.944000) -- 
		(60.754000,32.090000) -- 
		(60.627000,32.276000) -- 
		(60.513000,32.427000) -- 
		(60.466000,32.527000) -- 
		(60.414000,32.700000) -- 
		(60.412000,32.735000) -- 
		(60.431000,32.913000) -- 
		(60.441000,33.055000) -- 
		(60.453000,33.125000) -- 
		(60.527000,33.368000) -- 
		(60.612000,33.569000) -- 
		(60.632000,33.601000) -- 
		(60.765000,33.741000) -- 
		(60.823000,33.793000) -- 
		(60.888000,33.839000) -- 
		(60.995000,33.898000) -- 
		(61.190000,33.970000) -- 
		(61.311000,34.007000) -- 
		(61.477000,34.037000) -- 
		(61.562000,34.039000) -- 
		(61.603000,34.028000) -- 
		(61.717000,33.978000) -- 
		(61.853000,33.892000) -- 
		(61.937000,33.811000) -- 
		(61.979000,33.749000) -- 
		(62.007000,33.722000) -- 
		(62.069000,33.628000) -- 
		(62.088000,33.579000) -- 
		(62.130000,33.516000) -- 
		(62.166000,33.452000) -- 
		(62.317000,33.119000) -- 
		(62.446000,32.934000) -- 
		(62.504000,32.839000) -- 
		(62.630000,32.695000) -- 
		(62.837000,32.424000) -- 
		(62.890000,32.368000) -- 
		(62.938000,32.310000) -- 
		(63.080000,32.178000) -- 
		(63.245000,32.065000) -- 
		(63.280000,32.045000) -- 
		(63.475000,31.972000) -- 
		(63.555000,31.949000) -- 
		(63.680000,31.928000) -- 
		(63.923000,31.895000) -- 
		(63.948000,31.891000) -- 
		(63.972000,31.888000) -- 
		(63.978000,31.887000) -- 
		(64.014000,31.885000) -- 
		(64.310000,31.902000) -- 
		(64.463000,31.924000) -- 
		(64.893000,32.008000) -- 
		(65.087000,32.084000) -- 
		(65.121000,32.105000) -- 
		(65.184000,32.154000) -- 
		(65.255000,32.193000) -- 
		(65.285000,32.218000) -- 
		(65.340000,32.274000) -- 
		(65.385000,32.334000) -- 
		(65.404000,32.366000) -- 
		(65.431000,32.434000) -- 
		(65.445000,32.505000) -- 
		(65.446000,32.541000) -- 
		(65.434000,32.597000) -- 
		(65.424000,32.646000) -- 
		(65.394000,32.751000) -- 
		(65.322000,32.880000) -- 
		(65.223000,33.039000) -- 
		(65.128000,33.157000) -- 
		(64.951000,33.357000) -- 
		(64.898000,33.412000) -- 
		(64.774000,33.556000) -- 
		(64.688000,33.634000) -- 
		(64.352000,33.958000) -- 
		(64.201000,34.085000) -- 
		(64.087000,34.191000) -- 
		(64.062000,34.220000) -- 
		(63.906000,34.434000) -- 
		(63.892000,34.461000) -- 
		(63.806000,34.623000) -- 
		(63.743000,34.741000) -- 
		(63.705000,34.805000) -- 
		(63.679000,34.872000) -- 
		(63.648000,34.974000) -- 
		(63.617000,35.115000) -- 
		(63.617000,35.135000) -- 
		(63.608000,35.319000) -- 
		(63.603000,35.434000) -- 
		(63.606000,35.576000) -- 
		(63.622000,35.894000) -- 
		(63.661000,36.213000) -- 
		(63.702000,36.603000) -- 
		(63.729000,36.744000) -- 
		(63.802000,37.023000) -- 
		(63.825000,37.091000) -- 
		(63.920000,37.290000) -- 
		(63.941000,37.321000) -- 
		(63.998000,37.388000) -- 
		(64.040000,37.437000) -- 
		(64.130000,37.558000) -- 
		(64.209000,37.642000) -- 
		(64.413000,37.771000) -- 
		(64.530000,37.815000) -- 
		(64.696000,37.850000) -- 
		(64.781000,37.857000) -- 
		(64.823000,37.851000) -- 
		(64.988000,37.814000) -- 
		(65.060000,37.776000) -- 
		(65.114000,37.720000) -- 
		(65.161000,37.661000) -- 
		(65.177000,37.628000) -- 
		(65.183000,37.593000) -- 
		(65.186000,37.487000) -- 
		(65.044000,37.152000) -- 
		(65.014000,37.012000) -- 
		(65.002000,36.978000) -- 
		(64.989000,36.837000) -- 
		(64.994000,36.695000) -- 
		(65.001000,36.659000) -- 
		(65.129000,36.397000) -- 
		(65.170000,36.334000) -- 
		(65.199000,36.308000) -- 
		(65.328000,36.168000) -- 
		(65.415000,36.089000) -- 
		(65.820000,35.770000) -- 
		(65.889000,35.728000) -- 
		(66.254000,35.485000) -- 
		(66.282000,35.458000) -- 
		(66.646000,35.154000) -- 
		(67.088000,34.820000) -- 
		(67.263000,34.720000) -- 
		(67.331000,34.677000) -- 
		(67.440000,34.621000) -- 
		(67.517000,34.592000) -- 
		(67.638000,34.556000) -- 
		(67.723000,34.551000) -- 
		(68.101000,34.601000) -- 
		(68.179000,34.630000) -- 
		(68.328000,34.699000) -- 
		(68.362000,34.721000) -- 
		(68.416000,34.775000) -- 
		(68.485000,34.866000) -- 
		(68.557000,34.995000) -- 
		(68.582000,35.064000) -- 
		(68.594000,35.171000) -- 
		(68.591000,35.242000) -- 
		(68.581000,35.277000) -- 
		(68.532000,35.376000) -- 
		(68.455000,35.502000) -- 
		(68.405000,35.560000) -- 
		(68.373000,35.584000) -- 
		(68.223000,35.653000) -- 
		(68.103000,35.685000) -- 
		(67.810000,35.727000) -- 
		(67.770000,35.738000) -- 
		(67.647000,35.762000) -- 
		(67.563000,35.774000) -- 
		(67.440000,35.800000) -- 
		(67.280000,35.847000) -- 
		(67.118000,35.889000) -- 
		(67.038000,35.914000) -- 
		(66.809000,36.010000) -- 
		(66.739000,36.050000) -- 
		(66.604000,36.138000) -- 
		(66.576000,36.165000) -- 
		(66.542000,36.231000) -- 
		(66.467000,36.435000) -- 
		(66.458000,36.471000) -- 
		(66.479000,36.545000) -- 
		(66.519000,36.648000) -- 
		(66.589000,36.738000) -- 
		(66.651000,36.795000) -- 
		(66.741000,36.863000) -- 
		(66.776000,36.883000) -- 
		(67.097000,37.005000) -- 
		(67.600000,37.179000) -- 
		(67.924000,37.269000) -- 
		(68.003000,37.296000) -- 
		(68.312000,37.416000) -- 
		(68.500000,37.500000) -- 
		(68.934000,37.651000) -- 
		(69.280000,37.787000) -- 
		(69.355000,37.822000) -- 
		(69.678000,37.997000) -- 
		(69.832000,38.058000) -- 
		(69.955000,38.087000) -- 
		(70.081000,38.104000) -- 
		(70.209000,38.106000) -- 
		(70.294000,38.103000) -- 
		(70.446000,38.039000) -- 
		(70.479000,38.016000) -- 
		(70.507000,37.989000) -- 
		(70.604000,37.871000) -- 
		(70.660000,37.822000) -- 
		(70.702000,37.834000) -- 
		(70.830000,37.845000) -- 
		(70.913000,37.860000) -- 
		(70.951000,37.855000) -- 
		(71.017000,37.808000) -- 
		(71.046000,37.782000) -- 
		(71.069000,37.752000) -- 
		(71.084000,37.712000) -- 
		(71.118000,37.654000) -- 
		(71.141000,37.624000) -- 
		(71.159000,37.592000) -- 
		(71.171000,37.558000) -- 
		(71.175000,37.523000) -- 
		(71.169000,37.489000) -- 
		(71.176000,37.453000) -- 
		(71.222000,37.121000) -- 
		(71.234000,37.035000) -- 
		(71.280000,37.074000) -- 
		(71.384000,36.953000) -- 
		(71.474000,36.903000) -- 
		(71.456000,36.829000) -- 
		(71.547000,36.640000) -- 
		(71.646000,36.315000) -- 
		(71.688000,35.848000) -- 
		(71.748000,35.592000) -- 
		(71.948000,35.437000) -- 
		(72.313000,35.165000) -- 
		(72.402000,35.163000) -- 
		(72.544000,35.029000) -- 
		(72.556000,34.963000) -- 
		(72.543000,34.821000) -- 
		(72.541000,34.714000) -- 
		(72.530000,34.607000) -- 
		(72.481000,34.325000) -- 
		(72.441000,34.262000) -- 
		(72.415000,34.235000) -- 
		(72.290000,34.137000) -- 
		(72.041000,33.999000) -- 
		(71.977000,33.953000) -- 
		(71.917000,33.903000) -- 
		(71.866000,33.845000) -- 
		(71.804000,33.713000) -- 
		(71.812000,33.535000) -- 
		(71.833000,33.465000) -- 
		(71.850000,33.432000) -- 
		(71.899000,33.374000) -- 
		(72.011000,33.221000) -- 
		(72.237000,33.008000) -- 
		(72.357000,32.905000) -- 
		(72.441000,32.824000) -- 
		(72.645000,32.640000) -- 
		(72.808000,32.475000) -- 
		(72.986000,32.229000) -- 
		(73.016000,32.163000) -- 
		(73.077000,31.991000) -- 
		(73.102000,31.886000) -- 
		(73.114000,31.780000) -- 
		(73.118000,31.636000) -- 
		(73.110000,31.601000) -- 
		(73.018000,31.363000) -- 
		(72.931000,31.202000) -- 
		(72.696000,30.868000) -- 
		(72.677000,30.835000) -- 
		(72.579000,30.720000) -- 
		(72.513000,30.628000) -- 
		(72.421000,30.466000) -- 
		(72.326000,30.267000) -- 
		(72.243000,29.953000) -- 
		(72.241000,29.810000) -- 
		(72.253000,29.524000) -- 
		(72.252000,29.417000) -- 
		(72.232000,29.168000) -- 
		(72.218000,29.098000) -- 
		(72.196000,29.029000) -- 
		(72.138000,28.894000) -- 
		(72.103000,28.829000) -- 
		(72.079000,28.800000) -- 
		(72.048000,28.775000) -- 
		(71.995000,28.720000) -- 
		(71.728000,28.543000) -- 
		(71.542000,28.458000) -- 
		(71.386000,28.401000) -- 
		(71.345000,28.391000) -- 
		(71.262000,28.381000) -- 
		(71.178000,28.376000) -- 
		(71.052000,28.390000) -- 
		(70.928000,28.410000) -- 
		(70.847000,28.429000) -- 
		(70.751000,28.444000) -- 
		(70.598000,28.469000) -- 
		(70.555000,28.470000) -- 
		(70.300000,28.462000) -- 
		(70.259000,28.454000) -- 
		(70.189000,28.414000) -- 
		(70.037000,28.289000) -- 
		(70.016000,28.258000) -- 
		(69.983000,28.192000) -- 
		(69.871000,27.850000) -- 
		(69.796000,27.681000) -- 
		(69.721000,27.554000) -- 
		(69.657000,27.462000) -- 
		(69.495000,27.254000) -- 
		(69.413000,27.172000) -- 
		(69.327000,27.093000) -- 
		(69.127000,26.908000) -- 
		(69.042000,26.824000) -- 
		(68.936000,26.718000) -- 
		(68.793000,26.541000) -- 
		(68.741000,26.450000) -- 
		(68.683000,26.348000) -- 
		(68.654000,26.281000) -- 
		(68.605000,26.182000) -- 
		(68.530000,26.055000) -- 
		(68.372000,25.843000) -- 
		(68.217000,25.674000) -- 
		(68.100000,25.525000) -- 
		(68.049000,25.469000) -- 
		(67.865000,25.321000) -- 
		(67.811000,25.267000) -- 
		(67.780000,25.241000) -- 
		(67.743000,25.206000) -- 
		(67.644000,25.105000) -- 
		(67.543000,24.948000) -- 
		(67.530000,24.914000) -- 
		(67.494000,24.849000) -- 
		(67.438000,24.735000) -- 
		(67.372000,24.601000) -- 
		(67.349000,24.554000) -- 
		(67.287000,24.384000) -- 
		(67.268000,24.314000) -- 
		(67.244000,24.246000) -- 
		(67.230000,24.140000) -- 
		(67.225000,24.104000) -- 
		(67.228000,23.996000) -- 
		(67.256000,23.820000) -- 
		(67.268000,23.785000) -- 
		(67.289000,23.754000) -- 
		(67.361000,23.625000) -- 
		(67.478000,23.476000) -- 
		(67.729000,23.236000) -- 
		(67.870000,23.121000) -- 
		(67.911000,23.087000) -- 
		(68.048000,22.951000) -- 
		(68.210000,22.743000) -- 
		(68.260000,22.685000) -- 
		(68.329000,22.555000) -- 
		(68.371000,22.454000) -- 
		(68.415000,22.279000) -- 
		(68.418000,22.244000) -- 
		(68.394000,22.139000) -- 
		(68.354000,22.038000) -- 
		(68.311000,21.976000) -- 
		(68.161000,21.805000) -- 
		(67.878000,21.537000) -- 
		(67.599000,21.303000) -- 
		(67.455000,21.185000) -- 
		(67.179000,20.961000) -- 
		(66.967000,20.839000) -- 
		(66.930000,20.822000) -- 
		(66.692000,20.742000) -- 
		(66.575000,20.713000) -- 
		(66.284000,20.641000) -- 
		(66.163000,20.608000) -- 
		(66.011000,20.545000) -- 
		(65.937000,20.510000) -- 
		(65.897000,20.496000) -- 
		(65.829000,20.454000) -- 
		(65.632000,20.316000) -- 
		(65.564000,20.274000) -- 
		(65.465000,20.158000) -- 
		(65.391000,20.029000) -- 
		(65.362000,19.962000) -- 
		(65.337000,19.812000) -- 
		(65.320000,19.715000) -- 
		(65.320000,19.643000) -- 
		(65.371000,19.425000) -- 
		(65.390000,19.393000) -- 
		(65.625000,19.095000) -- 
		(65.676000,19.038000) -- 
		(65.788000,18.933000) -- 
		(65.939000,18.809000) -- 
		(65.996000,18.757000) -- 
		(66.179000,18.608000) -- 
		(66.243000,18.561000) -- 
		(66.344000,18.495000) -- 
		(66.441000,18.425000) -- 
		(66.591000,18.299000) -- 
		(66.615000,18.270000) -- 
		(66.811000,18.083000) -- 
		(66.863000,18.027000) -- 
		(66.919000,17.974000) -- 
		(67.021000,17.859000) -- 
		(67.101000,17.733000) -- 
		(67.114000,17.699000) -- 
		(67.150000,17.635000) -- 
		(67.181000,17.531000) -- 
		(67.194000,17.460000) -- 
		(67.188000,17.353000) -- 
		(67.176000,17.282000) -- 
		(67.137000,17.143000) -- 
		(67.068000,17.012000) -- 
		(66.785000,16.572000) -- 
		(66.680000,16.376000) -- 
		(66.653000,16.314000) -- 
		(66.566000,16.191000) -- 
		(66.488000,16.063000) -- 
		(66.327000,15.656000) -- 
		(66.305000,15.514000) -- 
		(66.295000,15.300000) -- 
		(66.308000,15.158000) -- 
		(66.316000,14.981000) -- 
		(66.330000,14.875000) -- 
		(66.339000,14.840000) -- 
		(66.367000,14.773000) -- 
		(66.413000,14.635000) -- 
		(66.417000,14.627000) -- 
		(66.481000,14.511000) -- 
		(66.520000,14.441000) -- 
		(66.593000,14.354000) -- 
		(66.646000,14.298000) -- 
		(66.842000,14.109000) -- 
		(66.901000,14.061000) -- 
		(67.086000,13.909000) -- 
		(67.230000,13.811000) -- 
		(67.285000,13.774000) -- 
		(67.355000,13.734000) -- 
		(67.646000,13.583000) -- 
		(67.761000,13.530000) -- 
		(67.907000,13.462000) -- 
		(68.023000,13.415000) -- 
		(68.277000,13.323000) -- 
		(68.337000,13.301000) -- 
		(68.537000,13.240000) -- 
		(68.660000,13.197000) -- 
		(68.934000,13.103000) -- 
		(69.053000,13.065000) -- 
		(69.216000,13.019000) -- 
		(69.419000,12.967000) -- 
		(69.707000,12.906000) -- 
		(69.787000,12.881000) -- 
		(70.151000,12.783000) -- 
		(70.479000,12.708000) -- 
		(70.564000,12.698000) -- 
		(70.620000,12.695000) -- 
		(70.691000,12.692000) -- 
		(70.819000,12.691000) -- 
		(70.903000,12.703000) -- 
		(71.025000,12.735000) -- 
		(71.102000,12.767000) -- 
		(71.294000,12.908000) -- 
		(71.328000,12.942000) -- 
		(71.349000,12.963000) -- 
		(71.461000,13.114000) -- 
		(71.482000,13.143000) -- 
		(71.570000,13.269000) -- 
		(71.613000,13.330000) -- 
		(71.685000,13.409000) -- 
		(71.769000,13.500000) -- 
		(71.800000,13.526000) -- 
		(71.866000,13.571000) -- 
		(71.977000,13.624000) -- 
		(72.267000,13.687000) -- 
		(72.310000,13.688000) -- 
		(72.383000,13.676000) -- 
		(72.436000,13.667000) -- 
		(72.557000,13.633000) -- 
		(72.594000,13.615000) -- 
		(72.763000,13.505000) -- 
		(72.882000,13.402000) -- 
		(72.931000,13.344000) -- 
		(72.942000,13.326000) -- 
		(73.045000,13.152000) -- 
		(73.060000,13.119000) -- 
		(73.109000,12.951000) -- 
		(73.121000,12.911000) -- 
		(73.147000,12.806000) -- 
		(73.160000,12.735000) -- 
		(73.165000,12.664000) -- 
		(73.167000,12.550000) -- 
		(73.170000,12.343000) -- 
		(73.142000,12.146000) -- 
		(73.134000,12.094000) -- 
		(73.119000,12.023000) -- 
		(73.103000,11.990000) -- 
		(72.961000,11.770000) -- 
		(72.856000,11.657000) -- 
		(72.792000,11.610000) -- 
		(72.595000,11.513000) -- 
		(72.558000,11.494000) -- 
		(72.271000,11.389000) -- 
		(72.244000,11.379000) -- 
		(71.948000,11.297000) -- 
		(71.840000,11.267000) -- 
		(71.210000,11.163000) -- 
		(70.997000,11.132000) -- 
		(70.831000,11.108000) -- 
		(70.584000,11.053000) -- 
		(70.514000,11.035000) -- 
		(70.463000,11.021000) -- 
		(70.306000,10.970000) -- 
		(70.193000,10.943000) -- 
		(69.394000,10.751000) -- 
		(69.160000,10.695000) -- 
		(68.554000,10.525000) -- 
		(68.193000,10.419000) -- 
		(67.948000,10.356000) -- 
		(67.749000,10.294000) -- 
		(67.587000,10.248000) -- 
		(67.439000,10.197000) -- 
		(67.193000,10.113000) -- 
		(67.042000,10.047000) -- 
		(66.971000,10.007000) -- 
		(66.824000,9.935000) -- 
		(66.698000,9.858000) -- 
		(66.584000,9.788000) -- 
		(66.513000,9.749000) -- 
		(66.411000,9.686000) -- 
		(66.149000,9.504000) -- 
		(65.990000,9.384000) -- 
		(65.803000,9.190000) -- 
		(65.660000,9.058000) -- 
		(65.635000,9.029000) -- 
		(65.619000,8.996000) -- 
		(65.550000,8.806000) -- 
		(65.523000,8.624000) -- 
		(65.505000,8.374000) -- 
		(65.483000,8.233000) -- 
		(65.484000,8.197000) -- 
		(65.526000,7.998000) -- 
		(65.536000,7.951000) -- 
		(65.650000,7.606000) -- 
		(65.667000,7.573000) -- 
		(65.847000,7.374000) -- 
		(65.877000,7.348000) -- 
		(65.929000,7.292000) -- 
		(66.177000,7.047000) -- 
		(66.251000,6.960000) -- 
		(66.417000,6.797000) -- 
		(66.490000,6.709000) -- 
		(66.595000,6.597000) -- 
		(66.690000,6.478000) -- 
		(66.854000,6.314000) -- 
		(66.926000,6.226000) -- 
		(67.010000,6.061000) -- 
		(67.048000,5.997000) -- 
		(67.284000,5.657000) -- 
		(67.356000,5.490000) -- 
		(67.415000,5.318000) -- 
		(67.494000,5.191000) -- 
		(67.620000,5.047000) -- 
		(67.806000,4.851000) -- 
		(68.134000,4.570000) -- 
		(68.300000,4.458000) -- 
		(68.547000,4.317000) -- 
		(68.742000,4.244000) -- 
		(68.783000,4.234000) -- 
		(68.910000,4.217000) -- 
		(68.952000,4.218000) -- 
		(69.036000,4.232000) -- 
		(69.197000,4.281000) -- 
		(69.236000,4.296000) -- 
		(69.309000,4.333000) -- 
		(69.479000,4.441000) -- 
		(69.544000,4.487000) -- 
		(69.581000,4.505000) -- 
		(69.831000,4.700000) -- 
		(69.899000,4.743000) -- 
		(70.181000,4.904000) -- 
		(70.433000,5.038000) -- 
		(70.849000,5.223000) -- 
		(71.172000,5.313000) -- 
		(71.296000,5.342000) -- 
		(71.464000,5.366000) -- 
		(71.635000,5.368000) -- 
		(72.134000,5.276000) -- 
		(72.215000,5.255000) -- 
		(72.656000,5.117000) -- 
		(72.940000,5.039000) -- 
		(73.428000,4.912000) -- 
		(74.067000,4.708000) -- 
		(74.751000,4.501000) -- 
		(74.828000,4.472000) -- 
		(74.902000,4.435000) -- 
		(75.144000,4.288000) -- 
		(75.184000,4.275000) -- 
		(75.331000,4.213000) -- 
		(75.337000,4.210000) -- 
		(75.418000,4.190000) -- 
		(75.516000,4.122000) -- 
		(75.558000,4.089000) -- 
		(75.559000,4.088000) -- 
		(75.578000,4.073000) -- 
		(75.598000,4.042000) -- 
		(75.657000,3.990000) -- 
		(75.692000,3.969000) -- 
		(75.768000,3.937000) -- 
		(75.804000,3.918000) -- 
		(75.901000,3.848000) -- 
		(75.994000,3.774000) -- 
		(76.081000,3.695000) -- 
		(76.326000,3.404000) -- 
		(76.381000,3.349000) -- 
		(76.407000,3.328000) -- 
		(76.442000,3.300000) -- 
		(76.666000,3.138000) -- 
		(76.781000,3.032000) -- 
		(76.871000,2.956000) -- 
		(76.973000,2.843000) -- 
		(77.068000,2.771000) -- 
		(77.201000,2.683000) -- 
		(77.712000,2.363000) -- 
		(77.787000,2.331000) -- 
		(77.842000,2.303000) -- 
		(77.897000,2.276000) -- 
		(78.171000,2.182000) -- 
		(78.321000,2.115000) -- 
		(78.393000,2.076000) -- 
		(78.670000,1.768000) -- 
		(78.803000,1.621000) -- 
		(79.099000,1.230000) -- 
		(79.242000,1.054000) -- 
		(79.321000,0.971000) -- 
		(79.352000,0.946000) -- 
		(79.485000,0.810000) -- 
		(79.515000,0.785000) -- 
		(79.809000,0.577000) -- 
		(79.913000,0.516000) -- 
		(80.091000,0.420000) -- 
		(80.195000,0.357000) -- 
		(80.329000,0.268000) -- 
		(80.583000,0.078000) -- 
		(80.670000,0.000000) -- 
		(80.796000,-0.144000) -- 
		(80.836000,-0.207000) -- 
		(80.927000,-0.328000) -- 
		(81.055000,-0.513000) -- 
		(81.194000,-0.812000) -- 
		(81.195000,-0.814000) -- 
		(81.384000,-1.131000) -- 
		(81.533000,-1.426000) -- 
		(81.742000,-1.778000) -- 
		(81.902000,-2.109000) -- 
		(81.990000,-2.311000) -- 
		(82.050000,-2.427000) -- 
		(82.125000,-2.574000) -- 
		(82.147000,-2.604000) -- 
		(82.138000,-2.639000) -- 
		(82.135000,-2.854000) -- 
		(82.145000,-2.961000) -- 
		(82.137000,-3.031000) -- 
		(82.122000,-3.102000) -- 
		(82.089000,-3.349000) -- 
		(81.997000,-3.625000) -- 
		(81.978000,-3.657000) -- 
		(81.948000,-3.723000) -- 
		(81.911000,-3.787000) -- 
		(81.896000,-3.821000) -- 
		(81.665000,-4.164000) -- 
		(81.522000,-4.341000) -- 
		(81.359000,-4.530000) -- 
		(81.272000,-4.630000) -- 
		(80.966000,-5.017000) -- 
		(80.930000,-5.073000) -- 
		(80.721000,-5.394000) -- 
		(80.659000,-5.528000) -- 
		(80.578000,-5.731000) -- 
		(80.417000,-6.023000) -- 
		(80.348000,-6.174000) -- 
		(80.264000,-6.357000) -- 
		(80.156000,-6.551000) -- 
		(80.025000,-6.816000) -- 
		(79.986000,-6.910000) -- 
		(79.927000,-7.053000) -- 
		(79.891000,-7.166000) -- 
		(79.911000,-7.197000) -- 
		(79.942000,-7.265000) -- 
		(80.020000,-7.580000) -- 
		(80.058000,-7.864000) -- 
		(80.071000,-7.985000) -- 
		(80.086000,-8.134000) -- 
		(80.102000,-8.283000) -- 
		(80.103000,-8.292000) -- 
		(80.123000,-8.614000) -- 
		(80.120000,-8.650000) -- 
		(80.108000,-8.704000) -- 
		(80.104000,-8.720000) -- 
		(80.047000,-8.892000) -- 
		(80.146000,-9.009000) -- 
		(80.235000,-9.095000) -- 
		(80.315000,-9.171000) -- 
		(80.363000,-9.230000) -- 
		(80.392000,-9.256000) -- 
		(80.526000,-9.345000) -- 
		(80.551000,-9.357000) -- 
		(80.563000,-9.363000) -- 
		(80.603000,-9.375000) -- 
		(80.632000,-9.400000) -- 
		(80.792000,-9.568000) -- 
		(80.876000,-9.648000) -- 
		(80.952000,-9.734000) -- 
		(80.974000,-9.764000) -- 
		(81.153000,-9.916000) -- 
		(81.261000,-10.025000) -- 
		(81.294000,-10.062000) -- 
		(81.338000,-10.110000) -- 
		(81.436000,-10.227000) -- 
		(81.489000,-10.283000) -- 
		(81.577000,-10.362000) -- 
		(81.634000,-10.408000) -- 
		(81.700000,-10.461000) -- 
		(81.749000,-10.493000) -- 
		(81.870000,-10.569000) -- 
		(81.906000,-10.588000) -- 
		(82.140000,-10.675000) -- 
		(82.219000,-10.704000) -- 
		(82.299000,-10.729000) -- 
		(82.383000,-10.742000) -- 
		(82.468000,-10.748000) -- 
		(82.658000,-10.725000) -- 
		(82.765000,-10.719000) -- 
		(82.869000,-10.729000) -- 
		(82.977000,-10.739000) -- 
		(83.032000,-10.752000) -- 
		(83.059000,-10.758000) -- 
		(83.526000,-10.799000) -- 
		(84.451000,-10.930000) -- 
		(84.580000,-10.948000) -- 
		(84.707000,-10.959000) -- 
		(84.916000,-10.991000) -- 
		(84.999000,-11.000000) -- 
		(85.114000,-10.953000) -- 
		(85.156000,-10.946000) -- 
		(85.269000,-10.897000) -- 
		(85.312000,-10.890000) -- 
		(85.353000,-10.894000) -- 
		(85.377000,-10.921000) -- 
		(85.390000,-10.955000) -- 
		(85.407000,-11.025000) -- 
		(85.408000,-11.061000) -- 
		(85.364000,-11.209000) -- 
		(85.411000,-11.259000) -- 
		(85.608000,-11.535000) -- 
		(85.634000,-11.564000) -- 
		(85.839000,-11.791000) -- 
		(85.947000,-11.898000) -- 
		(86.060000,-12.011000) -- 
		(86.183000,-12.116000) -- 
		(86.448000,-12.343000) -- 
		(86.602000,-12.534000) -- 
		(86.734000,-12.699000) -- 
		(86.786000,-12.755000) -- 
		(86.948000,-12.965000) -- 
		(87.198000,-13.380000) -- 
		(87.423000,-13.691000) -- 
		(87.626000,-13.958000) -- 
		(87.795000,-14.207000) -- 
		(87.886000,-14.369000) -- 
		(88.019000,-14.670000) -- 
		(88.031000,-14.705000) -- 
		(88.082000,-14.803000) -- 
		(88.250000,-15.091000) -- 
		(88.511000,-15.500000) -- 
		(88.527000,-15.533000) -- 
		(88.535000,-15.568000) -- 
		(88.567000,-15.592000) -- 
		(88.595000,-15.619000) -- 
		(88.615000,-15.651000) -- 
		(88.626000,-15.685000) -- 
		(88.719000,-15.859000) -- 
		(88.730000,-15.881000) -- 
		(88.834000,-16.037000) -- 
		(89.001000,-16.199000) -- 
		(89.208000,-16.392000) -- 
		(89.259000,-16.439000) -- 
		(89.507000,-16.685000) -- 
		(89.848000,-17.095000) -- 
		(90.158000,-17.479000) -- 
		(90.228000,-17.608000) -- 
		(90.303000,-17.814000) -- 
		(90.387000,-18.091000) -- 
		(90.425000,-18.303000) -- 
		(90.442000,-18.415000) -- 
		(90.460000,-18.527000) -- 
		(90.471000,-18.601000) -- 
		(90.472000,-18.603000) -- 
		(90.477000,-18.628000) -- 
		(90.507000,-18.772000) -- 
		(90.515000,-18.811000) -- 
		(90.520000,-18.829000) -- 
		(90.537000,-18.903000) -- 
		(90.418000,-18.915000) -- 
		(87.987000,-19.158000) -- 
		(86.879000,-19.266000) -- 
		(86.023000,-19.315000) -- 
		(83.262000,-19.472000) -- 
		(81.021000,-19.599000) -- 
		(80.321000,-19.697000) -- 
		(77.782000,-20.048000) -- 
		(77.348000,-19.784000) -- 
		(76.914000,-19.812000) -- 
		(75.224000,-19.918000) -- 
		(74.116000,-20.169000) -- 
		(73.195000,-20.377000) -- 
		(73.036000,-20.078000) -- 
		(72.317000,-20.148000) -- 
		(70.921000,-20.569000) -- 
		(70.669000,-20.920000) -- 
		(70.102000,-21.704000) -- 
		(62.995000,-25.119000) -- 
		(60.437000,-26.348000) -- 
		(60.335000,-26.397000) -- 
		(60.233000,-26.445000) -- 
		(58.621000,-27.204000) -- 
		(58.478000,-27.272000) -- 
		(58.336000,-27.339000) -- 
		(58.245000,-27.382000) -- 
		(56.633000,-28.141000) -- 
		(55.794000,-28.538000) -- 
		(53.800000,-29.476000) -- 
		(53.027000,-28.357000) -- 
		(52.977000,-28.285000) -- 
		(48.981000,-30.122000) -- 
		(45.815000,-31.577000) -- 
		(45.804000,-33.365000) -- 
		(44.066000,-33.379000) -- 
		(39.329000,-33.415000) -- 
		(39.187000,-33.416000) -- 
		(39.045000,-33.417000) -- 
		(38.979000,-33.363000) -- 
		(38.928000,-33.306000) -- 
		(38.863000,-33.173000) -- 
		(38.848000,-33.103000) -- 
		(38.849000,-33.094000) -- 
		(38.850000,-33.031000) -- 
		(38.985000,-32.363000) -- 
		(38.998000,-32.256000) -- 
		(39.005000,-32.150000) -- 
		(39.000000,-32.078000) -- 
		(38.967000,-31.939000) -- 
		(38.941000,-31.833000) -- 
		(38.904000,-31.731000) -- 
		(38.861000,-31.630000) -- 
		(38.833000,-31.572000) -- 
		(38.670000,-31.234000) -- 
		(38.457000,-30.846000) -- 
		(38.206000,-30.355000) -- 
		(38.016000,-30.075000) -- 
		(37.988000,-30.048000) -- 
		(37.885000,-29.985000) -- 
		(37.772000,-29.950000) -- 
		(37.687000,-29.945000) -- 
		(37.666000,-29.951000) -- 
		(37.563000,-29.971000) -- 
		(37.444000,-30.011000) -- 
		(37.402000,-30.020000) -- 
		(37.241000,-30.068000) -- 
		(37.048000,-30.142000) -- 
		(36.753000,-30.287000) -- 
		(36.329000,-30.524000) -- 
		(36.192000,-30.609000) -- 
		(35.930000,-30.728000) -- 
		(35.772000,-30.781000) -- 
		(35.267000,-30.855000) -- 
		(35.099000,-30.875000) -- 
		(34.847000,-30.915000) -- 
		(34.299000,-30.985000) -- 
		(33.804000,-31.088000) -- 
		(33.510000,-31.134000) -- 
		(33.467000,-31.138000) -- 
		(33.382000,-31.133000) -- 
		(32.807000,-30.995000) -- 
		(32.651000,-30.940000) -- 
		(32.575000,-30.907000) -- 
		(32.419000,-30.852000) -- 
		(32.233000,-30.767000) -- 
		(32.193000,-30.754000) -- 
		(32.079000,-30.706000) -- 
		(31.934000,-30.633000) -- 
		(31.782000,-30.569000) -- 
		(31.718000,-30.523000) -- 
		(31.429000,-30.313000) -- 
		(31.313000,-30.223000) -- 
		(30.941000,-29.909000) -- 
		(30.751000,-29.765000) -- 
		(30.726000,-29.737000) -- 
		(30.659000,-29.694000) -- 
		(30.354000,-29.369000) -- 
		(30.250000,-29.303000) -- 
		(30.144000,-29.160000) -- 
		(30.035000,-29.012000) -- 
		(29.906000,-28.808000) -- 
		(29.724000,-28.566000) -- 
		(29.510000,-28.335000) -- 
		(29.335000,-28.192000) -- 
		(29.127000,-27.983000) -- 
		(29.049000,-27.928000) -- 
		(29.005000,-27.882000) -- 
		(28.790000,-27.659000) -- 
		(28.640000,-27.582000) -- 
		(28.498000,-27.494000) -- 
		(28.017000,-27.340000) -- 
		(27.887000,-27.208000) -- 
		(27.764000,-27.032000) -- 
		(27.660000,-26.724000) -- 
		(27.544000,-26.570000) -- 
		(27.446000,-26.493000) -- 
		(27.115000,-26.146000) -- 
		(26.830000,-25.970000) -- 
		(26.479000,-25.855000) -- 
		(26.278000,-25.739000) -- 
		(26.051000,-25.552000) -- 
		(25.837000,-25.255000) -- 
		(25.801000,-25.236000) -- 
		(25.762000,-25.224000) -- 
		(25.706000,-25.173000) -- 
		(25.685000,-25.142000) -- 
		(25.494000,-25.000000) -- 
		(25.317000,-24.903000) -- 
		(25.244000,-24.870000) -- 
		(25.085000,-24.768000) -- 
		(25.008000,-24.718000) -- 
		(24.972000,-24.701000) -- 
		(24.903000,-24.659000) -- 
		(24.792000,-24.609000) -- 
		(24.648000,-24.535000) -- 
		(24.517000,-24.443000) -- 
		(24.446000,-24.404000) -- 
		(24.408000,-24.389000) -- 
		(24.383000,-24.374000) -- 
		(24.338000,-24.348000) -- 
		(23.817000,-23.980000) -- 
		(23.742000,-23.945000) -- 
		(23.602000,-23.909000) -- 
		(23.603000,-23.873000) -- 
		(23.603000,-23.834000) -- 
		(23.604000,-23.765000) -- 
		(23.604000,-23.759000) -- 
		(23.618000,-23.732000) -- 
		(23.644000,-23.686000) -- 
		(23.721000,-23.545000) -- 
		(23.746000,-23.498000) -- 
		(24.034000,-23.478000) -- 
		(24.111000,-23.469000) -- 
		(24.269000,-23.434000) -- 
		(24.412000,-23.389000) -- 
		(24.480000,-23.364000) -- 
		(24.615000,-23.297000) -- 
		(24.697000,-23.240000) -- 
		(24.763000,-23.188000) -- 
		(24.875000,-23.082000) -- 
		(24.964000,-22.972000) -- 
		(25.011000,-22.896000) -- 
		(25.055000,-22.813000) -- 
		(25.160000,-22.647000) -- 
		(25.199000,-22.586000) -- 
		(25.215000,-22.560000) -- 
		(25.371000,-22.293000) -- 
		(25.422000,-22.203000) -- 
		(25.430000,-22.190000) -- 
		(25.454000,-22.150000) -- 
		(25.461000,-22.136000) -- 
		(25.557000,-21.972000) -- 
		(25.767000,-21.612000) -- 
		(25.767000,-21.611000) -- 
		(25.844000,-21.480000) -- 
		(25.939000,-21.314000) -- 
		(26.005000,-21.201000) -- 
		(26.013000,-21.187000) -- 
		(26.200000,-20.861000) -- 
		(26.265000,-20.747000) -- 
		(26.680000,-20.033000) -- 
		(27.056000,-19.383000) -- 
		(27.397000,-18.801000) -- 
		(27.607000,-18.453000) -- 
		(27.960000,-17.912000) -- 
		(28.411000,-17.220000) -- 
		(28.554000,-16.999000) -- 
		(28.643000,-16.860000) -- 
		(28.672000,-16.794000) -- 
		(28.729000,-16.653000) -- 
		(28.808000,-16.456000) -- 
		(28.859000,-16.357000) -- 
		(28.908000,-16.299000) -- 
		(28.920000,-16.282000) -- 
		(29.100000,-16.118000) -- 
		(29.196000,-15.898000) -- 
		(29.230000,-15.888000) -- 
		(29.331000,-15.857000) -- 
		(29.364000,-15.844000) -- 
		(29.492000,-15.804000) -- 
		(29.875000,-15.682000) -- 
		(30.002000,-15.641000) -- 
		(29.959000,-15.491000) -- 
		(29.912000,-15.327000) -- 
		(29.824000,-15.038000) -- 
		(29.788000,-14.921000) -- 
		(29.777000,-14.886000) -- 
		(29.772000,-14.870000) -- 
		(29.757000,-14.822000) -- 
		(29.751000,-14.805000) -- 
		(29.915000,-14.780000) -- 
		(30.285000,-14.412000) -- 
		(30.481000,-14.161000) -- 
		(30.558000,-14.060000) -- 
		(30.746000,-13.774000) -- 
		(30.993000,-13.032000) -- 
		(31.142000,-12.812000) -- 
		(31.292000,-12.691000) -- 
		(31.493000,-12.575000) -- 
		(31.895000,-12.284000) -- 
		(32.259000,-11.982000) -- 
		(32.301000,-11.883000) -- 
		(32.402000,-11.635000) -- 
		(32.460000,-11.470000) -- 
		(32.713000,-11.261000) -- 
		(32.914000,-11.135000) -- 
		(33.142000,-10.992000) -- 
		(33.356000,-10.838000) -- 
		(33.402000,-10.739000) -- 
		(33.562000,-10.051000) -- 
		(33.642000,-9.705000) -- 
		(33.785000,-9.458000) -- 
		(33.967000,-9.320000) -- 
		(34.058000,-9.232000) -- 
		(34.194000,-9.134000) -- 
		(34.259000,-9.106000) -- 
		(34.986000,-8.639000) -- 
		(35.259000,-8.485000) -- 
		(35.733000,-8.314000) -- 
		(35.837000,-8.287000) -- 
		(36.239000,-8.150000) -- 
		(36.473000,-8.078000) -- 
		(36.720000,-7.919000) -- 
		(36.769000,-7.879000) -- 
		(37.018000,-7.671000) -- 
		(37.070000,-7.600000) -- 
		(37.155000,-7.440000) -- 
		(37.246000,-7.132000) -- 
		(37.246000,-6.874000) -- 
		(37.208000,-6.702000) -- 
		(37.189000,-6.612000) -- 
		(37.142000,-6.401000) -- 
		(37.148000,-6.340000) -- 
		(37.156000,-6.247000) -- 
		(37.181000,-5.961000) -- 
		(37.357000,-5.686000) -- 
		(37.404000,-5.660000) -- 
		(37.404000,-5.632000) -- 
		(37.394000,-5.597000) -- 
		(37.393000,-5.562000) -- 
		(37.436000,-5.502000) -- 
		(37.462000,-5.473000) -- 
		(37.583000,-5.285000) -- 
		(37.637000,-5.230000) -- 
		(37.647000,-5.213000) -- 
		(37.671000,-5.165000) -- 
		(37.693000,-5.059000) -- 
		(37.711000,-4.947000) -- 
		(37.745000,-4.743000) -- 
		(37.695000,-4.685000) -- 
		(37.630000,-4.521000) -- 
		(37.581000,-4.465000) -- 
		(37.532000,-4.410000) -- 
		(37.493000,-4.273000) -- 
		(37.468000,-4.207000) -- 
		(37.447000,-4.180000) -- 
		(37.415000,-4.157000) -- 
		(37.205000,-3.979000) -- 
		(37.082000,-3.880000) -- 
		(36.795000,-3.665000) -- 
		(36.706000,-3.589000) -- 
		(36.554000,-3.421000) -- 
		(36.507000,-3.362000) -- 
		(36.471000,-3.297000) -- 
		(36.419000,-3.190000) -- 
		(36.391000,-3.132000) -- 
		(36.318000,-3.003000) -- 
		(36.268000,-2.905000) -- 
		(36.142000,-2.764000) -- 
		(36.120000,-2.734000) -- 
		(35.987000,-2.599000) -- 
		(35.929000,-2.550000) -- 
		(35.859000,-2.509000) -- 
		(35.746000,-2.459000) -- 
		(35.739000,-2.458000) -- 
		(35.669000,-2.430000) -- 
		(35.544000,-2.409000) -- 
		(35.460000,-2.400000) -- 
		(35.289000,-2.399000) -- 
		(35.119000,-2.405000) -- 
		(34.996000,-2.429000) -- 
		(34.789000,-2.460000) -- 
		(34.747000,-2.462000) -- 
		(34.708000,-2.478000) -- 
		(34.674000,-2.500000) -- 
		(34.637000,-2.491000) -- 
		(34.561000,-2.488000) -- 
		(34.313000,-2.480000) -- 
		(34.269000,-2.471000) -- 
		(34.198000,-2.500000) -- 
		(34.070000,-2.493000) -- 
		(33.987000,-2.483000) -- 
		(33.774000,-2.482000) -- 
		(33.603000,-2.474000) -- 
		(33.304000,-2.470000) -- 
		(33.231000,-2.467000) -- 
		(33.007000,-2.456000) -- 
		(32.153000,-2.465000) -- 
		(31.619000,-2.486000) -- 
		(30.964000,-2.511000) -- 
		(30.669000,-2.531000) -- 
		(30.584000,-2.532000) -- 
		(29.869000,-2.584000) -- 
		(29.785000,-2.593000) -- 
		(29.574000,-2.606000) -- 
		(29.235000,-2.618000) -- 
		(29.222000,-2.610000) -- 
		(29.201000,-2.597000) -- 
		(29.125000,-2.606000) -- 
		(28.937000,-2.617000) -- 
		(28.511000,-2.633000) -- 
		(28.086000,-2.649000) -- 
		(27.893000,-2.606000) -- 
		(27.492000,-2.517000) -- 
		(27.255000,-2.462000) -- 
		(27.028000,-2.396000) -- 
		(26.522000,-2.154000) -- 
		(26.385000,-2.039000) -- 
		(26.376000,-2.027000) -- 
		(26.282000,-1.896000) -- 
		(26.249000,-1.731000) -- 
		(26.246000,-1.648000) -- 
		(26.243000,-1.533000) -- 
		(26.327000,-1.236000) -- 
		(26.433000,-1.012000) -- 
		(26.489000,-0.890000) -- 
		(26.626000,-0.736000) -- 
		(26.976000,-0.411000) -- 
		(27.353000,-0.114000) -- 
		(27.548000,0.029000) -- 
		(27.587000,0.034000) -- 
		(27.970000,0.243000) -- 
		(27.989000,0.254000) -- 
		(28.379000,0.369000) -- 
		(28.574000,0.375000) -- 
		(28.620000,0.363000) -- 
		(28.768000,0.325000) -- 
		(29.035000,0.281000) -- 
		(29.275000,0.303000) -- 
		(29.538000,0.320000) -- 
		(29.613000,0.325000) -- 
		(29.851000,0.396000) -- 
		(30.119000,0.479000) -- 
		(30.431000,0.628000) -- 
		(30.742000,0.820000) -- 
		(31.190000,1.046000) -- 
		(31.457000,1.161000) -- 
		(31.944000,1.293000) -- 
		(32.392000,1.474000) -- 
		(33.125000,1.810000) -- 
		(33.281000,1.865000) -- 
		(33.290000,1.870000) -- 
		(33.723000,2.139000) -- 
		(33.879000,2.277000) -- 
		(34.099000,2.480000) -- 
		(34.132000,2.535000) -- 
		(34.229000,2.674000) -- 
		(34.359000,2.860000) -- 
		(34.431000,2.953000) -- 
		(34.788000,3.223000) -- 
		(34.807000,3.244000) -- 
		(34.846000,3.250000) -- 
		(35.022000,3.365000) -- 
		(35.180000,3.449000) -- 
		(35.548000,3.651000) -- 
		(35.775000,3.805000) -- 
		(35.872000,3.921000) -- 
		(35.989000,4.157000) -- 
		(36.054000,4.432000) -- 
		(36.087000,4.619000) -- 
		(36.054000,4.927000) -- 
		(35.983000,6.175000) -- 
		(35.931000,6.373000) -- 
		(35.717000,6.538000) -- 
		(35.602000,6.603000) -- 
		(35.581000,6.615000) -- 
		(35.289000,6.692000) -- 
		(34.834000,6.747000) -- 
		(34.522000,6.868000) -- 
		(34.399000,6.912000) -- 
		(34.349000,6.935000) -- 
		(34.328000,6.945000) -- 
		(34.289000,7.044000) -- 
		(34.290000,7.083000) -- 
		(34.291000,7.140000) -- 
		(34.238000,7.139000) -- 
		(33.982000,7.770000) -- 
		(33.738000,7.767000) -- 
		(32.388000,7.772000) -- 
		(30.810000,7.800000) -- 
		(29.874000,7.823000) -- 
		(25.651000,7.963000) -- 
		(21.384000,8.103000) -- 
		(21.201000,8.657000) -- 
		(20.881000,9.639000) -- 
		(20.561000,10.661000) -- 
		(19.829000,12.430000) -- 
		(19.828000,12.431000) -- 
		(19.533000,13.070000) -- 
		(18.963000,14.531000) -- 
		(18.859000,14.778000) -- 
		(18.845000,14.851000) -- 
		(18.819000,14.987000) -- 
		(18.786000,15.306000) -- 
		(18.959000,15.322000) -- 
		(20.631000,15.786000) -- 
		(21.768000,16.224000) -- 
		(22.483000,16.541000) -- 
		(22.932000,16.827000) -- 
		(23.096000,17.175000) -- 
		(23.152000,17.811000) -- 
		(23.143000,18.636000) -- 
		(23.135000,18.702000) -- 
		(23.142000,18.742000) -- 
		(23.129000,20.005000) -- 
		(23.189000,20.329000) -- 
		(23.179000,21.510000) -- 
		(23.133000,22.194000) -- 
		(23.067000,22.496000) -- 
		(23.036000,22.639000) -- 
		(22.946000,23.342000) -- 
		(23.038000,23.691000) -- 
		(23.187000,24.202000) -- 
		(23.223000,24.359000) -- 
		(23.240000,24.433000) -- 
		(23.292000,24.655000) -- 
		(23.308000,24.731000) -- 
		(23.311000,24.747000) -- 
		(23.376000,24.859000) -- 
		(23.420000,24.919000) -- 
		(23.522000,25.076000) -- 
		(23.546000,25.152000) -- 
		(23.560000,25.201000);
	\filldraw [draw=black, ultra thick, fill=gold]
		(-42.690000,52.356000) -- 
		(-43.384000,52.739000) -- 
		(-43.545000,52.844000) -- 
		(-43.712000,52.877000) -- 
		(-43.933000,52.935000) -- 
		(-44.267000,53.039000) -- 
		(-44.651000,53.150000) -- 
		(-44.927000,53.227000) -- 
		(-45.187000,53.304000) -- 
		(-45.450000,53.360000) -- 
		(-45.930000,53.357000) -- 
		(-45.937000,53.404000) -- 
		(-45.991000,53.952000) -- 
		(-45.993000,54.205000) -- 
		(-45.915000,54.361000) -- 
		(-45.558000,54.625000) -- 
		(-45.383000,54.761000) -- 
		(-45.227000,54.884000) -- 
		(-44.909000,54.991000) -- 
		(-44.620000,55.026000) -- 
		(-44.197000,55.050000) -- 
		(-43.829000,55.103000) -- 
		(-43.451000,55.185000) -- 
		(-42.783000,55.462000) -- 
		(-42.544000,55.523000) -- 
		(-42.014000,55.542000) -- 
		(-41.670000,55.489000) -- 
		(-41.139000,55.328000) -- 
		(-40.511000,55.110000) -- 
		(-40.165000,54.990000) -- 
		(-39.692000,54.816000) -- 
		(-39.076000,54.653000) -- 
		(-38.894000,54.626000) -- 
		(-38.500000,53.903000) -- 
		(-37.318000,51.734000) -- 
		(-36.925000,51.012000) -- 
		(-36.777000,51.024000) -- 
		(-36.599000,51.039000) -- 
		(-36.333000,51.028000) -- 
		(-36.186000,51.022000) -- 
		(-36.029000,51.014000) -- 
		(-35.994000,51.014000) -- 
		(-35.558000,51.041000) -- 
		(-35.402000,51.051000) -- 
		(-35.344000,51.053000) -- 
		(-35.170000,51.064000) -- 
		(-35.113000,51.068000) -- 
		(-35.035000,51.072000) -- 
		(-34.839000,51.084000) -- 
		(-34.802000,51.086000) -- 
		(-34.725000,51.091000) -- 
		(-34.546000,51.094000) -- 
		(-34.057000,51.104000) -- 
		(-34.057000,51.105000) -- 
		(-34.012000,51.104000) -- 
		(-33.835000,51.110000) -- 
		(-33.618000,51.141000) -- 
		(-33.211000,51.139000) -- 
		(-31.335000,51.132000) -- 
		(-30.710000,51.132000) -- 
		(-30.273000,51.129000) -- 
		(-28.962000,51.126000) -- 
		(-28.525000,51.125000) -- 
		(-28.047000,51.126000) -- 
		(-27.681000,51.127000) -- 
		(-26.612000,51.132000) -- 
		(-26.135000,51.134000) -- 
		(-25.472000,51.137000) -- 
		(-24.742000,51.141000) -- 
		(-23.610000,51.137000) -- 
		(-23.483000,51.135000) -- 
		(-23.305000,51.136000) -- 
		(-23.143000,51.144000) -- 
		(-23.062000,51.154000) -- 
		(-22.934000,51.179000) -- 
		(-22.902000,51.199000) -- 
		(-22.862000,51.260000) -- 
		(-22.827000,51.318000) -- 
		(-22.810000,51.387000) -- 
		(-22.798000,51.525000) -- 
		(-22.802000,51.852000) -- 
		(-22.815000,52.655000) -- 
		(-22.835000,53.665000) -- 
		(-22.847000,54.271000) -- 
		(-22.857000,54.912000) -- 
		(-22.887000,56.790000) -- 
		(-22.885000,56.837000) -- 
		(-22.876000,57.100000) -- 
		(-22.844000,57.387000) -- 
		(-22.832000,57.476000) -- 
		(-22.728000,57.449000) -- 
		(-22.643000,57.454000) -- 
		(-22.558000,57.456000) -- 
		(-22.473000,57.465000) -- 
		(-22.388000,57.464000) -- 
		(-21.970000,57.388000) -- 
		(-21.888000,57.369000) -- 
		(-21.687000,57.310000) -- 
		(-21.531000,57.253000) -- 
		(-21.424000,57.194000) -- 
		(-21.314000,57.141000) -- 
		(-21.237000,57.112000) -- 
		(-20.933000,57.011000) -- 
		(-20.880000,56.995000) -- 
		(-20.606000,56.897000) -- 
		(-20.046000,56.636000) -- 
		(-19.900000,56.561000) -- 
		(-19.764000,56.474000) -- 
		(-19.585000,56.377000) -- 
		(-19.554000,56.353000) -- 
		(-19.473000,56.270000) -- 
		(-19.413000,56.220000) -- 
		(-19.366000,56.160000) -- 
		(-19.094000,55.937000) -- 
		(-19.059000,55.916000) -- 
		(-18.995000,55.869000) -- 
		(-18.819000,55.713000) -- 
		(-18.571000,55.519000) -- 
		(-18.395000,55.362000) -- 
		(-18.125000,55.086000) -- 
		(-18.030000,54.967000) -- 
		(-17.780000,54.636000) -- 
		(-17.694000,54.473000) -- 
		(-17.635000,54.378000) -- 
		(-17.623000,54.344000) -- 
		(-17.589000,54.279000) -- 
		(-17.561000,54.066000) -- 
		(-17.551000,54.031000) -- 
		(-17.534000,53.999000) -- 
		(-17.267000,53.633000) -- 
		(-17.124000,53.337000) -- 
		(-17.043000,53.135000) -- 
		(-17.005000,53.049000) -- 
		(-16.970000,52.968000) -- 
		(-16.952000,52.922000) -- 
		(-16.878000,52.730000) -- 
		(-16.655000,52.230000) -- 
		(-16.492000,51.937000) -- 
		(-16.427000,51.767000) -- 
		(-16.408000,51.697000) -- 
		(-16.395000,51.627000) -- 
		(-16.383000,51.593000) -- 
		(-16.335000,51.493000) -- 
		(-16.276000,51.398000) -- 
		(-16.273000,51.365000) -- 
		(-16.289000,51.332000) -- 
		(-16.298000,51.297000) -- 
		(-16.300000,51.265000) -- 
		(-16.255000,51.204000) -- 
		(-16.230000,51.117000) -- 
		(-16.170000,50.913000) -- 
		(-16.178000,50.858000) -- 
		(-16.137000,50.842000) -- 
		(-16.103000,50.821000) -- 
		(-16.081000,50.791000) -- 
		(-16.065000,50.758000) -- 
		(-16.006000,50.663000) -- 
		(-15.998000,50.651000) -- 
		(-15.900000,50.509000) -- 
		(-15.799000,50.352000) -- 
		(-15.754000,50.291000) -- 
		(-15.607000,50.115000) -- 
		(-15.517000,50.039000) -- 
		(-15.484000,50.016000) -- 
		(-15.116000,49.833000) -- 
		(-15.008000,49.774000) -- 
		(-14.785000,49.669000) -- 
		(-14.741000,49.643000) -- 
		(-14.681000,49.608000) -- 
		(-14.553000,49.610000) -- 
		(-14.468000,49.603000) -- 
		(-14.426000,49.595000) -- 
		(-14.352000,49.560000) -- 
		(-14.218000,49.509000) -- 
		(-14.121000,49.471000) -- 
		(-14.082000,49.460000) -- 
		(-14.044000,49.443000) -- 
		(-14.014000,49.417000) -- 
		(-13.895000,49.269000) -- 
		(-13.857000,49.251000) -- 
		(-13.208000,49.067000) -- 
		(-13.119000,49.052000) -- 
		(-13.040000,49.041000) -- 
		(-12.999000,49.029000) -- 
		(-12.958000,49.022000) -- 
		(-11.977000,48.961000) -- 
		(-11.678000,48.980000) -- 
		(-11.636000,48.986000) -- 
		(-11.304000,49.057000) -- 
		(-11.144000,49.107000) -- 
		(-10.833000,49.227000) -- 
		(-10.606000,49.328000) -- 
		(-10.236000,49.506000) -- 
		(-9.960000,49.675000) -- 
		(-9.796000,49.839000) -- 
		(-9.721000,49.932000) -- 
		(-9.536000,50.164000) -- 
		(-9.439000,50.361000) -- 
		(-9.349000,50.709000) -- 
		(-9.289000,50.760000) -- 
		(-9.292000,50.786000) -- 
		(-9.323000,50.815000) -- 
		(-9.315000,51.014000) -- 
		(-9.310000,51.118000) -- 
		(-9.310000,51.129000) -- 
		(-9.337000,51.169000) -- 
		(-9.306000,51.198000) -- 
		(-9.313000,51.413000) -- 
		(-9.317000,51.519000) -- 
		(-9.334000,52.016000) -- 
		(-9.335000,52.159000) -- 
		(-9.296000,52.334000) -- 
		(-9.259000,52.543000) -- 
		(-9.247000,52.647000) -- 
		(-9.248000,52.683000) -- 
		(-9.293000,52.893000) -- 
		(-9.308000,53.210000) -- 
		(-9.302000,53.280000) -- 
		(-9.297000,53.489000) -- 
		(-9.258000,53.627000) -- 
		(-9.253000,53.662000) -- 
		(-9.233000,53.731000) -- 
		(-9.216000,53.764000) -- 
		(-9.134000,53.888000) -- 
		(-8.918000,54.132000) -- 
		(-8.804000,54.262000) -- 
		(-8.784000,54.294000) -- 
		(-8.668000,54.443000) -- 
		(-8.653000,54.476000) -- 
		(-8.620000,54.497000) -- 
		(-8.597000,54.527000) -- 
		(-8.525000,54.631000) -- 
		(-8.383000,54.956000) -- 
		(-8.324000,55.159000) -- 
		(-8.285000,55.412000) -- 
		(-8.194000,55.687000) -- 
		(-8.142000,56.358000) -- 
		(-8.078000,56.578000) -- 
		(-8.039000,56.683000) -- 
		(-7.850000,57.090000) -- 
		(-7.350000,57.420000) -- 
		(-6.869000,57.601000) -- 
		(-6.317000,57.783000) -- 
		(-5.694000,57.931000) -- 
		(-5.336000,57.975000) -- 
		(-5.076000,57.992000) -- 
		(-4.466000,57.992000) -- 
		(-4.095000,57.953000) -- 
		(-3.023000,57.794000) -- 
		(-3.021000,57.793000) -- 
		(-2.829000,57.739000) -- 
		(-2.517000,57.657000) -- 
		(-2.088000,57.552000) -- 
		(-1.412000,57.335000) -- 
		(-0.737000,57.118000) -- 
		(-0.399000,56.986000) -- 
		(-0.217000,56.953000) -- 
		(0.128000,56.871000) -- 
		(0.186000,56.821000) -- 
		(1.089000,56.629000) -- 
		(1.347000,56.568000) -- 
		(1.882000,56.442000) -- 
		(2.135000,56.415000) -- 
		(2.642000,56.338000) -- 
		(2.856000,56.327000) -- 
		(4.143000,56.239000) -- 
		(4.422000,56.250000) -- 
		(6.001000,56.222000) -- 
		(6.734000,56.175000) -- 
		(7.047000,56.157000) -- 
		(7.273000,56.110000) -- 
		(7.365000,56.091000) -- 
		(7.703000,55.959000) -- 
		(7.846000,55.876000) -- 
		(8.067000,55.684000) -- 
		(8.359000,55.502000) -- 
		(8.463000,55.480000) -- 
		(8.814000,55.436000) -- 
		(8.893000,55.408000) -- 
		(8.928000,55.389000) -- 
		(9.051000,55.362000) -- 
		(9.093000,55.356000) -- 
		(9.519000,55.328000) -- 
		(9.603000,55.332000) -- 
		(9.688000,55.330000) -- 
		(9.943000,55.309000) -- 
		(10.028000,55.298000) -- 
		(10.077000,55.286000) -- 
		(10.233000,55.254000) -- 
		(10.347000,55.206000) -- 
		(10.418000,55.165000) -- 
		(10.595000,55.014000) -- 
		(10.721000,54.871000) -- 
		(10.780000,54.737000) -- 
		(10.797000,54.667000) -- 
		(10.793000,54.490000) -- 
		(10.718000,54.102000) -- 
		(10.679000,53.923000) -- 
		(10.650000,53.787000) -- 
		(10.597000,53.578000) -- 
		(10.531000,53.408000) -- 
		(10.487000,53.307000) -- 
		(10.467000,53.275000) -- 
		(10.423000,53.174000) -- 
		(10.412000,53.140000) -- 
		(10.398000,53.069000) -- 
		(10.323000,52.865000) -- 
		(10.303000,52.795000) -- 
		(10.287000,52.762000) -- 
		(10.244000,52.701000) -- 
		(10.119000,52.603000) -- 
		(10.071000,52.545000) -- 
		(10.032000,52.482000) -- 
		(9.940000,52.361000) -- 
		(9.788000,52.189000) -- 
		(9.542000,51.943000) -- 
		(9.422000,51.841000) -- 
		(9.354000,51.798000) -- 
		(9.323000,51.773000) -- 
		(9.223000,51.706000) -- 
		(9.014000,51.583000) -- 
		(8.942000,51.545000) -- 
		(8.715000,51.445000) -- 
		(8.577000,51.408000) -- 
		(8.470000,51.382000) -- 
		(8.387000,51.366000) -- 
		(8.230000,51.308000) -- 
		(8.087000,51.231000) -- 
		(7.928000,51.111000) -- 
		(7.810000,51.010000) -- 
		(7.726000,50.931000) -- 
		(7.678000,50.874000) -- 
		(7.516000,50.712000) -- 
		(7.417000,50.652000) -- 
		(7.171000,50.438000) -- 
		(7.073000,50.509000) -- 
		(7.008000,50.476000) -- 
		(6.987000,50.454000) -- 
		(6.872000,50.339000) -- 
		(6.742000,50.245000) -- 
		(6.469000,50.102000) -- 
		(6.429000,50.088000) -- 
		(6.268000,50.036000) -- 
		(6.242000,49.987000) -- 
		(5.436000,49.745000) -- 
		(5.196000,49.613000) -- 
		(4.637000,49.354000) -- 
		(4.124000,49.019000) -- 
		(4.111000,48.991000) -- 
		(3.630000,48.705000) -- 
		(3.423000,48.591000) -- 
		(2.974000,48.348000) -- 
		(2.747000,48.166000) -- 
		(2.513000,47.820000) -- 
		(2.292000,47.660000) -- 
		(2.123000,47.506000) -- 
		(1.812000,47.303000) -- 
		(1.604000,47.149000) -- 
		(1.143000,46.824000) -- 
		(0.681000,46.511000) -- 
		(0.545000,46.379000) -- 
		(0.071000,45.967000) -- 
		(-0.092000,45.813000) -- 
		(-0.312000,45.620000) -- 
		(-0.540000,45.351000) -- 
		(-0.748000,45.142000) -- 
		(-0.890000,44.966000) -- 
		(-0.975000,44.806000) -- 
		(-1.079000,44.515000) -- 
		(-1.195000,44.306000) -- 
		(-1.338000,44.097000) -- 
		(-1.468000,43.998000) -- 
		(-1.780000,43.761000) -- 
		(-2.098000,43.541000) -- 
		(-2.358000,43.371000) -- 
		(-2.624000,43.140000) -- 
		(-2.923000,42.766000) -- 
		(-3.196000,42.271000) -- 
		(-3.267000,42.083000) -- 
		(-3.332000,41.913000) -- 
		(-3.598000,41.534000) -- 
		(-3.832000,41.308000) -- 
		(-4.729000,40.676000) -- 
		(-4.871000,40.588000) -- 
		(-5.222000,40.395000) -- 
		(-5.501000,40.170000) -- 
		(-5.591000,40.017000) -- 
		(-5.657000,39.906000) -- 
		(-5.742000,39.532000) -- 
		(-5.715000,39.197000) -- 
		(-5.702000,39.169000) -- 
		(-5.553000,38.806000) -- 
		(-5.423000,38.553000) -- 
		(-4.741000,37.850000) -- 
		(-4.137000,37.415000) -- 
		(-3.851000,37.234000) -- 
		(-3.598000,37.041000) -- 
		(-3.403000,36.904000) -- 
		(-3.366000,36.881000) -- 
		(-2.974000,36.646000) -- 
		(-2.539000,36.338000) -- 
		(-2.168000,36.145000) -- 
		(-1.922000,35.980000) -- 
		(-1.863000,35.953000) -- 
		(-1.662000,35.799000) -- 
		(-1.428000,35.639000) -- 
		(-1.006000,35.376000) -- 
		(-0.265000,35.002000) -- 
		(-0.122000,34.925000) -- 
		(0.820000,34.496000) -- 
		(1.027000,34.430000) -- 
		(2.197000,33.979000) -- 
		(3.067000,33.726000) -- 
		(3.892000,33.561000) -- 
		(4.243000,33.528000) -- 
		(5.555000,33.320000) -- 
		(6.165000,33.243000) -- 
		(6.568000,33.171000) -- 
		(7.295000,33.061000) -- 
		(7.529000,33.056000) -- 
		(8.107000,33.006000) -- 
		(8.458000,33.016000) -- 
		(8.854000,33.028000) -- 
		(10.134000,33.171000) -- 
		(11.238000,33.424000) -- 
		(11.549000,33.540000) -- 
		(11.686000,33.617000) -- 
		(11.920000,33.666000) -- 
		(12.166000,33.765000) -- 
		(12.550000,33.901000) -- 
		(12.900000,34.024000) -- 
		(13.331000,34.277000) -- 
		(13.368000,34.299000) -- 
		(13.751000,34.431000) -- 
		(14.544000,34.733000) -- 
		(14.797000,34.865000) -- 
		(15.180000,35.135000) -- 
		(15.570000,35.321000) -- 
		(15.696000,35.352000) -- 
		(16.154000,35.470000) -- 
		(16.551000,35.497000) -- 
		(16.960000,35.475000) -- 
		(17.291000,35.398000) -- 
		(17.596000,35.305000) -- 
		(17.720000,35.261000) -- 
		(18.006000,35.123000) -- 
		(18.070000,35.057000) -- 
		(18.135000,35.035000) -- 
		(18.441000,34.838000) -- 
		(18.584000,34.733000) -- 
		(18.752000,34.585000) -- 
		(19.006000,34.249000) -- 
		(19.136000,33.732000) -- 
		(19.168000,33.331000) -- 
		(19.149000,33.215000) -- 
		(18.993000,32.918000) -- 
		(18.830000,32.665000) -- 
		(18.681000,32.324000) -- 
		(18.590000,32.160000) -- 
		(18.434000,31.791000) -- 
		(18.148000,31.329000) -- 
		(17.869000,30.933000) -- 
		(17.680000,30.614000) -- 
		(17.453000,30.389000) -- 
		(17.239000,30.163000) -- 
		(17.012000,29.877000) -- 
		(16.706000,29.520000) -- 
		(16.641000,29.416000) -- 
		(16.258000,29.080000) -- 
		(15.862000,28.712000) -- 
		(15.693000,28.437000) -- 
		(15.635000,28.239000) -- 
		(15.602000,28.041000) -- 
		(15.589000,26.903000) -- 
		(15.596000,26.716000) -- 
		(15.579000,26.612000) -- 
		(15.565000,26.579000) -- 
		(15.584000,26.154000) -- 
		(15.628000,25.761000) -- 
		(15.652000,25.584000) -- 
		(15.676000,25.334000) -- 
		(15.696000,24.802000) -- 
		(15.707000,24.625000) -- 
		(15.715000,24.341000) -- 
		(15.724000,24.199000) -- 
		(15.720000,23.985000) -- 
		(15.686000,23.630000) -- 
		(15.682000,23.309000) -- 
		(15.699000,23.167000) -- 
		(15.714000,22.988000) -- 
		(15.745000,22.740000) -- 
		(15.757000,22.598000) -- 
		(15.783000,22.458000) -- 
		(15.863000,21.858000) -- 
		(15.885000,21.428000) -- 
		(15.937000,20.893000) -- 
		(15.951000,20.321000) -- 
		(15.998000,19.533000) -- 
		(16.052000,19.108000) -- 
		(16.135000,18.724000) -- 
		(16.139000,18.688000) -- 
		(16.164000,18.584000) -- 
		(16.211000,18.447000) -- 
		(16.230000,18.377000) -- 
		(16.391000,17.970000) -- 
		(16.499000,17.737000) -- 
		(16.599000,17.541000) -- 
		(16.653000,17.444000) -- 
		(16.694000,17.383000) -- 
		(16.759000,17.261000) -- 
		(16.850000,17.091000) -- 
		(16.950000,16.937000) -- 
		(16.985000,16.873000) -- 
		(17.131000,16.700000) -- 
		(17.306000,16.459000) -- 
		(17.354000,16.401000) -- 
		(17.453000,16.245000) -- 
		(17.642000,16.056000) -- 
		(17.790000,15.928000) -- 
		(17.846000,15.875000) -- 
		(18.252000,15.557000) -- 
		(18.485000,15.401000) -- 
		(18.515000,15.377000) -- 
		(18.536000,15.346000) -- 
		(18.594000,15.295000) -- 
		(18.647000,15.294000) -- 
		(18.786000,15.306000) -- 
		(18.819000,14.987000) -- 
		(18.845000,14.851000) -- 
		(18.859000,14.778000) -- 
		(18.963000,14.531000) -- 
		(19.533000,13.070000) -- 
		(19.828000,12.431000) -- 
		(19.829000,12.430000) -- 
		(20.561000,10.661000) -- 
		(20.881000,9.639000) -- 
		(21.201000,8.657000) -- 
		(21.384000,8.103000) -- 
		(20.421000,8.134000) -- 
		(19.378000,8.168000) -- 
		(18.702000,8.197000) -- 
		(18.030000,8.218000) -- 
		(17.532000,8.238000) -- 
		(16.568000,8.277000) -- 
		(15.971000,8.300000) -- 
		(15.715000,8.311000) -- 
		(14.694000,8.340000) -- 
		(14.177000,8.346000) -- 
		(14.054000,8.349000) -- 
		(13.579000,8.342000) -- 
		(13.373000,8.339000) -- 
		(13.048000,8.326000) -- 
		(12.831000,8.323000) -- 
		(12.243000,8.301000) -- 
		(11.640000,8.279000) -- 
		(10.972000,8.240000) -- 
		(8.238000,8.047000) -- 
		(6.903000,7.954000) -- 
		(5.755000,7.871000) -- 
		(2.308000,7.626000) -- 
		(1.158000,7.545000) -- 
		(-0.113000,7.454000) -- 
		(-2.113000,7.314000) -- 
		(-2.622000,7.281000) -- 
		(-3.562000,7.234000) -- 
		(-3.931000,7.220000) -- 
		(-4.034000,7.217000) -- 
		(-4.606000,7.209000) -- 
		(-5.049000,7.206000) -- 
		(-5.207000,7.205000) -- 
		(-5.375000,7.203000) -- 
		(-5.879000,7.201000) -- 
		(-6.048000,7.200000) -- 
		(-10.528000,7.183000) -- 
		(-11.773000,7.178000) -- 
		(-13.603000,7.179000) -- 
		(-14.371000,7.175000) -- 
		(-15.146000,7.179000) -- 
		(-16.270000,7.178000) -- 
		(-17.675000,7.163000) -- 
		(-19.603000,7.157000) -- 
		(-20.728000,7.149000) -- 
		(-20.757000,7.149000) -- 
		(-21.856000,7.154000) -- 
		(-22.272000,7.157000) -- 
		(-22.550000,7.170000) -- 
		(-22.829000,7.194000) -- 
		(-23.342000,7.249000) -- 
		(-23.705000,7.299000) -- 
		(-23.975000,7.344000) -- 
		(-24.273000,7.401000) -- 
		(-24.981000,7.558000) -- 
		(-26.525000,7.939000) -- 
		(-29.023000,8.562000) -- 
		(-29.578000,8.684000) -- 
		(-29.889000,8.740000) -- 
		(-30.283000,8.793000) -- 
		(-30.669000,8.832000) -- 
		(-30.875000,8.844000) -- 
		(-31.038000,8.900000) -- 
		(-31.033000,8.872000) -- 
		(-31.030000,8.854000) -- 
		(-31.029000,8.853000) -- 
		(-30.988000,8.600000) -- 
		(-30.984000,8.579000) -- 
		(-30.973000,8.511000) -- 
		(-30.674000,7.687000) -- 
		(-30.010000,6.382000) -- 
		(-29.661000,5.792000) -- 
		(-29.016000,4.663000) -- 
		(-27.895000,2.691000) -- 
		(-27.850000,2.600000) -- 
		(-27.622000,2.137000) -- 
		(-27.457000,1.708000) -- 
		(-27.300000,1.243000) -- 
		(-27.160000,0.757000) -- 
		(-27.143000,0.691000) -- 
		(-26.935000,-0.162000) -- 
		(-26.672000,-1.063000) -- 
		(-26.427000,-1.812000) -- 
		(-26.175000,-2.450000) -- 
		(-25.376000,-4.304000) -- 
		(-25.084000,-5.061000) -- 
		(-25.023000,-5.218000) -- 
		(-24.909000,-5.629000) -- 
		(-24.834000,-6.026000) -- 
		(-24.684000,-7.054000) -- 
		(-24.757000,-7.054000) -- 
		(-24.979000,-7.054000) -- 
		(-25.053000,-7.052000) -- 
		(-25.316000,-7.051000) -- 
		(-25.669000,-7.048000) -- 
		(-26.105000,-7.046000) -- 
		(-26.284000,-7.045000) -- 
		(-26.369000,-7.045000) -- 
		(-26.802000,-5.484000) -- 
		(-27.081000,-4.042000) -- 
		(-27.097000,-3.961000) -- 
		(-27.204000,-3.718000) -- 
		(-27.336000,-3.417000) -- 
		(-27.662000,-2.680000) -- 
		(-27.812000,-2.370000) -- 
		(-27.976000,-2.032000) -- 
		(-28.139000,-1.693000) -- 
		(-28.180000,-1.607000) -- 
		(-28.485000,-1.190000) -- 
		(-29.692000,0.075000) -- 
		(-30.496000,0.845000) -- 
		(-30.975000,1.350000) -- 
		(-31.043000,1.422000) -- 
		(-31.317000,1.726000) -- 
		(-32.553000,1.983000) -- 
		(-34.063000,2.144000) -- 
		(-35.459000,2.114000) -- 
		(-35.994000,2.037000) -- 
		(-36.358000,1.986000) -- 
		(-39.345000,1.398000) -- 
		(-39.695000,1.342000) -- 
		(-39.900000,1.952000) -- 
		(-39.955000,2.094000) -- 
		(-40.091000,2.449000) -- 
		(-40.282000,3.095000) -- 
		(-40.282000,4.167000) -- 
		(-40.693000,4.912000) -- 
		(-41.008000,5.486000) -- 
		(-41.554000,6.476000) -- 
		(-41.681000,6.635000) -- 
		(-41.869000,6.873000) -- 
		(-42.437000,7.591000) -- 
		(-43.211000,8.676000) -- 
		(-44.171000,9.488000) -- 
		(-44.141000,10.075000) -- 
		(-44.622000,11.620000) -- 
		(-44.960000,12.610000) -- 
		(-45.651000,8.942000) -- 
		(-46.881000,8.455000) -- 
		(-48.409000,7.705000) -- 
		(-49.347000,7.192000) -- 
		(-49.364000,7.183000) -- 
		(-50.568000,6.500000) -- 
		(-51.740000,6.190000) -- 
		(-51.930000,6.139000) -- 
		(-52.472000,5.995000) -- 
		(-53.263000,6.199000) -- 
		(-53.774000,6.331000) -- 
		(-53.896000,6.779000) -- 
		(-54.075000,7.429000) -- 
		(-54.276000,8.864000) -- 
		(-54.576000,12.073000) -- 
		(-54.452000,13.815000) -- 
		(-54.442000,13.963000) -- 
		(-54.400000,14.493000) -- 
		(-54.387000,14.653000) -- 
		(-54.370000,14.861000) -- 
		(-54.354000,15.070000) -- 
		(-52.466000,15.750000) -- 
		(-50.562000,16.342000) -- 
		(-48.946000,16.837000) -- 
		(-47.282000,17.686000) -- 
		(-45.074000,18.966000) -- 
		(-44.210000,19.589000) -- 
		(-43.122000,20.453000) -- 
		(-42.082000,21.573000) -- 
		(-41.090000,22.933000) -- 
		(-39.842000,24.837000) -- 
		(-38.868000,26.335000) -- 
		(-38.834000,26.389000) -- 
		(-37.778000,27.941000) -- 
		(-36.689000,29.717000) -- 
		(-36.660000,29.756000) -- 
		(-36.610000,29.845000) -- 
		(-36.592000,29.918000) -- 
		(-36.290000,31.189000) -- 
		(-36.018000,32.549000) -- 
		(-35.762000,33.700000) -- 
		(-36.050000,36.069000) -- 
		(-36.850000,38.181000) -- 
		(-37.771000,39.471000) -- 
		(-37.810000,39.525000) -- 
		(-37.992000,39.664000) -- 
		(-38.770000,40.261000) -- 
		(-38.972000,40.407000) -- 
		(-39.474000,40.773000) -- 
		(-41.074000,41.316000) -- 
		(-41.704000,41.484000) -- 
		(-42.642000,41.732000) -- 
		(-44.914000,41.988000) -- 
		(-45.282000,41.893000) -- 
		(-46.722000,41.765000) -- 
		(-49.106000,41.397000) -- 
		(-49.570000,41.205000) -- 
		(-49.973000,40.958000) -- 
		(-50.762000,40.479000) -- 
		(-51.550000,39.999000) -- 
		(-51.846000,39.822000) -- 
		(-51.846000,39.821000) -- 
		(-51.970000,39.748000) -- 
		(-53.330000,38.709000) -- 
		(-54.322000,38.085000) -- 
		(-55.634000,37.716000) -- 
		(-56.946000,37.525000) -- 
		(-57.794000,37.749000) -- 
		(-59.699000,38.116000) -- 
		(-61.571000,38.565000) -- 
		(-63.539000,39.333000) -- 
		(-64.883000,39.861000) -- 
		(-65.843000,40.181000) -- 
		(-67.347000,40.645000) -- 
		(-68.755000,41.141000) -- 
		(-69.489000,41.658000) -- 
		(-69.891000,41.941000) -- 
		(-70.371000,42.692000) -- 
		(-70.643000,43.397000) -- 
		(-70.467000,44.468000) -- 
		(-69.939000,45.125000) -- 
		(-69.107000,45.669000) -- 
		(-67.619000,46.037000) -- 
		(-67.027000,46.213000) -- 
		(-65.921000,46.410000) -- 
		(-64.056000,46.653000) -- 
		(-63.264000,46.759000) -- 
		(-62.077000,46.946000) -- 
		(-61.641000,47.015000) -- 
		(-61.140000,47.100000) -- 
		(-60.452000,47.216000) -- 
		(-60.082000,47.270000) -- 
		(-59.987000,47.284000) -- 
		(-58.824000,47.656000) -- 
		(-58.736000,47.684000) -- 
		(-58.138000,47.876000) -- 
		(-57.437000,48.103000) -- 
		(-56.946000,48.260000) -- 
		(-52.978000,49.843000) -- 
		(-48.786000,46.998000) -- 
		(-47.298000,45.989000) -- 
		(-46.594000,45.669000) -- 
		(-46.578000,45.653000) -- 
		(-45.170000,45.349000) -- 
		(-44.842000,45.362000) -- 
		(-44.402000,45.381000) -- 
		(-44.050000,45.444000) -- 
		(-42.194000,45.973000) -- 
		(-41.154000,46.453000) -- 
		(-40.354000,47.236000) -- 
		(-39.894000,48.523000) -- 
		(-39.925000,48.832000) -- 
		(-39.804000,49.723000) -- 
		(-40.114000,49.892000) -- 
		(-41.138000,50.932000) -- 
		(-41.938000,51.780000) -- 
		(-42.690000,52.356000);
	\filldraw [draw=black, ultra thick, fill=blue]
		(180.537000,411.553000) -- 
		(180.522000,411.382000) -- 
		(180.521000,411.216000) -- 
		(180.584000,410.608000) -- 
		(180.615000,410.110000) -- 
		(180.606000,410.089000) -- 
		(180.584000,409.155000) -- 
		(180.524000,408.779000) -- 
		(180.515000,408.756000) -- 
		(180.366000,408.501000) -- 
		(180.259000,408.198000) -- 
		(180.043000,407.818000) -- 
		(180.001000,407.766000) -- 
		(179.947000,407.686000) -- 
		(179.911000,407.591000) -- 
		(179.905000,407.528000) -- 
		(179.925000,407.518000) -- 
		(180.142000,407.523000) -- 
		(180.148000,407.462000) -- 
		(180.169000,407.404000) -- 
		(180.185000,407.217000) -- 
		(180.222000,407.013000) -- 
		(180.284000,406.838000) -- 
		(180.382000,406.611000) -- 
		(180.471000,406.375000) -- 
		(180.504000,406.327000) -- 
		(180.545000,406.217000) -- 
		(180.570000,406.162000) -- 
		(180.578000,406.121000) -- 
		(180.611000,406.073000) -- 
		(180.669000,405.948000) -- 
		(180.758000,405.789000) -- 
		(180.799000,405.747000) -- 
		(180.824000,405.700000) -- 
		(180.865000,405.652000) -- 
		(180.890000,405.604000) -- 
		(180.939000,405.541000) -- 
		(180.964000,405.473000) -- 
		(180.991000,405.323000) -- 
		(181.000000,405.097000) -- 
		(180.996000,405.073000) -- 
		(180.963000,404.976000) -- 
		(180.873000,404.879000) -- 
		(180.783000,404.813000) -- 
		(180.717000,404.777000) -- 
		(180.595000,404.742000) -- 
		(180.503000,404.726000) -- 
		(180.171000,404.707000) -- 
		(179.624000,404.665000) -- 
		(179.550000,404.651000) -- 
		(179.395000,404.561000) -- 
		(179.379000,404.541000) -- 
		(179.363000,404.479000) -- 
		(179.363000,404.417000) -- 
		(179.388000,404.363000) -- 
		(179.398000,404.323000) -- 
		(179.470000,404.217000) -- 
		(179.494000,404.163000) -- 
		(179.731000,404.040000) -- 
		(179.968000,403.902000) -- 
		(180.116000,403.761000) -- 
		(180.295000,403.576000) -- 
		(180.467000,403.449000) -- 
		(180.541000,403.409000) -- 
		(180.566000,403.361000) -- 
		(180.578000,403.302000) -- 
		(180.590000,403.272000) -- 
		(180.604000,403.212000) -- 
		(180.615000,403.134000) -- 
		(180.624000,403.065000) -- 
		(180.616000,402.921000) -- 
		(180.592000,402.852000) -- 
		(180.543000,402.790000) -- 
		(180.519000,402.728000) -- 
		(180.535000,402.686000) -- 
		(180.621000,402.604000) -- 
		(180.658000,402.578000) -- 
		(180.723000,402.544000) -- 
		(180.829000,402.522000) -- 
		(180.870000,402.530000) -- 
		(180.935000,402.551000) -- 
		(181.139000,402.681000) -- 
		(181.196000,402.730000) -- 
		(181.220000,402.832000) -- 
		(181.224000,402.883000) -- 
		(181.207000,403.138000) -- 
		(181.121000,403.333000) -- 
		(181.101000,403.426000) -- 
		(181.088000,403.457000) -- 
		(181.113000,403.533000) -- 
		(181.120000,403.574000) -- 
		(181.169000,403.623000) -- 
		(181.243000,403.643000) -- 
		(181.345000,403.626000) -- 
		(181.790000,403.671000) -- 
		(181.879000,403.685000) -- 
		(181.929000,403.685000) -- 
		(182.279000,403.707000) -- 
		(182.366000,403.704000) -- 
		(182.542000,403.667000) -- 
		(182.786000,403.581000) -- 
		(182.827000,403.570000) -- 
		(182.933000,403.522000) -- 
		(183.179000,403.393000) -- 
		(183.260000,403.338000) -- 
		(183.367000,403.238000) -- 
		(183.416000,403.180000) -- 
		(183.490000,403.049000) -- 
		(183.507000,402.947000) -- 
		(183.503000,402.827000) -- 
		(183.490000,402.802000) -- 
		(183.475000,402.740000) -- 
		(183.417000,402.671000) -- 
		(183.344000,402.616000) -- 
		(183.205000,402.561000) -- 
		(183.035000,402.451000) -- 
		(183.018000,402.416000) -- 
		(183.027000,402.272000) -- 
		(183.084000,402.213000) -- 
		(183.304000,402.073000) -- 
		(183.476000,401.991000) -- 
		(183.590000,401.971000) -- 
		(183.615000,401.970000) -- 
		(183.656000,401.936000) -- 
		(183.722000,401.870000) -- 
		(183.746000,401.839000) -- 
		(183.853000,401.737000) -- 
		(183.951000,401.566000) -- 
		(183.967000,401.531000) -- 
		(183.980000,401.459000) -- 
		(184.025000,401.332000) -- 
		(184.033000,401.290000) -- 
		(184.058000,401.262000) -- 
		(184.108000,401.168000) -- 
		(184.181000,401.088000) -- 
		(184.303000,401.002000) -- 
		(184.409000,400.955000) -- 
		(184.502000,400.939000) -- 
		(184.627000,400.924000) -- 
		(184.753000,400.927000) -- 
		(184.818000,400.921000) -- 
		(185.320000,400.967000) -- 
		(185.679000,400.974000) -- 
		(185.958000,400.947000) -- 
		(186.109000,400.910000) -- 
		(186.182000,400.875000) -- 
		(186.329000,400.738000) -- 
		(186.395000,400.635000) -- 
		(186.485000,400.387000) -- 
		(186.501000,400.325000) -- 
		(186.526000,400.291000) -- 
		(186.564000,400.094000) -- 
		(186.540000,399.772000) -- 
		(186.532000,399.491000) -- 
		(186.541000,399.346000) -- 
		(186.537000,399.157000) -- 
		(186.542000,398.850000) -- 
		(186.531000,398.400000) -- 
		(186.547000,398.324000) -- 
		(186.572000,398.278000) -- 
		(186.663000,398.182000) -- 
		(186.783000,398.089000) -- 
		(186.826000,398.065000) -- 
		(186.867000,398.051000) -- 
		(186.900000,398.031000) -- 
		(187.087000,397.955000) -- 
		(187.218000,397.913000) -- 
		(187.316000,397.846000) -- 
		(187.374000,397.798000) -- 
		(187.398000,397.763000) -- 
		(187.373000,397.715000) -- 
		(187.308000,397.674000) -- 
		(187.203000,397.653000) -- 
		(187.162000,397.640000) -- 
		(187.121000,397.640000) -- 
		(187.055000,397.598000) -- 
		(187.044000,397.546000) -- 
		(187.047000,397.508000) -- 
		(187.107000,397.452000) -- 
		(187.391000,397.468000) -- 
		(187.432000,397.448000) -- 
		(187.489000,397.398000) -- 
		(187.480000,397.350000) -- 
		(187.481000,397.288000) -- 
		(187.506000,397.200000) -- 
		(187.531000,397.179000) -- 
		(187.547000,397.159000) -- 
		(187.587000,397.145000) -- 
		(187.691000,397.150000) -- 
		(187.742000,397.166000) -- 
		(187.767000,397.200000) -- 
		(187.812000,397.362000) -- 
		(187.824000,397.393000) -- 
		(187.832000,397.462000) -- 
		(187.791000,397.502000) -- 
		(187.799000,397.530000) -- 
		(187.824000,397.544000) -- 
		(187.847000,397.522000) -- 
		(187.906000,397.434000) -- 
		(187.922000,397.263000) -- 
		(187.866000,397.153000) -- 
		(187.768000,397.042000) -- 
		(187.751000,396.959000) -- 
		(187.751000,396.925000) -- 
		(187.768000,396.883000) -- 
		(187.866000,396.761000) -- 
		(187.948000,396.671000) -- 
		(187.997000,396.596000) -- 
		(188.005000,396.568000) -- 
		(188.087000,396.438000) -- 
		(188.121000,396.315000) -- 
		(188.154000,396.005000) -- 
		(188.147000,395.681000) -- 
		(188.115000,395.304000) -- 
		(188.144000,395.129000) -- 
		(188.201000,395.002000) -- 
		(188.213000,394.954000) -- 
		(188.238000,394.919000) -- 
		(188.247000,394.892000) -- 
		(188.279000,394.843000) -- 
		(188.345000,394.679000) -- 
		(188.330000,394.232000) -- 
		(188.305000,393.970000) -- 
		(188.315000,393.827000) -- 
		(188.322000,393.781000) -- 
		(188.344000,393.631000) -- 
		(188.388000,393.497000) -- 
		(188.684000,392.898000) -- 
		(188.734000,392.776000) -- 
		(188.758000,392.693000) -- 
		(188.758000,392.638000) -- 
		(188.750000,392.590000) -- 
		(188.718000,392.549000) -- 
		(188.404000,392.321000) -- 
		(188.245000,392.228000) -- 
		(187.919000,391.997000) -- 
		(187.731000,391.874000) -- 
		(187.609000,391.767000) -- 
		(187.430000,391.550000) -- 
		(187.373000,391.460000) -- 
		(187.340000,391.384000) -- 
		(187.316000,391.302000) -- 
		(187.292000,391.276000) -- 
		(187.276000,391.240000) -- 
		(187.243000,391.200000) -- 
		(187.235000,391.172000) -- 
		(187.161000,391.096000) -- 
		(187.072000,390.931000) -- 
		(186.983000,390.848000) -- 
		(186.958000,390.890000) -- 
		(186.644000,390.627000) -- 
		(186.592000,390.604000) -- 
		(186.534000,390.565000) -- 
		(186.494000,390.565000) -- 
		(186.388000,390.510000) -- 
		(186.312000,390.440000) -- 
		(186.304000,390.423000) -- 
		(186.339000,390.297000) -- 
		(186.789000,389.906000) -- 
		(186.806000,389.873000) -- 
		(186.822000,389.852000) -- 
		(187.003000,389.680000) -- 
		(187.019000,389.646000) -- 
		(187.101000,389.543000) -- 
		(187.133000,389.487000) -- 
		(187.149000,389.439000) -- 
		(187.174000,389.411000) -- 
		(187.191000,389.247000) -- 
		(187.215000,389.131000) -- 
		(187.248000,389.048000) -- 
		(187.273000,389.000000) -- 
		(187.330000,388.930000) -- 
		(187.556000,388.746000) -- 
		(187.715000,388.740000) -- 
		(187.772000,388.746000) -- 
		(187.821000,388.775000) -- 
		(187.870000,388.815000) -- 
		(188.008000,388.952000) -- 
		(188.126000,389.050000) -- 
		(188.211000,389.098000) -- 
		(188.350000,389.112000) -- 
		(188.436000,389.107000) -- 
		(188.478000,389.102000) -- 
		(188.514000,389.090000) -- 
		(188.539000,389.078000) -- 
		(188.653000,389.071000) -- 
		(188.718000,389.037000) -- 
		(188.866000,389.023000) -- 
		(188.938000,389.024000) -- 
		(189.012000,389.037000) -- 
		(189.086000,389.086000) -- 
		(189.192000,389.147000) -- 
		(189.294000,389.165000) -- 
		(189.449000,389.151000) -- 
		(189.559000,389.134000) -- 
		(189.746000,389.134000) -- 
		(189.964000,389.118000) -- 
		(189.991000,389.101000) -- 
		(190.074000,389.087000) -- 
		(190.094000,389.073000) -- 
		(190.139000,389.053000) -- 
		(190.287000,388.947000) -- 
		(190.393000,388.836000) -- 
		(190.401000,388.813000) -- 
		(190.442000,388.772000) -- 
		(190.458000,388.730000) -- 
		(190.479000,388.617000) -- 
		(190.509000,388.222000) -- 
		(190.509000,388.160000) -- 
		(190.420000,387.933000) -- 
		(190.329000,387.815000) -- 
		(190.200000,387.688000) -- 
		(190.151000,387.616000) -- 
		(190.126000,387.589000) -- 
		(190.111000,387.541000) -- 
		(190.086000,387.520000) -- 
		(190.069000,387.479000) -- 
		(190.046000,387.320000) -- 
		(190.054000,387.176000) -- 
		(190.070000,387.134000) -- 
		(190.209000,386.963000) -- 
		(190.414000,386.799000) -- 
		(190.451000,386.794000) -- 
		(190.512000,386.779000) -- 
		(190.651000,386.785000) -- 
		(190.733000,386.799000) -- 
		(190.790000,386.830000) -- 
		(191.042000,387.013000) -- 
		(191.100000,387.047000) -- 
		(191.221000,387.075000) -- 
		(191.254000,387.103000) -- 
		(191.311000,387.116000) -- 
		(191.410000,387.117000) -- 
		(191.459000,387.089000) -- 
		(191.507000,387.044000) -- 
		(191.548000,386.993000) -- 
		(191.623000,386.642000) -- 
		(191.623000,386.622000) -- 
		(191.599000,386.602000) -- 
		(191.590000,386.566000) -- 
		(191.509000,386.504000) -- 
		(191.321000,386.450000) -- 
		(191.288000,386.428000) -- 
		(191.183000,386.388000) -- 
		(191.102000,386.340000) -- 
		(190.955000,386.216000) -- 
		(190.955000,386.194000) -- 
		(190.930000,386.168000) -- 
		(190.889000,386.071000) -- 
		(190.906000,385.933000) -- 
		(191.078000,385.789000) -- 
		(191.433000,385.814000) -- 
		(191.511000,385.831000) -- 
		(191.560000,385.859000) -- 
		(191.837000,385.941000) -- 
		(191.991000,385.983000) -- 
		(192.017000,386.003000) -- 
		(192.098000,386.025000) -- 
		(192.155000,386.059000) -- 
		(192.196000,386.065000) -- 
		(192.277000,386.121000) -- 
		(192.661000,386.245000) -- 
		(192.767000,386.301000) -- 
		(192.865000,386.328000) -- 
		(192.889000,386.349000) -- 
		(192.947000,386.369000) -- 
		(193.110000,386.383000) -- 
		(193.196000,386.380000) -- 
		(193.224000,386.363000) -- 
		(193.298000,386.343000) -- 
		(193.395000,386.282000) -- 
		(193.445000,386.239000) -- 
		(193.461000,386.220000) -- 
		(193.478000,386.110000) -- 
		(193.511000,386.075000) -- 
		(193.593000,386.006000) -- 
		(193.649000,386.000000) -- 
		(193.723000,386.006000) -- 
		(193.788000,386.020000) -- 
		(193.837000,386.048000) -- 
		(193.952000,386.069000) -- 
		(194.115000,386.069000) -- 
		(194.197000,386.014000) -- 
		(194.229000,385.986000) -- 
		(194.245000,385.952000) -- 
		(194.235000,385.858000) -- 
		(194.182000,385.719000) -- 
		(194.091000,385.622000) -- 
		(194.091000,385.601000) -- 
		(194.051000,385.554000) -- 
		(194.034000,385.505000) -- 
		(194.010000,385.485000) -- 
		(193.994000,385.423000) -- 
		(193.969000,385.403000) -- 
		(193.961000,385.361000) -- 
		(193.888000,385.271000) -- 
		(193.881000,385.237000) -- 
		(193.815000,385.167000) -- 
		(193.536000,384.925000) -- 
		(193.503000,384.843000) -- 
		(193.514000,384.694000) -- 
		(193.604000,384.604000) -- 
		(193.703000,384.522000) -- 
		(193.744000,384.500000) -- 
		(194.226000,384.358000) -- 
		(194.266000,384.330000) -- 
		(194.445000,384.275000) -- 
		(194.846000,384.249000) -- 
		(195.119000,384.286000) -- 
		(195.319000,384.345000) -- 
		(195.588000,384.359000) -- 
		(195.932000,384.361000) -- 
		(196.111000,384.388000) -- 
		(196.189000,384.444000) -- 
		(196.355000,384.540000) -- 
		(196.413000,384.554000) -- 
		(196.519000,384.615000) -- 
		(196.576000,384.630000) -- 
		(196.625000,384.657000) -- 
		(196.691000,384.677000) -- 
		(196.715000,384.692000) -- 
		(196.992000,384.706000) -- 
		(197.050000,384.685000) -- 
		(197.140000,384.631000) -- 
		(197.222000,384.559000) -- 
		(197.263000,384.507000) -- 
		(197.320000,384.342000) -- 
		(197.356000,384.188000) -- 
		(197.393000,384.081000) -- 
		(197.427000,383.888000) -- 
		(197.452000,383.813000) -- 
		(197.477000,383.765000) -- 
		(197.550000,383.669000) -- 
		(197.656000,383.569000) -- 
		(197.665000,383.546000) -- 
		(197.681000,383.525000) -- 
		(197.763000,383.442000) -- 
		(197.960000,383.278000) -- 
		(198.405000,382.934000) -- 
		(198.483000,382.893000) -- 
		(198.581000,382.831000) -- 
		(198.769000,382.671000) -- 
		(198.851000,382.550000) -- 
		(198.876000,382.502000) -- 
		(198.913000,382.307000) -- 
		(198.950000,382.192000) -- 
		(198.966000,382.173000) -- 
		(198.991000,382.125000) -- 
		(199.048000,382.056000) -- 
		(199.285000,381.863000) -- 
		(199.466000,381.775000) -- 
		(199.580000,381.689000) -- 
		(199.638000,381.624000) -- 
		(199.827000,381.462000) -- 
		(200.097000,381.286000) -- 
		(200.227000,381.012000) -- 
		(200.416000,380.660000) -- 
		(200.652000,380.160000) -- 
		(200.952000,379.758000) -- 
		(201.108000,379.627000) -- 
		(201.206000,379.495000) -- 
		(201.441000,379.285000) -- 
		(201.617000,379.094000) -- 
		(201.741000,378.984000) -- 
		(201.794000,378.841000) -- 
		(201.852000,378.466000) -- 
		(201.898000,378.379000) -- 
		(201.963000,378.291000) -- 
		(202.055000,378.269000) -- 
		(202.185000,378.275000) -- 
		(202.361000,378.330000) -- 
		(202.668000,378.396000) -- 
		(202.804000,378.440000) -- 
		(202.963000,378.429000) -- 
		(203.163000,378.385000) -- 
		(203.307000,378.319000) -- 
		(203.365000,378.281000) -- 
		(203.528000,378.116000) -- 
		(203.541000,378.066000) -- 
		(203.535000,377.906000) -- 
		(203.516000,377.791000) -- 
		(203.437000,377.615000) -- 
		(203.307000,377.390000) -- 
		(203.320000,377.115000) -- 
		(203.412000,376.978000) -- 
		(203.562000,376.824000) -- 
		(203.660000,376.758000) -- 
		(203.705000,376.685000) -- 
		(203.712000,376.609000) -- 
		(203.425000,376.164000) -- 
		(203.308000,375.916000) -- 
		(203.250000,375.823000) -- 
		(203.191000,375.641000) -- 
		(203.145000,375.542000) -- 
		(203.061000,375.486000) -- 
		(202.859000,375.432000) -- 
		(202.761000,375.382000) -- 
		(202.637000,375.272000) -- 
		(202.565000,375.157000) -- 
		(202.461000,375.020000) -- 
		(202.298000,374.838000) -- 
		(202.090000,374.563000) -- 
		(201.901000,374.326000) -- 
		(201.595000,374.084000) -- 
		(201.477000,373.969000) -- 
		(201.425000,373.892000) -- 
		(201.393000,373.754000) -- 
		(201.400000,373.412000) -- 
		(202.013000,372.941000) -- 
		(202.143000,372.820000) -- 
		(202.235000,372.666000) -- 
		(202.391000,372.479000) -- 
		(202.619000,372.347000) -- 
		(202.776000,372.303000) -- 
		(203.076000,372.243000) -- 
		(203.428000,372.205000) -- 
		(203.676000,372.133000) -- 
		(203.813000,372.055000) -- 
		(203.885000,371.934000) -- 
		(203.917000,371.781000) -- 
		(204.015000,371.666000) -- 
		(204.165000,371.577000) -- 
		(204.276000,371.556000) -- 
		(204.406000,371.577000) -- 
		(204.569000,371.639000) -- 
		(204.824000,371.771000) -- 
		(205.097000,371.847000) -- 
		(205.280000,371.875000) -- 
		(205.475000,371.870000) -- 
		(205.645000,371.843000) -- 
		(205.978000,371.739000) -- 
		(206.147000,371.633000) -- 
		(206.441000,371.512000) -- 
		(206.597000,371.501000) -- 
		(206.740000,371.529000) -- 
		(207.281000,371.739000) -- 
		(207.836000,371.976000) -- 
		(208.311000,372.229000) -- 
		(208.566000,372.383000) -- 
		(208.618000,372.444000) -- 
		(208.624000,372.779000) -- 
		(208.617000,372.911000) -- 
		(208.819000,373.290000) -- 
		(208.995000,373.471000) -- 
		(209.178000,373.626000) -- 
		(209.633000,373.918000) -- 
		(209.640000,373.923000) -- 
		(210.031000,374.033000) -- 
		(210.357000,374.050000) -- 
		(210.488000,374.104000) -- 
		(210.618000,374.204000) -- 
		(210.853000,374.523000) -- 
		(210.937000,374.588000) -- 
		(211.042000,374.628000) -- 
		(211.374000,374.628000) -- 
		(211.485000,374.672000) -- 
		(211.550000,374.737000) -- 
		(211.596000,374.842000) -- 
		(211.602000,375.189000) -- 
		(211.687000,375.260000) -- 
		(211.785000,375.299000) -- 
		(212.058000,375.331000) -- 
		(212.287000,375.331000) -- 
		(212.515000,375.354000) -- 
		(212.736000,375.398000) -- 
		(212.899000,375.475000) -- 
		(213.049000,375.596000) -- 
		(213.258000,375.849000) -- 
		(213.427000,376.074000) -- 
		(213.590000,376.383000) -- 
		(213.662000,376.471000) -- 
		(213.922000,376.614000) -- 
		(213.991000,376.637000) -- 
		(214.079000,376.668000) -- 
		(214.216000,376.685000) -- 
		(214.614000,376.620000) -- 
		(215.285000,376.465000) -- 
		(215.879000,376.301000) -- 
		(216.068000,376.240000) -- 
		(216.224000,376.223000) -- 
		(216.342000,376.240000) -- 
		(216.485000,376.296000) -- 
		(216.674000,376.340000) -- 
		(217.150000,376.384000) -- 
		(217.280000,376.445000) -- 
		(217.548000,376.659000) -- 
		(217.821000,376.819000) -- 
		(218.310000,377.006000) -- 
		(218.799000,377.225000) -- 
		(218.995000,377.248000) -- 
		(219.190000,377.231000) -- 
		(219.360000,377.160000) -- 
		(219.530000,377.034000) -- 
		(219.673000,376.879000) -- 
		(219.856000,376.616000) -- 
		(220.006000,376.462000) -- 
		(220.162000,376.347000) -- 
		(220.606000,376.132000) -- 
		(220.919000,375.950000) -- 
		(220.991000,375.967000) -- 
		(221.069000,375.939000) -- 
		(221.128000,375.869000) -- 
		(221.206000,375.731000) -- 
		(221.284000,375.561000) -- 
		(221.389000,375.280000) -- 
		(221.376000,375.213000) -- 
		(221.395000,374.895000) -- 
		(221.317000,374.549000) -- 
		(221.376000,374.335000) -- 
		(221.591000,374.142000) -- 
		(221.761000,373.944000) -- 
		(221.846000,373.823000) -- 
		(221.904000,373.647000) -- 
		(221.892000,373.158000) -- 
		(221.859000,373.015000) -- 
		(221.755000,372.905000) -- 
		(221.559000,372.832000) -- 
		(221.286000,372.773000) -- 
		(221.025000,372.728000) -- 
		(220.842000,372.679000) -- 
		(220.764000,372.624000) -- 
		(220.692000,372.542000) -- 
		(220.608000,372.338000) -- 
		(220.588000,371.996000) -- 
		(220.556000,371.805000) -- 
		(220.588000,371.579000) -- 
		(220.628000,370.831000) -- 
		(220.674000,370.265000) -- 
		(220.739000,369.946000) -- 
		(220.811000,369.748000) -- 
		(220.915000,369.539000) -- 
		(221.215000,369.254000) -- 
		(221.450000,369.116000) -- 
		(221.731000,369.027000) -- 
		(221.867000,368.951000) -- 
		(222.070000,368.809000) -- 
		(222.311000,368.571000) -- 
		(222.722000,368.314000) -- 
		(223.080000,368.067000) -- 
		(223.335000,367.957000) -- 
		(223.570000,367.907000) -- 
		(224.293000,367.853000) -- 
		(224.606000,367.875000) -- 
		(224.854000,367.907000) -- 
		(225.408000,367.869000) -- 
		(226.151000,367.859000) -- 
		(226.634000,367.914000) -- 
		(227.625000,367.975000) -- 
		(227.859000,367.986000) -- 
		(228.114000,367.969000) -- 
		(228.479000,367.920000) -- 
		(228.916000,367.892000) -- 
		(229.248000,367.827000) -- 
		(229.685000,367.821000) -- 
		(230.428000,367.717000) -- 
		(230.519000,367.673000) -- 
		(230.630000,367.546000) -- 
		(230.696000,367.398000) -- 
		(230.715000,367.305000) -- 
		(230.722000,367.107000) -- 
		(230.657000,366.826000) -- 
		(230.481000,366.656000) -- 
		(229.888000,366.210000) -- 
		(229.496000,365.891000) -- 
		(228.910000,365.363000) -- 
		(228.578000,365.022000) -- 
		(228.350000,364.730000) -- 
		(228.226000,364.433000) -- 
		(228.148000,364.307000) -- 
		(228.063000,364.113000) -- 
		(227.998000,363.928000) -- 
		(228.011000,363.763000) -- 
		(228.044000,363.646000) -- 
		(228.141000,363.536000) -- 
		(228.298000,363.432000) -- 
		(228.467000,363.378000) -- 
		(228.728000,363.361000) -- 
		(228.996000,363.411000) -- 
		(229.224000,363.427000) -- 
		(229.380000,363.406000) -- 
		(229.491000,363.351000) -- 
		(229.538000,363.286000) -- 
		(229.556000,363.263000) -- 
		(229.621000,363.131000) -- 
		(229.621000,363.081000) -- 
		(229.550000,362.861000) -- 
		(229.198000,362.168000) -- 
		(228.957000,361.811000) -- 
		(228.631000,361.112000) -- 
		(228.481000,360.865000) -- 
		(228.227000,360.535000) -- 
		(227.856000,359.908000) -- 
		(227.687000,359.599000) -- 
		(227.635000,359.346000) -- 
		(227.641000,359.121000) -- 
		(227.700000,358.940000) -- 
		(227.791000,358.737000) -- 
		(227.922000,358.499000) -- 
		(228.059000,358.285000) -- 
		(228.750000,357.274000) -- 
		(228.809000,357.175000) -- 
		(229.024000,356.944000) -- 
		(229.298000,356.703000) -- 
		(229.565000,356.373000) -- 
		(229.670000,356.219000) -- 
		(229.737000,356.075000) -- 
		(229.800000,355.938000) -- 
		(230.133000,355.552000) -- 
		(230.335000,355.279000) -- 
		(230.524000,354.871000) -- 
		(230.557000,354.722000) -- 
		(230.531000,354.601000) -- 
		(230.485000,354.553000) -- 
		(230.224000,354.404000) -- 
		(229.964000,354.338000) -- 
		(229.664000,354.294000) -- 
		(229.286000,354.277000) -- 
		(228.992000,354.288000) -- 
		(228.601000,354.327000) -- 
		(228.080000,354.426000) -- 
		(227.441000,354.584000) -- 
		(226.769000,354.739000) -- 
		(226.515000,354.788000) -- 
		(226.372000,354.798000) -- 
		(225.766000,354.760000) -- 
		(225.446000,354.809000) -- 
		(224.768000,354.947000) -- 
		(224.605000,354.995000) -- 
		(224.344000,355.112000) -- 
		(224.116000,355.233000) -- 
		(223.855000,355.337000) -- 
		(223.458000,355.458000) -- 
		(223.119000,355.540000) -- 
		(222.773000,355.639000) -- 
		(222.584000,355.704000) -- 
		(221.958000,356.018000) -- 
		(221.802000,356.067000) -- 
		(221.306000,356.095000) -- 
		(220.869000,356.039000) -- 
		(220.576000,355.968000) -- 
		(219.670000,355.847000) -- 
		(219.266000,355.748000) -- 
		(218.829000,355.610000) -- 
		(218.230000,355.335000) -- 
		(218.132000,355.279000) -- 
		(218.054000,355.220000) -- 
		(217.891000,355.077000) -- 
		(217.853000,355.036000) -- 
		(217.793000,354.972000) -- 
		(217.676000,354.719000) -- 
		(217.637000,354.532000) -- 
		(217.572000,354.307000) -- 
		(217.383000,354.125000) -- 
		(217.311000,354.037000) -- 
		(217.311000,353.965000) -- 
		(217.377000,353.867000) -- 
		(217.566000,353.768000) -- 
		(217.807000,353.680000) -- 
		(218.022000,353.543000) -- 
		(218.257000,353.317000) -- 
		(218.309000,353.284000) -- 
		(218.407000,353.103000) -- 
		(218.433000,352.943000) -- 
		(218.335000,352.652000) -- 
		(218.186000,352.547000) -- 
		(218.049000,352.520000) -- 
		(217.866000,352.465000) -- 
		(217.768000,352.377000) -- 
		(217.717000,352.079000) -- 
		(217.769000,351.981000) -- 
		(217.932000,351.926000) -- 
		(218.108000,351.849000) -- 
		(218.271000,351.722000) -- 
		(218.342000,351.624000) -- 
		(218.388000,351.486000) -- 
		(218.401000,351.338000) -- 
		(218.356000,351.030000) -- 
		(218.219000,350.733000) -- 
		(218.180000,350.616000) -- 
		(218.174000,350.518000) -- 
		(218.219000,350.337000) -- 
		(218.389000,350.122000) -- 
		(218.552000,350.007000) -- 
		(218.630000,349.991000) -- 
		(218.721000,350.035000) -- 
		(218.760000,350.106000) -- 
		(218.786000,350.211000) -- 
		(218.878000,350.469000) -- 
		(218.930000,350.551000) -- 
		(219.008000,350.551000) -- 
		(219.021000,350.512000) -- 
		(219.015000,350.315000) -- 
		(218.989000,350.160000) -- 
		(219.073000,349.782000) -- 
		(219.158000,349.529000) -- 
		(219.191000,349.259000) -- 
		(219.269000,348.842000) -- 
		(219.328000,348.622000) -- 
		(219.354000,348.429000) -- 
		(219.550000,348.242000) -- 
		(219.817000,348.182000) -- 
		(220.032000,348.155000) -- 
		(220.104000,348.105000) -- 
		(220.130000,348.023000) -- 
		(220.039000,347.830000) -- 
		(219.987000,347.286000) -- 
		(220.059000,346.813000) -- 
		(220.111000,346.625000) -- 
		(220.196000,346.428000) -- 
		(220.261000,346.362000) -- 
		(220.379000,346.307000) -- 
		(220.776000,346.169000) -- 
		(220.998000,346.005000) -- 
		(221.070000,345.834000) -- 
		(221.096000,345.668000) -- 
		(221.031000,345.322000) -- 
		(220.758000,344.717000) -- 
		(220.689000,344.578000) -- 
		(220.412000,344.019000) -- 
		(220.224000,343.332000) -- 
		(220.093000,342.980000) -- 
		(219.976000,342.826000) -- 
		(219.527000,342.336000) -- 
		(219.123000,341.934000) -- 
		(219.024000,341.821000) -- 
		(218.980000,341.772000) -- 
		(218.928000,341.650000) -- 
		(218.871000,341.464000) -- 
		(218.792000,341.291000) -- 
		(218.764000,341.150000) -- 
		(218.804000,341.062000) -- 
		(218.940000,340.968000) -- 
		(219.258000,340.906000) -- 
		(219.798000,340.924000) -- 
		(220.038000,340.953000) -- 
		(220.406000,341.100000) -- 
		(220.736000,341.167000) -- 
		(220.923000,341.179000) -- 
		(221.164000,341.117000) -- 
		(221.338000,341.035000) -- 
		(221.554000,340.856000) -- 
		(221.765000,340.616000) -- 
		(221.881000,340.342000) -- 
		(221.898000,340.310000) -- 
		(222.067000,340.036000) -- 
		(222.157000,339.819000) -- 
		(222.314000,339.499000) -- 
		(222.459000,339.168000) -- 
		(222.501000,339.038000) -- 
		(222.676000,338.795000) -- 
		(222.854000,338.443000) -- 
		(223.031000,338.124000) -- 
		(223.232000,337.837000) -- 
		(223.283000,337.620000) -- 
		(223.208000,337.515000) -- 
		(223.138000,337.440000) -- 
		(223.009000,337.404000) -- 
		(222.885000,337.440000) -- 
		(222.702000,337.524000) -- 
		(222.358000,337.740000) -- 
		(222.226000,337.776000) -- 
		(222.076000,337.798000) -- 
		(221.973000,337.786000) -- 
		(221.889000,337.751000) -- 
		(221.859000,337.700000) -- 
		(221.960000,337.258000) -- 
		(222.044000,336.761000) -- 
		(222.121000,336.516000) -- 
		(222.190000,336.370000) -- 
		(222.277000,336.241000) -- 
		(222.661000,335.987000) -- 
		(223.266000,335.694000) -- 
		(223.626000,335.595000) -- 
		(223.873000,335.595000) -- 
		(224.009000,335.625000) -- 
		(224.182000,335.707000) -- 
		(224.323000,335.787000) -- 
		(224.639000,335.899000) -- 
		(224.794000,335.941000) -- 
		(225.078000,335.959000) -- 
		(225.307000,335.937000) -- 
		(225.426000,335.897000) -- 
		(225.518000,335.844000) -- 
		(225.592000,335.736000) -- 
		(225.667000,335.595000) -- 
		(225.661000,335.546000) -- 
		(225.717000,335.371000) -- 
		(225.806000,335.204000) -- 
		(226.012000,334.988000) -- 
		(226.211000,334.841000) -- 
		(226.881000,334.594000) -- 
		(227.169000,334.470000) -- 
		(227.455000,334.377000) -- 
		(227.931000,334.267000) -- 
		(228.329000,334.135000) -- 
		(228.511000,334.047000) -- 
		(228.922000,333.772000) -- 
		(229.222000,333.530000) -- 
		(229.548000,333.327000) -- 
		(229.685000,333.228000) -- 
		(229.769000,333.118000) -- 
		(229.711000,333.080000) -- 
		(229.561000,333.118000) -- 
		(229.326000,333.260000) -- 
		(229.085000,333.443000) -- 
		(228.961000,333.564000) -- 
		(228.857000,333.597000) -- 
		(228.733000,333.612000) -- 
		(228.766000,333.508000) -- 
		(228.794000,333.454000) -- 
		(228.837000,333.370000) -- 
		(229.026000,333.113000) -- 
		(229.202000,332.931000) -- 
		(229.515000,332.666000) -- 
		(229.952000,332.331000) -- 
		(230.389000,332.035000) -- 
		(231.119000,331.574000) -- 
		(231.471000,331.293000) -- 
		(231.797000,331.106000) -- 
		(232.351000,330.859000) -- 
		(232.514000,330.733000) -- 
		(233.016000,330.397000) -- 
		(233.368000,330.265000) -- 
		(233.504000,330.166000) -- 
		(233.622000,330.062000) -- 
		(233.752000,329.852000) -- 
		(234.039000,329.600000) -- 
		(234.150000,329.457000) -- 
		(234.326000,329.342000) -- 
		(234.482000,329.215000) -- 
		(234.958000,328.484000) -- 
		(235.219000,327.989000) -- 
		(235.304000,327.802000) -- 
		(235.415000,327.499000) -- 
		(235.571000,327.027000) -- 
		(235.597000,326.857000) -- 
		(235.581000,326.584000) -- 
		(235.571000,326.399000) -- 
		(235.532000,326.103000) -- 
		(235.493000,325.552000) -- 
		(235.572000,325.141000) -- 
		(235.708000,324.626000) -- 
		(235.780000,324.231000) -- 
		(235.799000,323.659000) -- 
		(235.780000,323.208000) -- 
		(235.806000,322.878000) -- 
		(235.910000,322.433000) -- 
		(235.956000,321.761000) -- 
		(235.989000,321.009000) -- 
		(235.904000,320.443000) -- 
		(235.755000,319.898000) -- 
		(235.651000,319.563000) -- 
		(235.647000,319.366000) -- 
		(235.644000,319.244000) -- 
		(235.684000,318.892000) -- 
		(235.807000,318.639000) -- 
		(235.977000,318.502000) -- 
		(236.159000,318.440000) -- 
		(236.381000,318.440000) -- 
		(236.700000,318.474000) -- 
		(237.111000,318.480000) -- 
		(237.612000,318.671000) -- 
		(238.368000,318.755000) -- 
		(238.409000,318.756000) -- 
		(238.915000,318.777000) -- 
		(239.202000,318.848000) -- 
		(239.639000,318.904000) -- 
		(239.723000,319.129000) -- 
		(239.834000,319.140000) -- 
		(240.043000,319.058000) -- 
		(240.420000,319.062000) -- 
		(240.877000,319.146000) -- 
		(241.606000,319.256000) -- 
		(242.245000,319.405000) -- 
		(243.685000,319.712000) -- 
		(244.213000,319.779000) -- 
		(244.643000,319.916000) -- 
		(244.884000,320.026000) -- 
		(245.053000,320.021000) -- 
		(245.333000,319.922000) -- 
		(245.646000,319.911000) -- 
		(246.408000,319.988000) -- 
		(247.177000,320.010000) -- 
		(247.712000,320.016000) -- 
		(248.037000,320.071000) -- 
		(248.161000,320.181000) -- 
		(248.259000,320.285000) -- 
		(248.441000,320.302000) -- 
		(248.787000,320.241000) -- 
		(248.930000,320.247000) -- 
		(249.217000,320.230000) -- 
		(249.510000,320.137000) -- 
		(249.790000,319.978000) -- 
		(249.999000,319.791000) -- 
		(250.244000,319.449000) -- 
		(250.318000,319.345000) -- 
		(250.488000,319.037000) -- 
		(250.774000,318.735000) -- 
		(251.055000,318.322000) -- 
		(251.274000,318.068000) -- 
		(251.367000,317.959000) -- 
		(251.609000,317.795000) -- 
		(251.643000,317.767000) -- 
		(251.739000,317.690000) -- 
		(251.838000,317.373000) -- 
		(251.921000,317.106000) -- 
		(252.084000,316.684000) -- 
		(252.078000,316.492000) -- 
		(251.961000,316.321000) -- 
		(251.459000,316.068000) -- 
		(251.185000,316.013000) -- 
		(249.002000,315.918000) -- 
		(248.742000,315.924000) -- 
		(248.455000,315.980000) -- 
		(248.195000,316.134000) -- 
		(247.810000,316.391000) -- 
		(247.549000,316.612000) -- 
		(247.419000,316.700000) -- 
		(247.217000,316.737000) -- 
		(246.976000,316.711000) -- 
		(246.637000,316.546000) -- 
		(246.018000,315.996000) -- 
		(245.699000,315.676000) -- 
		(245.432000,315.380000) -- 
		(245.269000,315.155000) -- 
		(245.165000,315.060000) -- 
		(244.807000,314.902000) -- 
		(244.631000,314.759000) -- 
		(244.461000,314.539000) -- 
		(244.298000,314.373000) -- 
		(244.110000,314.137000) -- 
		(243.921000,313.829000) -- 
		(243.641000,313.487000) -- 
		(243.406000,313.098000) -- 
		(243.250000,312.811000) -- 
		(243.217000,312.530000) -- 
		(243.263000,312.299000) -- 
		(243.393000,312.091000) -- 
		(243.693000,311.855000) -- 
		(244.006000,311.663000) -- 
		(244.188000,311.486000) -- 
		(244.332000,311.238000) -- 
		(244.403000,310.975000) -- 
		(244.391000,310.601000) -- 
		(244.267000,310.304000) -- 
		(244.084000,310.045000) -- 
		(244.039000,309.924000) -- 
		(244.039000,309.786000) -- 
		(244.163000,309.678000) -- 
		(244.378000,309.595000) -- 
		(244.677000,309.568000) -- 
		(245.010000,309.682000) -- 
		(245.459000,309.870000) -- 
		(245.765000,310.019000) -- 
		(246.059000,310.056000) -- 
		(246.443000,310.024000) -- 
		(246.879000,309.935000) -- 
		(247.283000,309.805000) -- 
		(247.498000,309.629000) -- 
		(247.557000,309.442000) -- 
		(247.583000,309.271000) -- 
		(247.512000,309.156000) -- 
		(247.140000,308.874000) -- 
		(246.802000,308.578000) -- 
		(246.652000,308.401000) -- 
		(246.600000,308.265000) -- 
		(246.632000,308.067000) -- 
		(246.763000,307.979000) -- 
		(247.043000,307.852000) -- 
		(247.232000,307.720000) -- 
		(247.434000,307.473000) -- 
		(247.727000,307.039000) -- 
		(247.903000,306.802000) -- 
		(248.359000,306.516000) -- 
		(248.574000,306.240000) -- 
		(248.731000,305.873000) -- 
		(248.704000,305.648000) -- 
		(248.555000,305.432000) -- 
		(247.695000,304.982000) -- 
		(247.473000,304.894000) -- 
		(247.284000,304.905000) -- 
		(247.050000,305.037000) -- 
		(246.880000,305.086000) -- 
		(246.704000,305.086000) -- 
		(246.542000,305.037000) -- 
		(246.385000,304.861000) -- 
		(246.333000,304.663000) -- 
		(246.314000,304.492000) -- 
		(246.340000,304.222000) -- 
		(246.438000,303.943000) -- 
		(246.464000,303.723000) -- 
		(246.431000,303.437000) -- 
		(246.125000,302.606000) -- 
		(245.923000,302.095000) -- 
		(245.572000,301.594000) -- 
		(245.469000,301.397000) -- 
		(245.402000,301.270000) -- 
		(245.207000,300.973000) -- 
		(244.999000,300.803000) -- 
		(244.790000,300.726000) -- 
		(244.613000,300.704000) -- 
		(244.471000,300.687000) -- 
		(244.295000,300.747000) -- 
		(244.073000,300.852000) -- 
		(243.943000,301.050000) -- 
		(243.897000,301.407000) -- 
		(243.819000,301.649000) -- 
		(243.695000,301.841000) -- 
		(243.487000,301.974000) -- 
		(242.972000,302.122000) -- 
		(242.712000,302.116000) -- 
		(242.484000,302.055000) -- 
		(241.702000,301.588000) -- 
		(241.448000,301.484000) -- 
		(241.318000,301.456000) -- 
		(241.174000,301.500000) -- 
		(240.738000,301.785000) -- 
		(240.614000,301.808000) -- 
		(240.392000,301.748000) -- 
		(240.184000,301.616000) -- 
		(239.904000,301.324000) -- 
		(239.735000,301.076000) -- 
		(239.559000,300.718000) -- 
		(239.435000,300.493000) -- 
		(239.259000,300.318000) -- 
		(239.031000,300.153000) -- 
		(238.836000,300.065000) -- 
		(238.569000,299.992000) -- 
		(238.328000,299.998000) -- 
		(238.217000,300.048000) -- 
		(238.080000,300.158000) -- 
		(237.995000,300.306000) -- 
		(237.950000,300.471000) -- 
		(237.891000,300.609000) -- 
		(237.774000,300.735000) -- 
		(237.657000,300.796000) -- 
		(237.487000,300.823000) -- 
		(237.253000,300.845000) -- 
		(236.823000,300.811000) -- 
		(236.484000,300.729000) -- 
		(236.119000,300.603000) -- 
		(235.370000,300.212000) -- 
		(234.875000,299.722000) -- 
		(234.628000,299.530000) -- 
		(234.302000,299.299000) -- 
		(233.683000,298.892000) -- 
		(233.481000,298.777000) -- 
		(233.351000,298.744000) -- 
		(233.038000,298.705000) -- 
		(232.819000,298.594000) -- 
		(232.511000,298.402000) -- 
		(232.400000,298.281000) -- 
		(232.374000,298.188000) -- 
		(232.511000,298.078000) -- 
		(232.713000,298.056000) -- 
		(233.052000,297.974000) -- 
		(233.397000,297.825000) -- 
		(233.638000,297.659000) -- 
		(233.814000,297.462000) -- 
		(233.970000,297.192000) -- 
		(234.003000,297.006000) -- 
		(233.971000,296.803000) -- 
		(233.827000,296.533000) -- 
		(233.606000,296.269000) -- 
		(233.560000,296.198000) -- 
		(233.573000,296.131000) -- 
		(233.743000,296.066000) -- 
		(234.068000,295.978000) -- 
		(234.225000,295.857000) -- 
		(234.355000,295.637000) -- 
		(234.394000,295.401000) -- 
		(234.375000,295.126000) -- 
		(234.297000,294.735000) -- 
		(234.310000,294.614000) -- 
		(234.401000,294.488000) -- 
		(234.610000,294.361000) -- 
		(234.740000,294.246000) -- 
		(234.870000,294.053000) -- 
		(234.870000,293.927000) -- 
		(234.955000,293.657000) -- 
		(235.072000,293.514000) -- 
		(235.274000,293.361000) -- 
		(235.359000,293.190000) -- 
		(235.379000,293.052000) -- 
		(235.333000,292.827000) -- 
		(235.340000,292.635000) -- 
		(235.386000,292.453000) -- 
		(235.496000,292.322000) -- 
		(235.822000,292.179000) -- 
		(235.972000,292.030000) -- 
		(236.122000,291.728000) -- 
		(236.148000,291.397000) -- 
		(236.246000,291.127000) -- 
		(236.402000,290.781000) -- 
		(236.578000,290.584000) -- 
		(236.904000,290.342000) -- 
		(237.340000,290.007000) -- 
		(237.992000,289.485000) -- 
		(238.324000,289.188000) -- 
		(238.559000,289.056000) -- 
		(238.976000,289.011000) -- 
		(239.119000,288.901000) -- 
		(239.223000,288.748000) -- 
		(239.301000,288.451000) -- 
		(239.425000,288.132000) -- 
		(239.608000,287.851000) -- 
		(239.803000,287.609000) -- 
		(239.999000,287.423000) -- 
		(240.129000,287.224000) -- 
		(240.207000,287.033000) -- 
		(240.246000,286.829000) -- 
		(240.351000,286.697000) -- 
		(240.500000,286.681000) -- 
		(240.728000,286.730000) -- 
		(240.950000,286.818000) -- 
		(241.093000,286.829000) -- 
		(241.230000,286.762000) -- 
		(241.367000,286.582000) -- 
		(241.999000,285.576000) -- 
		(242.201000,285.290000) -- 
		(242.416000,285.136000) -- 
		(242.618000,285.113000) -- 
		(242.833000,285.147000) -- 
		(243.054000,285.164000) -- 
		(243.289000,285.113000) -- 
		(243.445000,285.026000) -- 
		(243.517000,284.839000) -- 
		(243.497000,284.625000) -- 
		(243.419000,284.449000) -- 
		(243.340000,284.367000) -- 
		(243.181000,284.204000) -- 
		(242.931000,283.981000) -- 
		(242.748000,283.717000) -- 
		(242.703000,283.547000) -- 
		(242.729000,283.404000) -- 
		(242.866000,283.283000) -- 
		(243.048000,283.244000) -- 
		(243.211000,283.288000) -- 
		(243.419000,283.382000) -- 
		(243.673000,283.360000) -- 
		(243.862000,283.289000) -- 
		(243.927000,283.184000) -- 
		(243.901000,282.904000) -- 
		(243.914000,282.518000) -- 
		(243.993000,282.310000) -- 
		(244.142000,282.123000) -- 
		(244.585000,281.826000) -- 
		(245.022000,281.491000) -- 
		(245.413000,281.155000) -- 
		(245.478000,280.996000) -- 
		(245.478000,280.765000) -- 
		(245.296000,280.571000) -- 
		(245.120000,280.523000) -- 
		(244.827000,280.512000) -- 
		(244.403000,280.593000) -- 
		(244.019000,280.765000) -- 
		(243.772000,280.924000) -- 
		(243.615000,281.105000) -- 
		(243.511000,281.325000) -- 
		(243.387000,281.457000) -- 
		(243.276000,281.539000) -- 
		(243.140000,281.522000) -- 
		(243.068000,281.440000) -- 
		(243.094000,281.331000) -- 
		(243.290000,281.138000) -- 
		(243.257000,281.049000) -- 
		(243.120000,281.017000) -- 
		(242.801000,280.979000) -- 
		(242.612000,280.792000) -- 
		(242.378000,280.433000) -- 
		(242.183000,280.077000) -- 
		(241.994000,279.850000) -- 
		(241.805000,279.698000) -- 
		(241.662000,279.631000) -- 
		(241.505000,279.642000) -- 
		(241.219000,279.807000) -- 
		(241.036000,279.901000) -- 
		(240.834000,279.906000) -- 
		(240.613000,279.813000) -- 
		(240.398000,279.565000) -- 
		(240.112000,279.109000) -- 
		(240.047000,278.812000) -- 
		(240.073000,278.366000) -- 
		(240.021000,277.756000) -- 
		(240.060000,277.558000) -- 
		(240.158000,277.475000) -- 
		(240.353000,277.443000) -- 
		(240.548000,277.498000) -- 
		(240.861000,277.668000) -- 
		(241.402000,277.734000) -- 
		(241.942000,277.773000) -- 
		(242.131000,277.740000) -- 
		(242.281000,277.657000) -- 
		(242.418000,277.498000) -- 
		(242.554000,277.250000) -- 
		(242.782000,277.069000) -- 
		(243.049000,276.921000) -- 
		(243.427000,276.788000) -- 
		(243.720000,276.602000) -- 
		(243.961000,276.388000) -- 
		(244.281000,275.969000) -- 
		(244.430000,275.629000) -- 
		(244.671000,275.161000) -- 
		(244.802000,274.875000) -- 
		(244.867000,274.623000) -- 
		(245.108000,274.326000) -- 
		(245.401000,274.012000) -- 
		(245.571000,273.715000) -- 
		(246.040000,272.308000) -- 
		(246.092000,271.917000) -- 
		(246.046000,271.719000) -- 
		(245.988000,271.609000) -- 
		(245.864000,271.531000) -- 
		(245.682000,271.499000) -- 
		(245.454000,271.516000) -- 
		(245.050000,271.669000) -- 
		(244.640000,271.846000) -- 
		(244.425000,271.917000) -- 
		(244.210000,271.945000) -- 
		(244.027000,271.944000) -- 
		(243.793000,271.829000) -- 
		(243.565000,271.635000) -- 
		(243.279000,271.323000) -- 
		(243.038000,271.021000) -- 
		(242.855000,270.900000) -- 
		(242.510000,270.822000) -- 
		(242.185000,270.718000) -- 
		(241.950000,270.542000) -- 
		(241.768000,270.304000) -- 
		(241.736000,270.074000) -- 
		(241.749000,269.778000) -- 
		(241.846000,269.569000) -- 
		(242.029000,269.310000) -- 
		(242.335000,269.079000) -- 
		(242.530000,268.947000) -- 
		(242.732000,268.716000) -- 
		(242.765000,268.474000) -- 
		(242.706000,268.233000) -- 
		(242.491000,268.007000) -- 
		(242.198000,267.847000) -- 
		(241.827000,267.577000) -- 
		(241.645000,267.490000) -- 
		(241.391000,267.512000) -- 
		(241.170000,267.627000) -- 
		(240.955000,267.802000) -- 
		(240.740000,267.996000) -- 
		(240.590000,268.050000) -- 
		(240.453000,268.044000) -- 
		(240.290000,268.007000) -- 
		(240.206000,267.864000) -- 
		(239.809000,267.346000) -- 
		(239.581000,267.182000) -- 
		(239.327000,267.093000) -- 
		(239.255000,267.076000) -- 
		(239.092000,266.967000) -- 
		(238.988000,266.819000) -- 
		(238.936000,266.643000) -- 
		(238.969000,266.477000) -- 
		(239.027000,266.356000) -- 
		(239.021000,266.087000) -- 
		(239.027000,266.038000) -- 
		(239.053000,266.010000) -- 
		(239.216000,265.928000) -- 
		(239.464000,265.862000) -- 
		(239.737000,265.703000) -- 
		(239.829000,265.531000) -- 
		(239.816000,265.295000) -- 
		(239.744000,265.120000) -- 
		(239.588000,264.933000) -- 
		(239.580000,264.898000) -- 
		(239.549000,264.768000) -- 
		(239.464000,264.641000) -- 
		(239.327000,264.537000) -- 
		(239.490000,264.366000) -- 
		(239.555000,264.355000) -- 
		(239.855000,264.338000) -- 
		(239.972000,264.295000) -- 
		(239.985000,264.245000) -- 
		(239.966000,264.223000) -- 
		(239.790000,264.113000) -- 
		(239.737000,264.040000) -- 
		(239.173000,264.289000) -- 
		(239.149000,264.301000) -- 
		(238.919000,264.409000) -- 
		(238.690000,264.517000) -- 
		(237.683000,264.991000) -- 
		(236.023000,265.743000) -- 
		(234.694000,266.330000) -- 
		(232.920000,267.132000) -- 
		(232.177000,267.497000) -- 
		(230.463000,268.271000) -- 
		(228.390000,269.231000) -- 
		(225.915000,270.351000) -- 
		(223.583000,271.405000) -- 
		(220.780000,272.696000) -- 
		(219.941000,273.032000) -- 
		(219.291000,273.247000) -- 
		(218.556000,273.429000) -- 
		(218.368000,273.453000) -- 
		(217.481000,273.521000) -- 
		(215.174000,273.529000) -- 
		(213.888000,273.521000) -- 
		(212.455000,273.511000) -- 
		(211.002000,273.502000) -- 
		(209.983000,273.507000) -- 
		(209.982000,273.508000) -- 
		(209.344000,273.511000) -- 
		(208.155000,273.513000) -- 
		(206.721000,273.516000) -- 
		(206.380000,273.507000) -- 
		(205.997000,273.501000) -- 
		(205.356000,273.515000) -- 
		(205.014000,273.523000) -- 
		(203.828000,273.503000) -- 
		(201.617000,273.502000) -- 
		(200.497000,273.502000) -- 
		(200.399000,273.499000) -- 
		(199.964000,273.491000) -- 
		(197.049000,273.498000) -- 
		(192.461000,273.468000) -- 
		(191.763000,273.471000) -- 
		(188.886000,273.482000) -- 
		(188.282000,273.494000) -- 
		(187.928000,273.469000) -- 
		(187.237000,273.348000) -- 
		(186.849000,273.252000) -- 
		(186.583000,273.180000) -- 
		(185.991000,273.023000) -- 
		(184.686000,272.675000) -- 
		(184.081000,272.488000) -- 
		(183.721000,272.323000) -- 
		(182.640000,271.670000) -- 
		(182.342000,271.466000) -- 
		(181.306000,270.757000) -- 
		(180.996000,270.515000) -- 
		(180.094000,269.817000) -- 
		(178.814000,268.888000) -- 
		(177.630000,267.964000) -- 
		(177.054000,267.527000) -- 
		(176.623000,267.202000) -- 
		(175.707000,266.577000) -- 
		(175.164000,266.095000) -- 
		(175.003000,265.891000) -- 
		(174.418000,265.058000) -- 
		(174.138000,264.663000) -- 
		(173.425000,263.570000) -- 
		(172.683000,262.479000) -- 
		(172.594000,262.331000) -- 
		(171.813000,261.196000) -- 
		(171.070000,260.071000) -- 
		(171.017000,259.991000) -- 
		(169.954000,258.408000) -- 
		(169.483000,257.716000) -- 
		(169.264000,257.384000) -- 
		(168.450000,256.154000) -- 
		(168.370000,256.029000) -- 
		(168.233000,255.813000) -- 
		(167.943000,255.128000) -- 
		(167.613000,254.097000) -- 
		(167.569000,253.963000) -- 
		(167.247000,252.918000) -- 
		(167.155000,252.621000) -- 
		(167.138000,252.567000) -- 
		(166.799000,251.519000) -- 
		(166.685000,251.170000) -- 
		(166.271000,249.840000) -- 
		(165.987000,248.932000) -- 
		(165.843000,248.446000) -- 
		(165.770000,248.137000) -- 
		(165.704000,247.680000) -- 
		(165.682000,247.237000) -- 
		(165.627000,246.119000) -- 
		(165.627000,246.118000) -- 
		(165.610000,245.743000) -- 
		(165.597000,245.471000) -- 
		(165.575000,245.197000) -- 
		(165.541000,245.055000) -- 
		(165.412000,244.780000) -- 
		(165.277000,244.625000) -- 
		(165.138000,244.497000) -- 
		(164.928000,244.339000) -- 
		(164.662000,244.196000) -- 
		(164.652000,244.192000) -- 
		(164.494000,244.141000) -- 
		(164.134000,244.098000) -- 
		(163.746000,244.084000) -- 
		(163.056000,244.076000) -- 
		(162.936000,244.080000) -- 
		(162.934000,244.081000) -- 
		(162.360000,244.068000) -- 
		(161.707000,244.052000) -- 
		(160.530000,244.028000) -- 
		(160.530000,244.027000) -- 
		(159.896000,244.014000) -- 
		(159.748000,244.015000) -- 
		(159.369000,244.019000) -- 
		(159.094000,244.021000) -- 
		(157.331000,243.993000) -- 
		(157.174000,243.996000) -- 
		(156.594000,244.013000) -- 
		(156.387000,244.045000) -- 
		(156.166000,244.090000) -- 
		(155.771000,244.201000) -- 
		(155.446000,244.312000) -- 
		(154.843000,244.620000) -- 
		(151.947000,246.238000) -- 
		(150.269000,247.175000) -- 
		(150.400000,245.747000) -- 
		(150.406000,245.679000) -- 
		(150.568000,243.047000) -- 
		(150.607000,241.775000) -- 
		(150.611000,241.451000) -- 
		(150.617000,240.757000) -- 
		(150.614000,240.018000) -- 
		(150.610000,238.535000) -- 
		(150.595000,234.259000) -- 
		(150.595000,234.087000) -- 
		(150.588000,232.606000) -- 
		(150.590000,232.286000) -- 
		(150.593000,231.551000) -- 
		(150.595000,231.331000) -- 
		(150.597000,231.013000) -- 
		(150.571000,230.624000) -- 
		(150.573000,229.997000) -- 
		(150.573000,229.590000) -- 
		(150.575000,228.449000) -- 
		(150.564000,226.950000) -- 
		(150.563000,226.853000) -- 
		(150.545000,226.585000) -- 
		(150.483000,226.338000) -- 
		(150.424000,226.152000) -- 
		(150.341000,225.970000) -- 
		(150.286000,225.846000) -- 
		(150.243000,225.754000) -- 
		(150.088000,225.492000) -- 
		(150.019000,225.376000) -- 
		(149.531000,224.654000) -- 
		(149.363000,224.399000) -- 
		(148.626000,223.290000) -- 
		(148.365000,222.831000) -- 
		(148.176000,222.473000) -- 
		(148.045000,222.007000) -- 
		(147.987000,221.666000) -- 
		(148.002000,221.497000) -- 
		(148.059000,221.237000) -- 
		(148.193000,220.623000) -- 
		(148.314000,220.227000) -- 
		(148.360000,220.103000) -- 
		(148.419000,219.945000) -- 
		(148.565000,219.603000) -- 
		(148.632000,219.492000) -- 
		(148.726000,219.335000) -- 
		(148.916000,219.083000) -- 
		(149.297000,218.650000) -- 
		(149.515000,218.403000) -- 
		(149.914000,217.947000) -- 
		(150.355000,217.445000) -- 
		(150.414000,217.378000) -- 
		(150.677000,217.006000) -- 
		(150.698000,216.968000) -- 
		(150.841000,216.698000) -- 
		(150.842000,216.697000) -- 
		(150.897000,216.593000) -- 
		(151.100000,216.102000) -- 
		(151.237000,215.549000) -- 
		(151.277000,215.313000) -- 
		(151.327000,215.013000) -- 
		(151.361000,214.733000) -- 
		(151.391000,214.482000) -- 
		(151.394000,214.313000) -- 
		(151.397000,214.063000) -- 
		(151.392000,213.769000) -- 
		(151.318000,213.217000) -- 
		(151.285000,213.056000) -- 
		(151.199000,212.644000) -- 
		(151.120000,212.269000) -- 
		(151.105000,212.016000) -- 
		(151.084000,211.253000) -- 
		(151.084000,211.121000) -- 
		(151.083000,210.739000) -- 
		(151.082000,210.059000) -- 
		(151.081000,209.886000) -- 
		(151.069000,208.020000) -- 
		(151.064000,207.341000) -- 
		(151.054000,205.660000) -- 
		(151.056000,204.767000) -- 
		(151.051000,204.414000) -- 
		(151.043000,203.861000) -- 
		(151.037000,203.367000) -- 
		(151.036000,203.039000) -- 
		(151.029000,202.057000) -- 
		(151.027000,201.732000) -- 
		(151.028000,201.548000) -- 
		(151.028000,201.298000) -- 
		(150.998000,200.998000) -- 
		(150.979000,200.816000) -- 
		(150.858000,200.460000) -- 
		(150.748000,200.182000) -- 
		(150.651000,199.990000) -- 
		(150.604000,199.917000) -- 
		(150.343000,199.523000) -- 
		(149.919000,199.158000) -- 
		(149.408000,198.813000) -- 
		(148.310000,198.151000) -- 
		(147.472000,197.648000) -- 
		(147.122000,197.447000) -- 
		(146.070000,196.854000) -- 
		(145.962000,196.795000) -- 
		(144.387000,195.874000) -- 
		(143.279000,195.225000) -- 
		(142.018000,194.519000) -- 
		(141.889000,194.436000) -- 
		(140.934000,193.832000) -- 
		(140.519000,193.589000) -- 
		(139.765000,193.145000) -- 
		(138.466000,192.383000) -- 
		(137.490000,191.833000) -- 
		(137.468000,191.822000) -- 
		(136.739000,191.383000) -- 
		(135.840000,190.842000) -- 
		(133.219000,189.319000) -- 
		(132.927000,189.144000) -- 
		(132.678000,188.996000) -- 
		(132.241000,188.719000) -- 
		(131.686000,188.414000) -- 
		(131.020000,188.011000) -- 
		(130.035000,187.462000) -- 
		(129.157000,186.931000) -- 
		(128.003000,186.256000) -- 
		(126.922000,185.690000) -- 
		(126.299000,185.439000) -- 
		(126.002000,185.335000) -- 
		(125.907000,185.316000) -- 
		(125.907000,185.315000) -- 
		(125.768000,185.288000) -- 
		(124.151000,185.248000) -- 
		(122.931000,185.254000) -- 
		(121.099000,185.227000) -- 
		(120.660000,185.222000) -- 
		(120.356000,185.221000) -- 
		(118.340000,185.203000) -- 
		(117.776000,185.177000) -- 
		(117.198000,185.133000) -- 
		(116.558000,184.939000) -- 
		(116.296000,184.819000) -- 
		(116.137000,184.721000) -- 
		(115.794000,184.512000) -- 
		(115.658000,184.425000) -- 
		(115.499000,184.325000) -- 
		(115.096000,184.070000) -- 
		(114.479000,183.630000) -- 
		(114.326000,183.493000) -- 
		(114.134000,183.321000) -- 
		(113.952000,183.110000) -- 
		(113.702000,182.792000) -- 
		(113.581000,182.613000) -- 
		(113.502000,182.499000) -- 
		(113.502000,182.498000) -- 
		(113.256000,182.028000) -- 
		(113.081000,181.626000) -- 
		(112.872000,181.114000) -- 
		(112.872000,181.113000) -- 
		(112.601000,180.506000) -- 
		(112.404000,180.031000) -- 
		(112.365000,179.919000) -- 
		(112.134000,179.250000) -- 
		(111.921000,178.682000) -- 
		(111.714000,178.163000) -- 
		(111.424000,177.728000) -- 
		(111.132000,177.434000) -- 
		(109.697000,176.191000) -- 
		(109.605000,176.115000) -- 
		(108.635000,175.332000) -- 
		(108.464000,175.157000) -- 
		(108.362000,175.053000) -- 
		(108.105000,174.812000) -- 
		(107.826000,174.475000) -- 
		(107.733000,174.334000) -- 
		(107.624000,174.168000) -- 
		(107.412000,173.652000) -- 
		(107.305000,173.289000) -- 
		(107.215000,172.840000) -- 
		(107.194000,172.445000) -- 
		(107.118000,171.037000) -- 
		(107.037000,169.890000) -- 
		(107.021000,169.364000) -- 
		(106.981000,168.687000) -- 
		(106.898000,167.501000) -- 
		(106.868000,166.914000) -- 
		(106.837000,166.493000) -- 
		(106.722000,164.980000) -- 
		(106.722000,164.979000) -- 
		(106.692000,164.560000) -- 
		(106.698000,164.511000) -- 
		(106.104000,164.680000) -- 
		(105.416000,164.841000) -- 
		(104.878000,164.951000) -- 
		(104.877000,164.952000) -- 
		(104.321000,165.066000) -- 
		(103.360000,165.290000) -- 
		(103.219000,165.317000) -- 
		(102.450000,165.467000) -- 
		(102.274000,165.480000) -- 
		(102.273000,165.481000) -- 
		(102.025000,165.501000) -- 
		(101.203000,165.510000) -- 
		(100.561000,165.473000) -- 
		(99.915000,165.347000) -- 
		(99.416000,165.265000) -- 
		(98.855000,165.122000) -- 
		(98.663000,165.073000) -- 
		(97.933000,164.935000) -- 
		(97.546000,164.863000) -- 
		(97.086000,164.804000) -- 
		(96.607000,164.768000) -- 
		(95.962000,164.803000) -- 
		(95.365000,164.877000) -- 
		(95.207000,164.897000) -- 
		(95.055000,164.917000) -- 
		(94.005000,165.095000) -- 
		(93.406000,165.186000) -- 
		(92.819000,165.275000) -- 
		(91.439000,165.511000) -- 
		(87.703000,166.081000) -- 
		(86.823000,166.216000) -- 
		(85.145000,166.465000) -- 
		(85.085000,166.473000) -- 
		(84.903000,166.500000) -- 
		(84.842000,166.509000) -- 
		(84.450000,166.567000) -- 
		(80.704000,167.199000) -- 
		(79.801000,167.321000) -- 
		(78.969000,167.433000) -- 
		(77.472000,167.544000) -- 
		(76.970000,167.550000) -- 
		(74.724000,167.476000) -- 
		(73.196000,167.437000) -- 
		(73.196000,167.436000) -- 
		(72.877000,167.429000) -- 
		(72.528000,167.420000) -- 
		(72.528000,167.419000) -- 
		(72.223000,167.414000) -- 
		(72.223000,167.413000) -- 
		(70.552000,167.383000) -- 
		(68.818000,167.339000) -- 
		(68.178000,167.323000) -- 
		(67.009000,167.300000) -- 
		(66.616000,167.293000) -- 
		(66.162000,167.284000) -- 
		(64.821000,167.262000) -- 
		(64.797000,167.261000) -- 
		(64.342000,167.246000) -- 
		(63.405000,167.214000) -- 
		(62.511000,167.188000) -- 
		(61.186000,167.154000) -- 
		(58.536000,167.125000) -- 
		(55.885000,167.080000) -- 
		(55.818000,167.098000) -- 
		(55.801000,167.094000) -- 
		(55.751000,167.078000) -- 
		(55.373000,167.069000) -- 
		(53.016000,167.014000) -- 
		(51.788000,166.998000) -- 
		(51.788000,166.997000) -- 
		(50.792000,166.974000) -- 
		(50.397000,166.965000) -- 
		(49.003000,166.930000) -- 
		(47.969000,166.919000) -- 
		(47.437000,166.914000) -- 
		(46.322000,166.875000) -- 
		(45.652000,166.842000) -- 
		(44.997000,166.805000) -- 
		(44.552000,166.773000) -- 
		(43.003000,166.569000) -- 
		(41.746000,166.396000) -- 
		(41.288000,166.330000) -- 
		(41.155000,166.313000) -- 
		(39.914000,166.127000) -- 
		(39.455000,166.061000) -- 
		(39.039000,165.998000) -- 
		(38.025000,165.850000) -- 
		(37.969000,165.841000) -- 
		(37.969000,165.840000) -- 
		(37.788000,165.812000) -- 
		(37.371000,165.749000) -- 
		(36.357000,165.602000) -- 
		(36.357000,165.601000) -- 
		(35.921000,165.547000) -- 
		(35.652000,165.509000) -- 
		(35.232000,165.427000) -- 
		(34.731000,165.330000) -- 
		(34.275000,165.216000) -- 
		(34.275000,165.215000) -- 
		(34.176000,165.191000) -- 
		(34.176000,165.190000) -- 
		(32.991000,164.843000) -- 
		(31.903000,164.570000) -- 
		(30.741000,164.231000) -- 
		(30.741000,164.230000) -- 
		(30.059000,164.026000) -- 
		(29.413000,163.856000) -- 
		(28.966000,163.737000) -- 
		(28.837000,163.703000) -- 
		(28.007000,163.483000) -- 
		(28.007000,163.482000) -- 
		(27.843000,163.440000) -- 
		(27.602000,163.377000) -- 
		(26.878000,163.168000) -- 
		(26.532000,163.076000) -- 
		(26.420000,163.046000) -- 
		(25.922000,162.913000) -- 
		(25.414000,162.762000) -- 
		(25.414000,162.761000) -- 
		(25.158000,162.686000) -- 
		(25.052000,162.657000) -- 
		(24.646000,162.556000) -- 
		(24.592000,162.542000) -- 
		(24.410000,162.496000) -- 
		(23.931000,162.342000) -- 
		(23.754000,162.291000) -- 
		(22.001000,161.793000) -- 
		(21.220000,161.596000) -- 
		(20.860000,161.506000) -- 
		(20.375000,161.363000) -- 
		(20.023000,161.264000) -- 
		(18.966000,160.969000) -- 
		(18.613000,160.871000) -- 
		(17.999000,160.696000) -- 
		(17.813000,160.644000) -- 
		(17.294000,160.471000) -- 
		(16.566000,160.194000) -- 
		(16.248000,160.054000) -- 
		(16.203000,160.031000) -- 
		(16.020000,159.940000) -- 
		(15.651000,159.709000) -- 
		(14.935000,159.261000) -- 
		(14.507000,159.008000) -- 
		(13.794000,158.557000) -- 
		(12.941000,158.040000) -- 
		(12.498000,157.708000) -- 
		(11.477000,157.079000) -- 
		(11.336000,156.992000) -- 
		(11.196000,156.905000) -- 
		(11.135000,156.868000) -- 
		(11.010000,156.791000) -- 
		(10.885000,156.714000) -- 
		(9.327000,155.753000) -- 
		(7.525000,154.643000) -- 
		(6.769000,154.201000) -- 
		(6.059000,153.749000) -- 
		(4.650000,152.872000) -- 
		(4.630000,152.861000) -- 
		(3.976000,152.479000) -- 
		(3.238000,152.151000) -- 
		(3.007000,152.076000) -- 
		(2.721000,151.983000) -- 
		(2.068000,151.833000) -- 
		(1.644000,151.764000) -- 
		(1.277000,151.760000) -- 
		(0.491000,151.755000) -- 
		(-0.770000,151.762000) -- 
		(-1.508000,151.766000) -- 
		(-1.722000,151.760000) -- 
		(-2.270000,151.746000) -- 
		(-2.657000,151.709000) -- 
		(-3.036000,151.659000) -- 
		(-3.471000,151.561000) -- 
		(-3.985000,151.399000) -- 
		(-4.127000,151.355000) -- 
		(-5.597000,150.874000) -- 
		(-5.666000,150.851000) -- 
		(-6.043000,150.726000) -- 
		(-7.058000,150.390000) -- 
		(-7.461000,150.256000) -- 
		(-7.803000,150.143000) -- 
		(-7.803000,150.142000) -- 
		(-8.157000,150.025000) -- 
		(-8.157000,150.024000) -- 
		(-8.583000,149.885000) -- 
		(-10.925000,149.118000) -- 
		(-12.849000,148.473000) -- 
		(-13.166000,148.368000) -- 
		(-14.646000,147.882000) -- 
		(-14.968000,147.775000) -- 
		(-15.440000,147.621000) -- 
		(-15.931000,147.450000) -- 
		(-16.252000,147.339000) -- 
		(-16.475000,147.260000) -- 
		(-16.763000,147.162000) -- 
		(-17.122000,147.047000) -- 
		(-17.122000,147.046000) -- 
		(-17.150000,147.037000) -- 
		(-17.376000,146.966000) -- 
		(-17.766000,146.841000) -- 
		(-18.036000,146.755000) -- 
		(-18.937000,146.461000) -- 
		(-19.328000,146.335000) -- 
		(-19.585000,146.250000) -- 
		(-19.695000,146.213000) -- 
		(-20.796000,145.850000) -- 
		(-21.164000,145.730000) -- 
		(-21.495000,145.622000) -- 
		(-21.875000,145.497000) -- 
		(-22.429000,145.316000) -- 
		(-22.849000,145.168000) -- 
		(-24.010000,144.795000) -- 
		(-24.546000,144.624000) -- 
		(-24.728000,144.585000) -- 
		(-24.838000,144.562000) -- 
		(-25.082000,144.560000) -- 
		(-25.216000,144.577000) -- 
		(-25.391000,144.600000) -- 
		(-25.876000,144.692000) -- 
		(-26.664000,144.867000) -- 
		(-27.145000,144.975000) -- 
		(-27.374000,145.025000) -- 
		(-28.061000,145.177000) -- 
		(-28.290000,145.228000) -- 
		(-28.146000,145.457000) -- 
		(-27.866000,145.749000) -- 
		(-27.112000,146.140000) -- 
		(-26.722000,146.409000) -- 
		(-26.618000,146.552000) -- 
		(-26.540000,146.800000) -- 
		(-26.553000,146.943000) -- 
		(-26.638000,147.096000) -- 
		(-26.723000,147.184000) -- 
		(-26.749000,147.212000) -- 
		(-26.918000,147.294000) -- 
		(-26.983000,147.360000) -- 
		(-26.911000,147.800000) -- 
		(-26.931000,148.383000) -- 
		(-26.977000,148.548000) -- 
		(-27.061000,148.823000) -- 
		(-26.990000,149.307000) -- 
		(-27.075000,149.488000) -- 
		(-27.081000,149.769000) -- 
		(-26.939000,150.242000) -- 
		(-26.939000,150.555000) -- 
		(-27.056000,150.764000) -- 
		(-27.141000,150.989000) -- 
		(-27.102000,151.633000) -- 
		(-27.167000,151.968000) -- 
		(-27.226000,152.133000) -- 
		(-27.232000,152.221000) -- 
		(-27.063000,152.447000) -- 
		(-27.031000,152.524000) -- 
		(-27.063000,152.639000) -- 
		(-27.252000,152.710000) -- 
		(-27.447000,152.771000) -- 
		(-27.480000,152.826000) -- 
		(-27.499000,153.079000) -- 
		(-27.577000,153.244000) -- 
		(-27.688000,153.343000) -- 
		(-27.948000,153.392000) -- 
		(-28.260000,153.381000) -- 
		(-28.377000,153.430000) -- 
		(-28.429000,153.595000) -- 
		(-28.501000,153.755000) -- 
		(-28.605000,153.826000) -- 
		(-28.839000,153.876000) -- 
		(-28.895000,153.867000) -- 
		(-29.022000,153.866000) -- 
		(-29.104000,153.886000) -- 
		(-29.196000,153.899000) -- 
		(-29.357000,153.922000) -- 
		(-29.436000,153.949000) -- 
		(-29.581000,154.025000) -- 
		(-29.608000,154.053000) -- 
		(-29.747000,154.391000) -- 
		(-29.806000,154.487000) -- 
		(-29.882000,154.573000) -- 
		(-29.978000,154.644000) -- 
		(-30.082000,154.707000) -- 
		(-30.535000,154.904000) -- 
		(-30.607000,154.943000) -- 
		(-30.725000,155.048000) -- 
		(-30.751000,155.076000) -- 
		(-30.772000,155.107000) -- 
		(-30.807000,155.173000) -- 
		(-30.829000,155.202000) -- 
		(-30.866000,155.220000) -- 
		(-30.972000,155.282000) -- 
		(-31.013000,155.289000) -- 
		(-31.056000,155.290000) -- 
		(-31.270000,155.280000) -- 
		(-31.397000,155.279000) -- 
		(-31.481000,155.291000) -- 
		(-31.552000,155.332000) -- 
		(-31.701000,155.345000) -- 
		(-31.734000,155.348000) -- 
		(-31.897000,155.392000) -- 
		(-31.968000,155.497000) -- 
		(-32.027000,155.667000) -- 
		(-31.999000,155.851000) -- 
		(-31.988000,155.926000) -- 
		(-32.008000,155.981000) -- 
		(-32.092000,156.030000) -- 
		(-32.203000,156.035000) -- 
		(-32.539000,156.051000) -- 
		(-32.651000,156.057000) -- 
		(-32.775000,156.096000) -- 
		(-32.912000,156.261000) -- 
		(-32.971000,156.301000) -- 
		(-33.016000,156.332000) -- 
		(-33.217000,156.343000) -- 
		(-33.484000,156.376000) -- 
		(-33.575000,156.464000) -- 
		(-33.614000,156.563000) -- 
		(-33.595000,156.810000) -- 
		(-33.686000,156.931000) -- 
		(-33.759000,156.986000) -- 
		(-34.089000,157.241000) -- 
		(-34.135000,157.277000) -- 
		(-34.298000,157.470000) -- 
		(-34.402000,157.629000) -- 
		(-34.439000,157.760000) -- 
		(-34.441000,157.767000) -- 
		(-34.370000,157.954000) -- 
		(-34.129000,158.207000) -- 
		(-34.103000,158.317000) -- 
		(-34.142000,158.432000) -- 
		(-34.396000,158.735000) -- 
		(-34.559000,159.235000) -- 
		(-34.546000,159.437000) -- 
		(-34.539000,159.548000) -- 
		(-34.488000,159.823000) -- 
		(-34.514000,159.933000) -- 
		(-34.605000,160.032000) -- 
		(-34.911000,160.219000) -- 
		(-35.353000,160.461000) -- 
		(-35.502000,160.496000) -- 
		(-35.581000,160.516000) -- 
		(-35.737000,160.571000) -- 
		(-36.069000,160.746000) -- 
		(-36.192000,160.856000) -- 
		(-36.225000,160.994000) -- 
		(-36.264000,161.192000) -- 
		(-36.407000,161.351000) -- 
		(-36.479000,161.483000) -- 
		(-36.511000,161.676000) -- 
		(-36.466000,161.901000) -- 
		(-36.466000,162.099000) -- 
		(-36.440000,162.236000) -- 
		(-36.219000,162.418000) -- 
		(-36.076000,162.572000) -- 
		(-36.063000,162.682000) -- 
		(-36.063000,162.865000) -- 
		(-36.063000,162.902000) -- 
		(-36.122000,163.001000) -- 
		(-36.304000,163.100000) -- 
		(-36.447000,163.314000) -- 
		(-36.532000,163.358000) -- 
		(-36.776000,163.539000) -- 
		(-36.864000,163.605000) -- 
		(-36.959000,163.842000) -- 
		(-36.994000,163.930000) -- 
		(-37.079000,164.018000) -- 
		(-37.339000,164.117000) -- 
		(-37.417000,164.172000) -- 
		(-37.534000,164.375000) -- 
		(-37.716000,164.474000) -- 
		(-37.772000,164.508000) -- 
		(-37.885000,164.578000) -- 
		(-38.005000,164.778000) -- 
		(-38.240000,164.489000) -- 
		(-38.917000,163.664000) -- 
		(-38.963000,163.661000) -- 
		(-39.334000,163.637000) -- 
		(-40.222000,163.291000) -- 
		(-41.565000,162.455000) -- 
		(-48.024000,158.440000) -- 
		(-50.178000,157.102000) -- 
		(-51.611000,156.246000) -- 
		(-52.763000,155.559000) -- 
		(-55.912000,153.684000) -- 
		(-57.346000,152.830000) -- 
		(-57.957000,152.491000) -- 
		(-57.962000,152.486000) -- 
		(-59.756000,151.368000) -- 
		(-60.354000,150.996000) -- 
		(-60.341000,150.892000) -- 
		(-60.361000,150.656000) -- 
		(-60.517000,150.535000) -- 
		(-60.556000,150.436000) -- 
		(-60.439000,149.974000) -- 
		(-60.348000,149.721000) -- 
		(-60.303000,149.550000) -- 
		(-60.335000,149.341000) -- 
		(-60.297000,148.978000) -- 
		(-60.355000,148.665000) -- 
		(-60.414000,148.484000) -- 
		(-60.544000,148.242000) -- 
		(-60.603000,148.181000) -- 
		(-60.655000,148.121000) -- 
		(-60.798000,147.961000) -- 
		(-60.863000,147.769000) -- 
		(-61.006000,147.434000) -- 
		(-61.084000,147.142000) -- 
		(-61.072000,146.565000) -- 
		(-61.091000,146.295000) -- 
		(-61.033000,145.976000) -- 
		(-61.046000,145.800000) -- 
		(-60.741000,145.564000) -- 
		(-60.507000,145.283000) -- 
		(-60.364000,145.091000) -- 
		(-60.286000,144.794000) -- 
		(-60.266000,144.623000) -- 
		(-60.189000,144.370000) -- 
		(-59.883000,143.991000) -- 
		(-59.747000,143.864000) -- 
		(-59.656000,143.749000) -- 
		(-59.454000,143.589000) -- 
		(-59.415000,143.540000) -- 
		(-59.220000,143.336000) -- 
		(-58.999000,143.188000) -- 
		(-58.752000,142.847000) -- 
		(-58.408000,142.649000) -- 
		(-57.985000,142.385000) -- 
		(-57.855000,142.330000) -- 
		(-57.699000,142.181000) -- 
		(-57.569000,142.005000) -- 
		(-57.433000,141.807000) -- 
		(-57.336000,141.532000) -- 
		(-57.323000,141.444000) -- 
		(-57.206000,141.229000) -- 
		(-57.115000,141.158000) -- 
		(-56.959000,140.833000) -- 
		(-56.913000,140.696000) -- 
		(-56.887000,140.661000) -- 
		(-56.822000,140.575000) -- 
		(-56.744000,140.382000) -- 
		(-56.706000,140.245000) -- 
		(-56.556000,140.047000) -- 
		(-56.491000,139.909000) -- 
		(-56.433000,139.816000) -- 
		(-56.335000,139.508000) -- 
		(-56.186000,139.222000) -- 
		(-56.102000,139.073000) -- 
		(-55.926000,138.458000) -- 
		(-55.803000,138.150000) -- 
		(-55.725000,137.924000) -- 
		(-55.674000,137.843000) -- 
		(-55.920000,137.808000) -- 
		(-56.370000,137.787000) -- 
		(-58.680000,137.801000) -- 
		(-60.489000,137.815000) -- 
		(-61.853000,137.801000) -- 
		(-62.062000,137.803000) -- 
		(-64.265000,137.821000) -- 
		(-66.807000,137.805000) -- 
		(-67.707000,137.803000) -- 
		(-68.580000,137.803000) -- 
		(-68.911000,137.747000) -- 
		(-69.166000,137.689000) -- 
		(-69.681000,137.501000) -- 
		(-70.624000,137.195000) -- 
		(-70.649000,137.185000) -- 
		(-70.727000,137.161000) -- 
		(-70.754000,137.153000) -- 
		(-70.901000,137.105000) -- 
		(-71.084000,137.047000) -- 
		(-71.332000,137.019000) -- 
		(-71.570000,136.993000) -- 
		(-73.122000,136.943000) -- 
		(-73.719000,136.926000) -- 
		(-74.070000,136.965000) -- 
		(-74.605000,137.052000) -- 
		(-75.077000,137.198000) -- 
		(-75.547000,137.379000) -- 
		(-75.557000,137.384000) -- 
		(-75.708000,137.416000) -- 
		(-76.478000,137.395000) -- 
		(-78.622000,137.355000) -- 
		(-79.527000,137.345000) -- 
		(-81.225000,137.347000) -- 
		(-81.892000,137.348000) -- 
		(-83.120000,137.337000) -- 
		(-83.293000,137.254000) -- 
		(-83.307000,137.248000) -- 
		(-83.485000,137.206000) -- 
		(-83.812000,137.201000) -- 
		(-83.864000,137.207000) -- 
		(-83.956000,137.220000) -- 
		(-84.020000,137.297000) -- 
		(-84.035000,137.334000) -- 
		(-84.057000,137.388000) -- 
		(-84.047000,137.457000) -- 
		(-84.042000,137.499000) -- 
		(-85.921000,137.412000) -- 
		(-86.246000,137.343000) -- 
		(-86.725000,137.330000) -- 
		(-86.506000,138.929000) -- 
		(-86.490000,139.002000) -- 
		(-85.202000,144.963000) -- 
		(-83.920000,150.923000) -- 
		(-83.770000,151.617000) -- 
		(-83.621000,152.311000) -- 
		(-82.901000,156.290000) -- 
		(-82.147000,161.939000) -- 
		(-82.111000,162.212000) -- 
		(-82.151000,162.815000) -- 
		(-82.167000,163.054000) -- 
		(-82.183000,163.293000) -- 
		(-82.278000,164.714000) -- 
		(-82.373000,166.133000) -- 
		(-82.518000,168.325000) -- 
		(-83.239000,170.993000) -- 
		(-83.588000,171.365000) -- 
		(-83.592000,171.369000) -- 
		(-85.438000,173.304000) -- 
		(-85.728000,173.529000) -- 
		(-87.129000,174.641000) -- 
		(-89.754000,175.358000) -- 
		(-92.879000,176.387000) -- 
		(-95.638000,176.864000) -- 
		(-96.172000,176.875000) -- 
		(-96.902000,176.882000) -- 
		(-97.141000,176.889000) -- 
		(-97.629000,176.902000) -- 
		(-97.834000,176.906000) -- 
		(-98.621000,176.858000) -- 
		(-99.258000,176.819000) -- 
		(-100.464000,176.715000) -- 
		(-102.562000,176.160000) -- 
		(-105.493000,174.776000) -- 
		(-107.230000,173.689000) -- 
		(-107.823000,173.149000) -- 
		(-111.079000,170.200000) -- 
		(-112.929000,168.169000) -- 
		(-113.629000,166.172000) -- 
		(-114.435000,164.478000) -- 
		(-116.534000,162.470000) -- 
		(-119.091000,160.869000) -- 
		(-119.237000,160.865000) -- 
		(-119.634000,160.872000) -- 
		(-121.577000,161.341000) -- 
		(-123.480000,162.380000) -- 
		(-124.053000,162.705000) -- 
		(-124.083000,162.722000) -- 
		(-124.606000,163.034000) -- 
		(-126.268000,166.593000) -- 
		(-126.112000,167.289000) -- 
		(-126.000000,167.930000) -- 
		(-125.966000,168.127000) -- 
		(-125.374000,169.761000) -- 
		(-124.902000,171.052000) -- 
		(-124.404000,171.966000) -- 
		(-123.619000,173.407000) -- 
		(-122.569000,175.243000) -- 
		(-121.672000,177.210000) -- 
		(-121.545000,178.284000) -- 
		(-121.449000,180.328000) -- 
		(-120.466000,182.961000) -- 
		(-119.169000,186.002000) -- 
		(-117.759000,188.070000) -- 
		(-116.087000,189.985000) -- 
		(-114.929000,191.079000) -- 
		(-114.926000,191.080000) -- 
		(-114.925000,191.081000) -- 
		(-114.451000,191.507000) -- 
		(-114.121000,191.922000) -- 
		(-110.708000,194.817000) -- 
		(-109.192000,196.102000) -- 
		(-105.658000,198.846000) -- 
		(-103.779000,200.175000) -- 
		(-103.410000,200.417000) -- 
		(-103.158000,200.636000) -- 
		(-102.432000,201.058000) -- 
		(-102.075000,201.305000) -- 
		(-101.615000,201.694000) -- 
		(-101.032000,202.102000) -- 
		(-99.571000,202.958000) -- 
		(-98.214000,203.749000) -- 
		(-96.981000,204.548000) -- 
		(-96.211000,205.119000) -- 
		(-95.062000,205.853000) -- 
		(-94.498000,206.251000) -- 
		(-93.342000,207.122000) -- 
		(-92.920000,207.544000) -- 
		(-92.124000,208.339000) -- 
		(-91.111000,209.273000) -- 
		(-90.702000,209.904000) -- 
		(-90.044000,211.083000) -- 
		(-89.297000,212.267000) -- 
		(-88.818000,213.480000) -- 
		(-88.575000,213.821000) -- 
		(-88.216000,214.090000) -- 
		(-88.058000,214.333000) -- 
		(-87.919000,214.633000) -- 
		(-87.657000,215.054000) -- 
		(-87.311000,215.517000) -- 
		(-86.759000,216.247000) -- 
		(-86.236000,217.051000) -- 
		(-85.854000,217.747000) -- 
		(-85.395000,218.364000) -- 
		(-85.056000,219.423000) -- 
		(-84.777000,220.659000) -- 
		(-84.609000,221.684000) -- 
		(-84.632000,222.773000) -- 
		(-84.684000,223.949000) -- 
		(-84.516000,224.997000) -- 
		(-84.487000,225.756000) -- 
		(-84.545000,226.562000) -- 
		(-84.395000,227.507000) -- 
		(-83.994000,228.937000) -- 
		(-83.718000,230.447000) -- 
		(-83.463000,231.441000) -- 
		(-83.199000,233.127000) -- 
		(-83.233000,234.360000) -- 
		(-83.384000,235.134000) -- 
		(-83.462000,235.532000) -- 
		(-83.539000,235.929000) -- 
		(-83.555000,236.011000) -- 
		(-83.571000,236.091000) -- 
		(-83.646000,236.398000) -- 
		(-83.771000,236.852000) -- 
		(-84.139000,237.491000) -- 
		(-84.764000,238.489000) -- 
		(-84.933000,238.785000) -- 
		(-85.045000,238.980000) -- 
		(-85.363000,239.377000) -- 
		(-85.849000,240.072000) -- 
		(-86.235000,240.709000) -- 
		(-86.603000,241.831000) -- 
		(-86.871000,242.568000) -- 
		(-87.006000,243.519000) -- 
		(-87.156000,243.944000) -- 
		(-87.290000,244.313000) -- 
		(-87.575000,244.809000) -- 
		(-88.196000,245.703000) -- 
		(-88.799000,246.242000) -- 
		(-89.218000,246.639000) -- 
		(-89.788000,247.178000) -- 
		(-91.011000,248.085000) -- 
		(-92.218000,248.724000) -- 
		(-93.039000,249.079000) -- 
		(-94.296000,249.462000) -- 
		(-95.604000,249.604000) -- 
		(-96.475000,249.759000) -- 
		(-96.811000,249.773000) -- 
		(-97.447000,249.716000) -- 
		(-99.107000,249.461000) -- 
		(-100.263000,249.064000) -- 
		(-101.353000,248.539000) -- 
		(-102.739000,247.475000) -- 
		(-103.230000,247.078000) -- 
		(-103.967000,246.341000) -- 
		(-104.357000,245.927000) -- 
		(-105.425000,244.793000) -- 
		(-107.199000,242.961000) -- 
		(-107.410000,242.752000) -- 
		(-107.648000,242.516000) -- 
		(-108.114000,242.050000) -- 
		(-108.125000,242.039000) -- 
		(-109.275000,240.722000) -- 
		(-109.940000,239.886000) -- 
		(-110.636000,239.211000) -- 
		(-111.255000,238.504000) -- 
		(-111.775000,237.506000) -- 
		(-112.865000,235.603000) -- 
		(-113.595000,234.636000) -- 
		(-114.608000,233.482000) -- 
		(-115.468000,232.933000) -- 
		(-116.121000,232.646000) -- 
		(-116.831000,232.333000) -- 
		(-117.829000,232.218000) -- 
		(-118.628000,232.228000) -- 
		(-119.859000,232.401000) -- 
		(-120.710000,232.724000) -- 
		(-121.698000,233.327000) -- 
		(-122.316000,233.999000) -- 
		(-122.920000,235.033000) -- 
		(-123.400000,236.133000) -- 
		(-123.819000,237.588000) -- 
		(-123.970000,238.964000) -- 
		(-124.275000,240.500000) -- 
		(-124.304000,241.998000) -- 
		(-124.253000,242.451000) -- 
		(-123.750000,244.060000) -- 
		(-122.929000,246.107000) -- 
		(-122.041000,247.389000) -- 
		(-121.112000,248.446000) -- 
		(-119.831000,249.506000) -- 
		(-118.915000,250.103000) -- 
		(-116.949000,251.169000) -- 
		(-115.707000,251.608000) -- 
		(-113.157000,252.503000) -- 
		(-111.080000,253.151000) -- 
		(-108.626000,253.928000) -- 
		(-108.620000,253.930000) -- 
		(-108.463000,253.981000) -- 
		(-106.936000,254.463000) -- 
		(-106.169000,254.817000) -- 
		(-105.646000,254.927000) -- 
		(-105.075000,255.021000) -- 
		(-104.725000,255.079000) -- 
		(-104.174000,255.238000) -- 
		(-103.161000,255.675000) -- 
		(-102.435000,255.923000) -- 
		(-101.663000,256.189000) -- 
		(-99.787000,256.932000) -- 
		(-99.505000,257.073000) -- 
		(-98.862000,257.395000) -- 
		(-98.491000,257.580000) -- 
		(-97.483000,258.467000) -- 
		(-95.681000,260.499000) -- 
		(-94.335000,262.503000) -- 
		(-94.056000,263.038000) -- 
		(-93.262000,264.295000) -- 
		(-92.845000,265.210000) -- 
		(-92.079000,266.361000) -- 
		(-91.670000,267.147000) -- 
		(-91.451000,267.915000) -- 
		(-90.928000,268.765000) -- 
		(-90.237000,270.594000) -- 
		(-89.598000,272.026000) -- 
		(-89.267000,273.020000) -- 
		(-88.812000,274.669000) -- 
		(-88.164000,277.213000) -- 
		(-87.795000,278.660000) -- 
		(-87.149000,281.020000) -- 
		(-86.745000,283.007000) -- 
		(-86.638000,283.450000) -- 
		(-86.066000,285.818000) -- 
		(-86.065000,286.503000) -- 
		(-85.636000,288.796000) -- 
		(-85.349000,291.757000) -- 
		(-85.371000,292.748000) -- 
		(-85.323000,293.418000) -- 
		(-85.143000,295.024000) -- 
		(-85.174000,295.290000) -- 
		(-85.328000,296.110000) -- 
		(-85.343000,297.344000) -- 
		(-85.303000,298.681000) -- 
		(-85.334000,299.680000) -- 
		(-85.396000,300.793000) -- 
		(-85.579000,301.726000) -- 
		(-85.596000,302.355000) -- 
		(-85.624000,303.087000) -- 
		(-85.579000,303.374000) -- 
		(-85.501000,303.870000) -- 
		(-85.414000,304.523000) -- 
		(-85.380000,304.895000) -- 
		(-85.424000,305.670000) -- 
		(-85.427000,305.725000) -- 
		(-85.430000,305.781000) -- 
		(-85.425000,306.368000) -- 
		(-85.391000,306.523000) -- 
		(-85.350000,307.014000) -- 
		(-85.181000,307.353000) -- 
		(-85.322000,307.959000) -- 
		(-85.421000,308.562000) -- 
		(-85.616000,309.192000) -- 
		(-85.605000,309.250000) -- 
		(-85.977000,309.912000) -- 
		(-86.627000,310.774000) -- 
		(-87.014000,311.274000) -- 
		(-87.502000,311.504000) -- 
		(-88.646000,311.808000) -- 
		(-89.157000,311.743000) -- 
		(-89.632000,311.588000) -- 
		(-89.942000,311.473000) -- 
		(-90.134000,311.336000) -- 
		(-91.093000,310.655000) -- 
		(-91.454000,310.304000) -- 
		(-92.177000,309.606000) -- 
		(-92.899000,308.907000) -- 
		(-93.184000,308.400000) -- 
		(-93.543000,307.581000) -- 
		(-94.462000,305.804000) -- 
		(-94.836000,305.019000) -- 
		(-94.885000,304.917000) -- 
		(-95.931000,302.728000) -- 
		(-96.672000,300.958000) -- 
		(-97.554000,298.824000) -- 
		(-97.961000,298.098000) -- 
		(-98.191000,297.807000) -- 
		(-98.222000,297.768000) -- 
		(-98.340000,297.620000) -- 
		(-98.950000,297.208000) -- 
		(-99.553000,296.880000) -- 
		(-100.175000,296.756000) -- 
		(-101.289000,296.874000) -- 
		(-101.385000,296.884000) -- 
		(-101.724000,296.918000) -- 
		(-104.708000,297.560000) -- 
		(-105.349000,297.714000) -- 
		(-106.068000,297.974000) -- 
		(-106.539000,298.142000) -- 
		(-106.611000,298.175000) -- 
		(-106.959000,298.336000) -- 
		(-107.570000,298.484000) -- 
		(-108.222000,298.605000) -- 
		(-108.349000,298.629000) -- 
		(-108.479000,298.654000) -- 
		(-108.548000,298.676000) -- 
		(-108.646000,298.708000) -- 
		(-109.242000,298.905000) -- 
		(-109.486000,298.987000) -- 
		(-109.799000,299.091000) -- 
		(-109.956000,299.144000) -- 
		(-109.989000,299.156000) -- 
		(-111.521000,299.679000) -- 
		(-112.144000,300.005000) -- 
		(-113.671000,300.908000) -- 
		(-114.006000,301.132000) -- 
		(-114.747000,301.593000) -- 
		(-114.882000,301.720000) -- 
		(-115.021000,302.137000) -- 
		(-114.998000,302.566000) -- 
		(-114.755000,302.905000) -- 
		(-114.580000,303.091000) -- 
		(-114.078000,303.630000) -- 
		(-113.571000,304.183000) -- 
		(-113.165000,304.501000) -- 
		(-112.777000,304.666000) -- 
		(-112.038000,305.027000) -- 
		(-111.016000,305.584000) -- 
		(-109.805000,306.217000) -- 
		(-107.997000,307.137000) -- 
		(-106.749000,307.906000) -- 
		(-106.124000,308.193000) -- 
		(-105.539000,308.643000) -- 
		(-105.143000,308.921000) -- 
		(-104.402000,309.202000) -- 
		(-103.589000,309.724000) -- 
		(-103.088000,310.107000) -- 
		(-102.824000,310.263000) -- 
		(-102.294000,310.695000) -- 
		(-101.892000,311.246000) -- 
		(-101.519000,312.136000) -- 
		(-101.493000,312.215000) -- 
		(-101.103000,313.190000) -- 
		(-100.994000,313.566000) -- 
		(-100.848000,314.066000) -- 
		(-100.695000,314.589000) -- 
		(-100.691000,314.607000) -- 
		(-100.421000,315.800000) -- 
		(-100.360000,316.093000) -- 
		(-100.125000,317.261000) -- 
		(-100.071000,318.260000) -- 
		(-100.196000,319.335000) -- 
		(-100.416000,320.201000) -- 
		(-100.707000,320.770000) -- 
		(-101.147000,321.361000) -- 
		(-101.771000,321.848000) -- 
		(-102.992000,322.611000) -- 
		(-104.987000,323.780000) -- 
		(-106.132000,324.204000) -- 
		(-107.417000,324.707000) -- 
		(-108.257000,324.770000) -- 
		(-109.192000,324.758000) -- 
		(-109.996000,324.657000) -- 
		(-110.610000,324.652000) -- 
		(-111.119000,324.658000) -- 
		(-111.647000,324.622000) -- 
		(-112.117000,324.553000) -- 
		(-112.711000,324.482000) -- 
		(-112.790000,324.493000) -- 
		(-113.193000,324.553000) -- 
		(-113.825000,324.893000) -- 
		(-114.029000,325.149000) -- 
		(-114.464000,325.745000) -- 
		(-115.245000,326.815000) -- 
		(-115.771000,327.497000) -- 
		(-117.604000,329.604000) -- 
		(-119.055000,331.687000) -- 
		(-120.313000,334.340000) -- 
		(-120.900000,337.706000) -- 
		(-121.061000,340.276000) -- 
		(-120.694000,342.603000) -- 
		(-120.672000,342.744000) -- 
		(-120.665000,342.823000) -- 
		(-120.658000,342.904000) -- 
		(-120.463000,345.155000) -- 
		(-120.487000,346.549000) -- 
		(-120.677000,347.365000) -- 
		(-121.177000,348.684000) -- 
		(-121.504000,349.707000) -- 
		(-121.512000,350.489000) -- 
		(-121.345000,350.982000) -- 
		(-120.850000,351.971000) -- 
		(-119.593000,353.973000) -- 
		(-118.376000,356.115000) -- 
		(-118.043000,356.857000) -- 
		(-117.434000,358.220000) -- 
		(-116.421000,359.939000) -- 
		(-116.094000,360.545000) -- 
		(-115.390000,361.510000) -- 
		(-114.617000,362.494000) -- 
		(-113.826000,363.370000) -- 
		(-113.102000,364.402000) -- 
		(-111.934000,365.993000) -- 
		(-111.876000,366.048000) -- 
		(-110.153000,369.172000) -- 
		(-108.717000,372.088000) -- 
		(-107.588000,373.845000) -- 
		(-107.634000,374.066000) -- 
		(-107.624000,375.039000) -- 
		(-107.350000,375.379000) -- 
		(-106.717000,375.991000) -- 
		(-104.004000,377.812000) -- 
		(-103.521000,378.191000) -- 
		(-102.707000,379.131000) -- 
		(-102.040000,380.039000) -- 
		(-101.855000,381.129000) -- 
		(-101.967000,381.806000) -- 
		(-102.416000,382.701000) -- 
		(-102.578000,382.974000) -- 
		(-103.160000,384.120000) -- 
		(-104.117000,385.464000) -- 
		(-105.622000,386.887000) -- 
		(-107.520000,388.358000) -- 
		(-110.292000,390.163000) -- 
		(-111.217000,390.908000) -- 
		(-112.310000,391.583000) -- 
		(-114.073000,392.556000) -- 
		(-115.989000,393.336000) -- 
		(-119.680000,394.679000) -- 
		(-122.544000,396.091000) -- 
		(-123.570000,396.394000) -- 
		(-124.041000,396.469000) -- 
		(-125.167000,396.528000) -- 
		(-126.177000,396.401000) -- 
		(-126.935000,396.403000) -- 
		(-129.268000,396.413000) -- 
		(-137.798000,396.347000) -- 
		(-139.039000,396.314000) -- 
		(-139.772000,396.211000) -- 
		(-140.446000,396.028000) -- 
		(-141.050000,396.059000) -- 
		(-141.746000,396.293000) -- 
		(-142.597000,396.764000) -- 
		(-143.699000,397.832000) -- 
		(-143.934000,398.073000) -- 
		(-144.099000,398.241000) -- 
		(-144.546000,398.694000) -- 
		(-144.982000,399.250000) -- 
		(-145.148000,399.879000) -- 
		(-145.224000,400.428000) -- 
		(-144.864000,401.425000) -- 
		(-144.231000,402.318000) -- 
		(-143.803000,402.843000) -- 
		(-142.685000,403.449000) -- 
		(-141.696000,403.882000) -- 
		(-140.636000,404.233000) -- 
		(-140.191000,404.347000) -- 
		(-139.433000,404.448000) -- 
		(-138.721000,404.608000) -- 
		(-137.472000,404.886000) -- 
		(-135.924000,405.194000) -- 
		(-135.394000,405.286000) -- 
		(-134.350000,405.569000) -- 
		(-133.485000,405.832000) -- 
		(-132.616000,406.260000) -- 
		(-132.109000,406.663000) -- 
		(-131.804000,407.263000) -- 
		(-131.563000,408.975000) -- 
		(-130.868000,411.393000) -- 
		(-129.672000,413.602000) -- 
		(-129.087000,414.403000) -- 
		(-128.635000,414.712000) -- 
		(-128.366000,414.896000) -- 
		(-127.451000,415.759000) -- 
		(-126.597000,416.734000) -- 
		(-126.077000,416.876000) -- 
		(-123.890000,416.130000) -- 
		(-123.200000,415.924000) -- 
		(-122.510000,415.718000) -- 
		(-121.234000,415.549000) -- 
		(-120.257000,415.487000) -- 
		(-119.818000,415.509000) -- 
		(-119.664000,415.516000) -- 
		(-119.508000,415.531000) -- 
		(-118.586000,415.279000) -- 
		(-117.002000,414.938000) -- 
		(-115.748000,414.539000) -- 
		(-115.610000,414.496000) -- 
		(-114.900000,414.053000) -- 
		(-113.978000,413.435000) -- 
		(-112.081000,412.205000) -- 
		(-111.263000,411.805000) -- 
		(-109.073000,411.383000) -- 
		(-107.514000,411.447000) -- 
		(-105.626000,411.948000) -- 
		(-104.092000,412.626000) -- 
		(-102.428000,413.923000) -- 
		(-100.924000,415.620000) -- 
		(-100.430000,416.724000) -- 
		(-100.115000,418.302000) -- 
		(-100.157000,419.853000) -- 
		(-100.492000,420.765000) -- 
		(-102.280000,423.244000) -- 
		(-103.067000,423.765000) -- 
		(-104.295000,423.917000) -- 
		(-105.313000,423.807000) -- 
		(-105.794000,423.598000) -- 
		(-106.813000,423.152000) -- 
		(-107.998000,422.496000) -- 
		(-109.180000,421.851000) -- 
		(-110.626000,420.916000) -- 
		(-111.390000,420.308000) -- 
		(-111.874000,419.934000) -- 
		(-112.198000,419.668000) -- 
		(-112.202000,419.665000) -- 
		(-113.843000,418.791000) -- 
		(-116.824000,417.753000) -- 
		(-119.668000,417.231000) -- 
		(-121.350000,417.129000) -- 
		(-122.527000,417.191000) -- 
		(-122.759000,417.350000) -- 
		(-123.241000,417.685000) -- 
		(-123.332000,418.085000) -- 
		(-122.027000,419.168000) -- 
		(-120.490000,420.881000) -- 
		(-119.232000,422.883000) -- 
		(-118.295000,424.487000) -- 
		(-117.689000,425.814000) -- 
		(-117.545000,426.630000) -- 
		(-117.459000,427.535000) -- 
		(-117.518000,428.527000) -- 
		(-117.881000,430.339000) -- 
		(-118.578000,432.165000) -- 
		(-118.826000,432.985000) -- 
		(-118.902000,433.479000) -- 
		(-119.011000,434.186000) -- 
		(-119.526000,435.932000) -- 
		(-119.866000,436.956000) -- 
		(-120.530000,439.331000) -- 
		(-120.532000,439.335000) -- 
		(-120.647000,439.461000) -- 
		(-120.741000,439.597000) -- 
		(-120.838000,439.702000) -- 
		(-120.945000,439.817000) -- 
		(-120.999000,439.902000) -- 
		(-121.054000,440.000000) -- 
		(-121.073000,440.071000) -- 
		(-122.978000,443.643000) -- 
		(-124.768000,450.101000) -- 
		(-124.771000,450.127000) -- 
		(-124.764000,450.151000) -- 
		(-124.444000,451.095000) -- 
		(-124.123000,452.040000) -- 
		(-124.221000,452.150000) -- 
		(-124.319000,452.259000) -- 
		(-125.340000,453.415000) -- 
		(-125.441000,453.529000) -- 
		(-125.464000,453.583000) -- 
		(-125.471000,453.601000) -- 
		(-127.270000,457.981000) -- 
		(-127.284000,458.864000) -- 
		(-127.236000,459.667000) -- 
		(-127.157000,459.813000) -- 
		(-127.071000,459.895000) -- 
		(-126.748000,460.435000) -- 
		(-126.727000,460.469000) -- 
		(-126.726000,460.470000) -- 
		(-126.082000,461.488000) -- 
		(-125.719000,461.985000) -- 
		(-125.228000,462.329000) -- 
		(-124.771000,462.648000) -- 
		(-124.735000,462.666000) -- 
		(-124.582000,462.742000) -- 
		(-124.525000,462.773000) -- 
		(-124.468000,462.804000) -- 
		(-124.232000,462.926000) -- 
		(-123.997000,463.049000) -- 
		(-123.899000,463.092000) -- 
		(-123.466000,463.284000) -- 
		(-121.811000,463.714000) -- 
		(-119.301000,464.161000) -- 
		(-118.724000,464.626000) -- 
		(-118.088000,465.169000) -- 
		(-117.259000,465.842000) -- 
		(-116.269000,466.765000) -- 
		(-115.717000,467.576000) -- 
		(-115.164000,468.489000) -- 
		(-114.553000,469.371000) -- 
		(-113.095000,469.368000) -- 
		(-108.721000,469.361000) -- 
		(-107.263000,469.359000) -- 
		(-107.127000,469.358000) -- 
		(-106.719000,469.358000) -- 
		(-106.583000,469.358000) -- 
		(-106.477000,469.357000) -- 
		(-106.162000,469.356000) -- 
		(-106.058000,469.358000) -- 
		(-105.939000,469.355000) -- 
		(-105.588000,469.356000) -- 
		(-105.472000,469.358000) -- 
		(-105.363000,469.357000) -- 
		(-105.038000,469.357000) -- 
		(-104.930000,469.357000) -- 
		(-104.817000,469.356000) -- 
		(-104.649000,469.356000) -- 
		(-104.480000,469.356000) -- 
		(-104.368000,469.356000) -- 
		(-103.812000,469.355000) -- 
		(-103.255000,469.354000) -- 
		(-102.822000,469.353000) -- 
		(-99.431000,469.348000) -- 
		(-97.031000,469.347000) -- 
		(-86.933000,469.344000) -- 
		(-83.955000,469.343000) -- 
		(-82.213000,469.342000) -- 
		(-73.149000,469.340000) -- 
		(-73.148000,469.340000) -- 
		(-73.065000,469.339000) -- 
		(-65.480000,469.330000) -- 
		(-65.479000,469.329000) -- 
		(-63.609000,469.328000) -- 
		(-63.302000,469.327000) -- 
		(-61.537000,469.326000) -- 
		(-61.235000,469.325000) -- 
		(-57.479000,469.321000) -- 
		(-57.235000,469.320000) -- 
		(-54.928000,469.317000) -- 
		(-50.176000,469.311000) -- 
		(-45.799000,469.491000) -- 
		(-33.992000,469.472000) -- 
		(-23.273000,469.456000) -- 
		(-23.128000,469.456000) -- 
		(-17.283000,469.420000) -- 
		(-17.282000,469.419000) -- 
		(-15.975000,469.413000) -- 
		(-15.648000,469.412000) -- 
		(-10.023000,469.398000) -- 
		(-5.112000,469.385000) -- 
		(-5.022000,469.386000) -- 
		(9.379000,469.350000) -- 
		(12.421000,469.343000) -- 
		(12.639000,469.342000) -- 
		(13.205000,469.340000) -- 
		(23.385000,469.316000) -- 
		(23.743000,469.316000) -- 
		(24.652000,469.314000) -- 
		(25.059000,469.315000) -- 
		(25.122000,469.314000) -- 
		(28.776000,469.305000) -- 
		(33.946000,469.293000) -- 
		(36.900000,469.286000) -- 
		(37.117000,469.286000) -- 
		(37.359000,469.285000) -- 
		(37.963000,469.284000) -- 
		(38.293000,469.283000) -- 
		(41.604000,469.276000) -- 
		(42.039000,469.275000) -- 
		(46.484000,469.265000) -- 
		(46.485000,469.264000) -- 
		(59.519000,469.234000) -- 
		(61.614000,469.229000) -- 
		(61.815000,469.229000) -- 
		(61.816000,469.229000) -- 
		(67.443000,469.215000) -- 
		(67.444000,469.215000) -- 
		(71.926000,469.204000) -- 
		(72.067000,469.203000) -- 
		(74.409000,469.200000) -- 
		(74.684000,469.199000) -- 
		(74.836000,469.199000) -- 
		(81.415000,469.160000) -- 
		(89.138000,469.115000) -- 
		(100.064000,469.049000) -- 
		(100.274000,469.048000) -- 
		(100.425000,469.047000) -- 
		(100.962000,469.043000) -- 
		(105.655000,469.014000) -- 
		(123.016000,468.908000) -- 
		(123.347000,468.906000) -- 
		(124.131000,468.902000) -- 
		(124.469000,468.899000) -- 
		(124.769000,468.898000) -- 
		(124.991000,468.896000) -- 
		(128.128000,468.878000) -- 
		(131.565000,468.857000) -- 
		(131.838000,468.856000) -- 
		(132.265000,468.853000) -- 
		(136.594000,468.829000) -- 
		(154.271000,468.727000) -- 
		(154.434000,468.727000) -- 
		(154.776000,468.724000) -- 
		(154.826000,468.725000) -- 
		(154.922000,468.725000) -- 
		(167.413000,468.825000) -- 
		(168.502000,468.829000) -- 
		(168.842000,468.833000) -- 
		(174.590000,468.637000) -- 
		(175.084000,468.637000) -- 
		(178.221000,468.651000) -- 
		(178.642000,468.632000) -- 
		(178.985000,468.616000) -- 
		(179.006000,468.616000) -- 
		(179.013000,468.517000) -- 
		(179.020000,468.422000) -- 
		(179.020000,468.243000) -- 
		(179.000000,468.040000) -- 
		(178.874000,467.761000) -- 
		(178.842000,467.673000) -- 
		(178.826000,467.488000) -- 
		(178.851000,467.412000) -- 
		(178.908000,467.344000) -- 
		(179.071000,467.206000) -- 
		(179.137000,467.186000) -- 
		(179.362000,467.162000) -- 
		(179.627000,467.090000) -- 
		(179.710000,467.021000) -- 
		(179.725000,466.897000) -- 
		(179.676000,466.852000) -- 
		(179.522000,466.733000) -- 
		(179.383000,466.612000) -- 
		(179.359000,466.581000) -- 
		(179.350000,466.547000) -- 
		(179.326000,466.519000) -- 
		(179.302000,466.478000) -- 
		(179.284000,466.387000) -- 
		(179.303000,466.140000) -- 
		(179.385000,465.990000) -- 
		(179.401000,465.969000) -- 
		(179.663000,465.737000) -- 
		(179.905000,465.538000) -- 
		(179.973000,465.504000) -- 
		(180.039000,465.484000) -- 
		(180.219000,465.470000) -- 
		(180.259000,465.449000) -- 
		(180.390000,465.415000) -- 
		(180.713000,465.419000) -- 
		(180.766000,465.401000) -- 
		(180.831000,465.442000) -- 
		(180.896000,465.478000) -- 
		(180.937000,465.490000) -- 
		(181.011000,465.532000) -- 
		(181.084000,465.552000) -- 
		(181.166000,465.532000) -- 
		(181.265000,465.465000) -- 
		(181.297000,465.458000) -- 
		(181.329000,465.437000) -- 
		(181.378000,465.424000) -- 
		(181.419000,465.396000) -- 
		(181.649000,465.335000) -- 
		(181.718000,465.286000) -- 
		(181.821000,465.203000) -- 
		(181.795000,465.169000) -- 
		(181.804000,465.135000) -- 
		(181.805000,465.047000) -- 
		(181.821000,464.971000) -- 
		(181.853000,464.936000) -- 
		(182.001000,464.812000) -- 
		(182.098000,464.785000) -- 
		(182.244000,464.754000) -- 
		(182.385000,464.724000) -- 
		(182.540000,464.645000) -- 
		(182.720000,464.525000) -- 
		(182.740000,464.485000) -- 
		(182.770000,464.449000) -- 
		(182.802000,464.367000) -- 
		(182.835000,464.263000) -- 
		(182.852000,464.148000) -- 
		(182.876000,464.113000) -- 
		(183.015000,463.996000) -- 
		(183.204000,464.018000) -- 
		(183.269000,464.059000) -- 
		(183.395000,464.076000) -- 
		(183.530000,464.060000) -- 
		(183.555000,464.038000) -- 
		(183.587000,464.032000) -- 
		(183.759000,463.945000) -- 
		(183.987000,463.806000) -- 
		(184.397000,463.489000) -- 
		(184.520000,463.182000) -- 
		(184.545000,463.098000) -- 
		(184.569000,463.078000) -- 
		(184.610000,462.988000) -- 
		(184.655000,462.928000) -- 
		(184.872000,462.728000) -- 
		(185.028000,462.597000) -- 
		(185.105000,462.544000) -- 
		(185.281000,462.448000) -- 
		(185.289000,462.427000) -- 
		(185.338000,462.358000) -- 
		(185.305000,462.317000) -- 
		(185.290000,462.276000) -- 
		(185.265000,462.248000) -- 
		(185.241000,462.206000) -- 
		(185.208000,462.130000) -- 
		(185.188000,462.011000) -- 
		(185.213000,461.825000) -- 
		(185.240000,461.709000) -- 
		(185.246000,461.681000) -- 
		(185.296000,461.386000) -- 
		(185.324000,461.121000) -- 
		(185.325000,461.060000) -- 
		(185.288000,460.815000) -- 
		(185.227000,460.674000) -- 
		(185.170000,460.523000) -- 
		(185.146000,460.495000) -- 
		(185.130000,460.461000) -- 
		(185.105000,460.433000) -- 
		(185.081000,460.385000) -- 
		(184.878000,460.193000) -- 
		(184.331000,459.697000) -- 
		(184.200000,459.607000) -- 
		(184.135000,459.573000) -- 
		(184.086000,459.559000) -- 
		(184.029000,459.559000) -- 
		(183.972000,459.524000) -- 
		(183.834000,459.510000) -- 
		(183.597000,459.517000) -- 
		(183.302000,459.544000) -- 
		(183.131000,459.523000) -- 
		(183.106000,459.509000) -- 
		(183.042000,459.473000) -- 
		(182.854000,459.310000) -- 
		(182.846000,459.282000) -- 
		(182.805000,459.241000) -- 
		(182.764000,459.138000) -- 
		(182.748000,458.986000) -- 
		(182.757000,458.910000) -- 
		(182.810000,458.639000) -- 
		(182.826000,458.578000) -- 
		(182.856000,458.491000) -- 
		(182.893000,458.409000) -- 
		(182.921000,458.333000) -- 
		(182.946000,458.305000) -- 
		(182.970000,458.155000) -- 
		(182.971000,458.080000) -- 
		(182.958000,458.015000) -- 
		(182.874000,457.804000) -- 
		(182.853000,457.711000) -- 
		(182.800000,457.590000) -- 
		(182.780000,457.278000) -- 
		(182.806000,457.168000) -- 
		(182.859000,457.035000) -- 
		(182.900000,456.980000) -- 
		(183.006000,456.863000) -- 
		(183.039000,456.832000) -- 
		(183.096000,456.746000) -- 
		(183.154000,456.622000) -- 
		(183.186000,456.526000) -- 
		(183.203000,456.149000) -- 
		(183.148000,455.564000) -- 
		(183.197000,455.302000) -- 
		(183.247000,455.146000) -- 
		(183.287000,455.070000) -- 
		(183.312000,455.043000) -- 
		(183.353000,454.967000) -- 
		(183.394000,454.925000) -- 
		(183.410000,454.885000) -- 
		(183.451000,454.830000) -- 
		(183.484000,454.799000) -- 
		(183.508000,454.747000) -- 
		(183.578000,454.652000) -- 
		(183.697000,454.514000) -- 
		(183.770000,454.452000) -- 
		(183.771000,454.348000) -- 
		(183.755000,454.286000) -- 
		(183.624000,454.163000) -- 
		(183.501000,454.074000) -- 
		(183.300000,453.903000) -- 
		(183.249000,453.860000) -- 
		(183.258000,453.846000) -- 
		(183.258000,453.784000) -- 
		(183.233000,453.579000) -- 
		(183.242000,453.448000) -- 
		(183.258000,453.372000) -- 
		(183.274000,453.338000) -- 
		(183.291000,453.317000) -- 
		(183.553000,453.081000) -- 
		(183.692000,452.927000) -- 
		(183.692000,452.906000) -- 
		(183.719000,452.870000) -- 
		(183.737000,452.690000) -- 
		(183.700000,452.609000) -- 
		(183.685000,452.541000) -- 
		(183.440000,452.232000) -- 
		(183.383000,452.180000) -- 
		(183.269000,452.032000) -- 
		(183.122000,451.891000) -- 
		(183.058000,451.806000) -- 
		(182.882000,451.599000) -- 
		(182.568000,451.296000) -- 
		(182.512000,451.247000) -- 
		(182.425000,451.187000) -- 
		(182.259000,451.089000) -- 
		(182.210000,451.055000) -- 
		(182.169000,451.047000) -- 
		(182.096000,451.007000) -- 
		(182.071000,451.007000) -- 
		(182.039000,450.985000) -- 
		(181.997000,450.978000) -- 
		(181.957000,450.957000) -- 
		(181.830000,450.940000) -- 
		(181.786000,450.926000) -- 
		(181.655000,450.903000) -- 
		(181.548000,450.867000) -- 
		(181.433000,450.853000) -- 
		(181.271000,450.805000) -- 
		(181.120000,450.782000) -- 
		(180.920000,450.722000) -- 
		(180.834000,450.712000) -- 
		(180.757000,450.687000) -- 
		(180.365000,450.646000) -- 
		(180.265000,450.624000) -- 
		(180.047000,450.563000) -- 
		(180.014000,450.543000) -- 
		(179.956000,450.521000) -- 
		(179.924000,450.501000) -- 
		(179.884000,450.487000) -- 
		(179.824000,450.451000) -- 
		(179.753000,450.446000) -- 
		(179.606000,450.343000) -- 
		(179.597000,450.321000) -- 
		(179.602000,450.278000) -- 
		(179.629000,450.254000) -- 
		(179.651000,450.203000) -- 
		(179.642000,450.135000) -- 
		(179.623000,450.117000) -- 
		(179.492000,450.069000) -- 
		(179.383000,449.993000) -- 
		(179.208000,449.896000) -- 
		(179.125000,449.875000) -- 
		(179.069000,449.841000) -- 
		(179.027000,449.803000) -- 
		(178.882000,449.634000) -- 
		(178.792000,449.428000) -- 
		(178.809000,449.263000) -- 
		(178.825000,449.201000) -- 
		(178.850000,449.153000) -- 
		(178.891000,449.097000) -- 
		(179.112000,448.910000) -- 
		(179.161000,448.851000) -- 
		(179.414000,448.646000) -- 
		(179.627000,448.464000) -- 
		(179.643000,448.438000) -- 
		(179.865000,448.241000) -- 
		(180.012000,448.097000) -- 
		(180.168000,447.863000) -- 
		(180.181000,447.830000) -- 
		(180.193000,447.801000) -- 
		(180.208000,447.698000) -- 
		(180.193000,447.622000) -- 
		(180.115000,447.498000) -- 
		(180.055000,447.424000) -- 
		(179.818000,447.210000) -- 
		(179.680000,447.120000) -- 
		(179.606000,447.093000) -- 
		(179.483000,447.038000) -- 
		(179.312000,446.982000) -- 
		(179.190000,446.913000) -- 
		(179.076000,446.810000) -- 
		(179.003000,446.679000) -- 
		(178.954000,446.623000) -- 
		(178.938000,446.583000) -- 
		(178.914000,446.547000) -- 
		(178.889000,446.487000) -- 
		(178.865000,446.335000) -- 
		(178.824000,446.246000) -- 
		(178.791000,446.204000) -- 
		(178.628000,446.080000) -- 
		(178.556000,446.053000) -- 
		(178.498000,446.018000) -- 
		(178.433000,445.956000) -- 
		(178.360000,445.846000) -- 
		(178.311000,445.737000) -- 
		(178.295000,445.627000) -- 
		(178.295000,445.556000) -- 
		(178.320000,445.331000) -- 
		(178.312000,445.022000) -- 
		(178.322000,444.822000) -- 
		(178.301000,444.715000) -- 
		(178.216000,444.450000) -- 
		(178.111000,444.224000) -- 
		(178.103000,444.196000) -- 
		(178.070000,444.148000) -- 
		(178.029000,444.045000) -- 
		(177.989000,443.969000) -- 
		(177.934000,443.910000) -- 
		(177.625000,443.609000) -- 
		(177.394000,443.401000) -- 
		(177.345000,443.366000) -- 
		(177.211000,443.298000) -- 
		(177.032000,443.219000) -- 
		(176.951000,443.174000) -- 
		(176.870000,443.160000) -- 
		(176.797000,443.133000) -- 
		(176.764000,443.112000) -- 
		(176.724000,443.098000) -- 
		(176.692000,443.079000) -- 
		(176.585000,443.037000) -- 
		(176.530000,443.023000) -- 
		(176.424000,443.023000) -- 
		(176.310000,443.051000) -- 
		(176.281000,443.054000) -- 
		(176.253000,443.071000) -- 
		(176.220000,443.078000) -- 
		(176.131000,443.133000) -- 
		(175.984000,443.241000) -- 
		(175.952000,443.272000) -- 
		(175.936000,443.295000) -- 
		(175.886000,443.351000) -- 
		(175.879000,443.378000) -- 
		(175.851000,443.419000) -- 
		(175.797000,443.460000) -- 
		(175.773000,443.467000) -- 
		(175.741000,443.467000) -- 
		(175.702000,443.738000) -- 
		(175.670000,443.737000) -- 
		(175.570000,443.659000) -- 
		(175.536000,443.624000) -- 
		(175.512000,443.576000) -- 
		(175.485000,443.484000) -- 
		(175.453000,443.415000) -- 
		(175.474000,443.279000) -- 
		(175.461000,443.148000) -- 
		(175.477000,443.079000) -- 
		(175.504000,443.030000) -- 
		(175.536000,442.987000) -- 
		(175.776000,442.764000) -- 
		(175.835000,442.718000) -- 
		(175.860000,442.687000) -- 
		(175.975000,442.603000) -- 
		(175.980000,442.522000) -- 
		(175.972000,442.410000) -- 
		(175.909000,442.239000) -- 
		(175.868000,442.190000) -- 
		(175.833000,442.162000) -- 
		(175.809000,442.148000) -- 
		(175.717000,442.119000) -- 
		(175.546000,442.123000) -- 
		(175.471000,442.138000) -- 
		(175.243000,442.204000) -- 
		(175.145000,442.226000) -- 
		(175.070000,442.225000) -- 
		(174.985000,442.204000) -- 
		(174.729000,442.100000) -- 
		(174.661000,442.058000) -- 
		(174.612000,441.997000) -- 
		(174.496000,441.813000) -- 
		(174.462000,441.772000) -- 
		(174.388000,441.653000) -- 
		(174.347000,441.548000) -- 
		(174.309000,441.350000) -- 
		(174.287000,441.100000) -- 
		(174.295000,441.010000) -- 
		(174.320000,440.835000) -- 
		(174.343000,440.801000) -- 
		(174.370000,440.562000) -- 
		(174.339000,440.364000) -- 
		(174.294000,440.248000) -- 
		(174.135000,439.892000) -- 
		(174.065000,439.751000) -- 
		(173.987000,439.571000) -- 
		(173.969000,439.488000) -- 
		(173.994000,439.388000) -- 
		(174.011000,439.354000) -- 
		(174.109000,439.284000) -- 
		(174.209000,439.381000) -- 
		(174.302000,439.516000) -- 
		(174.334000,439.627000) -- 
		(174.376000,439.732000) -- 
		(174.403000,439.781000) -- 
		(174.519000,439.892000) -- 
		(174.576000,439.906000) -- 
		(174.676000,439.904000) -- 
		(174.834000,439.885000) -- 
		(174.867000,439.863000) -- 
		(174.901000,439.819000) -- 
		(174.892000,439.799000) -- 
		(174.900000,439.695000) -- 
		(174.891000,439.546000) -- 
		(174.865000,439.456000) -- 
		(174.833000,439.395000) -- 
		(174.633000,439.150000) -- 
		(174.583000,439.073000) -- 
		(174.552000,438.997000) -- 
		(174.532000,438.898000) -- 
		(174.508000,438.806000) -- 
		(174.517000,438.744000) -- 
		(174.541000,438.688000) -- 
		(174.724000,438.435000) -- 
		(174.765000,438.388000) -- 
		(174.807000,438.350000) -- 
		(174.816000,438.316000) -- 
		(174.889000,438.237000) -- 
		(175.057000,438.017000) -- 
		(175.098000,437.968000) -- 
		(175.165000,437.848000) -- 
		(175.196000,437.365000) -- 
		(175.184000,437.252000) -- 
		(175.145000,436.904000) -- 
		(175.120000,436.756000) -- 
		(175.067000,436.507000) -- 
		(175.028000,436.393000) -- 
		(174.921000,436.142000) -- 
		(174.895000,436.114000) -- 
		(174.854000,436.037000) -- 
		(174.828000,436.010000) -- 
		(174.804000,435.959000) -- 
		(174.795000,435.931000) -- 
		(174.779000,435.910000) -- 
		(174.759000,435.897000) -- 
		(174.679000,435.814000) -- 
		(174.538000,435.676000) -- 
		(174.267000,435.478000) -- 
		(174.115000,435.389000) -- 
		(173.863000,435.256000) -- 
		(173.789000,435.221000) -- 
		(173.664000,435.186000) -- 
		(173.497000,435.158000) -- 
		(173.381000,435.132000) -- 
		(173.207000,435.061000) -- 
		(173.173000,435.041000) -- 
		(173.031000,434.991000) -- 
		(172.998000,434.986000) -- 
		(172.732000,434.867000) -- 
		(172.675000,434.833000) -- 
		(172.516000,434.769000) -- 
		(172.426000,434.721000) -- 
		(172.401000,434.700000) -- 
		(172.352000,434.686000) -- 
		(172.316000,434.665000) -- 
		(172.085000,434.575000) -- 
		(172.059000,434.575000) -- 
		(171.950000,434.525000) -- 
		(171.395000,434.253000) -- 
		(171.229000,434.205000) -- 
		(171.195000,434.184000) -- 
		(171.046000,434.136000) -- 
		(170.896000,434.094000) -- 
		(170.783000,434.076000) -- 
		(170.625000,434.091000) -- 
		(170.587000,434.101000) -- 
		(170.346000,434.123000) -- 
		(170.202000,434.113000) -- 
		(170.139000,434.095000) -- 
		(170.096000,434.095000) -- 
		(169.974000,433.980000) -- 
		(169.881000,433.859000) -- 
		(169.673000,433.502000) -- 
		(169.606000,433.425000) -- 
		(169.565000,433.356000) -- 
		(169.557000,433.327000) -- 
		(169.532000,433.292000) -- 
		(169.472000,433.159000) -- 
		(169.449000,433.013000) -- 
		(169.449000,432.949000) -- 
		(169.478000,432.778000) -- 
		(169.614000,432.453000) -- 
		(169.585000,432.282000) -- 
		(169.564000,432.237000) -- 
		(169.546000,432.125000) -- 
		(169.547000,432.068000) -- 
		(169.585000,431.883000) -- 
		(169.760000,431.347000) -- 
		(169.835000,431.004000) -- 
		(169.869000,430.901000) -- 
		(169.913000,430.731000) -- 
		(169.927000,430.605000) -- 
		(169.902000,430.528000) -- 
		(169.769000,430.291000) -- 
		(169.645000,430.144000) -- 
		(169.595000,430.054000) -- 
		(169.551000,429.705000) -- 
		(169.515000,429.492000) -- 
		(169.497000,429.428000) -- 
		(169.451000,429.304000) -- 
		(169.385000,428.963000) -- 
		(169.382000,428.912000) -- 
		(169.335000,428.768000) -- 
		(169.323000,428.619000) -- 
		(169.489000,428.531000) -- 
		(169.483000,428.424000) -- 
		(169.499000,428.363000) -- 
		(169.549000,428.274000) -- 
		(169.631000,428.174000) -- 
		(169.770000,428.023000) -- 
		(169.843000,427.953000) -- 
		(169.893000,427.887000) -- 
		(170.147000,427.589000) -- 
		(170.229000,427.446000) -- 
		(170.254000,427.418000) -- 
		(170.369000,427.208000) -- 
		(170.501000,426.936000) -- 
		(170.510000,426.907000) -- 
		(170.542000,426.861000) -- 
		(170.551000,426.832000) -- 
		(170.583000,426.785000) -- 
		(170.617000,426.690000) -- 
		(170.639000,426.642000) -- 
		(170.660000,426.499000) -- 
		(170.654000,426.231000) -- 
		(170.607000,425.972000) -- 
		(170.574000,425.910000) -- 
		(170.506000,425.835000) -- 
		(170.413000,425.787000) -- 
		(170.300000,425.751000) -- 
		(170.069000,425.724000) -- 
		(169.700000,425.692000) -- 
		(169.643000,425.678000) -- 
		(169.578000,425.643000) -- 
		(169.546000,425.609000) -- 
		(169.530000,425.575000) -- 
		(169.530000,425.534000) -- 
		(169.554000,425.466000) -- 
		(169.579000,425.416000) -- 
		(169.595000,425.397000) -- 
		(169.670000,425.326000) -- 
		(169.767000,425.255000) -- 
		(169.923000,425.195000) -- 
		(169.971000,425.181000) -- 
		(170.057000,425.172000) -- 
		(170.333000,425.160000) -- 
		(170.449000,425.180000) -- 
		(170.621000,425.220000) -- 
		(170.710000,425.254000) -- 
		(170.743000,425.275000) -- 
		(170.929000,425.353000) -- 
		(171.098000,425.408000) -- 
		(171.211000,425.484000) -- 
		(171.260000,425.532000) -- 
		(171.283000,425.635000) -- 
		(171.282000,425.703000) -- 
		(171.258000,425.861000) -- 
		(171.205000,426.054000) -- 
		(171.185000,426.172000) -- 
		(171.176000,426.363000) -- 
		(171.179000,426.428000) -- 
		(171.220000,426.522000) -- 
		(171.268000,426.579000) -- 
		(171.502000,426.745000) -- 
		(171.558000,426.793000) -- 
		(171.606000,426.868000) -- 
		(171.662000,427.012000) -- 
		(171.670000,427.067000) -- 
		(171.695000,427.115000) -- 
		(171.767000,427.197000) -- 
		(171.919000,427.342000) -- 
		(172.032000,427.417000) -- 
		(172.138000,427.453000) -- 
		(172.269000,427.454000) -- 
		(172.406000,427.428000) -- 
		(172.488000,427.387000) -- 
		(172.610000,427.340000) -- 
		(172.642000,427.333000) -- 
		(172.741000,427.289000) -- 
		(172.888000,427.211000) -- 
		(173.017000,427.117000) -- 
		(173.230000,426.934000) -- 
		(173.497000,426.679000) -- 
		(173.566000,426.582000) -- 
		(173.677000,426.480000) -- 
		(173.779000,426.426000) -- 
		(173.851000,426.508000) -- 
		(173.875000,426.577000) -- 
		(173.803000,426.900000) -- 
		(173.766000,427.006000) -- 
		(173.758000,427.048000) -- 
		(173.683000,427.163000) -- 
		(173.666000,427.211000) -- 
		(173.650000,427.226000) -- 
		(173.577000,427.346000) -- 
		(173.486000,427.475000) -- 
		(173.461000,427.503000) -- 
		(173.445000,427.537000) -- 
		(173.338000,427.645000) -- 
		(173.249000,427.720000) -- 
		(172.849000,427.991000) -- 
		(172.864000,428.032000) -- 
		(172.897000,428.053000) -- 
		(173.084000,428.047000) -- 
		(173.157000,428.034000) -- 
		(173.279000,427.977000) -- 
		(173.426000,427.873000) -- 
		(173.524000,427.783000) -- 
		(173.655000,427.655000) -- 
		(173.795000,427.403000) -- 
		(173.820000,427.320000) -- 
		(173.853000,427.272000) -- 
		(173.886000,427.178000) -- 
		(173.934000,426.881000) -- 
		(173.959000,426.758000) -- 
		(173.982000,426.453000) -- 
		(173.981000,426.386000) -- 
		(173.928000,426.057000) -- 
		(173.912000,426.017000) -- 
		(173.864000,425.955000) -- 
		(173.840000,425.914000) -- 
		(173.792000,425.851000) -- 
		(173.639000,425.672000) -- 
		(173.623000,425.631000) -- 
		(173.497000,425.534000) -- 
		(173.372000,425.473000) -- 
		(173.324000,425.425000) -- 
		(173.307000,425.389000) -- 
		(173.309000,425.279000) -- 
		(173.325000,425.240000) -- 
		(173.359000,425.179000) -- 
		(173.473000,425.048000) -- 
		(173.645000,424.777000) -- 
		(173.662000,424.729000) -- 
		(173.737000,424.613000) -- 
		(173.753000,424.573000) -- 
		(173.785000,424.512000) -- 
		(173.842000,424.443000) -- 
		(173.852000,424.369000) -- 
		(173.842000,424.353000) -- 
		(173.832000,424.303000) -- 
		(173.820000,424.280000) -- 
		(173.829000,424.246000) -- 
		(173.829000,424.204000) -- 
		(173.797000,424.156000) -- 
		(173.788000,424.128000) -- 
		(173.764000,424.101000) -- 
		(173.685000,423.950000) -- 
		(173.628000,423.826000) -- 
		(173.616000,423.465000) -- 
		(173.633000,423.197000) -- 
		(173.603000,423.103000) -- 
		(173.553000,423.027000) -- 
		(173.509000,422.979000) -- 
		(173.409000,422.915000) -- 
		(172.841000,422.735000) -- 
		(172.550000,422.568000) -- 
		(172.493000,422.520000) -- 
		(172.438000,422.444000) -- 
		(172.410000,422.382000) -- 
		(172.351000,422.122000) -- 
		(172.320000,422.034000) -- 
		(172.271000,421.943000) -- 
		(172.203000,421.848000) -- 
		(172.086000,421.731000) -- 
		(172.045000,421.697000) -- 
		(171.957000,421.662000) -- 
		(171.745000,421.530000) -- 
		(171.657000,421.467000) -- 
		(171.617000,421.427000) -- 
		(171.569000,421.352000) -- 
		(171.554000,421.304000) -- 
		(171.529000,421.276000) -- 
		(171.457000,421.222000) -- 
		(171.441000,421.200000) -- 
		(171.424000,421.118000) -- 
		(171.457000,421.064000) -- 
		(171.514000,420.940000) -- 
		(171.572000,420.816000) -- 
		(171.539000,420.666000) -- 
		(171.532000,420.590000) -- 
		(171.548000,420.500000) -- 
		(171.548000,420.453000) -- 
		(171.529000,420.301000) -- 
		(171.476000,420.191000) -- 
		(171.403000,420.067000) -- 
		(171.305000,419.923000) -- 
		(171.297000,419.895000) -- 
		(171.175000,419.744000) -- 
		(171.159000,419.710000) -- 
		(171.126000,419.682000) -- 
		(171.101000,419.648000) -- 
		(171.028000,419.579000) -- 
		(170.922000,419.510000) -- 
		(170.767000,419.448000) -- 
		(170.735000,419.441000) -- 
		(170.636000,419.406000) -- 
		(170.457000,419.365000) -- 
		(170.269000,419.282000) -- 
		(170.188000,419.220000) -- 
		(170.082000,419.096000) -- 
		(170.066000,419.062000) -- 
		(170.050000,418.978000) -- 
		(170.042000,418.842000) -- 
		(170.010000,418.745000) -- 
		(169.953000,418.635000) -- 
		(169.928000,418.604000) -- 
		(169.766000,418.456000) -- 
		(169.741000,418.387000) -- 
		(169.737000,418.362000) -- 
		(169.754000,418.249000) -- 
		(169.807000,418.147000) -- 
		(169.889000,418.054000) -- 
		(170.044000,417.894000) -- 
		(170.389000,417.453000) -- 
		(170.413000,417.377000) -- 
		(170.403000,417.311000) -- 
		(170.364000,417.301000) -- 
		(170.291000,417.357000) -- 
		(170.201000,417.432000) -- 
		(170.144000,417.502000) -- 
		(170.119000,417.549000) -- 
		(170.037000,417.644000) -- 
		(169.939000,417.770000) -- 
		(169.922000,417.872000) -- 
		(169.856000,418.037000) -- 
		(169.815000,418.099000) -- 
		(169.758000,418.150000) -- 
		(169.685000,418.173000) -- 
		(169.619000,418.221000) -- 
		(169.431000,418.207000) -- 
		(169.366000,418.187000) -- 
		(169.276000,418.125000) -- 
		(169.236000,418.077000) -- 
		(169.219000,418.043000) -- 
		(169.195000,417.892000) -- 
		(169.212000,417.754000) -- 
		(169.286000,417.623000) -- 
		(169.416000,417.466000) -- 
		(169.663000,417.215000) -- 
		(169.974000,416.924000) -- 
		(170.072000,416.862000) -- 
		(170.169000,416.807000) -- 
		(170.300000,416.760000) -- 
		(170.439000,416.732000) -- 
		(170.529000,416.726000) -- 
		(170.611000,416.726000) -- 
		(170.750000,416.746000) -- 
		(170.893000,416.791000) -- 
		(171.183000,416.808000) -- 
		(171.232000,416.837000) -- 
		(171.272000,416.878000) -- 
		(171.280000,416.906000) -- 
		(171.304000,416.918000) -- 
		(171.305000,416.940000) -- 
		(171.280000,416.966000) -- 
		(171.230000,417.063000) -- 
		(171.239000,417.104000) -- 
		(171.279000,417.104000) -- 
		(171.394000,417.008000) -- 
		(171.533000,416.917000) -- 
		(171.607000,416.879000) -- 
		(171.729000,416.852000) -- 
		(171.758000,416.841000) -- 
		(171.810000,416.838000) -- 
		(171.872000,416.821000) -- 
		(171.974000,416.818000) -- 
		(172.138000,416.838000) -- 
		(172.562000,416.977000) -- 
		(172.602000,417.004000) -- 
		(172.660000,417.066000) -- 
		(172.684000,417.155000) -- 
		(172.717000,417.202000) -- 
		(172.724000,417.231000) -- 
		(172.765000,417.312000) -- 
		(172.797000,417.348000) -- 
		(172.851000,417.365000) -- 
		(172.891000,417.373000) -- 
		(172.985000,417.376000) -- 
		(173.067000,417.362000) -- 
		(173.194000,417.284000) -- 
		(173.230000,417.246000) -- 
		(173.312000,417.135000) -- 
		(173.321000,417.108000) -- 
		(173.387000,417.011000) -- 
		(173.485000,416.910000) -- 
		(173.525000,416.889000) -- 
		(173.665000,416.853000) -- 
		(173.868000,416.879000) -- 
		(173.950000,416.903000) -- 
		(174.015000,416.938000) -- 
		(174.040000,416.965000) -- 
		(174.068000,417.034000) -- 
		(174.113000,417.117000) -- 
		(174.153000,417.157000) -- 
		(174.219000,417.236000) -- 
		(174.300000,417.309000) -- 
		(174.415000,417.391000) -- 
		(174.488000,417.419000) -- 
		(174.610000,417.447000) -- 
		(174.643000,417.447000) -- 
		(174.684000,417.461000) -- 
		(174.880000,417.481000) -- 
		(175.218000,417.480000) -- 
		(175.314000,417.422000) -- 
		(175.378000,417.382000) -- 
		(175.443000,417.343000) -- 
		(175.489000,417.053000) -- 
		(175.517000,416.996000) -- 
		(175.567000,416.865000) -- 
		(175.616000,416.775000) -- 
		(175.912000,416.102000) -- 
		(175.928000,416.048000) -- 
		(175.953000,416.012000) -- 
		(176.085000,415.705000) -- 
		(176.241000,415.373000) -- 
		(176.282000,415.272000) -- 
		(176.273000,415.237000) -- 
		(176.314000,415.196000) -- 
		(176.355000,415.175000) -- 
		(176.371000,415.155000) -- 
		(176.421000,415.141000) -- 
		(176.453000,415.121000) -- 
		(176.650000,415.066000) -- 
		(177.046000,415.083000) -- 
		(177.181000,415.114000) -- 
		(177.315000,415.131000) -- 
		(177.466000,415.156000) -- 
		(177.727000,415.150000) -- 
		(178.454000,415.089000) -- 
		(178.519000,415.069000) -- 
		(178.736000,415.038000) -- 
		(179.100000,415.021000) -- 
		(179.189000,415.035000) -- 
		(179.255000,415.057000) -- 
		(179.499000,415.161000) -- 
		(179.646000,415.229000) -- 
		(179.817000,415.263000) -- 
		(179.956000,415.257000) -- 
		(179.997000,415.244000) -- 
		(180.022000,415.244000) -- 
		(180.063000,415.230000) -- 
		(180.095000,415.230000) -- 
		(180.266000,415.168000) -- 
		(180.332000,415.134000) -- 
		(180.365000,415.100000) -- 
		(180.438000,415.080000) -- 
		(180.578000,414.948000) -- 
		(180.626000,414.852000) -- 
		(180.660000,414.770000) -- 
		(180.705000,414.526000) -- 
		(180.722000,414.210000) -- 
		(180.610000,413.701000) -- 
		(180.577000,413.577000) -- 
		(180.542000,413.023000) -- 
		(180.550000,412.880000) -- 
		(180.542000,412.667000) -- 
		(180.559000,412.550000) -- 
		(180.547000,412.457000) -- 
		(180.543000,412.289000) -- 
		(180.571000,412.111000) -- 
		(180.537000,411.553000);
	\filldraw [draw=black, ultra thick, fill=blue]
		(-22.003000,-37.022000) -- 
		(-22.639000,-37.938000) -- 
		(-23.132000,-38.738000) -- 
		(-23.156000,-38.776000) -- 
		(-23.213000,-38.868000) -- 
		(-23.237000,-38.909000) -- 
		(-23.448000,-39.251000) -- 
		(-23.662000,-39.557000) -- 
		(-23.896000,-39.892000) -- 
		(-24.049000,-40.112000) -- 
		(-24.080000,-40.159000) -- 
		(-24.625000,-40.966000) -- 
		(-24.738000,-41.132000) -- 
		(-24.772000,-41.183000) -- 
		(-24.807000,-41.234000) -- 
		(-25.632000,-40.841000) -- 
		(-25.761000,-40.775000) -- 
		(-26.505000,-40.387000) -- 
		(-28.274000,-39.782000) -- 
		(-33.553000,-38.060000) -- 
		(-35.371000,-37.393000) -- 
		(-35.613000,-37.235000) -- 
		(-36.028000,-37.398000) -- 
		(-36.934000,-37.228000) -- 
		(-37.448000,-37.248000) -- 
		(-37.450000,-37.247000) -- 
		(-37.450000,-37.248000) -- 
		(-38.005000,-37.208000) -- 
		(-38.112000,-37.200000) -- 
		(-38.496000,-37.128000) -- 
		(-38.666000,-37.040000) -- 
		(-39.085000,-36.824000) -- 
		(-40.389000,-36.286000) -- 
		(-40.522000,-36.231000) -- 
		(-41.909000,-35.729000) -- 
		(-41.305000,-35.270000) -- 
		(-41.117000,-35.127000) -- 
		(-40.495000,-34.679000) -- 
		(-40.470000,-34.661000) -- 
		(-40.451000,-34.647000) -- 
		(-40.411000,-34.618000) -- 
		(-40.294000,-34.535000) -- 
		(-39.944000,-34.281000) -- 
		(-39.763000,-34.161000) -- 
		(-39.584000,-34.041000) -- 
		(-39.544000,-34.015000) -- 
		(-39.427000,-33.937000) -- 
		(-39.388000,-33.910000) -- 
		(-38.855000,-33.556000) -- 
		(-38.530000,-33.337000) -- 
		(-37.702000,-32.756000) -- 
		(-37.284000,-32.451000) -- 
		(-36.856000,-32.138000) -- 
		(-36.767000,-32.074000) -- 
		(-36.797000,-32.014000) -- 
		(-36.848000,-31.908000) -- 
		(-36.880000,-31.825000) -- 
		(-36.905000,-31.760000) -- 
		(-36.929000,-31.699000) -- 
		(-36.940000,-31.662000) -- 
		(-36.993000,-31.486000) -- 
		(-37.020000,-31.355000) -- 
		(-37.042000,-31.251000) -- 
		(-36.887000,-31.251000) -- 
		(-36.529000,-31.247000) -- 
		(-36.424000,-31.242000) -- 
		(-36.393000,-31.240000) -- 
		(-36.316000,-31.226000) -- 
		(-36.273000,-31.213000) -- 
		(-36.216000,-31.188000) -- 
		(-36.170000,-31.161000) -- 
		(-36.147000,-31.146000) -- 
		(-36.083000,-31.091000) -- 
		(-35.916000,-30.925000) -- 
		(-35.911000,-30.919000) -- 
		(-35.844000,-30.835000) -- 
		(-35.710000,-30.663000) -- 
		(-35.506000,-30.398000) -- 
		(-35.312000,-30.140000) -- 
		(-35.181000,-29.964000) -- 
		(-34.973000,-29.667000) -- 
		(-34.827000,-29.456000) -- 
		(-34.389000,-28.749000) -- 
		(-34.199000,-28.440000) -- 
		(-34.117000,-28.327000) -- 
		(-34.102000,-28.304000) -- 
		(-34.079000,-28.265000) -- 
		(-33.881000,-27.975000) -- 
		(-33.803000,-27.857000) -- 
		(-33.187000,-26.941000) -- 
		(-32.693000,-26.208000) -- 
		(-32.576000,-26.042000) -- 
		(-32.573000,-26.030000) -- 
		(-32.264000,-25.571000) -- 
		(-31.923000,-25.067000) -- 
		(-31.804000,-24.911000) -- 
		(-31.658000,-24.740000) -- 
		(-31.574000,-24.781000) -- 
		(-31.562000,-24.785000) -- 
		(-31.363000,-24.500000) -- 
		(-31.245000,-24.350000) -- 
		(-31.077000,-24.220000) -- 
		(-30.849000,-24.065000) -- 
		(-30.641000,-23.943000) -- 
		(-30.609000,-23.924000) -- 
		(-30.358000,-23.797000) -- 
		(-30.098000,-23.686000) -- 
		(-29.828000,-23.590000) -- 
		(-29.621000,-23.529000) -- 
		(-29.411000,-23.477000) -- 
		(-29.198000,-23.434000) -- 
		(-28.686000,-23.348000) -- 
		(-26.298000,-22.946000) -- 
		(-26.138000,-22.919000) -- 
		(-25.049000,-22.735000) -- 
		(-23.977000,-22.555000) -- 
		(-23.730000,-22.514000) -- 
		(-22.989000,-22.389000) -- 
		(-22.743000,-22.347000) -- 
		(-22.398000,-22.290000) -- 
		(-21.362000,-22.116000) -- 
		(-21.018000,-22.057000) -- 
		(-20.990000,-22.131000) -- 
		(-20.938000,-22.251000) -- 
		(-20.885000,-22.373000) -- 
		(-20.706000,-22.834000) -- 
		(-20.631000,-23.028000) -- 
		(-20.608000,-23.088000) -- 
		(-20.506000,-23.341000) -- 
		(-20.211000,-24.076000) -- 
		(-20.129000,-24.275000) -- 
		(-20.002000,-24.585000) -- 
		(-19.928000,-24.766000) -- 
		(-19.888000,-24.881000) -- 
		(-19.806000,-25.083000) -- 
		(-19.673000,-25.417000) -- 
		(-19.613000,-25.576000) -- 
		(-19.399000,-26.100000) -- 
		(-19.339000,-26.262000) -- 
		(-19.214000,-26.575000) -- 
		(-19.168000,-26.692000) -- 
		(-19.011000,-27.069000) -- 
		(-18.860000,-27.463000) -- 
		(-18.859000,-27.465000) -- 
		(-18.743000,-27.880000) -- 
		(-18.614000,-28.451000) -- 
		(-18.601000,-28.565000) -- 
		(-18.586000,-28.694000) -- 
		(-18.554000,-28.990000) -- 
		(-18.545000,-29.113000) -- 
		(-18.535000,-29.247000) -- 
		(-18.543000,-29.583000) -- 
		(-18.550000,-29.839000) -- 
		(-18.591000,-30.630000) -- 
		(-18.666000,-30.984000) -- 
		(-18.764000,-31.443000) -- 
		(-18.935000,-31.984000) -- 
		(-18.985000,-32.139000) -- 
		(-19.250000,-32.718000) -- 
		(-19.650000,-33.436000) -- 
		(-19.693000,-33.501000) -- 
		(-20.012000,-33.970000) -- 
		(-20.044000,-34.020000) -- 
		(-20.142000,-34.164000) -- 
		(-20.176000,-34.211000) -- 
		(-20.187000,-34.227000) -- 
		(-20.957000,-35.408000) -- 
		(-21.052000,-35.540000) -- 
		(-21.302000,-35.887000) -- 
		(-21.496000,-36.202000) -- 
		(-22.003000,-37.022000);
	\fill [fill=black] (-41.800000, 144.900000) circle (5) node [above=50, scale=6] {\color{white}Benton};
	\fill [fill=black] (-32.100000, -34.000000) circle (5) node [below left, scale=6] {\begin{tabular}{r}Bossier\cr City\cr\end{tabular}};
	\fill [fill=white] (195.900000, -17.300000) circle (5) node [right, scale=6] {Haughton};
	\fill [fill=white] (3.200000, 355.000000) circle (5) node [above, scale=6] {\color{white}Plain Dealing};
\end{tikzpicture}

			\input{Context_Contents}
		}
		&
		\adjustbox{valign=T}{
		}
		\cr
		\end{tabular}
		\end{center}

		\vskip -3.7in		
		{\Large Data Sources}
			
		Demographic Data:  
		
		\qquad 2017 American 
		
		\qquad \qquad Community Survey,
			
		\qquad US Census Bureau
		
		Map Coordinates:  
		
		\qquad TIGER/Line Shapefile, 2017, 
		
		\qquad U.S. Census Bureau	

	\end{block}

\vskip -24pt

% Connect the Dots
	\begin{block}{}
		\begin{tikzpicture}
			\node (0,0) [text width=28cm,fill=gold,rounded corners=.5cm, below right,drop shadow]
			{
				\centerline{
				\vrule width 0pt height 40pt depth 12pt
				\Large
				What \ Stories \ do \ These \ Maps \ Tell \ ?
				}
			};
		\end{tikzpicture}
		
		\vskip 48pt
		
		\begin{center}
		\begin{tabular}{p{4.5in}p{.5in}p{4in}}		
%		\adjustbox{valign=T}{
			\centerline{\Large Education}

		\vskip 6pt
		
		Percentage of adults over 25 who graduated from college
%		}
		&&
%		\adjustbox{valign=T}{
			\centerline{\Large Income}
		
		\vskip 6pt
			
			Average household income
%		}
		\end{tabular}
		\end{center}
		
		\vskip -36pt

		\begin{center}
		\begin{tabular}{p{5in}p{4in}}		
		\adjustbox{valign=T}{
			\input{Education_Contents}
		}
		&
		\adjustbox{valign=T}{
			\input{Income_Contents}
		}
		\end{tabular}
		\end{center}

	\end{block}

	


	\end{column}

% Third Column
	\begin{column}{.3\linewidth} {\color{white} xxx}

% Connect the Dots
	\begin{block}{}
		\begin{tikzpicture}
			\node (0,0) [text width=28cm,fill=blue,rounded corners=.5cm, below right,drop shadow]
			{
				\centerline{
				\vrule width 0pt height 40pt depth 12pt
				\Large\color{gold}
				Make \ Your \ Map, \  Tell  \ Your \ Story
				}
			};
		\end{tikzpicture}
	\end{block}

	\begin{block}{}
		\begin{tikzpicture}
			\node (0,0) [text width=28cm,fill=gold,rounded corners=.5cm, below right,drop shadow]
			{
				\centerline{
				\vrule width 0pt height 40pt depth 12pt
				\Large
				Option \ 1: \  Color \ by \ Numbers
				}
			};
		\end{tikzpicture}
		
		\
		
		\begin{itemize}
			\setlength\itemsep{0.4em}
			\item Decide which story you want to tell.
			\vskip 4pt
			\begin{itemize}
				\setlength\itemsep{0.2em}
				\item Population Density
				\item Immigrants
				\item Median Age
				\item Black or African American
				\item American Indian and Alaska Native
				\item Asian
				\item Hispanic or Latino Origin
				\item Speak Language Other than English at Home
				\item Bachelor's Degree
				\item Median Income
			\end{itemize}
			\item Get a map, data sheet, and colored pencils.
			\item Decide how you will use color.
			\vskip 4pt
			\begin{itemize}
				\setlength\itemsep{0.2em}
				\item How many colors?
				\item Which colors?
				\item What does each color represent?
				\item If using shading, what does the scale represent?
			\end{itemize}
			\item Choose symbols and text labels.
		\end{itemize}
		
	\end{block}

	\begin{block}{}
		\begin{tikzpicture}
			\node (0,0) [text width=28cm,fill=gold,rounded corners=.5cm, below right,drop shadow]
			{
				\centerline{
				\vrule width 0pt height 40pt depth 12pt
				\Large
				Option \ 2: \ Draw \ Your \ Own \ Map
				}
			};
		\end{tikzpicture}
		
		\
		
		\begin{itemize}
			\setlength\itemsep{0.4em}
			\item Decide which story you want to tell.
			\item Get a blank sheet and colored pencils.
			\item Draw your neighborhood, yard, house floor plan, \dots .
			\item Decide how you will use color. [See above.]
			\item Choose symbols and text labels.
		\end{itemize}
	\end{block}

			
			

% Contact Me
	\begin{block}{}
		\begin{tikzpicture}
			\node (0,0) [text width=28cm,fill=blue,rounded corners=.5cm, below right,drop shadow]
			{
				\Large
				\centerline{
				\vrule width 0pt height 40pt depth 12pt
				\color{gold}
					Contact \ Us
				}
			};
		\end{tikzpicture}

		\

		\setlength{\baselineskip}{1.2\baselineskip}
		Louisiana School for Math, Science, and the Arts
		
		Natchitoches, Louisiana
	
		{\tt\bf admissions@lsmsa.edu}


	\end{block}



	\end{column}

	\end{columns}


  \end{frame}
%	\end{Large}
\end{document}

% Connect the Dots
	\begin{block}{}
		\begin{tikzpicture}
			\node (0,0) [text width=28cm,fill=lightgreen,rounded corners=.5cm, below right,drop shadow]
			{
				\centerline{
				\vrule width 0pt height 40pt depth 12pt
				\Large
				Add \ Context \ ($\approx 20,000$ Vertices)
				}
			};
		\end{tikzpicture}

		\

			\input{Income_Contents}
		
%		\begin{center}
%		\input{Context_Contents}
%		\end{center}

	\end{block}




%%%%%%%%%%%%%%%%%%%%%%%%%%%%%%%%%%%%%%%%%%%%%%%%%%%%%%%%%%%%%%%%%%%%%%%%%%%%%%%%%%%%%%%%%%%%%%%%%%%%
%%% Local Variables: 
%%% mode: latex
%%% TeX-PDF-mode: t
