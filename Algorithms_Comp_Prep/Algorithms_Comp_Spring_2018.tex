%%%%%%%%%%%%%%%%%%%%
\section{Spring 2018}
% Copying Completed
% Index Completed

\subsection{Short Questions (Answer six of seven.)}

\begin{enumerate}
	% S18 #S1
	\item A problem with size $n$ follows a typical divide-and-conquer approach to have its time complexity of
	$$T(n) = T\left(\frac{n}{4} \right) + T \left( \frac{3n}{4} \right) + c \cdot n$$
	Solve $T(n)$.  (Show your work.)
	
	\index{Time Complexity!S18 \#S1}
	\index{Divide and Conquer!S18 \#S1}
	\index{Recurrence!S18 \#S1}
	
	% S18 #S2
	\item Given that for an open-address hash table with load factor $\displaystyle \alpha = \frac{n}{m} < 1$, the expected number of probes in unsuccessful search under uniform hashing is at most $\displaystyle \frac{1}{1-\alpha}$, how do you prove the expected number of probes in a successful probe under uniform hashing being at most $\displaystyle \frac{1}{\alpha} \ln \left( \frac{1}{1-\alpha}\right)$?  (Just give a proof sketch, explaining how many probes are needed to locate existing keys.)
	\index{Hash!Open Address!S18 \#S2}
	\index{Hash!Load Factor!S18 \#S2}
	\index{Hash!Probes!S18 \#S2}
	\index{Hash!Uniform Hashing!S18 \#S2}

	% S18 #S3
	\item (Lemma 16.2, \S 16.3, page 433) Sketch a proof of the Lemma below, using the tree provided.  
	
	Let $C$ be an alphabet in which each character $c \in C$ has frequency $c.freq$.  Let $x$ and $y$ be two characters in $C$ having the lowest frequencies.  Then there exists an optimal prefix code for $C$ in which the codewords for $x$ and $y$ have the same length and differ only in the last bit. 
	
\hfil	\begin{tikzpicture}[x=15mm, y=10mm]
		\node [circle, draw] (T) {$T$};
		\node [rectangle, draw, below right=of T] (x) {$x$}; 
		\node [circle, draw, below left=of T] (p) { \ };
		\node [circle, draw, below right=of p] (q) { \ };
		\node [rectangle, draw, below left=of 	p] (y) {$y$};
		\node [rectangle, draw, below left=of q] (a) {$a$};
		\node [rectangle, draw, below right=of q] (b) {$b$};
		\foreach \from/\to in {T/x, T/p, p/q, p/y, q/a, q/b}{
			\draw (\from) -- (\to);
		}
	\end{tikzpicture}
	
	\index{Huffman Code!Optimal Prefix Code!S18 \#S3}
	\index{Huffman Code!Codewords!S18 \#S3}

	% S18 #S4
	\item The Dijkstra's algorithm (DS) solves the single-source shortest-path problem in a weighted graph $G = (V,E)$ without negative weighted edges or cycles, by edge relaxation at one vertex at a time until all vertices are examined.  Given the graph $G$ below, follow DS to find the shortest paths from vertex $v_1$ to all other vertices, with all predecessor edges shaded and estimated distance values from $v_1$ to all vertices provided at the end.  Also list the sequence of vertices at which relaxation takes place.  
	
	What is the time complexity of DS for a general graph $G = (V,E)$ when candidate vertices are kept in an array?
	
\hfil	\begin{tikzpicture}[node distance = 2cm and 4cm]
		\node [circle, draw, label=above:$v_1$] (v1) {};
		\node [circle, draw, label=above:$v_2$, right=of v1] (v2) {};
		\node [circle, draw, label=above:$v_3$, right=of v2] (v3) {};
		\node [circle, draw, label=below:$v_4$, below= of v1] (v4) {};
		\node [circle, draw, label=below:$v_5$, right=of v4] (v5) {};
		\node [circle, draw, label=below:$v_6$, right=of v5] (v6) {};
		\foreach \from/\to/\label in {v1/v2/1, v1/v4/4, v2/v3/9, v4/v5/3, v5/v2/7}{
			\draw [-triangle 60] (\from) -- (\to) node [midway, rectangle, fill=white] {\label};
		}
		\foreach \from/\to/\label in {v1/v5/11, v2/v4/5}{
			\draw [-triangle 60] (\from) -- (\to) node [pos=0.7, rectangle, fill=white] {\label};
		}
		\foreach \from/\to/\label in {v3/v6/8, v6/v3/2}{
			\path [-triangle 60] (\from) edge [bend left] node [midway, rectangle, fill=white] {\label} (\to);
		}
		\foreach \from/\to/\label in {v2/v6/5}{
			\draw [triangle 60-triangle 60] (\from) -- (\to) node [midway, rectangle, fill=white] {\label};
		}
 	\end{tikzpicture}
	
	\index{Dijkstra's Algorithm!S18 \#S4}
	\index{Time Complexity!S18 \#S4}
	
	% S18 #S5
	\item \begin{enumerate}[label=\alph*.]
		\item Define height balanced binary tree
		\item Write a pseudo code to determine whether a tree is height balanced?
		\item Obtain a tight bound of your algorithm.
	\end{enumerate}
	
	\index{Binary Search Tree!Height Balanced!S18 \#S5}

\end{enumerate}

\subsection{Long Questions (Answer three of four.)}

\begin{enumerate}

	% S18 #L1
	\item Given the initial B-tree with the minimum node degree of $t=3$ below, show the results 
	
	\begin{enumerate}[label=\alph*.]
		\item After deleting the key of $M_2$, 
		\item Followed by inserting the key of $L$, 
		\item Then by deleting the key of $J_2$, 
		\item Then by inserting the key of $O_1$ with $O < O_1 < O_2$, and 
		\item Then by deleting $K$.  
	\end{enumerate}
	
	(Show the result after every deletion and after every insertion.)
	
\
	
\hfil	\begin{tikzpicture}[node distance=8 mm and 8 mm]
		\node [rectangle, draw] (a) at (0,0) {$M$};
		\node [rectangle, draw] (b) at (-3,-1) {$J K$};
		\node [rectangle, draw] (c) at (3,-1) {$N O$};
		\node [rectangle, draw, below left=of b] (d) {$A B$};
		\node [rectangle, draw, below=of b] (e) {$J_2 J_3$};
		\node [rectangle, draw, below right=of b] (f) {$K_4 K_6$};
		\node [rectangle, draw, below left=of c] (g) {$M_2 M_4$};
		\node [rectangle, draw, below=of c] (h) {$N_1 N_2 N_3$};
		\node [rectangle, draw, below right=of c] (i) {$O_2 O_3$};
		\foreach \from/\to in {a/b, a/c, b/d, b/e, b/f, c/g, c/h, c/i}
			\draw (\from) -- (\to);
	\end{tikzpicture}
	
	\index{Balanced Search Tree!S18 \#L1}

\

	% S18 #L2
	\item A Fibonacci min-heap relies on the procedure of CONSOLIDATE to merge min-heaps in the root list upon the operation of extracting the minimum node.  Given the following Fibonacci min-heap, show every consolidation step and the final heap result after $H.min$ is extracted, with the aid of $A$.  
	
	\index{Heaps!Fibonacci Min Heap!S18 \#L2}
	
\hfil \begin{tikzpicture}[x=13mm, y=13mm]
	\node [circle, draw] (7) at (0,0) {7};
	\node [circle, draw] (24) at (-1,-1) {24};
	\node [circle, draw] (17) at (0,-1) {17};
	\node [circle, draw] (26) at (-2,-2) {26};
	\node [circle, draw] (46) at (-1,-2) {46};
	\node [circle, draw] (30) at (0,-2) {30};
	\node [circle, draw] (35) at (-2,-3) {35};
	\node [circle, draw] (21) at (2,0) {21};
	\node [circle, draw] (18) at (3,0) {18};
	\node [circle, draw] (52) at (4,0) {52};
	\node [circle, draw] (39) at (3,-1) {39};
	\foreach \from/\to in {7/17, 7/24, 7/21, 24/26, 24/46, 17/30, 26/35, 21/18, 18/52, 18/39}
		\draw (\from) -- (\to);
	\node (a) at (-1,1) {$H.min$};
	\draw [-triangle 60] (a) -- (7);
	\node (b) at (2,1) {A};
	\node (c0) at (2.4,1.4) {0};
	\node (c1) at (2.8,1.4) {1};
	\node (c2) at (3.2,1.4) {2};
	\node (c3) at (3.6,1.4) {3};
	\draw (2.2,0.8) rectangle (3.8,1.2);
	\draw (2.6,0.8) -- (2.6,1.2);
	\draw (3.0,0.8) -- (3.0,1.2);
	\draw (3.4,0.8) -- (3.4,1.2);
\end{tikzpicture}	

	% S18 #L3
	\item \begin{enumerate}[label=\alph*.]
		\item To what extent the asymptotic upper bound and lower bound provide insight on running time of an algorithm.
		\item Compare and contrast asymptotic tight bound to the average running time of an algorithm.
		\item Consider the pseudo code of an algorithm given below.  
		\begin{enumerate}
			\item What the value $K$ in line 4 denote?
			\item What the value $m$ in line 8 denote?
			\item When the algorithm terminates, what does the value $m+K$ in line 9 denote?
			\item Find the asymptotic tight bound of Algorithm Test below.
		\end{enumerate}
	\end{enumerate}
	
	\verb|AlgorithmTest(n)|
	
	\begin{lstlisting}[label=Algorithm Test, mathescape=True, numbers=left]
$K=0$
for $i=1$ to $n$
	for $j=1$ to $i$
		$K = K+1$
$m=0$
for $i=1$ to $n-1$
	for $j=i+1$ to $n$
		$m = m+1$
return $(m+K)$
	\end{lstlisting}
	
	\index{Time Complexity!S18 \#L3}
	\index{Time Complexity!Big-$\Theta$!S18 \#L3}
	\index{Follow-the-Bouncing-Ball!S18 \#L3}
	
	% S18 #L4
	
	\item \begin{enumerate}[label=\alph*.]
		\item Define the following classes of a decision problem:  P, NP, and NP-completeness.
		\item Consider the 0-1 knapsack problem with $n$ objects each with its respective pre-defined profit.  The objective is to maximize the total profit that can be accommodated into a container of capacity $W$.  Define appropriate notations for weight and profit of objects, formulate the problem.
		\item Convert of the problem that you have defined in (b) into a decision problem.
		\item Show the problem that you have defined in (c) belongs to NP-class.
		\item Does the problem in (d) belong to the P-class or NP-completeness. (Justify your answer.)
		\item If principle of optimality be applicable to solve the problem defined in (c), formulate it.  Otherwise, explain why not.  
		\item What would be your explanation, if 0-1 knapsack problem is solved by dynamic programming in polynomial time?
	\end{enumerate}
	
	\index{NP!NP-Complete!S18 \#L4}
	\index{Knapsack Problem!S18 \#L4}
	\index{Decision Problem!S18 \#L4}
	\index{Principle of Optimality!S18 \#L4}
	\index{Dynamic Programming!S18 \#L4}
	
\end{enumerate}

