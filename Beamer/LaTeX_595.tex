\documentclass{beamer}
\usepackage{array}
\usepackage{mathtools}
\usepackage[normalem]{ulem}
\usepackage{tikz}
\usetikzlibrary{arrows}
\usetikzlibrary{overlay-beamer-styles}
\usetikzlibrary{positioning}
\usetikzlibrary{shapes, shapes.geometric, positioning}
\usetikzlibrary{calc,intersections}
\usepackage{tkz-euclide}
\usetikzlibrary{through}


\usepackage{pgfmath}
\usepackage{pgfplots}



\usepackage{amsmath}
\usepackage{amsfonts}
\usepackage{verbatim}
\usepackage{arcs}
\usepackage{setspace}
\usepackage{hyperref}

%\usepackage{enumerate}
%\usepackage{enumitem}
%\setlist{noitemsep}

\usepackage{listings}
\lstset{language=python}

\usepackage{array, multirow, bigdelim}
\usepackage{makeidx}

\setbeamertemplate{navigation symbols}{}
\usetheme{CambridgeUS}
\usecolortheme{Vermillion}

\usepackage{adjustbox}


%\DeclareMathOperator*{\argmax}{argmax}
\DeclareMathOperator*{\argmax}{arg\,max}
\DeclareMathOperator*{\argmin}{arg\,min}


%%%%%
\title{\LaTeX: What? Why? and How?}
\subtitle{CSCE 595 Presentation}
\author{Brad Burkman}
\date{31 January 2020}

\begin{document}
\logo{\includegraphics[height=1.0cm]{AcademicHorizontal_0.jpg}\hspace{12pt}}
\newcommand{\nologo}{\setbeamertemplate{logo}{}} 


\begin{frame}[t]
	\Large
	\maketitle
\end{frame}

\begin{frame}[t]
	\Large
	\tableofcontents
\end{frame}





%%%%%%%%%%%%
%
% End
%
%%%%%%%%%%%

\end{document}
%%%%%%%%%
%
% Generic structures
%
%%%%%%%%%%%



\begin{frame}[t]
	\frametitle{}
\Large

\end{frame}


\begin{enumerate}
	\item 
\end{enumerate}


\begin{itemize}
	\item 
\end{itemize}


\begin{tabular}{@{}rl}
	Given: & \cr
	Prove: & \cr
\end{tabular}


\begin{tikzpicture}[x=10mm, y=10mm]
	\coordinate () at ();
	\path () node [] {$$};
\end{tikzpicture}


\begin{tabular}{*5{@{\hspace{5pt}}>{$\displaystyle}r<{$\vrule width 0pt height 12pt depth 4pt}}}

\end{tabular}

